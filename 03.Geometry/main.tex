%\pdfminorversion=4
\documentclass[handout,fleqn,aspectratio=169]{beamer}

% when making printed slides
\usepackage{pgfpages}
\pgfpagesuselayout{resize to}[a4paper,landscape,border shrink=5mm]

\usepackage[english]{babel}
\usepackage{tikz}
\usepackage{courier}
\usepackage{array}
\usepackage{bold-extra}
%\usepackage{minted}
\usepackage[thicklines]{cancel}
\usepackage{fancyvrb}
\usepackage{kotex}
\usepackage{paralist}
\usepackage{collectbox}
\usepackage{bm}

\usepackage{mathrsfs}
\usepackage[reqno,disallowspaces]{mathtools}  % imports amsmath
\usepackage{amsfonts} %for Y&Y BSR AMS fonts
\usepackage{amssymb}
\usepackage{amscd}
%\usepackage{tikz,lipsum,lmodern}
\usepackage[most]{tcolorbox}
\usepackage{verbatim}
\mode<presentation>
{
  \usetheme{default}
  \usecolortheme{default}
  \usefonttheme{default}
  \setbeamertemplate{navigation symbols}{}
  \setbeamertemplate{caption}[numbered]
  \setbeamertemplate{footline}[frame number]  % or "page number"
  \setbeamercolor{frametitle}{fg=yellow}
  \setbeamercolor{footline}{fg=black}
} 

\setbeamercolor{block body alerted}{bg=alerted text.fg!10}
\setbeamercolor{block title alerted}{bg=alerted text.fg!20}
\setbeamercolor{block body}{bg=structure!10}
\setbeamercolor{block title}{bg=structure!20}
\setbeamercolor{block body example}{bg=green!10}
\setbeamercolor{block title example}{bg=green!20}
\setbeamertemplate{blocks}[rounded][shadow]

\xdefinecolor{dianablue}{rgb}{0.18,0.24,0.31}
\xdefinecolor{darkblue}{rgb}{0.1,0.1,0.7}
\xdefinecolor{darkgreen}{rgb}{0,0.5,0}
\xdefinecolor{darkgrey}{rgb}{0.35,0.35,0.35}
\xdefinecolor{darkorange}{rgb}{0.8,0.5,0}
\xdefinecolor{darkred}{rgb}{0.7,0,0}
\definecolor{darkgreen}{rgb}{0,0.6,0}
\definecolor{mauve}{rgb}{0.58,0,0.82}

\usetikzlibrary{shapes.callouts}

\makeatletter
\setbeamertemplate{footline}
{
  \leavevmode%
  \hbox{%
  \begin{beamercolorbox}[wd=.333333\paperwidth,ht=2.25ex,dp=1ex,center]{author in head/foot}%
    \usebeamerfont{author in head/foot}\insertsection
  \end{beamercolorbox}%
  \begin{beamercolorbox}[wd=.333333\paperwidth,ht=2.25ex,dp=1ex,center]{title in head/foot}%
    \usebeamerfont{title in head/foot}\insertsubsection
  \end{beamercolorbox}%
  \begin{beamercolorbox}[wd=.333333\paperwidth,ht=2.25ex,dp=1ex,right]{date in head/foot}%
    \usebeamerfont{date in head/foot}
    \insertshortdate{}\hspace*{2em}
    \insertframenumber{} / \inserttotalframenumber\hspace*{2ex} 
  \end{beamercolorbox}}%

  \vskip0pt%
}
\makeatother


\title[]{Lecture 3: Analytic Geometry}
\author{Yi, Yung (이융)}
\institute{Mathematics for Machine Learning\\ \url{https://yung-web.github.io/home/courses/mathml.html}
\\KAIST EE}
\date{\today}

%%%%%%%%%%%% real, integer notation
\newcommand{\real}{{\mathbb R}}
\newcommand{\realn}{{\mathbb R}^{n}}
\newcommand{\realm}{{\mathbb R}^{m}}
\newcommand{\realD}{{\mathbb R}^{D}}
\newcommand{\realM}{{\mathbb R}^{M}}
\newcommand{\realN}{{\mathbb R}^{N}}
\newcommand{\realnn}{{\mathbb R}^{n \times n}}
\newcommand{\realmm}{{\mathbb R}^{m \times m}}
\newcommand{\realmn}{{\mathbb R}^{m \times n}}
\newcommand{\realnm}{{\mathbb R}^{n \times m}}
\newcommand{\realDM}{{\mathbb R}^{D \times M}}
\newcommand{\realMD}{{\mathbb R}^{M \times D}}
\newcommand{\complex}{{\mathbb C}}
\newcommand{\integer}{{\mathbb Z}}
\newcommand{\natu}{{\mathbb N}}


%%% set, vector, matrix
\newcommand{\set}[1]{\ensuremath{\mathcal #1}}
\newcommand{\sets}[1]{\ensuremath{\{#1 \}}}
\renewcommand{\vec}[1]{\bm{#1}}
\newcommand{\mat}[1]{\bm{#1}}

%%%% vector
\def\vx{\vec{x}}
\def\vy{\vec{y}}
\def\vz{\vec{z}}
\def\vf{\vec{f}}
\def\ve{\vec{e}}
\def\vr{\vec{r}}
\def\vb{\vec{b}}
\def\vc{\vec{c}}
\def\vd{\vec{d}}
\def\vm{\vec{m}}
\def\vu{\vec{u}}
\def\vv{\vec{v}}
\def\vw{\vec{w}}
\def\vX{\vec{X}}
\def\vY{\vec{Y}}
\def\vZ{\vec{Z}}
\def\vth{\vec{\theta}}
\def\vmu{\vec{\mu}}
\def\vnu{\vec{\nu}}
\def\vlam{\vec{\lambda}}
\def\vep{\vec{\epsilon}}
\def\vpi{\vec{\pi}}
\def\vphi{\vec{\phi}}
\def\vxi{\vec{\xi}}
\def\valpha{\vec{\alpha}}
\def\vgamma{\vec{\gamma}}

%%%% Well-used matrices
\def\mA{\mat{A}}
\def\mB{\mat{B}}
\def\mC{\mat{C}}
\def\mD{\mat{D}}
\def\mI{\mat{I}}
\def\mJ{\mat{J}}
\def\mK{\mat{K}}
\def\mE{\mat{E}}
\def\mP{\mat{P}}
\def\mQ{\mat{Q}}
\def\mU{\mat{U}}
\def\mV{\mat{V}}
\def\mR{\mat{R}}
\def\mS{\mat{S}}
\def\mX{\mat{X}}
\def\msig{\mat{\Sigma}}
\def\mPhi{\mat{\Phi}}


\usepackage{amsmath}
%%%%% vector caculus useful macro
% ...\d, which typesets a derivative. ex: \d{y}{x}, instead of \frac{dx}{dy}.
\renewcommand{\d}[2]{\frac{\text{d} #1}{\text{d} #2}}


% ...similar for double-derivatives. ex: \dd{y}{x}.
\newcommand{\dd}[2]{\frac{\text{d}^2 #1}{\text{d} #2^2}}

% ...similar for partial derivatives. ex: \pd{y}{x}.
\newcommand{\pd}[2]{\frac{\partial #1}{\partial #2}}


% ...similar for partial double derivatives. ex: \pdd{y}{x}.
\newcommand{\pdd}[2]{\frac{\partial^2 #1}{\partial #2^2}}
% pdd with argument
\newcommand{\pdda}[3]{\frac{\partial^2 #1}{\partial #2 \partial #3}}

\usepackage{xparse}

%%%% caligraphic fonts
\def\cL{\ensuremath{{\cal L}}}
\def\cN{\ensuremath{{\cal N}}}
\def\cD{\ensuremath{{\cal D}}}
\def\cC{\ensuremath{{\cal C}}}
\def\cX{\ensuremath{{\cal X}}}
\def\cY{\ensuremath{{\cal Y}}}

%%% big parenthesis
\def\Bl{\Bigl}
\def\Br{\Bigr}
\def\lf{\left}
\def\ri{\right}


%%% floor notations
\newcommand{\lfl}{{\lfloor}}
\newcommand{\rfl}{{\rfloor}}
\newcommand{\floor}[1]{{\lfloor #1 \rfloor}}

%%% gradient
\newcommand{\grad}[1]{\nabla #1}
\newcommand{\hess}[1]{\text{H} #1}

%%% definition
%\newcommand{\eqdef}{\ensuremath{\triangleq}}
\newcommand{\eqdef}{\ensuremath{:=}}
%%% imply
\newcommand{\imp}{\Longrightarrow}



\newcommand{\separator}{
%  \begin{center}
    \par\noindent\rule{\columnwidth}{0.3mm}
%  \end{center}
}

\newcommand{\mynote}[1]{{\it \color{red} [#1]}}







%%% equation alignment
\newcommand{\aleq}[1]{\begin{align*}#1\end{align*}}

%%%%%%%%%%%%%%%% colored emphasized font, blanked words

\newcommand{\empr}[1]{{\color{red}\emph{#1}}}
\newcommand{\empb}[1]{{\color{blue}\emph{#1}}}
\newcommand{\redf}[1]{{\color{red} #1}}
\newcommand{\bluef}[1]{{\color{blue} #1}}
\newcommand{\grayf}[1]{{\color{gray} #1}}
\newcommand{\magenf}[1]{{\color{magenta} #1}}
\newcommand{\greenf}[1]{{\color{green} #1}}
\newcommand{\cyanf}[1]{{\color{cyan} #1}}
\newcommand{\orangef}[1]{{\color{orange} #1}}

\newcommand{\blk}[1]{\underline{\mbox{\hspace{#1}}}}


\newcommand{\redblk}[1]{\framebox{\color{red} #1}}
\newcommand{\redblank}[2]{\framebox{\onslide<#1->{\color{red} #2}}}
\newcommand{\blueblk}[1]{\framebox{\color{blue} #1}}
\newcommand{\blueblank}[2]{\framebox{\onslide<#1->{\color{blue} #2}}}



\makeatletter
\newcommand{\mybox}{%
    \collectbox{%
        \setlength{\fboxsep}{1pt}%
        \fbox{\BOXCONTENT}%
    }%
}
\makeatother

\makeatletter
\newcommand{\lecturemark}{%
    \collectbox{%
        \setlength{\fboxsep}{1pt}%
        \fcolorbox{red}{yellow}{\BOXCONTENT}%
    }%
}
\makeatother

\newcommand{\mycolorbox}[1]{
\begin{tcolorbox}[colback=red!5!white,colframe=red!75!black]
#1
\end{tcolorbox}
}
%%%% figure inclusion
\newcommand{\mypic}[2]{
\begin{center}
\includegraphics[width=#1\textwidth]{#2}
\end{center}
}

\newcommand{\myinlinepic}[2]{
\makebox[0cm][r]{\raisebox{-4ex}{\includegraphics[height=#1]{#2}}}
}




%%%% itemized and enumerated list
\newcommand{\bci}{\begin{compactitem}}
\newcommand{\eci}{\end{compactitem}}
\newcommand{\bce}{\begin{compactenum}}
\newcommand{\ece}{\end{compactenum}}


%%%% making 0.5/0.5 two columns
%%%% how to use: first number: length of separation bar
% \mytwocols{0.6}
% {
% contents in the left column
% }
% {
% contents in the right column
% }
%%%%

\newcommand{\mytwocols}[3]{
\begin{columns}[T] \column{.499\textwidth} #2 \column{.001\textwidth} \rule{.3mm}{{#1}\textheight} \column{.499\textwidth} #3 \end{columns}}

\newcommand{\mythreecols}[4]{
\begin{columns}[T] \column{.31\textwidth} #2 \column{.001\textwidth} \rule{.3mm}{{#1}\textheight} \column{.31\textwidth} #3 \column{.001\textwidth} \rule{.3mm}{{#1}\textheight} \column{.31\textwidth} #4  \end{columns}}

\newcommand{\mysmalltwocols}[3]{
\begin{columns}[T] \column{.4\textwidth} #2 \column{.001\textwidth} \rule{.3mm}{{#1}\textheight} \column{.4\textwidth} #3 \end{columns}}

%%%% making two columns with customized ratios
%%%% how to use: 
%first parameter: length of separation bar
%second parameter: ratio of left column
%third parameter: ratio of right column
% \mytwocols{0.6}{0.7}{0.29}
% {
% contents in the left column
% }
% {
% contents in the right column
% }
%%%%
\newcommand{\myvartwocols}[5]{
\begin{columns}[T] \column{#2\textwidth} {#4} \column{.01\textwidth} \rule{.3mm}{{#1}\textheight} \column{#3\textwidth} {#5} \end{columns}}

%%% making my block in beamer
%%% first parameter: title of block
%%% second parameter: contents of block
\newcommand{\myblock}[2]{
\begin{block}{#1} {#2}  \end{block}}

%%% independence notation
\newcommand{\indep}{\perp \!\!\! \perp}

%%%% probability with different shapes (parenthesis or bracket) and different sizes
%%% `i' enables us to insert the subscript to the probability 
\newcommand{\bprob}[1]{\mathbb{P}\Bl[ #1 \Br]}
\newcommand{\prob}[1]{\mathbb{P}[ #1 ]}
\newcommand{\cbprob}[1]{\mathbb{P}\Bl( #1 \Br)}
\newcommand{\cprob}[1]{\mathbb{P}( #1 )}
\newcommand{\probi}[2]{\mathbb{P}_{#1}[ #2 ]}
\newcommand{\bprobi}[2]{\mathbb{P}_{#1}\Bl[ #2 \Br]}
\newcommand{\cprobi}[2]{\mathbb{P}_{#1}( #2 )}
\newcommand{\cbprobi}[2]{\mathbb{P}_{#1}\Bl( #2 \Br)}

%%%% expectation with different shapes (parenthesis or bracket) and different sizes
%%% `i' enables us to insert the subscript to the expectation
\newcommand{\expect}[1]{\mathbb{E}[ #1 ]}
\newcommand{\cexpect}[1]{\mathbb{E}( #1 )}
\newcommand{\bexpect}[1]{\mathbb{E}\Bl[ #1 \Br]}
\newcommand{\cbexpect}[1]{\mathbb{E}\Bl( #1 \Br)}
\newcommand{\bbexpect}[1]{\mathbb{E}\lf[ #1 \ri]}
\newcommand{\expecti}[2]{\mathbb{E}_{#1}[ #2 ]}
\newcommand{\bexpecti}[2]{\mathbb{E}_{#1}\Bl[ #2 \Br]}
\newcommand{\bbexpecti}[2]{\mathbb{E}_{#1}\lf[ #2 \ri]}

%%%% variance
\newcommand{\var}[1]{\text{var}[ #1 ]}
\newcommand{\bvar}[1]{\text{var}\Bl[ #1 \Br]}
\newcommand{\cvar}[1]{\text{var}( #1 )}
\newcommand{\cbvar}[1]{\text{var}\Bl( #1 \Br)}

%%%% covariance
\newcommand{\cov}[1]{\text{cov}( #1 )}
\newcommand{\bcov}[1]{\text{cov}\Bl( #1 \Br)}

%%% Popular pmf, pdf notation to avoid long typing
\newcommand{\px}{\ensuremath{p_X(x)}}
\newcommand{\py}{\ensuremath{p_Y(y)}}
\newcommand{\pz}{\ensuremath{p_Z(z)}}
\newcommand{\pxA}{\ensuremath{p_{X|A}(x)}}
\newcommand{\pyA}{\ensuremath{p_{Y|A}(y)}}
\newcommand{\pzA}{\ensuremath{p_{Z|A}(z)}}
\newcommand{\pxy}{\ensuremath{p_{X,Y}(x,y)}}
\newcommand{\pxcy}{\ensuremath{p_{X|Y}(x|y)}}
\newcommand{\pycx}{\ensuremath{p_{Y|X}(y|x)}}

\newcommand{\fx}{\ensuremath{f_X(x)}}
\newcommand{\Fx}{\ensuremath{F_X(x)}}
\newcommand{\fy}{\ensuremath{f_Y(y)}}
\newcommand{\Fy}{\ensuremath{F_Y(y)}}
\newcommand{\fz}{\ensuremath{f_Z(z)}}
\newcommand{\Fz}{\ensuremath{F_Z(z)}}
\newcommand{\fxA}{\ensuremath{f_{X|A}(x)}}
\newcommand{\fyA}{\ensuremath{f_{Y|A}(y)}}
\newcommand{\fzA}{\ensuremath{f_{Z|A}(z)}}
\newcommand{\fxy}{\ensuremath{f_{X,Y}(x,y)}}
\newcommand{\Fxy}{\ensuremath{F_{X,Y}(x,y)}}
\newcommand{\fxcy}{\ensuremath{f_{X|Y}(x|y)}}
\newcommand{\fycx}{\ensuremath{f_{Y|X}(y|x)}}

\newcommand{\fth}{\ensuremath{f_\Theta(\theta)}}
\newcommand{\fxcth}{\ensuremath{f_{X|\Theta}(x|\theta)}}
\newcommand{\fthcx}{\ensuremath{f_{\Theta|X}(\theta|x)}}

\newcommand{\pkcth}{\ensuremath{p_{X|\Theta}(k|\theta)}}
\newcommand{\fthck}{\ensuremath{f_{\Theta|X}(\theta|k)}}


%%%% indicator
\newcommand{\indi}[1]{\mathbf{1}_{ #1 }}

%%%% exponential rv.
\newcommand{\elambdax}{\ensuremath{e^{-\lambda x}}}

%%%% normal  rv.
\newcommand{\stdnormal}{\ensuremath{\frac{1}{\sqrt{2\pi}} e^{-x^2/2}}}
\newcommand{\gennormal}{\ensuremath{\frac{1}{\sigma\sqrt{2\pi}} e^{-(x-\mu)^2/2}}}

%%%%%% estimator, estimate
\newcommand{\hth}{\ensuremath{\hat{\theta}}}
\newcommand{\hTH}{\ensuremath{\hat{\Theta}}}
\newcommand{\MAP}{\ensuremath{\text{MAP}}}
\newcommand{\LMS}{\ensuremath{\text{LMS}}}
\newcommand{\LLMS}{\ensuremath{\text{L}}}
\newcommand{\ML}{\ensuremath{\text{ML}}}

%%%% colored text
\newcommand{\red}[1]{\color{red}#1} 
\newcommand{\cyan}[1]{\color{cyan}#1} 
\newcommand{\magenta}[1]{\color{magenta}#1} 
\newcommand{\blue}[1]{\color{blue}#1} 
\newcommand{\green}[1]{\color{green}#1} 
\newcommand{\white}[1]{\color{white}#1} 
\newcommand{\gray}[1]{\color{gray}#1} 

%%% definition
\newcommand{\defi}{{\color{red} Definition.} } 
\newcommand{\exam}{{\color{red} Example.} } 
\newcommand{\question}{{\color{red} Question.} } 
\newcommand{\thm}{{\color{red} Theorem.} } 
\newcommand{\background}{{\color{red} Background.} } 
\newcommand{\msg}{{\color{red} Message.} } 


\def\ml{\text{ML}}
\def\map{\text{MAP}}

%%%%%%%%%%%%%%%%%%%%%%% old macros that you can ignore %%%%%%%%%%%%%%%%%%%%%%%%

% \def\un{\underline}
% \def\ov{\overline}


% \newcommand{\beq}{\begin{eqnarray*}}
% \newcommand{\eeq}{\end{eqnarray*}}
% \newcommand{\beqn}{\begin{eqnarray}}
% \newcommand{\eeqn}{\end{eqnarray}}
% \newcommand{\bemn}{\begin{multiline}}
% \newcommand{\eemn}{\end{multiline}}
% \newcommand{\beal}{\begin{align}}
% \newcommand{\eeal}{\end{align}}
% \newcommand{\beas}{\begin{align*}}
% \newcommand{\eeas}{\end{align*}}



% \newcommand{\bd}{\begin{displaymath}}
% \newcommand{\ed}{\end{displaymath}}
% \newcommand{\bee}{\begin{equation}}
% \newcommand{\eee}{\end{equation}}


% \newcommand{\vs}{\vspace{0.2in}}
% \newcommand{\hs}{\hspace{0.5in}}
% \newcommand{\el}{\end{flushleft}}
% \newcommand{\bl}{\begin{flushleft}}
% \newcommand{\bc}{\begin{center}}
% \newcommand{\ec}{\end{center}}
% \newcommand{\remove}[1]{}

% \newtheorem{theorem}{Theorem}
% \newtheorem{corollary}{Corollary}
% \newtheorem{prop}{Proposition}
% \newtheorem{lemma}{Lemma}
% \newtheorem{defi}{Definition}
% \newtheorem{assum}{Assumption}
% \newtheorem{example}{Example}
% \newtheorem{property}{Property}
% \newtheorem{remark}{Remark}

% \newcommand{\separator}{
%   \begin{center}
%     \rule{\columnwidth}{0.3mm}
%   \end{center}
% }

% \newenvironment{separation}
% { \vspace{-0.3cm}
%   \separator
%   \vspace{-0.25cm}
% }
% {
%   \vspace{-0.5cm}
%   \separator
%   \vspace{-0.15cm}
% }

% \def\A{\mathcal A}
% \def\oA{\overline{\mathcal A}}
% \def\S{\mathcal S}
% \def\D{\mathcal D}
% \def\eff{{\rm Eff}}
% \def\bD{\bm{D}}
% \def\cU{{\cal U}}
% \def\bbs{{\mathbb{s}}}
% \def\bbS{{\mathbb{S} }}
% \def\cM{{\cal M}}
% \def\bV{{\bm{V}}}
% \def\cH{{\cal H}}
% \def\ch{{\cal h}}
% \def\cR{{\cal R}}
% \def\cV{{\cal V}}
% \def\cA{{\cal A}}
% \def\cX{{\cal X}}
% \def\cN{{\cal N}}
% \def\cJ{{\cal J}}
% \def\cK{{\cal K}}
% \def\cL{{\cal L}}
% \def\cI{{\cal I}}
% \def\cY{{\cal Y}}
% \def\cZ{{\cal Z}}
% \def\cC{{\cal C}}
% \def\cR{{\cal R}}
% \def\id{{\rm Id}}
% \def\st{{\rm st}}
% \def\cF{{\cal F}}
% \def\bz{{\bm z}}
% \def\cG{{\cal G}}
% \def\N{\mathbb{N}}
% \def\bbh{\mathbb{h}}
% \def\bbH{\mathbb{H}}
% \def\bbi{\mathbb{i}}
% \def\bbI{\mathbb{I}}
% \def\R{\mathbb{R}}
% \def\bbR{\mathbb{R}}
% \def\bbr{\mathbb{r}}
% \def\cB{{\cal B}}
% \def\cP{{\cal P}}
% \def\cS{{\cal S}}
% \def\bW{{\bm W}}
% \def\bc{{\bm c}}

% %\def\and{\quad\mbox{and}\quad}
% \def\ind{{\bf 1}}


% \def\bmg{{\bm{\gamma}}}
% \def\bmr{{\bm{\rho}}}
% \def\bmq{{\bm{q}}}
% \def\bmt{{\bm{\tau}}}
% \def\bmn{{\bm{n}}}
% \def\bmcapn{{\bm{N}}}
% \def\bmrho{{\bm{\rho}}}

% \def\igam{\underline{\gamma}(\lambda)}
% \def\sgam{\overline{\gamma}(\lambda)}
% \def\ovt{\overline{\theta}}
% \def\ovT{\overline{\Theta}}
% \def\PP{{\mathrm P}}
% \def\EE{{\mathrm E}}
% \def\iskip{{\vskip -0.4cm}}
% \def\siskip{{\vskip -0.2cm}}

% \def\bp{\noindent{\it Proof.}\ }
% \def\ep{\hfill $\Box$}



%%%%%%%%%% linear algebra macros %%%%%%%%%%%%%%%%%%%%%%%

%--------linsys
%  Use as \begin{linsys}{3}
%           x &+ &3y &+ &a &= &7 \\
%           x &- &3y &+ &a &= &7
%         \end{linsys}
% Remark: TeXbook pp. 167-170 says to put a medmuskip around a +; and that's
% 4/18-ths of an em.  Why does 2/18-ths of an em work?  I don't know, but
% comparing to a regular displayed equation suggests it is right.
% (darseneau says LaTeX puts in half an \arraycolsep.)
\newenvironment{linsys}[2][m]{%
\setlength{\arraycolsep}{.1111em} % p. 170 TeXbook; a medmuskip
\begin{array}[#1]{@{}*{#2}{rc}r@{}}
}{%
\end{array}}

\newsavebox\boxofmathplus
\sbox{\boxofmathplus}{$+$}
\newcommand{\spaceforemptycolumn}{\makebox[\wd\boxofmathplus]{\ }}

%--------grstep
% For denoting a Gauss' reduction step.
% Use as: \grstep{\rho_1+\rho_3} or \grstep[2\rho_5 \\ 3\rho_6]{\rho_1+\rho_3}
% \newcommand{\grstep}[2][\relax]{%
%    \ensuremath{\mathrel{
%        \mathop{\longrightarrow}\limits^{#2\mathstrut}_{
%                                    \begin{subarray}{l} #1 \end{subarray}}}}}

% Advantage of length formulation is that between adjacent
% \grstep's you can add \hspace{-\grsteplength} to make it look not too wide
\newlength{\grsteplength}
\setlength{\grsteplength}{1.5ex plus .1ex minus .1ex}

\newcommand{\grstep}[2][\relax]{%
   \ensuremath{\mathrel{
       \hspace{\grsteplength}\mathop{\longrightarrow}\limits^{#2\mathstrut}_{
                                     \begin{subarray}{l} #1 \end{subarray}}\hspace{\grsteplength}}}}
% If two or more \grsteps are in a row then they need to be tightened
\newcommand{\repeatedgrstep}[2][\relax]{\hspace{-\grsteplength}\grstep[#1]{#2}}

% row swap operation: \rho_1\swap\rho_2
\newcommand{\swap}{\leftrightarrow}

%-------------amatrix
% Augmented matrix.  Usage (note the argument does not count the aug col):
% \begin{amatrix}{2}
%   1  2  3 \\  4  5  6
% \end{amatrix}
\newenvironment{amatrix}[1]{%
  \left(\begin{array}{@{}*{#1}{c}|c@{}}
}{%
  \end{array}\right)
}



%-------------pmat
% For matrices with arguments.
% Usage: \begin{pmat}{c|c|c} 1 &2 &3 \end{pmat}
\newenvironment{pmat}[1]{
  \left(\begin{array}{@{}#1@{}}
}{\end{array}\right)
}



%-------------misc matrices
% \newenvironment{mat}{\left(\begin{array}}{\end{array}\right)}
\newenvironment{detmat}{\left|\begin{array}}{\end{array}\right|}
\newcommand{\deter}[1]{ \mathchoice{\left|#1\right|}{|#1|}{|#1|}{|#1|} }
\newcommand{\generalmatrix}[3]{ %arg1: low-case letter, arg2: rows, arg3: cols 
               \left(
                  \begin{array}{cccc}
                    #1_{1,1}  &#1_{1,2}  &\ldots  &#1_{1,#2}  \\
                    #1_{2,1}  &#1_{2,2}  &\ldots  &#1_{2,#2}  \\
                              &\vdots                         \\
                    #1_{#3,1} &#1_{#3,2} &\ldots  &#1_{#3,#2}
                  \end{array}
               \right)  }

\newcommand{\generaldet}[3]{ %arg1: low-case letter, arg2: rows, arg3: cols 
               \left|
                  \begin{array}{cccc}
                    #1_{11}  &#1_{12}  &\ldots  &#1_{1 #2}  \\
                    #1_{21}  &#1_{22}  &\ldots  &#1_{2 #2}  \\
                              &\vdots                         \\
                    #1_{#3 1} &#1_{#3 2} &\ldots  &#1_{#3 #2}
                  \end{array}
               \right|  }

% With mathtools we can have column entries right flushed
% There is an optional argument \begin{mat}[r]{3} .. \end{mat} for
% right-flushed columns.  Perhaps the rule is that numbers are better 
% right-flushed but if there are any letters it is better centered?
\newenvironment{nmat}[1][c]{\begin{pmatrix*} % disable optional arg [#1] 
      }{\end{pmatrix*}}
% If mat starts with &\vdots get an error; why?  No apparent macro fix, according to texexchange
\newenvironment{vmat}[1][c]{\begin{vmatrix*} % disable optional arg [#1] 
      }{\end{vmatrix*}}
\newenvironment{amat}[2][c]{%
  % disable optional arg \left(\begin{array}{@{}*{#2}{#1}|#1@{}}
  \left(\begin{array}{@{}*{#2}{c}|#1@{}}
}{%
  \end{array}\right)
}
% \newcommand\vdotswithin[1]{% Taken from mathtools.dtx because my TL is not 2011
%   {\mathmakebox[\widthof{\ensuremath{{}#1{}}}][c]{{\vdots}}}}


%------------colvec and rowvec
% Column vector and row vector.  Usage:
%  \colvec{1  \\ 2 \\ 3 \\ 4} and \rowvec{1  &2  &3}
% Colvec takes an optional argument \colvec[r]{x_1 \\ 0}.  Perhaps 
% digits look better right aligned, but if there are any letters it
% needs to be centered?
\newcommand{\colvec}[2][c]{\begin{nmat}[#1] #2 \end{nmat}}
\newcommand{\smallcolvec}[1]{\left(\begin{smallmatrix} #1 \end{smallmatrix}\right)}
% For row vectors, cannot do \newcommand{\rowvec}[1]{\begin{mat} #1 \end{mat}}
% since the delimiters come out too large.
\newcommand{\rowvec}[1]{\setlength{\arraycolsep}{3pt}\left(\begin{matrix} #1 \end{matrix}\right)}



%-------------making aligned columns
% Usage: \begin{aligncolondecimal}{2} 1.2 \\ .33 \end{aligncolondecimal}
% (negative argument centers decimal pt in column).  Also Usage:
% \begin{aligncolondecimal}[0em]{2} 1.2 \\ .33 \end{aligncolondecimal}
% to make the left and right LaTeX-array padding disappear.
\RequirePackage{array}\RequirePackage{dcolumn}
\newenvironment{aligncolondecimal}[2][.1111em]{%
\setlength{\arraycolsep}{#1}
\newcolumntype{.}{D{.}{.}{#2}}\begin{array}{.}}{%
\end{array}}

% Matrix and vector, with numbers centered on decimal point
% Usage: \begin{dmat}{D{.}{.}{1}D{.}{.}{3}}  0  &.123 \\ .2 &.456 \end{dmat}
%  (in the D{.}{.}{number} that is the number of decimal places)
\newlength{\dmatcolsep}\setlength{\dmatcolsep}{5pt}
\newenvironment{dmat}[2][\dmatcolsep]{%
  \setlength{\arraycolsep}{#1}
  \left(\begin{array}{@{}#2@{}}
}{%
  \end{array}\right)}
% Usage: \dcolvec[2]{1.23 \\ 4.56} where the optional argument is the number
% of decimal places.
\newcommand{\dcolvec}[2][-1]{\left(\begin{array}{@{}D{.}{.}{#1}@{}} #2 \end{array}\right)}

%\newcommand{\trans}[1]{ {{#1}^{\mathsf{T}}} } 
\newcommand{\trans}[1]{ {#1}^{\mathsf{T}} } 
\newcommand{\inv}[1]{ {#1}^{-1} } 
\newcommand{\spn}[1]{\ensuremath{\text{span}[#1]} } 
\newcommand{\rk}[1]{\ensuremath{\text{rk}(#1)} } 
\newcommand{\dimm}[1]{\ensuremath{\text{dim}(#1)} } 
\newcommand{\img}[1]{\ensuremath{\text{Im}(#1)} } 
%\newcommand{\norm}[1]{\ensuremath{\left || #1 \right ||} } 
\newcommand{\norm}[1]{\ensuremath{\left \lVert #1 \right \rVert} } 
% orthogonal complement
\newcommand{\ocomp}[1]{\ensuremath{#1^{\bot}} } 
\newcommand{\inner}[2]{\ensuremath{\left\langle #1, #2 \right\rangle} } 
\DeclareMathOperator{\tr}{tr}


% \NewDocumentCommand{\grad}{e{_^}}{%
%   \mathop{}\!% \mathop for good spacing before \nabla
%   \nabla
%   \IfValueT{#1}{_{\!#1}}% tuck in the subscript
%   \IfValueT{#2}{^{#2}}% possible superscript
% }
% \begin{equation*}
%          \begin{nmat}[r]
% 1 &2 &13 \\
%           4  &5  &6
%          \end{nmat}
%       \end{equation*}

% \begin{equation*}
%          \begin{amat}{2}
%           1  &2  &3  \\
%           4  &5  &6
%          \end{amat}
%       \end{equation*}
       
%       \begin{equation*}
%          \begin{pmat}{c|c|c}
% 1 &2 &3 \\
%           4  &5  &6
%          \end{pmat}
%       \end{equation*}

% \begin{equation*}
%          \begin{vmat}
% a &c \\
%           b  &d
%          \end{vmat}
%          =ad-bc
% \end{equation*}

%  \begin{equation*}
%   \vec{v}=\colvec{-1  \\ -0.5  \\ 0}
% \end{equation*}

%  \begin{equation*}
%   \vec{v}=\rowvec{-1  & -0.5  & 0}
% \end{equation*}



\begin{document}

%itemshape
\setbeamertemplate{itemize item}{\scriptsize\raise1.25pt\hbox{\donotcoloroutermaths$\bullet$}}
\setbeamertemplate{itemize subitem}{\tiny\raise1.5pt\hbox{\donotcoloroutermaths$\circ$}}
\setbeamertemplate{itemize subsubitem}{\tiny\raise1.5pt\hbox{\donotcoloroutermaths$\blacktriangleright$}}
%default value for spacing
\plitemsep 0.1in
\pltopsep 0.03in
\setlength{\parskip}{0.15in}
%\setlength{\parindent}{-0.5in}
\setlength{\abovedisplayskip}{0.07in}
\setlength{\belowdisplayskip}{0.07in}
\setlength{\mathindent}{0cm}
\setbeamertemplate{frametitle continuation}{[\insertcontinuationcount]}

\setlength{\leftmargini}{0.5cm}
\setlength{\leftmarginii}{0.5cm}

\setlength{\fboxrule}{0.05pt}
\setlength{\fboxsep}{5pt}


%%%%%%% This should be placed at the end of this file
\logo{\pgfputat{\pgfxy(0.11, 7.4)}{\pgfbox[right,base]{\tikz{\filldraw[fill=dianablue, draw=none] (0 cm, 0 cm) rectangle (50 cm, 1 cm);}\mbox{\hspace{-8 cm}\includegraphics[height=0.7 cm]{../kaist_ee.png}
}}}}

\begin{frame}
  \titlepage
\end{frame}

\logo{\pgfputat{\pgfxy(0.11, 7.4)}{\pgfbox[right,base]{\tikz{\filldraw[fill=dianablue, draw=none] (0 cm, 0 cm) rectangle (50 cm, 1 cm);}\mbox{\hspace{-8 cm}\includegraphics[height=0.7 cm]{../kaist_ee.png}
}}}}

% rule color - gray
\makeatletter
\let\old@rule\@rule
\def\@rule[#1]#2#3{\textcolor{gray}{\old@rule[#1]{#2}{#3}}}
\makeatother





% START START START START START START START START START START START START START
%%%%%%%%%%%%%%%%%%%%%%%%%%%%%%%%%%%%%%%%%%%%%%%%%%%%%%
\begin{frame}{Roadmap}

\plitemsep 0.1in

\bce[(1)] 
\item Norms

\item Inner Products

\item Lengths and Distances

\item Angles and Orthogonality

\item Orthonormal Basis

\item Orthogonal Complement

\item Inner Product of Functions

\item Orthogonal Projections

\item Rotations

\ece
\end{frame}

%%%%%%%%%%%%%%%%%%%%%%%%%%%%%%%%%%%%%%%%%%%%%%%%%%%%%%
\section{L3(1)}
\begin{frame}{Roadmap}

\plitemsep 0.1in

\bce[(1)] 
\item \redf{Norms}

\item \grayf{Inner Products

\item Lengths and Distances

\item Angles and Orthogonality

\item Orthonormal Basis

\item Orthogonal Complement

\item Inner Product of Functions

\item Orthogonal Projections

\item Rotations}

\ece
\end{frame}

%%%%%%%%%%%%%%%%%%%%%%%%%%%%%%%%%%%%%%%%%%%%%%%%%%%%%%
\begin{frame}{Norm}

\plitemsep 0.1in

\bci 
\item A notion of the length of vectors

\item \defi A norm on a vector space $V$ is a function $\norm{\cdot}: V \mapsto \real,$ such that for all $\lambda \in \real$ the following hold:

\bci
\item \bluef{Absolutely homogeneous}: $\norm{\lambda \vec{x}} = |\lambda| \norm{\vec{x}}$
\item \bluef{Triangle inequality}: $\norm{\vec{x} + \vec{y}} \le \norm{\vec{x}} + \norm{\vec{y}} $
\item \bluef{Positive definite}: $\norm{\vec{x}} \ge 0$ and $\norm{\vec{x}} \Longleftrightarrow \vec{x} = \vec{0}$
\eci
\eci
\end{frame}

%%%%%%%%%%%%%%%%%%%%%%%%%%%%%%%%%%%%%%%%%%%%%%%%%%%%%%
\begin{frame}{Example for $V \in \real^n$}

\plitemsep 0.1in

\bci 
\item \bluef{Manhattan Norm} (also called $\ell_1$ norm) For $\vec{x}= [x_1, \cdots, x_n] \in \real^n,$
$$
\norm{\vec{x}}_1 \eqdef = \sum_{i=1}^n |x_i|
$$
\item \bluef{Euclidean Norm} (also called $\ell_2$ norm) For $\vec{x} \in \real^n,$
$$
\norm{\vec{x}}_2 \eqdef = \sqrt{\sum_{i=1}^n x_i^2} = \sqrt{\trans{\vec{x}} \vec{x}}
$$

\eci
\end{frame}


%%%%%%%%%%%%%%%%%%%%%%%%%%%%%%%%%%%%%%%%%%%%%%%%%%%%%%
\begin{frame}{Roadmap}

\plitemsep 0.1in

\bce[(1)] 
\item \grayf{Norms}

\item \redf{Inner Products}

\item \grayf{Lengths and Distances

\item Angles and Orthogonality

\item Orthonormal Basis

\item Orthogonal Complement

\item Inner Product of Functions

\item Orthogonal Projections

\item Rotations}

\ece
\end{frame}

%%%%%%%%%%%%%%%%%%%%%%%%%%%%%%%%%%%%%%%%%%%%%%%%%%%%%%
\section{L3(2)}
\begin{frame}{Motivation}

\plitemsep 0.1in

\bci 
\item Need to talk about the length of a vector and the angle or distance between two vectors, where vectors are defined in abstract vector spaces

\item To this end, we define the notion of \bluef{inner product} in an abstract manner.

\item Dot product: A kind of inner product in vector space $\real^n$. $\trans{\vec{x}} \vec{y} = \sum_{i=1}^n x_i y_i$ 


\bigskip
\item \question How can we generalize this and do a similar thing in some other vector spaces?
\eci
\end{frame}

%%%%%%%%%%%%%%%%%%%%%%%%%%%%%%%%%%%%%%%%%%%%%%%%%%%%%%
\begin{frame}{Formal Definition}

\plitemsep 0.1in

\bci 
\item An inner product is a mapping $\inner{\cdot}{\cdot}: V \times V \mapsto \real$ that satisfies the following conditions for all vectors $\vec{u},\vec{v},\vec{w} \in V$ and all scalars $\lambda \in \real$:

\medskip
\bce
\item $\inner{\vec{u}+ \vec{v}}{\vec{w}} = \inner{\vec{u}}{\vec{w}} + \inner{\vec{v}}{\vec{w}}$
\item $\inner{\lambda \vec{v}}{\vec{w}} = \lambda \inner{\vec{v}}{\vec{w}}$
\item $\inner{\vec{v}}{\vec{w}} = \inner{\vec{w}}{\vec{v}}$
\item $\inner{\vec{v}}{\vec{v}} \ge 0$ and equal iff $\vec{v}=\vec{0}$
\ece
\medskip

\item The pair $(V,\inner{\cdot}{\cdot})$ is called an \bluef{inner product space}.

\eci
\end{frame}

%%%%%%%%%%%%%%%%%%%%%%%%%%%%%%%%%%%%%%%%%%%%%%%%%%%%%%
\begin{frame}{Example}

\plitemsep 0.3in

\bci 

\item \exam $V=\real^n$ and the dot product $\inner{\vec{x}}{\vec{y}} \eqdef \trans{\vec{x}}\vec{y}$

\item \exam $V=\real^2$ and $\inner{\vec{x}}{\vec{y}} \eqdef x_1y_1 - (x_1y_2 + x_2y_1) + 2x_2y_2$

\item \exam $V=\{\text{continuous functions in $\real$ over $[a,b]$} \},$ $\inner{u}{v} \eqdef \int_a^b u(x)v(x) dx$
\eci
\end{frame}

%%%%%%%%%%%%%%%%%%%%%%%%%%%%%%%%%%%%%%%%%%%%%%%%%%%%%%
\begin{frame}{Symmetric, Positive Definite Matrix}

\plitemsep 0.1in

\bci 
\item A square matrix $\mat{A} \in \real^{n \times n}$ that satisfies the following is called \bluef{symmetric, positive definite} (or just positive definite): 
$$
\forall \vec{x} \in V \setminus \{\vec{0} \}: \trans{\vec{x}} \mat{A} \vec{x} > 0.
$$
If only $\ge$ in the above holds, then $\mat{A}$ is called \bluef{symmetric, positive semidefinite.} 

\bigskip
\item $\mat{A}_1 = \begin{nmat}
9 & 6 \cr
6 & 5
\end{nmat}
$ is positive definite.

\item $\mat{A}_2 = \begin{nmat}
9 & 6 \cr
6 & 3
\end{nmat}
$ is not positive definite.



\eci
\end{frame}

%%%%%%%%%%%%%%%%%%%%%%%%%%%%%%%%%%%%%%%%%%%%%%%%%%%%%%
\begin{frame}{Inner Product and Positive Definite Matrix (1)}

\plitemsep 0.2in

\bci 
\item Consider an $n$-dimensional vector space $V$ with an inner product $\inner{\cdot}{\cdot}$ and an ordered basis $B=(\vec{b}_1, \ldots, \vec{b}_n)$ of $V.$

\item Any $\vec{x},\vec{y} \in V$ can be represented as: $\vec{x}=\sum_{i=1}^n \psi_i \vec{b}_i$ and $\vec{y}=\sum_{i=j}^n \lambda_j \vec{b}_j$ for some $\psi_i$ and $\lambda_j,$ $i,j=1, \ldots, n.$
\aleq{
\inner{\vec{x}}{\vec{y}} = \inner{\sum_{i=1}^n \psi_i\vec{b}_i}{\sum_{i=j}^n \lambda_j \vec{b}_j} = 
\sum_{i=1}^n \sum_{j=1}^n \psi_i \inner{\vec{b}_i}{\vec{b}_j} \lambda_j = \trans{\hat{\vec{x}}} \mat{A} \hat{\vec{y}},
}
where $\mat{A}_{ij} = \inner{\vec{b}_i}{\vec{b}_j}$ and $\hat{\vec{x}}$ and $\hat{\vec{y}}$ are the coordinates w.r.t. $B.$

\eci
\end{frame}

%%%%%%%%%%%%%%%%%%%%%%%%%%%%%%%%%%%%%%%%%%%%%%%%%%%%%%
\begin{frame}{Inner Product and Positive Definite Matrix (2)}

\plitemsep 0.2in

\bci 

\item Then, if $\forall \vec{x} \in V \setminus \{\vec{0} \}: \trans{\vec{x}} \mat{A} \vec{x} > 0$ (i.e., $\mat{A}$ is symmetric, positive definite), $\bluef{\trans{\hat{\vec{x}}} \mat{A} \hat{\vec{y}}}$ legitimately defines an inner product (w.r.t. $B$)

\item Properties
\bci
\item The kernel of $\mat{A}$ is only $\{\vec{0} \}$, because $\trans{\vec{x}} \mat{A} \vec{x} > 0$ for all $\vec{x} \neq \vec{0}  \implies$ $\mat{A} \vec{x} \neq \vec{0}$ if $\vec{x} \neq \vec{0}.$
\item The diagonal elements $a_{ii}$ of $\mat{A}$ are all positive, because $a_{ii} = \trans{\vec{e}_i} \mat{A} \vec{e}_i >0.$
\eci
\eci
\end{frame}


%%%%%%%%%%%%%%%%%%%%%%%%%%%%%%%%%%%%%%%%%%%%%%%%%%%%%%
\section{L3(3)}
\begin{frame}{Roadmap}

\plitemsep 0.1in

\bce[(1)] 
\item \grayf{Norms

\item Inner Products}

\item \redf{Lengths and Distances

\item Angles and Orthogonality}

\item \grayf{Orthonormal Basis

\item Orthogonal Complement

\item Inner Product of Functions

\item Orthogonal Projections

\item Rotations}

\ece
\end{frame}

%%%%%%%%%%%%%%%%%%%%%%%%%%%%%%%%%%%%%%%%%%%%%%%%%%%%%%

\begin{frame}{Length}

\plitemsep 0.2in

\bci 

\item Inner product naturally induces a norm by defining:
$$
\norm{x} \eqdef \sqrt{\inner{\vec{x}}{\vec{x}}}
$$

\item Not every norm is induced by an inner product

\item \redf{Cachy-Schwarz inequality.} For the induced norm by the inner product, 
$$
|\inner{\vec{x}}{\vec{y}}| \le \norm{\vec{x}} \ \norm{\vec{y}}
$$

\eci
\end{frame}

%%%%%%%%%%%%%%%%%%%%%%%%%%%%%%%%%%%%%%%%%%%%%%%%%%%%%%
\begin{frame}{Distance}

\plitemsep 0.1in

\bci 

\item Now, we can introduce a notion of distance using a norm as:

\medskip
\redf{Distance}. $d(\vec{x},\vec{y}) \eqdef \norm{\vec{x} - \vec{y}} = \sqrt{\inner{\vec{x}-\vec{y}}{\vec{x}-\vec{y}}} $

\item If the dot product is used as an inner product in $\real^n,$ it is \bluef{Euclidian distance.}

\item \redf{Note.} The distance between two vectors does \bluef{NOT} necessarily require the notion of norm. Norm is just sufficient. 

\item Generally, if the following is satisfied, it is a suitable notion of distance, called \bluef{metric}. 
\bci
\item \bluef{\em Positive definite}. $d(\vec{x},\vec{y}) \ge 0$ for all $\vec{x},\vec{y}$ and $d(\vec{x},\vec{y}) = 0 \Longleftrightarrow \vec{x}=\vec{y}$ 
\item \bluef{\em Symmetric}. $d(\vec{x},\vec{y}) = d(\vec{y},\vec{x})$
\item \bluef{\em Triangle inequality}. $d(\vec{x},\vec{z}) \le d(\vec{x},\vec{y}) + d(\vec{y},\vec{z})$
\eci
\eci
\end{frame}

%%%%%%%%%%%%%%%%%%%%%%%%%%%%%%%%%%%%%%%%%%%%%%%%%%%%%%
\section{L3(4)}
\begin{frame}{Angle,  Orthogonal, and Orthonormal}

\plitemsep 0.1in

\bci 

\item Using C-S inequality, $$-1 \le \frac{\inner{\vec{x}}{\vec{y}}}{\norm{\vec{x}} \ \norm{\vec{y}}} \le 1$$

\item Then, there exists a unique $\omega \in [0,\pi]$ with $$\cos \omega = \frac{\inner{\vec{x}}{\vec{y}}}{\norm{\vec{x}} \ \norm{\vec{y}}}$$

\item We define $\omega$ as the \bluef{angle} between $\vec{x}$ and $\vec{y}.$

\item \defi If $\inner{\vec{x}}{\vec{y}} = 0,$ in other words their angle is $\pi/2,$ we say that they are \bluef{orthogonal}, denoted by $\vec{x} \perp \vec{y}.$ Additionally, if $\norm{x} = \norm{y} =1,$ they are \bluef{orthonormal}.
\eci
\end{frame}

%%%%%%%%%%%%%%%%%%%%%%%%%%%%%%%%%%%%%%%%%%%%%%%%%%%%%%
\begin{frame}{Example}

\plitemsep 0.15in

\bci 

\item Orthogonality is defined by a given inner product. Thus, different inner products may lead to different results about orthogonality. 

\item \exam Consider two vectors $\vec{x}=\colvec{1 \\1 }$ and $\vec{y}=\colvec{-1 \\ 1 }$

\item Using the dot product as the inner product, they are orthogonal.

\item However, using $\inner{\vec{x}}{\vec{y}} = \trans{\vec{x}}
\begin{nmat}
2 & 0 \cr
0 & 1
\end{nmat} \vec{y}$, they are not orthogonal. 
\aleq{
\cos \omega = \frac{\inner{\vec{x}}{\vec{y}}}{\norm{\vec{x}} \ \norm{\vec{y}}} = -\frac{1}{3} \implies \omega \approx 1.91 \text{ rad } \approx 109.5\text{\textdegree}
}
\eci


\end{frame}

%%%%%%%%%%%%%%%%%%%%%%%%%%%%%%%%%%%%%%%%%%%%%%%%%%%%%%
\begin{frame}{Orthogonal Matrix}

\plitemsep 0.05in

\bci 

\item \defi A square matrix $\mat{A} \in \real^{n \times n}$ is an \bluef{orthogonal matrix}, iff its columns (or rows) are \bluef{orthonormal} so that 
$$
\mat{A} \trans{\mat{A}} = I = \trans{\mat{A}}\mat{A}, \text{ implying } \inv{\mat{A}} = \trans{\mat{A}}.
$$
\vspace{-0.3cm}
\bci
\item We can use \bluef{$\inv{\mat{A}} = \trans{\mat{A}}$} for the definition of orthogonal matrices. 
\item Fact 1. $\mA,\mB$: orthogonal $\implies$ $\mA\mB$: orthogonal
\item Fact 2. $\mA$: orthogonal $\implies$ $\det(\mA) = \pm 1$
\eci


\item The linear mapping $\Phi$ by orthogonal matrices preserve \bluef{length} and \bluef{angle} (for the dot product)
\aleq{
\norm{\Phi(\mA)} = \norm{\mat{A}\vec{x}}^2 = \trans{(\mat{A}\vec{\vec{x}})} (\mat{A} \vec{x}) = \trans{\vec{x}} \trans{\mat{A}} \mat{A} \vec{x} = \trans{\vec{x}} \vec{x} = \norm{\vec{x}}^2
}
\vspace{-0.7cm}
\aleq{
\cos \omega = \frac{\trans{(\mat{A}\vec{x})} (\mat{A}\vec{y})}{\norm{\mat{A}\vec{x}} \ \norm{\mat{A}\vec{y}}} = 
\frac
{
\trans{\vec{x}} \trans{\mat{A}} \mat{A} \vec{y}
}
{
\sqrt{\trans{\vec{x}} \trans{\mat{A}} \mat{A} \vec{x} \trans{\vec{y}} \trans{\mat{A}} \mat{A} \vec{y}
}
}
= \frac{\trans{\vec{x}} \vec{y}}{\norm{\vec{x}} \ \norm{\vec{y}}}
}

\eci
\end{frame}



%%%%%%%%%%%%%%%%%%%%%%%%%%%%%%%%%%%%%%%%%%%%%%%%%%%%%%
\section{L3(5)}
\begin{frame}{Roadmap}

\plitemsep 0.1in

\bce[(1)] 
\item \grayf{Norms

\item Inner Products

\item Lengths and Distances

\item Angles and Orthogonality}

\item \redf{Orthonormal Basis

\item Orthogonal Complement

\item Inner Product of Functions}

\item \grayf{Orthogonal Projections

\item Rotations}

\ece
\end{frame}



%%%%%%%%%%%%%%%%%%%%%%%%%%%%%%%%%%%%%%%%%%%%%%%%%%%%%%
\begin{frame}{Orthonormal Basis}

\plitemsep 0.1in

\bci 

\item Basis that is orthonormal, i.e., they are all orthogonal to each other and their lengths are 1. 

\item Standard basis in $\real^n,$ $\{\vec{e}_1, \ldots, \vec{e}_n \},$ is orthonormal.


\item \question How to obtain an orthonormal basis?

\bigskip
\mycolorbox{
\item[1.] Use Gaussian elimination to find a basis for a vector space spanned by a set of vectors.
\bci
\item Given a set $\{\vec{b}_1, \ldots, \vec{b}_n \}$ of unorthogonal and unnormalized basis vectors. Apply Gaussian elimination to the augmented matrix $(\mat{B}\trans{\mat{B}}|\mat{B})$
\eci

\item[2.] Constructive way: Gram-Schmidt process (we will cover this later)
}
\eci
\end{frame}

%%%%%%%%%%%%%%%%%%%%%%%%%%%%%%%%%%%%%%%%%%%%%%%%%%%%%%
\section{L3(6)}
\begin{frame}{Orthogonal Complement (1)}

\plitemsep 0.1in

\bci 

\item Consider $D$-dimensional vector space $V$ and $M$-dimensional subspace  $W \subset V.$ The \bluef{orthogonal complement} $\ocomp{U}$ is a $(D-M)$-dimensional subspace of $V$ and contains all vectors in $V$ that are orthogonal to every vector in $U.$

\item $U \cap \ocomp{U} = \vec{0}$

\item Any vector $x \in V$ can be uniquely decomposed into:
\aleq{
\vec{x} = \sum_{m=1}^M \lambda_m \vec{b}_m + \sum_{j=1}^{D-M} \psi_j \ocomp{\vec{b}}_j, \quad \lambda_m, \psi_j \in \real,
}
where $(\vec{b}_1 \ldots, \vec{b}_M)$ and $(\ocomp{\vec{b}}_1, \ldots, \ocomp{\vec{b}}_{D-M} )$ are the \bluef{bases} of $U$ and $\ocomp{U},$ respectively. 
\eci
\end{frame}

%%%%%%%%%%%%%%%%%%%%%%%%%%%%%%%%%%%%%%%%%%%%%%%%%%%%%%
\begin{frame}{Orthogonal Complement (2)}

\plitemsep 0.1in

\vspace{-0.3cm}
\begin{center}
\mypic{0.35}{L3_ocomp.png}
\end{center}
\vspace{-0.5cm}
\bci 
\item The vector $\vw$ with $\norm{\vw}=1,$ which is orthogonal to $U$, is the basis of $\ocomp{U}.$
\item Such $\vw$ is called \bluef{normal vector} to $U.$

\item For a linear mapping represented by a matrix $\mat{A} \in \real^{m \times n},$ the solution space of $\mat{A} \vec{x} =0$ is $\ocomp{\text{row}(\mat{A})},$ where $\text{row}(\mat{A})$ is the row space of $\mat{A}$ (i.e., span of row vectors).

In other words, $\ocomp{\text{row}(\mat{A})} = \ker(\mat{A})$
\eci


\end{frame}

%%%%%%%%%%%%%%%%%%%%%%%%%%%%%%%%%%%%%%%%%%%%%%%%%%%%%%
\section{L3(7)}
\begin{frame}{Inner Product of Functions}

\plitemsep 0.15in

\bci 

\item \redf{Remind:} $V=\{\text{continuous functions in $\real$ over $[a,b]$} \},$ the following is a proper inner product. 
\bluef{$$\inner{u}{v} \eqdef \int_a^b u(x)v(x) dx$$}

\item \exam Choose $u(x) = \sin(x)$ and $v(x)= \cos(x),$ where we select $a=-\pi$ and $b=\pi.$ Then, since $f(x) = u(x)v(x)$ is odd (i.e., $f(-x) = -f(x)$), 
$$
\int_{-\pi}^\pi u(x) v(x) dx =0.
$$

\item Thus, $u$ and $v$ are orthogonal. 

\item Similarly, $\{1, \cos(x), \cos(2x), \cos(3x), \ldots,  \}$ is orthogonal over $[-\pi,\pi].$
\eci


\end{frame}



%%%%%%%%%%%%%%%%%%%%%%%%%%%%%%%%%%%%%%%%%%%%%%%%%%%%%%
\section{L3(8)}
\begin{frame}{Roadmap}

\plitemsep 0.1in

\bce[(1)] 
\item \grayf{Norms

\item Inner Products

\item Lengths and Distances

\item Angles and Orthogonality

\item Orthonormal Basis

\item Orthogonal Complement

\item Inner Product of Functions}

\item \redf{Orthogonal Projections}

\item \grayf{Rotations}

\ece
\end{frame}

%%%%%%%%%%%%%%%%%%%%%%%%%%%%%%%%%%%%%%%%%%%%%%%%%%%%%%
\begin{frame}{Projection: Motivation}

\plitemsep 0.05in

\bci 

\item Big data: high dimensional

\item However, most information is contained in a few dimensions

\item \bluef{Projection}: A process of reducing the dimensions (hopefully) without loss of much information\footnote{In \lecturemark{L10}, we will formally study this with the topic of PCA (Principal Component Analysis).}

\item \exam Projection of 2D dataset onto 1D subspace

\centering
\mypic{0.4}{L3_projection_ex.png}
\eci


\end{frame}

%%%%%%%%%%%%%%%%%%%%%%%%%%%%%%%%%%%%%%%%%%%%%%%%%%%%%%
\begin{frame}{Projection onto Lines (1D Subspaces)}

\plitemsep 0.1in

\bci 
\item Consider a 1D subspace $U \subset \real^n$ spanned by the basis $\vec{b}.$ 

\item For $\vx \in \realn,$ what is its projection \bluef{$\pi_U(\vec{x})$} onto $U$ (assume the dot product)?
\myvartwocols{0.3}{0.7}{0.29}
{
\small
\aleq{
&\inner{\vec{x} - \pi_U(\vec{x})}{\vec{b}} = 0 \xleftrightarrow{\pi_U(\vec{x}) = \lambda \vec{b}} \inner{\vec{x} - \lambda \vec{b}}{\vec{b}}=0\cr
& \implies \lambda = \frac{\inner{\vec{b}}{\vec{x}}}{\norm{\vec{b}}^2} = \frac{\trans{\vec{b}}\vec{x}}{\norm{\vec{b}}^2}, \ \text{and} \ \pi_U(\vec{x}) = \lambda \vec{b} =  \bluef{\frac{\trans{\vec{b}}\vec{x}}{\norm{\vec{b}}^2} \vec{b}}
}
}
{
\vspace{-0.2cm}
\mypic{0.8}{L3_projection_1D.png}
}
\vspace{-0.5cm}
\item Projection matrix \redf{$\mat{P}_\pi \in \realnn$}  in $\pi_U(\vec{x}) = \mat{P}_\pi \vec{x}$
\aleq{
\pi_U(\vec{x}) = \lambda \vec{b} =  \vec{b} \lambda =  \frac{\vec{b}\trans{\vec{b}}}{\norm{\vec{b}}^2} \vec{x}, \quad \mat{P}_\pi = \bluef{\frac{\vec{b}\trans{\vec{b}}}{\norm{\vec{b}}^2}}
}
\eci

\end{frame}

%%%%%%%%%%%%%%%%%%%%%%%%%%%%%%%%%%%%%%%%%%%%%%%%%%%%%%
\begin{frame}{Inner Product and Projection}

\plitemsep 0.1in

\bci 
\item We project $\vx$ onto $\vb$, and let $\pi_{\vb}(\vx)$ be the projected vector. 


\item \question Understanding the inner project $\inner{\vx}{\vb}$ from the projection perspective?
\mycolorbox{
$$
\inner{\vx}{\vb} = \norm{\pi_{\vb}(\vx)} \times \norm{\vb}
$$
}
\mytwocols{0.4}
{
\item In other words, the inner product of $\vx$ and $\vb$ is the product of (\bluef{length of the projection of $\vx$ onto $\vb$}) $\times$ (\bluef{length of $\vb$})
}
{
\vspace{-0.2cm}
\mypic{0.6}{L3_projection_1D.png}
}

\eci

\end{frame}

%%%%%%%%%%%%%%%%%%%%%%%%%%%%%%%%%%%%%%%%%%%%%%%%%%%%%%
\begin{frame}{Example}

\plitemsep 0.1in

\bci 
\item $\vec{b} = \colvec{1 \\ 2 \\ 2}$
\aleq{
\mat{P}_\pi = \frac{\vec{b}\trans{\vec{b}}}{\norm{\vec{b}}^2} = \frac{1}{9}\colvec{1\\2\\2}\rowvec{1 & 2 & 2} = \frac{1}{9}
\begin{nmat}
1&2&2 \cr
2&4&4 \cr
2&4&4 
\end{nmat}
}
For $\vec{x} = \colvec{1\\1\\1},$
\aleq{
\pi_U(\vec{x}) = \mat{P}_\pi \vec{x} = \frac{1}{9}
\begin{nmat}
1&2&2 \cr
2&4&4 \cr
2&4&4 
\end{nmat} \colvec{1\\1\\1} = \frac{1}{9} \colvec{5\\10\\10} \in \spn{\colvec{1\\2\\2}} 
}
\eci

\end{frame}

%%%%%%%%%%%%%%%%%%%%%%%%%%%%%%%%%%%%%%%%%%%%%%%%%%%%%%
\begin{frame}{Projection onto General Subspaces}

\plitemsep 0.1in


%\item Compare the results:

\mytwocols{0.4}
{
\bci
\item $\realn \rightarrow$ 1-Dim
\item A basis vector $\vec{b}$ in 1D subspace
\eci
\centering
$$
\pi_U(\vec{x}) = \bluef{\frac{\vec{b}\trans{\vec{b}}\vec{x}}{\trans{\vec{b}}\vec{b}}}, \ \lambda = \frac{\trans{\vec{b}}\vec{x}}{\trans{\vec{b}}\vec{b}}
$$
$$
\mat{P}_\pi  = \redf{\frac{\vec{b}\trans{\vec{b}}}{\trans{\vec{b}}\vec{b} }}
$$
}
{
\bci
\item $\realn \rightarrow$ $m$-Dim, $(m < n)$
\item A basis matrix $B=\rowvec{\vec{b}_1, \cdots, \vec{b}_m} \in \real^{n \times m}$ 
\eci
$$
\pi_U(\vec{x}) = \bluef{\mB\inv{(\trans{\mB}\mB)}\trans{\mB} \vec{x}}, \ 
\vlam = \inv{(\trans{\mB}\mB)}\trans{\mB} \vec{x}
$$
$$
\mat{P}_\pi  = \redf{\mB\inv{(\trans{\mB}\mB)}\trans{\mB} }
$$
}
\vspace{-0.4cm}
\bci
\item $\lambda \in \real^{1}$ and $\vlam \in \realm$ are the coordinates in the projected spaces, respectively. 
\item $\inv{(\trans{\mB}\mB)}\trans{\mB}$ is called \bluef{pseudo-inverse}.
\item How to derive is analogous to the case of 1-D lines (see pp. 71).
\eci
\end{frame}

%%%%%%%%%%%%%%%%%%%%%%%%%%%%%%%%%%%%%%%%%%%%%%%%%%%%%%
\begin{frame}{Example: Projection onto 2D Subspace}

\plitemsep 0.1in
\small
\bci 
\item $U = \spn{\colvec{1\\1\\1}, \colvec{0\\1\\2}} \subset \real^3$ and $\vec{x} = \colvec{6\\0\\0}$. Check that $\{ \trans{\rowvec{1&1&1}}, \trans{\rowvec{0&1&2}}\}$ is a basis.
\item Let $\mat{B} = \begin{nmat}
1&0\cr
1&1\cr
1&2
\end{nmat}.$ Then, $\trans{\mat{B}}\mat{B} = 
\begin{nmat}
1&1&2\cr
0&1&2
\end{nmat}
\begin{nmat}
1&0\cr
1&1\cr
1&2
\end{nmat}
= 
\begin{nmat}
3&3\cr
3&5
\end{nmat}
$
\item Can see that $\mat{P}_\pi  = \mB\inv{(\trans{\mB}\mB)}\trans{\mB} = \dfrac{1}{6}
\begin{nmat}
5&2&-1\cr
2&2&2\cr
-1&2&5
\end{nmat}
$, and $\pi_U(\vec{x}) = \dfrac{1}{6}
\begin{nmat}
5&2&-1\cr
2&2&2\cr
-1&2&5
\end{nmat} \colvec{6\\0\\0} = \colvec{5\\2\\-1}$

\eci

\end{frame}

%%%%%%%%%%%%%%%%%%%%%%%%%%%%%%%%%%%%%%%%%%%%%%%%%%%%%%
\begin{frame}{Gram-Schmidt Orthogonalization Method (G-S method)}

\plitemsep 0.05in

\bci 
\item Constructively transform any basis $(\vb_1, \ldots, \vb_n)$ of $n$-dimensional vector space $V$ into an orthogonal/orthonormal basis $(\vu_1, \ldots, \vu_n)$ of $V$ 

\item Iteratively construct as follows
\mycolorbox{
\vspace{-0.2cm}
\aleq{
\vu_1 &\eqdef \vb_1 \cr
\vu_k &\eqdef \vb_k - \pi_{\text{span}[\vu_1, \ldots, \vu_{k-1}]}(\vb_k), \ k=2, \ldots, n \qquad \qquad (*)
}
}
%\item In $(*)$
% \mytwocols{0.3}
% {
% \bci
% \item $\pi_{\text{span}[\vu_1, \ldots, \vu_{k-1}]}(\vb_k)$: projection of $\vb_k$ onto the subspace spanned by $[\vu_1, \ldots, \vu_{k-1}]$
% \item Then, $\vu_k$ becomes orthogonal to $\text{span}[\vu_1, \ldots, \vu_{k-1}]$
% \eci
% }
% {
% }
\eci
\vspace{-0.3cm}
\mypic{0.9}{L3_gramschmidt.png}

\end{frame}

%%%%%%%%%%%%%%%%%%%%%%%%%%%%%%%%%%%%%%%%%%%%%%%%%%%%%%
\begin{frame}{Example: G-S method}

\plitemsep 0.15in

\bci 
\item A basis $(\vb_1, \vb_2) \in \real^2,$ $\vb_1 = \colvec{2 \\ 0}$ and $\vb_2 = \colvec{1 \\1}$

\item $\vu_1 = \vb_1 = \colvec{2 \\ 0}$ and
\aleq{
\vu_2 = \vb_2 - \pi_{\text{span}[\vu_1]}(\vb_2) = \frac{\vu_1\trans{\vu_2}}{\norm{\vu_1}} \vb_2
= \colvec{1\\1} - \begin{nmat}
1 & 0 \cr
0 & 0
\end{nmat}
\colvec{1 \\1} = \colvec{0 \\1}
}

\item $\vu_1$ and $\vu_2$ are orthogonal. If we want them to be orthonormal, then just normaliation would do the job. 
\eci
\end{frame}

%%%%%%%%%%%%%%%%%%%%%%%%%%%%%%%%%%%%%%%%%%%%%%%%%%%%%%
\begin{frame}{Projection onto Affine Subspaces}

\begin{center}
    \mypic{0.7}{L3_projection_affine.png}
\end{center}

\plitemsep 0.05in
\vspace{-0.5cm}
\bci 
\item Affine space: $L = \vec{x}_0 + U$
\item Affine subspaces are not vector spaces
\item Idea: (i) move $\vec{x}$ to a point in $U$, (ii) do the projection, (iii) move back to $L$
\bluef{$$\pi_L(\vec{x}) = \vec{x}_0 + \pi_{U}(\vec{x} - \vec{x}_0)$$}
\eci

\end{frame}



%%%%%%%%%%%%%%%%%%%%%%%%%%%%%%%%%%%%%%%%%%%%%%%%%%%%%%
\section{L3(9)}
\begin{frame}{Roadmap}

\plitemsep 0.1in

\bce[(1)] 
\item \grayf{Norms

\item Inner Products

\item Lengths and Distances

\item Angles and Orthogonality

\item Orthonormal Basis

\item Orthogonal Complement

\item Inner Product of Functions

\item Orthogonal Projections}

\item \redf{Rotations}

\ece
\end{frame}

%%%%%%%%%%%%%%%%%%%%%%%%%%%%%%%%%%%%%%%%%%%%%%%%%%%%%%
\begin{frame}{Rotation}

\plitemsep 0.07in

\bci 
\item Length and angle preservation: two properties of linear mappings with \bluef{orthogonal matrices}. Let's look at some of their special cases. 

\item A linear mapping that rotates the given coordinate system by an angle $\theta.$

\item Basis change
\item $\vec{e}_1 = \colvec{1 \\ 0} \rightarrow \colvec{\cos\theta \\ \sin\theta}$ and $\vec{e}_2 = \colvec{0 \\ 1} \rightarrow \colvec{-\sin\theta \\ \cos\theta}$

\item Rotation matrix $\vec{R}(\theta) = \begin{nmat}
\cos\theta & -\sin\theta \cr
\sin\theta & \cos\theta 
\end{nmat}$

\item Properties
\bci
\item Preserves distance: $\norm{\vec{x} - \vec{y}} = \norm{\mat{R}_\theta(\vec{x}) - \mat{R}_\theta(\vec{y})}$ 
\item Preserves angle
\eci
\eci

\end{frame}


%%%%%%%%%%%%%%%%%%%%%%%%%%%%%%%%%%%%%%%%%%%%%%%%%%%%%%
\begin{frame}{}
\vspace{2cm}
\LARGE Questions?


\end{frame}

\begin{frame}{Review Questions}
% \tableofcontents
%\plitemsep 0.1in
\bce[1)]
\item 

\ece
\end{frame}


\end{document}
