%\pdfminorversion=4
\documentclass[handout,fleqn,aspectratio=169]{beamer}

% when making printed slides
\usepackage{pgfpages}
\pgfpagesuselayout{resize to}[a4paper,landscape,border shrink=5mm]

\usepackage[english]{babel}
\usepackage{tikz}
\usepackage{courier}
\usepackage{array}
\usepackage{bold-extra}
%\usepackage{minted}
\usepackage[thicklines]{cancel}
\usepackage{fancyvrb}
\usepackage{kotex}
\usepackage{paralist}
\usepackage{collectbox}
\usepackage{bm}

\usepackage{mathrsfs}
\usepackage[reqno,disallowspaces]{mathtools}  % imports amsmath
\usepackage{amsfonts} %for Y&Y BSR AMS fonts
\usepackage{amssymb}
\usepackage{amscd}
%\usepackage{tikz,lipsum,lmodern}
\usepackage[most]{tcolorbox}
\usepackage{verbatim}
\mode<presentation>
{
  \usetheme{default}
  \usecolortheme{default}
  \usefonttheme{default}
  \setbeamertemplate{navigation symbols}{}
  \setbeamertemplate{caption}[numbered]
  \setbeamertemplate{footline}[frame number]  % or "page number"
  \setbeamercolor{frametitle}{fg=yellow}
  \setbeamercolor{footline}{fg=black}
} 

\setbeamercolor{block body alerted}{bg=alerted text.fg!10}
\setbeamercolor{block title alerted}{bg=alerted text.fg!20}
\setbeamercolor{block body}{bg=structure!10}
\setbeamercolor{block title}{bg=structure!20}
\setbeamercolor{block body example}{bg=green!10}
\setbeamercolor{block title example}{bg=green!20}
\setbeamertemplate{blocks}[rounded][shadow]

\xdefinecolor{dianablue}{rgb}{0.18,0.24,0.31}
\xdefinecolor{darkblue}{rgb}{0.1,0.1,0.7}
\xdefinecolor{darkgreen}{rgb}{0,0.5,0}
\xdefinecolor{darkgrey}{rgb}{0.35,0.35,0.35}
\xdefinecolor{darkorange}{rgb}{0.8,0.5,0}
\xdefinecolor{darkred}{rgb}{0.7,0,0}
\definecolor{darkgreen}{rgb}{0,0.6,0}
\definecolor{mauve}{rgb}{0.58,0,0.82}

\usetikzlibrary{shapes.callouts}

\makeatletter
\setbeamertemplate{footline}
{
  \leavevmode%
  \hbox{%
  \begin{beamercolorbox}[wd=.333333\paperwidth,ht=2.25ex,dp=1ex,center]{author in head/foot}%
    \usebeamerfont{author in head/foot}\insertsection
  \end{beamercolorbox}%
  \begin{beamercolorbox}[wd=.333333\paperwidth,ht=2.25ex,dp=1ex,center]{title in head/foot}%
    \usebeamerfont{title in head/foot}\insertsubsection
  \end{beamercolorbox}%
  \begin{beamercolorbox}[wd=.333333\paperwidth,ht=2.25ex,dp=1ex,right]{date in head/foot}%
    \usebeamerfont{date in head/foot}
    \insertshortdate{}\hspace*{2em}
    \insertframenumber{} / \inserttotalframenumber\hspace*{2ex} 
  \end{beamercolorbox}}%

  \vskip0pt%
}
\makeatother

\title[]{Lecture 2: Linear Algebra}
\author{Yi, Yung (이융)}
\institute{Mathematics for Machine Learning\\ \url{https://yung-web.github.io/home/courses/mathml.html}
\\KAIST EE}
\date{\today}

%%%%%%%%%%%% real, integer notation
\newcommand{\real}{{\mathbb R}}
\newcommand{\realn}{{\mathbb R}^{n}}
\newcommand{\realm}{{\mathbb R}^{m}}
\newcommand{\realD}{{\mathbb R}^{D}}
\newcommand{\realM}{{\mathbb R}^{M}}
\newcommand{\realN}{{\mathbb R}^{N}}
\newcommand{\realnn}{{\mathbb R}^{n \times n}}
\newcommand{\realmm}{{\mathbb R}^{m \times m}}
\newcommand{\realmn}{{\mathbb R}^{m \times n}}
\newcommand{\realnm}{{\mathbb R}^{n \times m}}
\newcommand{\realDM}{{\mathbb R}^{D \times M}}
\newcommand{\realMD}{{\mathbb R}^{M \times D}}
\newcommand{\complex}{{\mathbb C}}
\newcommand{\integer}{{\mathbb Z}}
\newcommand{\natu}{{\mathbb N}}


%%% set, vector, matrix
\newcommand{\set}[1]{\ensuremath{\mathcal #1}}
\newcommand{\sets}[1]{\ensuremath{\{#1 \}}}
\renewcommand{\vec}[1]{\bm{#1}}
\newcommand{\mat}[1]{\bm{#1}}

%%%% vector
\def\vx{\vec{x}}
\def\vy{\vec{y}}
\def\vz{\vec{z}}
\def\vf{\vec{f}}
\def\ve{\vec{e}}
\def\vr{\vec{r}}
\def\vb{\vec{b}}
\def\vc{\vec{c}}
\def\vd{\vec{d}}
\def\vm{\vec{m}}
\def\vu{\vec{u}}
\def\vv{\vec{v}}
\def\vw{\vec{w}}
\def\vX{\vec{X}}
\def\vY{\vec{Y}}
\def\vZ{\vec{Z}}
\def\vth{\vec{\theta}}
\def\vmu{\vec{\mu}}
\def\vnu{\vec{\nu}}
\def\vlam{\vec{\lambda}}
\def\vep{\vec{\epsilon}}
\def\vpi{\vec{\pi}}
\def\vphi{\vec{\phi}}
\def\vxi{\vec{\xi}}
\def\valpha{\vec{\alpha}}
\def\vgamma{\vec{\gamma}}

%%%% Well-used matrices
\def\mA{\mat{A}}
\def\mB{\mat{B}}
\def\mC{\mat{C}}
\def\mD{\mat{D}}
\def\mI{\mat{I}}
\def\mJ{\mat{J}}
\def\mK{\mat{K}}
\def\mE{\mat{E}}
\def\mP{\mat{P}}
\def\mQ{\mat{Q}}
\def\mU{\mat{U}}
\def\mV{\mat{V}}
\def\mR{\mat{R}}
\def\mS{\mat{S}}
\def\mX{\mat{X}}
\def\msig{\mat{\Sigma}}
\def\mPhi{\mat{\Phi}}


\usepackage{amsmath}
%%%%% vector caculus useful macro
% ...\d, which typesets a derivative. ex: \d{y}{x}, instead of \frac{dx}{dy}.
\renewcommand{\d}[2]{\frac{\text{d} #1}{\text{d} #2}}


% ...similar for double-derivatives. ex: \dd{y}{x}.
\newcommand{\dd}[2]{\frac{\text{d}^2 #1}{\text{d} #2^2}}

% ...similar for partial derivatives. ex: \pd{y}{x}.
\newcommand{\pd}[2]{\frac{\partial #1}{\partial #2}}


% ...similar for partial double derivatives. ex: \pdd{y}{x}.
\newcommand{\pdd}[2]{\frac{\partial^2 #1}{\partial #2^2}}
% pdd with argument
\newcommand{\pdda}[3]{\frac{\partial^2 #1}{\partial #2 \partial #3}}

\usepackage{xparse}

%%%% caligraphic fonts
\def\cL{\ensuremath{{\cal L}}}
\def\cN{\ensuremath{{\cal N}}}
\def\cD{\ensuremath{{\cal D}}}
\def\cC{\ensuremath{{\cal C}}}
\def\cX{\ensuremath{{\cal X}}}
\def\cY{\ensuremath{{\cal Y}}}

%%% big parenthesis
\def\Bl{\Bigl}
\def\Br{\Bigr}
\def\lf{\left}
\def\ri{\right}


%%% floor notations
\newcommand{\lfl}{{\lfloor}}
\newcommand{\rfl}{{\rfloor}}
\newcommand{\floor}[1]{{\lfloor #1 \rfloor}}

%%% gradient
\newcommand{\grad}[1]{\nabla #1}
\newcommand{\hess}[1]{\text{H} #1}

%%% definition
%\newcommand{\eqdef}{\ensuremath{\triangleq}}
\newcommand{\eqdef}{\ensuremath{:=}}
%%% imply
\newcommand{\imp}{\Longrightarrow}



\newcommand{\separator}{
%  \begin{center}
    \par\noindent\rule{\columnwidth}{0.3mm}
%  \end{center}
}

\newcommand{\mynote}[1]{{\it \color{red} [#1]}}







%%% equation alignment
\newcommand{\aleq}[1]{\begin{align*}#1\end{align*}}

%%%%%%%%%%%%%%%% colored emphasized font, blanked words

\newcommand{\empr}[1]{{\color{red}\emph{#1}}}
\newcommand{\empb}[1]{{\color{blue}\emph{#1}}}
\newcommand{\redf}[1]{{\color{red} #1}}
\newcommand{\bluef}[1]{{\color{blue} #1}}
\newcommand{\grayf}[1]{{\color{gray} #1}}
\newcommand{\magenf}[1]{{\color{magenta} #1}}
\newcommand{\greenf}[1]{{\color{green} #1}}
\newcommand{\cyanf}[1]{{\color{cyan} #1}}
\newcommand{\orangef}[1]{{\color{orange} #1}}

\newcommand{\blk}[1]{\underline{\mbox{\hspace{#1}}}}


\newcommand{\redblk}[1]{\framebox{\color{red} #1}}
\newcommand{\redblank}[2]{\framebox{\onslide<#1->{\color{red} #2}}}
\newcommand{\blueblk}[1]{\framebox{\color{blue} #1}}
\newcommand{\blueblank}[2]{\framebox{\onslide<#1->{\color{blue} #2}}}



\makeatletter
\newcommand{\mybox}{%
    \collectbox{%
        \setlength{\fboxsep}{1pt}%
        \fbox{\BOXCONTENT}%
    }%
}
\makeatother

\makeatletter
\newcommand{\lecturemark}{%
    \collectbox{%
        \setlength{\fboxsep}{1pt}%
        \fcolorbox{red}{yellow}{\BOXCONTENT}%
    }%
}
\makeatother

\newcommand{\mycolorbox}[1]{
\begin{tcolorbox}[colback=red!5!white,colframe=red!75!black]
#1
\end{tcolorbox}
}
%%%% figure inclusion
\newcommand{\mypic}[2]{
\begin{center}
\includegraphics[width=#1\textwidth]{#2}
\end{center}
}

\newcommand{\myinlinepic}[2]{
\makebox[0cm][r]{\raisebox{-4ex}{\includegraphics[height=#1]{#2}}}
}




%%%% itemized and enumerated list
\newcommand{\bci}{\begin{compactitem}}
\newcommand{\eci}{\end{compactitem}}
\newcommand{\bce}{\begin{compactenum}}
\newcommand{\ece}{\end{compactenum}}


%%%% making 0.5/0.5 two columns
%%%% how to use: first number: length of separation bar
% \mytwocols{0.6}
% {
% contents in the left column
% }
% {
% contents in the right column
% }
%%%%

\newcommand{\mytwocols}[3]{
\begin{columns}[T] \column{.499\textwidth} #2 \column{.001\textwidth} \rule{.3mm}{{#1}\textheight} \column{.499\textwidth} #3 \end{columns}}

\newcommand{\mythreecols}[4]{
\begin{columns}[T] \column{.31\textwidth} #2 \column{.001\textwidth} \rule{.3mm}{{#1}\textheight} \column{.31\textwidth} #3 \column{.001\textwidth} \rule{.3mm}{{#1}\textheight} \column{.31\textwidth} #4  \end{columns}}

\newcommand{\mysmalltwocols}[3]{
\begin{columns}[T] \column{.4\textwidth} #2 \column{.001\textwidth} \rule{.3mm}{{#1}\textheight} \column{.4\textwidth} #3 \end{columns}}

%%%% making two columns with customized ratios
%%%% how to use: 
%first parameter: length of separation bar
%second parameter: ratio of left column
%third parameter: ratio of right column
% \mytwocols{0.6}{0.7}{0.29}
% {
% contents in the left column
% }
% {
% contents in the right column
% }
%%%%
\newcommand{\myvartwocols}[5]{
\begin{columns}[T] \column{#2\textwidth} {#4} \column{.01\textwidth} \rule{.3mm}{{#1}\textheight} \column{#3\textwidth} {#5} \end{columns}}

%%% making my block in beamer
%%% first parameter: title of block
%%% second parameter: contents of block
\newcommand{\myblock}[2]{
\begin{block}{#1} {#2}  \end{block}}

%%% independence notation
\newcommand{\indep}{\perp \!\!\! \perp}

%%%% probability with different shapes (parenthesis or bracket) and different sizes
%%% `i' enables us to insert the subscript to the probability 
\newcommand{\bprob}[1]{\mathbb{P}\Bl[ #1 \Br]}
\newcommand{\prob}[1]{\mathbb{P}[ #1 ]}
\newcommand{\cbprob}[1]{\mathbb{P}\Bl( #1 \Br)}
\newcommand{\cprob}[1]{\mathbb{P}( #1 )}
\newcommand{\probi}[2]{\mathbb{P}_{#1}[ #2 ]}
\newcommand{\bprobi}[2]{\mathbb{P}_{#1}\Bl[ #2 \Br]}
\newcommand{\cprobi}[2]{\mathbb{P}_{#1}( #2 )}
\newcommand{\cbprobi}[2]{\mathbb{P}_{#1}\Bl( #2 \Br)}

%%%% expectation with different shapes (parenthesis or bracket) and different sizes
%%% `i' enables us to insert the subscript to the expectation
\newcommand{\expect}[1]{\mathbb{E}[ #1 ]}
\newcommand{\cexpect}[1]{\mathbb{E}( #1 )}
\newcommand{\bexpect}[1]{\mathbb{E}\Bl[ #1 \Br]}
\newcommand{\cbexpect}[1]{\mathbb{E}\Bl( #1 \Br)}
\newcommand{\bbexpect}[1]{\mathbb{E}\lf[ #1 \ri]}
\newcommand{\expecti}[2]{\mathbb{E}_{#1}[ #2 ]}
\newcommand{\bexpecti}[2]{\mathbb{E}_{#1}\Bl[ #2 \Br]}
\newcommand{\bbexpecti}[2]{\mathbb{E}_{#1}\lf[ #2 \ri]}

%%%% variance
\newcommand{\var}[1]{\text{var}[ #1 ]}
\newcommand{\bvar}[1]{\text{var}\Bl[ #1 \Br]}
\newcommand{\cvar}[1]{\text{var}( #1 )}
\newcommand{\cbvar}[1]{\text{var}\Bl( #1 \Br)}

%%%% covariance
\newcommand{\cov}[1]{\text{cov}( #1 )}
\newcommand{\bcov}[1]{\text{cov}\Bl( #1 \Br)}

%%% Popular pmf, pdf notation to avoid long typing
\newcommand{\px}{\ensuremath{p_X(x)}}
\newcommand{\py}{\ensuremath{p_Y(y)}}
\newcommand{\pz}{\ensuremath{p_Z(z)}}
\newcommand{\pxA}{\ensuremath{p_{X|A}(x)}}
\newcommand{\pyA}{\ensuremath{p_{Y|A}(y)}}
\newcommand{\pzA}{\ensuremath{p_{Z|A}(z)}}
\newcommand{\pxy}{\ensuremath{p_{X,Y}(x,y)}}
\newcommand{\pxcy}{\ensuremath{p_{X|Y}(x|y)}}
\newcommand{\pycx}{\ensuremath{p_{Y|X}(y|x)}}

\newcommand{\fx}{\ensuremath{f_X(x)}}
\newcommand{\Fx}{\ensuremath{F_X(x)}}
\newcommand{\fy}{\ensuremath{f_Y(y)}}
\newcommand{\Fy}{\ensuremath{F_Y(y)}}
\newcommand{\fz}{\ensuremath{f_Z(z)}}
\newcommand{\Fz}{\ensuremath{F_Z(z)}}
\newcommand{\fxA}{\ensuremath{f_{X|A}(x)}}
\newcommand{\fyA}{\ensuremath{f_{Y|A}(y)}}
\newcommand{\fzA}{\ensuremath{f_{Z|A}(z)}}
\newcommand{\fxy}{\ensuremath{f_{X,Y}(x,y)}}
\newcommand{\Fxy}{\ensuremath{F_{X,Y}(x,y)}}
\newcommand{\fxcy}{\ensuremath{f_{X|Y}(x|y)}}
\newcommand{\fycx}{\ensuremath{f_{Y|X}(y|x)}}

\newcommand{\fth}{\ensuremath{f_\Theta(\theta)}}
\newcommand{\fxcth}{\ensuremath{f_{X|\Theta}(x|\theta)}}
\newcommand{\fthcx}{\ensuremath{f_{\Theta|X}(\theta|x)}}

\newcommand{\pkcth}{\ensuremath{p_{X|\Theta}(k|\theta)}}
\newcommand{\fthck}{\ensuremath{f_{\Theta|X}(\theta|k)}}


%%%% indicator
\newcommand{\indi}[1]{\mathbf{1}_{ #1 }}

%%%% exponential rv.
\newcommand{\elambdax}{\ensuremath{e^{-\lambda x}}}

%%%% normal  rv.
\newcommand{\stdnormal}{\ensuremath{\frac{1}{\sqrt{2\pi}} e^{-x^2/2}}}
\newcommand{\gennormal}{\ensuremath{\frac{1}{\sigma\sqrt{2\pi}} e^{-(x-\mu)^2/2}}}

%%%%%% estimator, estimate
\newcommand{\hth}{\ensuremath{\hat{\theta}}}
\newcommand{\hTH}{\ensuremath{\hat{\Theta}}}
\newcommand{\MAP}{\ensuremath{\text{MAP}}}
\newcommand{\LMS}{\ensuremath{\text{LMS}}}
\newcommand{\LLMS}{\ensuremath{\text{L}}}
\newcommand{\ML}{\ensuremath{\text{ML}}}

%%%% colored text
\newcommand{\red}[1]{\color{red}#1} 
\newcommand{\cyan}[1]{\color{cyan}#1} 
\newcommand{\magenta}[1]{\color{magenta}#1} 
\newcommand{\blue}[1]{\color{blue}#1} 
\newcommand{\green}[1]{\color{green}#1} 
\newcommand{\white}[1]{\color{white}#1} 
\newcommand{\gray}[1]{\color{gray}#1} 

%%% definition
\newcommand{\defi}{{\color{red} Definition.} } 
\newcommand{\exam}{{\color{red} Example.} } 
\newcommand{\question}{{\color{red} Question.} } 
\newcommand{\thm}{{\color{red} Theorem.} } 
\newcommand{\background}{{\color{red} Background.} } 
\newcommand{\msg}{{\color{red} Message.} } 


\def\ml{\text{ML}}
\def\map{\text{MAP}}

%%%%%%%%%%%%%%%%%%%%%%% old macros that you can ignore %%%%%%%%%%%%%%%%%%%%%%%%

% \def\un{\underline}
% \def\ov{\overline}


% \newcommand{\beq}{\begin{eqnarray*}}
% \newcommand{\eeq}{\end{eqnarray*}}
% \newcommand{\beqn}{\begin{eqnarray}}
% \newcommand{\eeqn}{\end{eqnarray}}
% \newcommand{\bemn}{\begin{multiline}}
% \newcommand{\eemn}{\end{multiline}}
% \newcommand{\beal}{\begin{align}}
% \newcommand{\eeal}{\end{align}}
% \newcommand{\beas}{\begin{align*}}
% \newcommand{\eeas}{\end{align*}}



% \newcommand{\bd}{\begin{displaymath}}
% \newcommand{\ed}{\end{displaymath}}
% \newcommand{\bee}{\begin{equation}}
% \newcommand{\eee}{\end{equation}}


% \newcommand{\vs}{\vspace{0.2in}}
% \newcommand{\hs}{\hspace{0.5in}}
% \newcommand{\el}{\end{flushleft}}
% \newcommand{\bl}{\begin{flushleft}}
% \newcommand{\bc}{\begin{center}}
% \newcommand{\ec}{\end{center}}
% \newcommand{\remove}[1]{}

% \newtheorem{theorem}{Theorem}
% \newtheorem{corollary}{Corollary}
% \newtheorem{prop}{Proposition}
% \newtheorem{lemma}{Lemma}
% \newtheorem{defi}{Definition}
% \newtheorem{assum}{Assumption}
% \newtheorem{example}{Example}
% \newtheorem{property}{Property}
% \newtheorem{remark}{Remark}

% \newcommand{\separator}{
%   \begin{center}
%     \rule{\columnwidth}{0.3mm}
%   \end{center}
% }

% \newenvironment{separation}
% { \vspace{-0.3cm}
%   \separator
%   \vspace{-0.25cm}
% }
% {
%   \vspace{-0.5cm}
%   \separator
%   \vspace{-0.15cm}
% }

% \def\A{\mathcal A}
% \def\oA{\overline{\mathcal A}}
% \def\S{\mathcal S}
% \def\D{\mathcal D}
% \def\eff{{\rm Eff}}
% \def\bD{\bm{D}}
% \def\cU{{\cal U}}
% \def\bbs{{\mathbb{s}}}
% \def\bbS{{\mathbb{S} }}
% \def\cM{{\cal M}}
% \def\bV{{\bm{V}}}
% \def\cH{{\cal H}}
% \def\ch{{\cal h}}
% \def\cR{{\cal R}}
% \def\cV{{\cal V}}
% \def\cA{{\cal A}}
% \def\cX{{\cal X}}
% \def\cN{{\cal N}}
% \def\cJ{{\cal J}}
% \def\cK{{\cal K}}
% \def\cL{{\cal L}}
% \def\cI{{\cal I}}
% \def\cY{{\cal Y}}
% \def\cZ{{\cal Z}}
% \def\cC{{\cal C}}
% \def\cR{{\cal R}}
% \def\id{{\rm Id}}
% \def\st{{\rm st}}
% \def\cF{{\cal F}}
% \def\bz{{\bm z}}
% \def\cG{{\cal G}}
% \def\N{\mathbb{N}}
% \def\bbh{\mathbb{h}}
% \def\bbH{\mathbb{H}}
% \def\bbi{\mathbb{i}}
% \def\bbI{\mathbb{I}}
% \def\R{\mathbb{R}}
% \def\bbR{\mathbb{R}}
% \def\bbr{\mathbb{r}}
% \def\cB{{\cal B}}
% \def\cP{{\cal P}}
% \def\cS{{\cal S}}
% \def\bW{{\bm W}}
% \def\bc{{\bm c}}

% %\def\and{\quad\mbox{and}\quad}
% \def\ind{{\bf 1}}


% \def\bmg{{\bm{\gamma}}}
% \def\bmr{{\bm{\rho}}}
% \def\bmq{{\bm{q}}}
% \def\bmt{{\bm{\tau}}}
% \def\bmn{{\bm{n}}}
% \def\bmcapn{{\bm{N}}}
% \def\bmrho{{\bm{\rho}}}

% \def\igam{\underline{\gamma}(\lambda)}
% \def\sgam{\overline{\gamma}(\lambda)}
% \def\ovt{\overline{\theta}}
% \def\ovT{\overline{\Theta}}
% \def\PP{{\mathrm P}}
% \def\EE{{\mathrm E}}
% \def\iskip{{\vskip -0.4cm}}
% \def\siskip{{\vskip -0.2cm}}

% \def\bp{\noindent{\it Proof.}\ }
% \def\ep{\hfill $\Box$}



%%%%%%%%%% linear algebra macros %%%%%%%%%%%%%%%%%%%%%%%

%--------linsys
%  Use as \begin{linsys}{3}
%           x &+ &3y &+ &a &= &7 \\
%           x &- &3y &+ &a &= &7
%         \end{linsys}
% Remark: TeXbook pp. 167-170 says to put a medmuskip around a +; and that's
% 4/18-ths of an em.  Why does 2/18-ths of an em work?  I don't know, but
% comparing to a regular displayed equation suggests it is right.
% (darseneau says LaTeX puts in half an \arraycolsep.)
\newenvironment{linsys}[2][m]{%
\setlength{\arraycolsep}{.1111em} % p. 170 TeXbook; a medmuskip
\begin{array}[#1]{@{}*{#2}{rc}r@{}}
}{%
\end{array}}

\newsavebox\boxofmathplus
\sbox{\boxofmathplus}{$+$}
\newcommand{\spaceforemptycolumn}{\makebox[\wd\boxofmathplus]{\ }}

%--------grstep
% For denoting a Gauss' reduction step.
% Use as: \grstep{\rho_1+\rho_3} or \grstep[2\rho_5 \\ 3\rho_6]{\rho_1+\rho_3}
% \newcommand{\grstep}[2][\relax]{%
%    \ensuremath{\mathrel{
%        \mathop{\longrightarrow}\limits^{#2\mathstrut}_{
%                                    \begin{subarray}{l} #1 \end{subarray}}}}}

% Advantage of length formulation is that between adjacent
% \grstep's you can add \hspace{-\grsteplength} to make it look not too wide
\newlength{\grsteplength}
\setlength{\grsteplength}{1.5ex plus .1ex minus .1ex}

\newcommand{\grstep}[2][\relax]{%
   \ensuremath{\mathrel{
       \hspace{\grsteplength}\mathop{\longrightarrow}\limits^{#2\mathstrut}_{
                                     \begin{subarray}{l} #1 \end{subarray}}\hspace{\grsteplength}}}}
% If two or more \grsteps are in a row then they need to be tightened
\newcommand{\repeatedgrstep}[2][\relax]{\hspace{-\grsteplength}\grstep[#1]{#2}}

% row swap operation: \rho_1\swap\rho_2
\newcommand{\swap}{\leftrightarrow}

%-------------amatrix
% Augmented matrix.  Usage (note the argument does not count the aug col):
% \begin{amatrix}{2}
%   1  2  3 \\  4  5  6
% \end{amatrix}
\newenvironment{amatrix}[1]{%
  \left(\begin{array}{@{}*{#1}{c}|c@{}}
}{%
  \end{array}\right)
}



%-------------pmat
% For matrices with arguments.
% Usage: \begin{pmat}{c|c|c} 1 &2 &3 \end{pmat}
\newenvironment{pmat}[1]{
  \left(\begin{array}{@{}#1@{}}
}{\end{array}\right)
}



%-------------misc matrices
% \newenvironment{mat}{\left(\begin{array}}{\end{array}\right)}
\newenvironment{detmat}{\left|\begin{array}}{\end{array}\right|}
\newcommand{\deter}[1]{ \mathchoice{\left|#1\right|}{|#1|}{|#1|}{|#1|} }
\newcommand{\generalmatrix}[3]{ %arg1: low-case letter, arg2: rows, arg3: cols 
               \left(
                  \begin{array}{cccc}
                    #1_{1,1}  &#1_{1,2}  &\ldots  &#1_{1,#2}  \\
                    #1_{2,1}  &#1_{2,2}  &\ldots  &#1_{2,#2}  \\
                              &\vdots                         \\
                    #1_{#3,1} &#1_{#3,2} &\ldots  &#1_{#3,#2}
                  \end{array}
               \right)  }

\newcommand{\generaldet}[3]{ %arg1: low-case letter, arg2: rows, arg3: cols 
               \left|
                  \begin{array}{cccc}
                    #1_{11}  &#1_{12}  &\ldots  &#1_{1 #2}  \\
                    #1_{21}  &#1_{22}  &\ldots  &#1_{2 #2}  \\
                              &\vdots                         \\
                    #1_{#3 1} &#1_{#3 2} &\ldots  &#1_{#3 #2}
                  \end{array}
               \right|  }

% With mathtools we can have column entries right flushed
% There is an optional argument \begin{mat}[r]{3} .. \end{mat} for
% right-flushed columns.  Perhaps the rule is that numbers are better 
% right-flushed but if there are any letters it is better centered?
\newenvironment{nmat}[1][c]{\begin{pmatrix*} % disable optional arg [#1] 
      }{\end{pmatrix*}}
% If mat starts with &\vdots get an error; why?  No apparent macro fix, according to texexchange
\newenvironment{vmat}[1][c]{\begin{vmatrix*} % disable optional arg [#1] 
      }{\end{vmatrix*}}
\newenvironment{amat}[2][c]{%
  % disable optional arg \left(\begin{array}{@{}*{#2}{#1}|#1@{}}
  \left(\begin{array}{@{}*{#2}{c}|#1@{}}
}{%
  \end{array}\right)
}
% \newcommand\vdotswithin[1]{% Taken from mathtools.dtx because my TL is not 2011
%   {\mathmakebox[\widthof{\ensuremath{{}#1{}}}][c]{{\vdots}}}}


%------------colvec and rowvec
% Column vector and row vector.  Usage:
%  \colvec{1  \\ 2 \\ 3 \\ 4} and \rowvec{1  &2  &3}
% Colvec takes an optional argument \colvec[r]{x_1 \\ 0}.  Perhaps 
% digits look better right aligned, but if there are any letters it
% needs to be centered?
\newcommand{\colvec}[2][c]{\begin{nmat}[#1] #2 \end{nmat}}
\newcommand{\smallcolvec}[1]{\left(\begin{smallmatrix} #1 \end{smallmatrix}\right)}
% For row vectors, cannot do \newcommand{\rowvec}[1]{\begin{mat} #1 \end{mat}}
% since the delimiters come out too large.
\newcommand{\rowvec}[1]{\setlength{\arraycolsep}{3pt}\left(\begin{matrix} #1 \end{matrix}\right)}



%-------------making aligned columns
% Usage: \begin{aligncolondecimal}{2} 1.2 \\ .33 \end{aligncolondecimal}
% (negative argument centers decimal pt in column).  Also Usage:
% \begin{aligncolondecimal}[0em]{2} 1.2 \\ .33 \end{aligncolondecimal}
% to make the left and right LaTeX-array padding disappear.
\RequirePackage{array}\RequirePackage{dcolumn}
\newenvironment{aligncolondecimal}[2][.1111em]{%
\setlength{\arraycolsep}{#1}
\newcolumntype{.}{D{.}{.}{#2}}\begin{array}{.}}{%
\end{array}}

% Matrix and vector, with numbers centered on decimal point
% Usage: \begin{dmat}{D{.}{.}{1}D{.}{.}{3}}  0  &.123 \\ .2 &.456 \end{dmat}
%  (in the D{.}{.}{number} that is the number of decimal places)
\newlength{\dmatcolsep}\setlength{\dmatcolsep}{5pt}
\newenvironment{dmat}[2][\dmatcolsep]{%
  \setlength{\arraycolsep}{#1}
  \left(\begin{array}{@{}#2@{}}
}{%
  \end{array}\right)}
% Usage: \dcolvec[2]{1.23 \\ 4.56} where the optional argument is the number
% of decimal places.
\newcommand{\dcolvec}[2][-1]{\left(\begin{array}{@{}D{.}{.}{#1}@{}} #2 \end{array}\right)}

%\newcommand{\trans}[1]{ {{#1}^{\mathsf{T}}} } 
\newcommand{\trans}[1]{ {#1}^{\mathsf{T}} } 
\newcommand{\inv}[1]{ {#1}^{-1} } 
\newcommand{\spn}[1]{\ensuremath{\text{span}[#1]} } 
\newcommand{\rk}[1]{\ensuremath{\text{rk}(#1)} } 
\newcommand{\dimm}[1]{\ensuremath{\text{dim}(#1)} } 
\newcommand{\img}[1]{\ensuremath{\text{Im}(#1)} } 
%\newcommand{\norm}[1]{\ensuremath{\left || #1 \right ||} } 
\newcommand{\norm}[1]{\ensuremath{\left \lVert #1 \right \rVert} } 
% orthogonal complement
\newcommand{\ocomp}[1]{\ensuremath{#1^{\bot}} } 
\newcommand{\inner}[2]{\ensuremath{\left\langle #1, #2 \right\rangle} } 
\DeclareMathOperator{\tr}{tr}


% \NewDocumentCommand{\grad}{e{_^}}{%
%   \mathop{}\!% \mathop for good spacing before \nabla
%   \nabla
%   \IfValueT{#1}{_{\!#1}}% tuck in the subscript
%   \IfValueT{#2}{^{#2}}% possible superscript
% }
% \begin{equation*}
%          \begin{nmat}[r]
% 1 &2 &13 \\
%           4  &5  &6
%          \end{nmat}
%       \end{equation*}

% \begin{equation*}
%          \begin{amat}{2}
%           1  &2  &3  \\
%           4  &5  &6
%          \end{amat}
%       \end{equation*}
       
%       \begin{equation*}
%          \begin{pmat}{c|c|c}
% 1 &2 &3 \\
%           4  &5  &6
%          \end{pmat}
%       \end{equation*}

% \begin{equation*}
%          \begin{vmat}
% a &c \\
%           b  &d
%          \end{vmat}
%          =ad-bc
% \end{equation*}

%  \begin{equation*}
%   \vec{v}=\colvec{-1  \\ -0.5  \\ 0}
% \end{equation*}

%  \begin{equation*}
%   \vec{v}=\rowvec{-1  & -0.5  & 0}
% \end{equation*}




\begin{document}

%itemshape
\setbeamertemplate{itemize item}{\scriptsize\raise1.25pt\hbox{\donotcoloroutermaths$\bullet$}}
\setbeamertemplate{itemize subitem}{\tiny\raise1.5pt\hbox{\donotcoloroutermaths$\circ$}}
\setbeamertemplate{itemize subsubitem}{\tiny\raise1.5pt\hbox{\donotcoloroutermaths$\blacktriangleright$}}
%default value for spacing
\plitemsep 0.1in
\pltopsep 0.03in
\setlength{\parskip}{0.15in}
%\setlength{\parindent}{-0.5in}
\setlength{\abovedisplayskip}{0.07in}
\setlength{\belowdisplayskip}{0.07in}
\setlength{\mathindent}{0cm}
\setbeamertemplate{frametitle continuation}{[\insertcontinuationcount]}

\setlength{\leftmargini}{0.5cm}
\setlength{\leftmarginii}{0.5cm}

\setlength{\fboxrule}{0.05pt}
\setlength{\fboxsep}{5pt}


%%%%%%% This should be placed at the end of this file
\logo{\pgfputat{\pgfxy(0.11, 7.4)}{\pgfbox[right,base]{\tikz{\filldraw[fill=dianablue, draw=none] (0 cm, 0 cm) rectangle (50 cm, 1 cm);}\mbox{\hspace{-8 cm}\includegraphics[height=0.7 cm]{../kaist_ee.png}
}}}}

\begin{frame}
  \titlepage
\end{frame}

\logo{\pgfputat{\pgfxy(0.11, 7.4)}{\pgfbox[right,base]{\tikz{\filldraw[fill=dianablue, draw=none] (0 cm, 0 cm) rectangle (50 cm, 1 cm);}\mbox{\hspace{-8 cm}\includegraphics[height=0.7 cm]{../kaist_ee.png}
}}}}

% rule color - gray
\makeatletter
\let\old@rule\@rule
\def\@rule[#1]#2#3{\textcolor{gray}{\old@rule[#1]{#2}{#3}}}
\makeatother







% START START START START START START START START START START START START START
%%%%%%%%%%%%%%%%%%%%%%%%%%%%%%%%%%%%%%%%%%%%%%%%%%%%%%

\begin{frame}{Roadmap}

\plitemsep 0.1in

\bce[(1)] 
\item Systems of Linear Equations

\item Matrices

\item Solving Systems of Linear Equations

\item Vector Spaces

\item Linear Independence

\item Basis and Rank

\item Linear Mappings

\item Affine Spaces

\ece
\end{frame}

%%%%%%%%%%%%%%%%%%%%%%%%%%%%%%%%%%%%%%%%%%%%%%%%%%%%%%
\section{L2(1)}
\begin{frame}{Roadmap}

\plitemsep 0.1in

\bce[(1)] 
\item \redf{Systems of Linear Equations}

\item \grayf{Matrices

\item Solving Systems of Linear Equations

\item Vector Spaces

\item Linear Independence

\item Basis and Rank

\item Linear Mappings

\item Affine Spaces
}
\ece
\end{frame}


%%%%%%%%%%%%%%%%%%%%%%%%%%%%%%%%%%%%%%%%%%%%%%%%%%%%%%
\begin{frame}{Linear Algebra}

\plitemsep 0.1in

\bci 
\item Algebra: a set of objects and a set of rules or operations to manipulate those objects

\item Linear algebra
\plitemsep 0.05in
\bci
\item Object: vectors $\vec{v}$
\item Operations: their additions ($\vec{v} + \vec{w}$) and scalar multiplication ($k\vec{v}$)
\eci

\item Examples

\mysmalltwocols{0.3}
{
\bci
\item Geometric vectors

- High school physics
\item Polynomials
\item Audio signals
\item Elements of $\real^n$
\eci
}
{
\centering
\mypic{0.97}{L2_vector_ex.png}
}

\eci
\end{frame}

%%%%%%%%%%%%%%%%%%%%%%%%%%%%%%%%%%%%%%%%%%%%%%%%%%%%%%
\begin{frame}{System of Linear Equations}

\plitemsep 0.1in


\bci 
\item For unknown variables $(x_1, \cdots, x_n) \in \realn,$
\aleq{
a_{11} x_1 + \cdots &+ a_{1n}x_n = b_1 \cr
&\vdots \cr
a_{m1} x_1 + \cdots &+ a_{mn}x_n = b_m 
}

\item Three cases of solutions

\mythreecols{0.25}
{
- No solution
$$
\begin{linsys}{3}
x_1 &+ &x_2 &+ &x_3 &= &3 \cr
x_1 &- &x_2 &+ &2x_3 &= &2 \cr
2x_1 & & &+ &3x_3 &= &1 
\end{linsys}
$$
}
{
- Unique solution
$$
\begin{linsys}{3}
x_1 &+ &x_2 &+ &x_3 &= &3 \cr
x_1 &- &x_2 &+ &2x_3 &= &2 \cr
 & &x_2 &+ &3x_3 &= &1 
\end{linsys}
$$
}
{
- Infinitely many solutions 
$$
\begin{linsys}{3}
x_1 &+ &x_2 &+ &x_3 &= &3 \cr
x_1 &- &x_2 &+ &2x_3 &= &2 \cr
2x_1 & & &+ &3x_3 &= &5 
\end{linsys}
$$
}
\item \question Under what conditions, one of the above three cases occur?
\eci
\end{frame}

%%%%%%%%%%%%%%%%%%%%%%%%%%%%%%%%%%%%%%%%%%%%%%%%%%%%%%
\begin{frame}{Matrix Representation}

\plitemsep 0.1in

\bci 
\item A collection of linear equations
\aleq{
a_{11} x_1 + \cdots &+ a_{1n}x_n = b_1 \cr
&\vdots \cr
a_{m1} x_1 + \cdots &+ a_{mn}x_n = b_m 
}

\item Matrix representations: 
\aleq{
\colvec{a_{11} \\ \vdots \\ a_{m1}} x_1 + \cdots + \colvec{a_{1n} \\ \vdots \\ a_{mn}} x_n = 
\colvec{b_1 \\ \vdots \\ b_m} \Longleftrightarrow 
\underbrace{
\begin{nmat}
 a_{11} &\cdots &a_{1n} \\
\vdots & & \vdots \\
a_{m1}  & \cdots  &a_{mn}
\end{nmat}}_{\mat{A}} \underbrace{\colvec{x_1 \\ \vdots \\ x_n}}_{\vec{x}} = 
\underbrace{\colvec{b_1 \\ \vdots \\ b_m}}_{\vec{b}}
}
\item Understanding $\mat{A}$ is the key to answering various questions about this linear system $\mat{A}\vec{x} = \vec{b}$.

\eci
\end{frame}


%%%%%%%%%%%%%%%%%%%%%%%%%%%%%%%%%%%%%%%%%%%%%%%%%%%%%%
\section{L2(2)}
\begin{frame}{Roadmap}

\plitemsep 0.1in

\bce[(1)] 
\item \grayf{Systems of Linear Equations}

\item \redf{Matrices}

\item \grayf{Solving Systems of Linear Equations

\item Vector Spaces

\item Linear Independence

\item Basis and Rank

\item Linear Mappings

\item Affine Spaces
}
\ece
\end{frame}

%%%%%%%%%%%%%%%%%%%%%%%%%%%%%%%%%%%%%%%%%%%%%%%%%%%%%%
\begin{frame}{Matrix: Addition and Multiplication}

\plitemsep 0.1in

\bci 
\item For two matrices $\mat{A} \in \real^{m \times n}$ and $\mat{B} \in \real^{m \times n}$, 
\aleq{
\mat{A} + \mat{B} \eqdef \begin{nmat}
 a_{11} + b_{11} &\cdots &a_{1n} + b_{1n} \\
\vdots & & \vdots \\
a_{m1} + b_{m1}  & \cdots  &a_{mn} + b_{mn}
\end{nmat}
\in \real^{m \times n}
}

\item For two matrices $\mat{A} \in \real^{m \times n}$ and $\mat{B} \in \real^{n \times k}$, 
the elements $c_{ij}$ of the product $\mat{C} = \mat{A}\mat{B} \in \real^{m \times k}$ is:
$$
c_{ij} = \sum_{l=1}^n a_{il} b_{lj}, \quad i = 1, \ldots, m, \quad j = 1, \ldots, k.
$$

\item \exam 
$\mat{A} = 
\begin{nmat} 
 1 &2 &3 \\
 3  &2  &1
\end{nmat}
$ and 
$\mat{B} = 
\begin{nmat} 
 0 &2 \\
 1  &-1\\
 0  & 1
\end{nmat}
$, compute $\mat{AB}$ and $\mat{BA}.$
\eci
\end{frame}

%%%%%%%%%%%%%%%%%%%%%%%%%%%%%%%%%%%%%%%%%%%%%%%%%%%%%%
\begin{frame}{Identity Matrix and Matrix Properties}

\plitemsep 0.1in

\bci 
\item A square matrix\footnote{\# of rows = \# of cols} $\mat{I}_n$ with $I_{ii} = 1$ and $I_{ij=0}$ for $i \neq j,$ where $n$ is the number of rows and columns. For example, 
$$
\mat{I}_2 = 
\begin{nmat} 
 1 &0 \\
 0  &1
\end{nmat}
, \quad 
\mat{I}_4 = 
\begin{nmat} 
 1 &0 & 0 & 0\\
 0  &1 & 0 & 0 \\
 0  &0 & 1 & 0 \\
 0  &0 & 0 & 1
\end{nmat}
$$

\bigskip
\item \bluef{Associativity}: For $\mat{A} \in \real^{m \times n}$, $\mat{B} \in \real^{n \times p},$ $\mat{C} \in \real^{p \times q},$ $(\mat{AB})\mat{C} = \mat{A}(\mat{BC})$

\item \bluef{Distributivity}: For $\mat{A}, \mat{B} \in \real^{m \times n},$ and $\mat{C}, \mat{D} \in \real^{n \times p},$\\ \hspace{2.2cm}(i) $(\mat{A} + \mat{B})\mat{C} = \mat{AC} + \mat{BC}$ and (ii) $\mat{A}(\mat{C} + \mat{D}) = \mat{AC} + \mat{AD}$

\item \bluef{Multiplication with the identity matrix}: For $\mat{A} \in \real^{m \times n}$, $\mat{I}_m \mat{A} = \mat{A} \mat{I}_n = \mat{A}$

\eci
\end{frame}

%%%%%%%%%%%%%%%%%%%%%%%%%%%%%%%%%%%%%%%%%%%%%%%%%%%%%%
\begin{frame}{Inverse and Transpose}

\mytwocols{0.75}
{
\plitemsep 0.1in
\bci 
\item For a square matrix $\mat{A} \in \real^{n \times n},$ $\mat{B}$ is the \redf{inverse} of $A$, denoted by $\inv{\mat{A}},$ if $$\mat{AB} = \mat{I}_n = \mat{BA}.$$

\item Called \bluef{regular/invertible/nonsingular}, if it exists.  

\item If it exists, it is unique. 

\item How to compute?  For $2\times 2$ matrix, 
\aleq{
\inv{\mat{A}} = \frac{1}{a_{11}a_{22} - a_{12}a_{21}} 
\begin{nmat} 
 a_{22} &-a_{12} \\
 -a_{21}  &a_{11}
\end{nmat}
}
\eci
}
{
\plitemsep 0.1in
\bci 
\item For a matrix $\mat{A} \in \real^{m \times n},$ $\mat{B} \in \real^{n \times m}$ with $b_{ij} = a_{ji}$ is the \redf{transpose} of $\mat{A},$ which we denote by $\trans{\mat{A}}.$ 

\item \exam For 
$\mat{A} = 
\begin{nmat} 
 0 &2 \\
 1  &-1\\
 0  & 1
\end{nmat}
$,
\aleq{
\trans{\mat{A}} = \begin{nmat} 
 0 &1 & 0 \\
 2  &-1 & 1
\end{nmat}
}

\item If $\mat{A} = \trans{\mat{A}},$ $\mat{A}$ is called \redf{symmetric}.
\eci
}

\end{frame}

%%%%%%%%%%%%%%%%%%%%%%%%%%%%%%%%%%%%%%%%%%%%%%%%%%%%%%
\begin{frame}{Inverse and Transpose: More Properties}

\plitemsep 0.2in
\bci 
\item $\mat{A}\inv{\mat{A}} = \mat{I} = \inv{\mat{A}}\mat{A}$

\item $\inv{(\mat{AB})} = \inv{\mat{B}} \inv{\mat{A}}$

\item $\inv{(\mat{A} + \mat{B})} \neq \inv{\mat{A}} + \inv{\mat{B}}$

\item $\trans{(\trans{\mA})} = \mA$  

\item $\trans{(\mA + \mB)} = \trans{\mA} + \trans{\mB}$

\item $\trans{(\mA\mB)} = \trans{\mB}\trans{\mA}$

\item If $\mA$ is invertible, so is $\trans{\mA}.$
\eci

\end{frame}

%%%%%%%%%%%%%%%%%%%%%%%%%%%%%%%%%%%%%%%%%%%%%%%%%%%%%%
\begin{frame}{Scalar Multiplication}

\plitemsep 0.1in
\bci 
    \item Multiplication by a scalar $\lambda \in \real$ to $\mA \in \real^{m \times n}$
    \item \exam For 
$\mat{A} = 
\begin{nmat} 
 0 &2 \\
 1  &-1\\
 0  & 1
\end{nmat}
$,
$
3 \times \mA = \begin{nmat} 
 0 &6 \\
 3  &-3\\
 0  & 3
\end{nmat}
$

\bigskip

\item \bluef{Associativity}
\bci
\item $(\lambda \psi) \mC = \lambda (\psi \mC)$
\item $\lambda(\mB \mC) = (\lambda \mB) \mC = \mB (\lambda \mC) = (\mB \mC) \lambda$
\item $\trans{(\lambda \mC)} = \trans{\mC} \trans{\lambda} = \trans{\mC} \lambda = \lambda \trans{\mC}$
\eci

\item \bluef{Distributivity}
\bci
\item $(\lambda + \psi) \mC = \lambda \mC + \psi \mC$
\item $\lambda(\mB + \mC) = \lambda \mB + \lambda \mC$
\eci


\eci

\end{frame}


%%%%%%%%%%%%%%%%%%%%%%%%%%%%%%%%%%%%%%%%%%%%%%%%%%%%%%
\section{L2(3)}
\begin{frame}{Roadmap}

\plitemsep 0.1in

\bce[(1)] 
\item \grayf{Systems of Linear Equations

\item Matrices}

\item \redf{Solving Systems of Linear Equations}

\item \grayf{Vector Spaces

\item Linear Independence

\item Basis and Rank

\item Linear Mappings

\item Affine Spaces
}
\ece
\end{frame}

%%%%%%%%%%%%%%%%%%%%%%%%%%%%%%%%%%%%%%%%%%%%%%%%%%%%%%
\begin{frame}{Example}

\mysmalltwocols{0.2}
{
\begin{equation*}
  \begin{linsys}{3}
       -3x   &   &   &+  &2z  &=  &-1  \\
         x   &-  &2y &+  &2z  &=  &-5/3  \\
        -x   &-  &4y &+  &6z  &=  &-13/3
  \end{linsys}
\end{equation*}
}
{
\bci
\item $\rho_i$: $i$-th equation
\item Express the equation as its \redf{augmented matrix}.
\eci
}
\vspace{-0.5cm}
\begin{eqnarray*}
    \begin{amat}[r]{3}
     -3  &0  &2  &-1  \\
      1  &-2 &2  &-5/3  \\
     -1  &-4 &6  &-13/3
    \end{amat}
  &\grstep[-(1/3)\rho_1+\rho_3]{(1/3)\rho_1+\rho_2}
  &\begin{amat}[r]{3}
     -3  &0  &2    &-1  \\
      0  &-2 &8/3  &-2  \\
      0  &-4 &16/3 &-4
    \end{amat}                      \\
  &\grstep{-2\rho_2+\rho_3}
  &\begin{amat}[r]{3}
     -3  &0  &2    &-1  \\
      0  &-2 &8/3  &-2  \\
      0  &0  &0    &0
    \end{amat} 
\end{eqnarray*}
The two nonzero rows give
$-3x+2z=-1$ and $-2y+(8/3)z=-2$.

\footnotetext{Examples from this slide to the next several slides come from Jim Hefferson's Linear Algebra book.}

\end{frame}


%%%%%%%%%%%%%%%%%%%%%%%%%%%%%%%%%%%%%%%%%%%%%%%%%%%%%%
\begin{frame}
\small
\noindent - Parametrizing
$-3x+2z=-1$ and $-2y+(8/3)z=-2$
gives:
\mysmalltwocols{0.2}
{
\vspace{-0.4cm}
\aleq{
  x &= (1/3)+(2/3)z  \\
  y &= 1+(4/3)z   \\
  z &= z
}
}
{
%The solution set in vectors.
\vspace{-0.1cm}
\aleq{
  \{\colvec{x \\ y \\ z}=\colvec{1/3 \\ 1 \\ 0}+\colvec{2/3 \\ 4/3 \\ 1}z 
                 \mid z\in \real \}
}
}
This helps us understand the set of solutions, e.g., each value of $z$ gives a different solution.
\begin{center}\renewcommand{\arraystretch}{1.2}%
  \begin{tabular}{r|cccc}
    $z$\,                    &$0$  &$1$  &$2$  &$-1/2$ \\
    \cline{2-5}
    solution $\colvec{x \\ y \\ z}$ &$\colvec{1/3 \\ 1 \\ 0}$ 
                           &$\colvec{1 \\ 7/3 \\ 1}$
                           &$\colvec{5/3 \\ 11/3 \\ 2}$
                           &$\colvec{0 \\ 1/3 \\ -1/2}$
    \setbox0=\hbox{$\colvec{1 \\ 7/3 \\ 1}$}
      \rule{0em}{1.1\ht0}
  \end{tabular}
\end{center}
\vspace{-0.9cm}
\end{frame}

%%%%%%%%%%%%%%%%%%%%%%%%%%%%%%%%%%%%%%%%%%%%%%%%%%%%%%
\begin{frame}{Form of solution sets} 

\bci
\item The system $
  \begin{linsys}{4}
    x &+  &2y  &- &z  &  &  &= &2 \\
   2x &-  &y   &- &2z &+ &w &= &5
  \end{linsys}
$ \hspace{0.5cm} reduces in this way.
\begin{equation*}
    \begin{amat}[r]{4}
      1  &2  &-1  &0  &2  \\
      2  &-1 &-2  &1  &5  
    \end{amat}
  \grstep{-2\rho_1+\rho_2}
  \begin{amat}[r]{4}
      1  &2  &-1  &0  &2  \\
      0  &-5 &0   &1  &1  
    \end{amat}
\end{equation*}
\item It 
has solutions of this form. 
\begin{equation*}
     \colvec{x  \\  y  \\  z  \\  w}
     =
     \colvec{12/5 \\ -1/5 \\ 0 \\ 0}
       +\colvec{1 \\ 0 \\ 1 \\ 0}z
       +\colvec{-2/5 \\ 1/5 \\ 0 \\ 1}w
   \qquad
   \text{for $z,w\in\Re$}
\end{equation*}
\item Note that taking $z=w=0$ 
shows that the first vector is a \redf{particular solution} of the
system.
\eci
\end{frame}

%%%%%%%%%%%%%%%%%%%%%%%%%%%%%%%%%%%%%%%%%%%%%%%%%%%%%%
\begin{frame}{General $=$ Particular $+$ Homogeneous} 

\bci
\item General approach
\bce
\item Find a particular solution to $\mA \vec{x} = \vec{b}$
\item Find all solutions to the homogeneous equation $\mA \vec{x} = \vec{0}$
\bci
\item $\vec{0}$ is a trivial solution 
\eci
\item Combine the solutions from steps 1. and 2. to the general solution
\ece

\item Questions: A formal algorithm that performs the above?
\bci
\item \redf{Gauss-Jordan method:} convert into a ``beautiful" form 

(formally \bluef{reduced row-echelon} form)

\item Elementary transformations: (i) row swapping (ii) multiply by a constant (iii) row addition
\eci

\item Such a form allows an algorithmic way of solving linear equations

\eci
\end{frame}

%%%%%%%%%%%%%%%%%%%%%%%%%%%%%%%%%%%%%%%%%%%%%%%%%%%%%%
\begin{frame}{Example: Unique Solution} 

\plitemsep 0.1in
\small
\bci
\item Start as usual by getting echelon form.
\aleq{
&  \begin{linsys}{3}
    x  &+  &y  &-  &z  &=  &2   \\
   2x  &-  &y  &   &   &=  &-1  \\
    x  &-  &2y &+  &2z &=  &-1
  \end{linsys}
  \grstep[-1\rho_1+\rho_3]{-2\rho_1+\rho_2}
  \begin{linsys}{3}
    x  &+  &y  &-  &z  &=  &2   \\
       &-  &3y &+  &2z &=  &-5  \\
       &-  &3y &+  &3z &=  &-3
  \end{linsys}                         
  \grstep{-1\rho_2+\rho_3}
  \begin{linsys}{3}
    x  &+  &y  &-  &z  &=  &2   \\
       &-  &3y &+  &2z &=  &-5  \\
       &   &   &   &z  &=  &2
  \end{linsys}
}
\vspace{-0.4cm}
\item Make all the leading entries one.
\aleq{
 & \grstep{(-1/3)\rho_2}
  \begin{linsys}{3}
    x  &+  &y  &-  &z      &=  &2   \\
       &   &y  &-  &(2/3)z &=  &5/3  \\
       &   &   &   &z      &=  &2
  \end{linsys}
}
\vspace{-0.4cm}
\item Finish by using the leading entries to eliminate upwards,
until we can read off the solution.
\aleq{
  \begin{linsys}{3}
    x  &+  &y  &-  &z      &=  &2   \\
       &   &y  &-  &(2/3)z &=  &5/3  \\
       &   &   &   &z      &=  &2
  \end{linsys}
  \grstep[(2/3)\rho_3+\rho_2]{\rho_3+\rho_1}
  \begin{linsys}{3}
    x  &+  &y  &\spaceforemptycolumn   &       &=  &4   \\
       &   &y  &   &       &=  &3  \\
       &   &   &   &z      &=  &2
  \end{linsys}                                                
  \grstep{-\rho_2+\rho_1}
  \begin{linsys}{3}
    x  &\spaceforemptycolumn   &   &\spaceforemptycolumn   &       &=  &1   \\
       &   &y  &   &       &=  &3  \\
       &   &   &   &z      &=  &2
  \end{linsys}
}
\eci
\end{frame}

%%%%%%%%%%%%%%%%%%%%%%%%%%%%%%%%%%%%%%%%%%%%%%%%%%%%%%
\begin{frame}{Example: Infinite Number of Solutions} 

\mytwocols{0.8}
{
\plitemsep 0.01in
\small
%\exam 
$$
  \begin{linsys}{4}
    x  &-  &y  &   &   &-  &2w  &=  &2   \\
    x  &+  &y  &+  &3z &+  &w   &=  &1  \\
       &-  &y  &+  &z  &-  &w   &=  &0
  \end{linsys}
$$

\bci
\item Start by getting echelon form and turn the leading entries to $1$'s.
%\hspace*{-2em}
\aleq{
 & \grstep{-1\rho_1+\rho_2}
  \begin{amat}{4}
    1  &-1  &0  &-2  &2  \\
    0  &2   &3  &3   &-1   \\
    0  &-1  &1  &-1  &0
  \end{amat} \cr
&  \grstep{(1/2)\rho_2+\rho_3} 
  \begin{amat}{4}
    1  &-1  &0    &-2   &2  \\
    0  &2   &3    &3    &-1   \\
    0  &0   &5/2  &1/2  &-1/2
  \end{amat}\cr
&  \grstep[(2/5)\rho_3]{(1/2)\rho_2}
  \begin{amat}{4}
    1  &-1  &0    &-2    &2  \\
    0  &1   &3/2  &3/2   &-1/2   \\
    0  &0   &1    &1/5   &-1/5
  \end{amat}
}

\eci
}
{
\plitemsep 0.1in
\small
\bci
\item Eliminate upwards.
\aleq{
 & \grstep{-(3/2)\rho_3+\rho_2}
  \begin{amat}{4}
    1  &-1  &0    &-2    &2  \\
    0  &1   &0    &6/5   &-1/5   \\
    0  &0   &1    &1/5   &-1/5
  \end{amat} \cr
&  \grstep{\rho_2+\rho_1}
  \begin{amat}{4}
    1  &0   &0    &-4/5  &9/5  \\
    0  &1   &0    &6/5   &-1/5   \\
    0  &0   &1    &1/5   &-1/5
  \end{amat}
}
\item The parameterized solution set is:
\begin{equation*}
  \{\colvec{9/5 \\ -1/5 \\ -1/5 \\ 0}
       +\colvec{4/5  \\ -6/5 \\ -1/5 \\ 1} w
       \mid w\in\real \}
\end{equation*}

\eci
}

\end{frame}

%%%%%%%%%%%%%%%%%%%%%%%%%%%%%%%%%%%%%%%%%%%%%%%%%%%%%%
\begin{frame}{Cases of Solution Sets} 

\Large
\begin{center} 
\begin{tabular}{r@{}c}
  &\hspace*{2.5em}\begin{tabular}{c}
      \textit{number of solutions of the} \\[-.5ex]
      \textit{homogeneous system}
    \end{tabular}                            \\[1.7ex]
  \begin{tabular}{r@{\hspace*{.5em}}}
     \ \\[.6ex]
     \textit{particular} \\[-.55ex]
     \textit{solution}   \\[-.5ex]
     \textit{exists?}
  \end{tabular}
  &\begin{tabular}{r|c@{\hspace*{1em}}c} % \cline{2-3}
     \multicolumn{1}{c}{\ }
         &\textit{\bluef{one}}    &\textit{\bluef{infinitely many}}                    \\
     \cline{2-3}
     \textit{\bluef{yes}}
        &\rule{0ex}{16pt}\begin{tabular}{@{}c@{}} \redf{unique} \\[-.5ex] solution \end{tabular}
        &\begin{tabular}{@{}c@{}} \redf{infinitely many} \\[-.5ex] solutions \end{tabular}
         \\[2ex] % \hline
     \textit{\bluef{no}}
        &\begin{tabular}{@{}c@{}} \redf{no} \\[-.5ex] solutions \end{tabular}
        &\begin{tabular}{@{}c@{}} \redf{no} \\[-.5ex] solutions \end{tabular}
         \\ % \hline
   \end{tabular}
\end{tabular}
\end{center}

\end{frame}

%%%%%%%%%%%%%%%%%%%%%%%%%%%%%%%%%%%%%%%%%%%%%%%%%%%%%%
\begin{frame}{Algorithms for Solving System of Linear Equations} 

\plitemsep 0.05in

\bce
\item Pseudo-inverse
\aleq{
\mA \vec{x} = \vec{b} \Longleftrightarrow \trans{\mA} \mA \vec{x} = \trans{\mA} \vec{b} 
\Longleftrightarrow \vec{x} = \inv{(\trans{\mA} \mA)} \trans{\mA} \vec{b}
}
\vspace{-0.6cm}
\bci
\item $\inv{(\trans{\mA} \mA)} \trans{\mA}$: \bluef{\em Moore-Penrose pseudo-inverse}
\item many computations: matrix product, inverse, etc
\eci

\item Gaussian elimination
\bci
\item intuitive and constructive way
\item cubic complexity (in terms of \# of simultaneous equations)
\eci

\item Iterative methods
\bci
\item practical ways to solve indirectly

\item[(a)] stationary iterative methods: Richardson method, Jacobi method, Gaus-Seidel method, successive over-relaxation method
\item[(b)] Krylov subspace methods: conjugate gradients, generalized minimal residual, biconjugate gradients  
\eci

\ece
\end{frame}


%%%%%%%%%%%%%%%%%%%%%%%%%%%%%%%%%%%%%%%%%%%%%%%%%%%%%%
\section{L2(4)}
\begin{frame}{Roadmap}

\plitemsep 0.1in

\bce[(1)] 
\item \grayf{Systems of Linear Equations

\item Matrices

\item Solving Systems of Linear Equations}

\item \redf{Vector Spaces}

\item \grayf{Linear Independence

\item Basis and Rank

\item Linear Mappings

\item Affine Spaces
}
\ece
\end{frame}

%%%%%%%%%%%%%%%%%%%%%%%%%%%%%%%%%%%%%%%%%%%%%%%%%%%%%%
\begin{frame}{Group} 

\plitemsep 0.1in

\bci
\item A set $\set{G}$ and an operation $\otimes: \set{G} \times \set{G} \mapsto \set{G}.$ $G \eqdef (\set{G},\otimes)$ is called a \redf{group}, if:

\bce
\item \bluef{Closure.} $\forall x,y \in \set{G},$ $x \otimes y \in \set{G}$
\item \bluef{Associativity.} $\forall x,y,z \in \set{G}$, $(x \otimes y) \otimes z = x \otimes (y \otimes z)$
\item \bluef{Neutral element.} $\exists e \in \set{G},$ $\forall x \in \set{G},$ $x \otimes e = x$ and $e \otimes x = x$
\item \bluef{Inverse element.} $\forall x \in \set{G},$ $\exists y \in \set{G},$  $x \otimes y = e$ and $y \otimes x = e.$ We often use $x^{-1} =y.$
\ece

\vspace{1cm}

\item $G = (\set{G},\otimes)$ is an \redf{Abelian group,} if the following is additionally met:
\bci
\item \bluef{Communicativity.} $\forall x,y \in \set{G},$ $x \otimes y = y \otimes x$
\eci
\eci
\end{frame}

%%%%%%%%%%%%%%%%%%%%%%%%%%%%%%%%%%%%%%%%%%%%%%%%%%%%%%
\begin{frame}{Examples} 

\plitemsep 0.05in

\bci
\item $(\integer,+)$ is an Abelian group
\item $(\natu \cup \{ 0\}, +)$ is not a group (because inverses are missing)
\item $(\integer,\cdot)$ is not a group
\item $(\real,\cdot)$ is not a group (because of no inverse for 0)
\item $(\real^n,+)$, $(\integer^n,+)$ are Abelian, if $+$ is defined componentwise

\item $(\real^{m \times n},+)$ is Abelian (with componentwise $+$)
\item $(\real^{n \times n},\cdot)$
\bci
\item Closure and associativity follow directly 
\item Neutral element: $\mat{I}_n$
\item The inverse $\inv{\mA}$ may exist or not. So, generally, it is not a group. 
However, the set of invertible matrices in $\real^{n \times n}$ with matrix multiplication  is a group, called \bluef{general linear group.}

\eci
\eci
\end{frame}

%%%%%%%%%%%%%%%%%%%%%%%%%%%%%%%%%%%%%%%%%%%%%%%%%%%%%%
\begin{frame}{Vector Spaces} 

\plitemsep 0.1in

\defi A real-valued vector space $V=(\set{V},+,\cdot)$ is a set $\set{V}$ with two operations
\aleq{
(a) \quad + &: \set{V} \times \set{V} \mapsto \set{V} \quad \text{(vector addition)}\cr
(b) \quad \ \cdot &:  \real \times \set{V} \mapsto \set{V} \quad \text{(scalar multiplication)},
}
where
\vspace{-0.2cm}
\bce
\item  $(\set{V},+)$ is an Abelian group
\item \bluef{Distributivity.} 
\bci
\item $\forall \lambda \in \real, \vec{x},\vec{y} \in \set{V},$ $\lambda \cdot(\vec{x} + \vec{y}) = \lambda \cdot \vec{x} + \lambda \vec{y}$
\item $\forall \lambda, \psi \in \real, \vec{x} \in \set{V},$ $(\lambda + \psi) \cdot \vec{x} = \lambda \cdot \vec{x} + \psi \cdot \vec{x}$
\eci
\item \bluef{Associativity.} $\forall \lambda, \psi \in \real, \vec{x \in \set{V}},$ $\lambda \cdot (\psi \cdot \vec{x}) = (\lambda \psi) \cdot \vec{x}$
\item \bluef{Neutral element.} $\forall \vec{x} \in \set{V},$ $1 \cdot \vec{x} = \vec{x}$
\ece

%  A vector space (over $\real$) consists of a set $\set{V}$ along with
% two operations `+' and `\( \cdot \)', if, 
% for all vectors $\vec{v},\vec{w},\vec{u}\in \set{V},$
% and all scalars $r,s\in \real$:

% %\vspace{-0.5cm}
% \bce
% \item $\set{V}$ is \redf{closed} under
%   vector addition:  $\vec{v}+\vec{w}\in \set{V}$
% \item vector addition is commutative:  $\vec{v}+\vec{w}=\vec{w}+\vec{v} $
% \item vector addition is associative: $(\vec{v}+\vec{w})+\vec{u}=\vec{v}+(\vec{w}+\vec{u}) $
%  \item there is a \redf{zero vector}
%       $\vec{0} \in \set{V}$  such that
%       $\vec{v}+\vec{0}=\vec{v},$ for all $\vec{v}\in \set{V}$ 
%  \item each $\vec{v}\in \set{V} $ has an
%      \redf{additive inverse}
%      $ \vec{w}\in \set{V} $ such that $ \vec{w}+\vec{v}=\vec{0} $
%  \item  $\set{V}$ is closed under
%      scalar multiplication, that is,
%   $ r\cdot\vec{v}\in \set{V} $
%  \item scalar multiplication distributes over addition of scalars
%   $(r+s)\cdot\vec{v}=r\cdot\vec{v}+s\cdot\vec{v} $
%  \item scalar multiplication distributes over vector addition
%   $ r\cdot(\vec{v}+\vec{w})=r\cdot\vec{v}+r\cdot\vec{w} $
%  \item ordinary multiplication of scalars associates with
%   scalar multiplication $ (rs)\cdot\vec{v} =r\cdot(s\cdot\vec{v}) $
%  \item multiplication by the scalar~$1$ is the
%   identity operation $ 1\cdot\vec{v}=\vec{v}$.
% \ece


\end{frame}

%%%%%%%%%%%%%%%%%%%%%%%%%%%%%%%%%%%%%%%%%%%%%%%%%%%%%%
\begin{frame}{Example} 

\plitemsep 0.1in

\bci
\item  $\set{V} = \real^n$ with 
\bci
\item Vector addition: $\vec{x} + \vec{y} = (x_1 + y_1, \ldots, x_n + y_n)$ 
\item Scalar multiplication: $\lambda \vec{x} = (\lambda x_1, \ldots, \lambda x_n)$
\eci

\bigskip
\item $\set{V} = \real^{m \times n}$ with 
\bci
\item Vector addition: $\mat{A} + \vec{B} 
= 
\begin{nmat}
a_{11} + b_{11} & \cdots & a_{1n} + b_{1n} \cr
\vdots & & \vdots \cr
a_{m1} + b_{m1} & \cdots & a_{mn} + b_{mn}
\end{nmat}
$ 
\item Scalar multiplication: $\lambda\mat{A}
= 
\begin{nmat}
\lambda a_{11}  & \cdots & \lambda a_{1n} \cr
\vdots & & \vdots \cr
\lambda a_{m1}  & \cdots & \lambda a_{mn} 
\end{nmat}
$ 
\eci
\eci
\end{frame}

%%%%%%%%%%%%%%%%%%%%%%%%%%%%%%%%%%%%%%%%%%%%%%%%%%%%%%
\begin{frame}{Vector Subspaces} 

\defi Consider a vector space $V=(\set{V},+,\cdot)$ and $\set{U} \subset \set{V}.$
Then, $U=(\set{U},+,\cdot)$ is called \redf{vector subspace} (simply linear subspace or subspace) of $V$ if $U$ is a vector space with two operations `$+$' and `$\cdot$' restricted to $\set{U} \times \set{U}$ and $\real \times \set{U}.$

\bigskip



\plitemsep 0.1in

Examples

\bci

\item For every vector space $V,$ $V$ and $\{\vec{0} \}$ are the trivial subspaces. 

\item The solution set of $\mA \vec{x} = \vec{0}$ is the subspace of $\real^n.$

\item The solution of $\mA \vec{x} = \vec{b}$ ($\vec{b} \neq \vec{0}$) is not a subspace of $\real^n.$

\item The intersection of arbitrarily many subspaces is a subspace itself. 
\eci
\end{frame}


%%%%%%%%%%%%%%%%%%%%%%%%%%%%%%%%%%%%%%%%%%%%%%%%%%%%%%
\section{L2(5)}
\begin{frame}{Roadmap}

\plitemsep 0.1in

\bce[(5)] 
\item \grayf{Systems of Linear Equations

\item Matrices

\item Solving Systems of Linear Equations

\item Vector Spaces}

\item \redf{Linear Independence}

\item \grayf{Basis and Rank

\item Linear Mappings

\item Affine Spaces
}
\ece
\end{frame}

%%%%%%%%%%%%%%%%%%%%%%%%%%%%%%%%%%%%%%%%%%%%%%%%%%%%%%
\begin{frame}{Linear Independence}

\plitemsep 0.1in

\bci 
\item \defi For a vector space $V$ and vectors $\vec{x}_1, \ldots, \vec{x}_n \in V,$ every $\vec{v} \in V$ of the form
$
\vec{v} = \lambda_1 \vec{x}_1 + \cdots + \lambda_k \vec{x}_k
$
with $\lambda_1, \ldots, \lambda_k \in \real$ is a \bluef{linear combination} of the vectors  $\vec{x}_1, \ldots, \vec{x}_n \in V.$
\item \defi If there is a non-trivial linear combination such that $\vec{0}=\sum_{i=1}^k \lambda_i \vec{x}_i$ with at least one $\lambda_i \neq 0,$ the vectors $\vec{x}_1, \ldots, \vec{x}_n$ are \bluef{linearly dependent}. If only the trivial solution exists, i.e., $\lambda_1 = \ldots = \lambda_k =0,$ $\vec{x}_1, \ldots, \vec{x}_n$ are \bluef{linearly independent.}

\item \redf{Meaning.} A set of linearly independent vectors consists of vectors that have no redundancy.  

\item \redf{Useful fact.} The vectors $\{\vec{x}_1, \ldots, \vec{x}_n \}$ are linearly dependent, iff
(at least) one of them is a linear combination of the others. 
\bci
\item $x-2y = 2$ and $2x-4y=4$ are linearly dependent. 
\eci

\eci


\end{frame}

%%%%%%%%%%%%%%%%%%%%%%%%%%%%%%%%%%%%%%%%%%%%%%%%%%%%%%
\begin{frame}{Checking Linear Independence}

\plitemsep 0.1in

\bci 
\item Gauss elimination to get the row echelon form

\item \bluef{All column vectors are linearly independent iff all columns are pivot columns (why?).}

\item \exam
{\small
\aleq{
&\vec{x}_1 = \colvec{1 \\ 2 \\ -3 \\ 4}, \quad \vec{x}_2 = \colvec{1 \\ 1 \\ 0 \\ 2}, \quad \vec{x}_3 = \colvec{-1 \\ -2 \\ 1 \\ 1} \cr
&
\begin{nmat}
1 & 1 & -1 \cr
2 & 1 & -2 \cr
-3& 0 & 1 \cr 
4 & 2 & 1 
\end{nmat}
\quad \leadsto \cdots \leadsto \quad
\begin{nmat}
1 & 1 & -1 \cr
0 & 1 & 0 \cr
0& 0 & 1 \cr 
0 & 0 & 0 
\end{nmat}
}}
\item Every column is a pivot column. Thus, $\vec{x}_1,$ $\vec{x}_2,$ $\vec{x}_3$ are linearly independent.  
\eci

\end{frame}

%%%%%%%%%%%%%%%%%%%%%%%%%%%%%%%%%%%%%%%%%%%%%%%%%%%%%%
\begin{frame}{Linear Combinations of Linearly Independent Vectors}

\plitemsep 0.1in

\bci 
\item Vector space $V$ with $k$ linearly independent vectors $\vec{b}_1, \vec{b}_2, \ldots, \vec{b}_k$

\item $m$ linear combinations $\vec{x}_1, \vec{x}_2, \ldots, \vec{x}_m.$ \redf{(Q)} Are they linearly independent?

\bigskip
\mysmalltwocols{0.25}
{
\small
\vspace{-0.3cm}
\aleq{
\vec{x}_1 &=\lambda_{11} \vec{b}_1 + \lambda_{21} \vec{b}_2+ \cdots + \lambda_{k1} \vec{b}_k \cr
\vdots & \cr
\vec{x}_m &=\lambda_{1m} \vec{b}_1 + \lambda_{2m} \vec{b}_2+ \cdots + \lambda_{km} \vec{b}_k 
}
}
{
\small
\vspace{-0.3cm}
\aleq{
\vec{x}_j &= \overbrace{\rowvec{\vec{b}_1, & \cdots, & \vec{b}_k}}^{\vec{B}} \overbrace{\colvec{\lambda_{1j} \\ \vdots \\ \lambda_{kj}}}^{\vec{\lambda}_j}, \quad 
\vec{x}_j = \vec{B} \vec{\lambda}_j
}
}
\bigskip

\item $\sum_{j=1}^m \psi_j \vec{x}_j = \sum_{j=1}^m \psi_j \vec{B} \vec{\lambda}_j = \vec{B} \sum_{j=1}^m \psi_j \vec{\lambda}_j$
\item $\{\vec{x}\}$ linearly independent $\Longleftrightarrow$ $\{\vec{\lambda}\}$ linearly independent
\eci

\end{frame}

%%%%%%%%%%%%%%%%%%%%%%%%%%%%%%%%%%%%%%%%%%%%%%%%%%%%%%
\begin{frame}{Example}

\plitemsep 0.1in

$$
\begin{linsys}{4}
\vec{x}_1 &=& \vec{b}_1 &-& 2\vec{b}_2 &+& \vec{b}_3 &-& \vec{b}_4 \cr
\vec{x}_2 &=& -4\vec{b}_1 &-& 2\vec{b}_2 &&  &+& 4\vec{b}_4 \cr
\vec{x}_3 &=& 2\vec{b}_1 &+& 3\vec{b}_2 &-& \vec{b}_3 &-& 3\vec{b}_4 \cr
\vec{x}_4 &=& 17\vec{b}_1 &-& 10\vec{b}_2 &+& 11\vec{b}_3 &+& \vec{b}_4 \cr
\end{linsys}
$$
\aleq{
\mat{A} = \rowvec{\vec{\lambda}_1 & \vec{\lambda}_2 & \vec{\lambda}_3 & \vec{\lambda}_4} =
\begin{nmat}
1 & -4 & 2 & 17 \cr
-2 & -2 & 3 & -10 \cr
1 & 0 & -1 & 11 \cr
-1 & -4 & -3 & 1 
\end{nmat}
\leadsto \cdots \leadsto
\begin{nmat}
1 & 0 & 0 & -7 \cr
0 & 1 & 0 & -15 \cr
0 & 0 & 1 & -18 \cr
0 & 0 & 0 & 0 
\end{nmat}
}

\bci 
\item The last column is not a pivot column. Thus, $\vec{x}_1, \vec{x}_2, \vec{x}_3, \vec{x}_3$ are linearly dependent. 
\eci

\end{frame}

%%%%%%%%%%%%%%%%%%%%%%%%%%%%%%%%%%%%%%%%%%%%%%%%%%%%%%
\section{L2(6)}
\begin{frame}{Roadmap}

\plitemsep 0.1in

\bce[(1)] 
\item \grayf{Systems of Linear Equations

\item Matrices

\item Solving Systems of Linear Equations

\item Vector Spaces

\item Linear Independence}

\item \redf{Basis and Rank}

\item \grayf{Linear Mappings

\item Affine Spaces
}
\ece
\end{frame}


%%%%%%%%%%%%%%%%%%%%%%%%%%%%%%%%%%%%%%%%%%%%%%%%%%%%%%
\begin{frame}{Generating Set and Basis}

\plitemsep 0.1in

\bci 
\item \defi A vector space $V=(\set{V}, +, \cdot)$ and a set of vectors $\set{A} = \{\vec{x}_1, \ldots, \vec{x}_k \} \subset \set{V}.$
\vspace{-0.3cm}
\bci
\item If every $v \in \set{V}$ can be expressed as a linear combination of $\vec{x}_1, \ldots, \vec{x}_k,$ $\set{A}$ is called a \bluef{generating set} of $V.$
\item The set of all linear combinations of $\set{A}$ is called the \bluef{span} of $\set{A}.$
\item If $\set{A}$ spans the vector space $V,$ we use $V = \spn{\set{A}}$ or $V = \spn{\vec{x}_1, \ldots, \vec{x}_k}$
\eci

\item \defi The minimal generating set $\set{B}$ of $V$ is called \bluef{basis} of $V.$ We call each element of $\set{B}$ \bluef{basis vector}. The number of basis vectors is called \bluef{dimension} of $V.$

\item Properties
\bci
\item $\set{B}$ is a maximally\footnote{Adding any other vector to this set will make it linearly dependent.} linearly independent set of vectors in $V.$
\item Every vector $x \in V$ is a linear combination of $\set{B}$, which is unique. 
\eci
\eci

\end{frame}


%%%%%%%%%%%%%%%%%%%%%%%%%%%%%%%%%%%%%%%%%%%%%%%%%%%%%%
\begin{frame}{Examples}

\plitemsep 0.1in

\bci 
\item Different bases $\real^3$
\aleq{
&\set{B}_1 = \{
\colvec{1 \\ 0 \\ 0}, \colvec{0 \\ 1 \\ 0}, \colvec{0 \\ 0 \\ 1} 
\}, 
\set{B}_2 = \{
\colvec{1 \\ 0 \\ 0}, \colvec{1 \\ 1 \\ 0}, \colvec{1 \\ 1 \\ 1} 
\}, \cr
&\set{B}_3 = \{
\colvec{0.5 \\ 0.8 \\ 0.4}, \colvec{1.8 \\ 0.3 \\ 0.3}, \colvec{-2.2 \\ -1.3 \\ 3.5} 
\} 
}
\item Linearly independent, but not maximal. Thus, not a basis. 
\aleq{
\set{A} = \{
\colvec{1 \\ 2 \\ 3 \\ 4},\colvec{2\\ -1 \\ 0 \\2},\colvec{1 \\ 1\\ 0 \\ -4} 
\}
}

\eci

\end{frame}


%%%%%%%%%%%%%%%%%%%%%%%%%%%%%%%%%%%%%%%%%%%%%%%%%%%%%%
\begin{frame}{Determining a Basis}

\plitemsep 0.1in

\bci 
\item Want to find a basis of a subspace $U = \spn{\vec{x}_1, \vec{x}_2, \ldots, \vec{x}_m}$

\bce
\item Construct a matrix $\mat{A} = \rowvec{\vec{x}_1 & \vec{x}_2 & \cdots & \vec{x}_m}$
\item Find the row-echelon form of $\mat{A}.$
\item Collect the pivot columns.
\ece

\item Logic: Collect $\vec{x}_i$ so that we have only trivial solution. Pivot columns tell us which set of vectors is linearly independent. 

\item See example 2.17 (pp. 35)
\eci

\end{frame}



%%%%%%%%%%%%%%%%%%%%%%%%%%%%%%%%%%%%%%%%%%%%%%%%%%%%%%
\begin{frame}{Rank (1)}

\plitemsep 0.15in

\bci 
\item \defi The \bluef{rank}  of $\mat{A} \in \real^{m \times n}$ denoted by $\rk{\mat{A}}$ is \# of linearly independent columns 
\bci
\item Same as the number of linearly independent rows
\eci

\item $\mat{A} = 
\begin{nmat}
1 & 2 & 1 \cr
-2 & -3 & 1 \cr
3 & 5 & 0 
\end{nmat}$
$\leadsto \cdots \leadsto$
$\begin{nmat}
1 & 2 & 1 \cr
0 & 1 & 3 \cr
0 & 0 & 0 
\end{nmat}$

\bigskip
Thus, $\rk{\mat{A}}=2.$

\item $\rk{\mat{A}} = \rk{\mat{\trans{\mat{A}}}}$
\eci

\end{frame}

%%%%%%%%%%%%%%%%%%%%%%%%%%%%%%%%%%%%%%%%%%%%%%%%%%%%%%
\begin{frame}{Rank (2)}

\plitemsep 0.2in

\bci 

\item The \bluef{columns} (resp. \redf{rows}) of $\mat{A}$ span a subspace \bluef{$U$} (resp. \redf{$W$}) with  \bluef{$\dimm{U} = \rk{\mat{A}}$} (resp. \redf{$\dimm{W} = \rk{\mat{A}}$}), and 
a basis of \bluef{$U$} (resp. \bluef{$W$}) can be found by Gauss elimination of \bluef{$\mat{A}$} (resp. \redf{$\mat{\trans{A}}$}).

\item For all $\mat{A} \in \real^{n \times n},$ $\rk{\mat{A}} =n,$ iff $\mat{A}$ is regular (invertible). 

\item The linear system $\mat{A} \vec{x} = \vec{b}$ is solvable, iff $\rk{\mat{A}} = \rk{\mat{A|b}}.$

\item For $\mat{A} \in \real^{m \times n},$ the subspace of solutions for $\mat{A} \vec{x} = \vec{0}$ possesses dimension $n - \rk{\mat{A}}.$

\item $\mat{A} \in \real^{m \times n}$ has \bluef{full rank} if its rank equals the largest possible rank for a matrix of the same dimensions. The rank of the full-rank matrix $\mat{A}$ is \bluef{$\min(\text{\# of cols, \# of rows}).$}

\eci

\end{frame}

%%%%%%%%%%%%%%%%%%%%%%%%%%%%%%%%%%%%%%%%%%%%%%%%%%%%%%
\section{L2(7)}
\begin{frame}{Roadmap}

\plitemsep 0.1in

\bce[(1)] 
\item \grayf{Systems of Linear Equations

\item Matrices

\item Solving Systems of Linear Equations

\item Vector Spaces

\item Linear Independence

\item Basis and Rank}

\item \redf{Linear Mappings}

\item \grayf{Affine Spaces}

\ece
\end{frame}


%%%%%%%%%%%%%%%%%%%%%%%%%%%%%%%%%%%%%%%%%%%%%%%%%%%%%%
\begin{frame}{Linear Mapping (1)}

\plitemsep 0.1in

\bci 
\item Interest: A mapping that preserves the structure of the vector space
\item \defi For vector spaces $V,W,$ a mapping $\Phi: V \mapsto W$ is called a \bluef{linear mapping} (or homomorphism/linear transformation), if, for all $\vec{x}, \vec{y} \in V$ and all $\lambda \in \real,$
\bci
\item $\Phi(\vec{x} + \vec{y}) = \Phi(\vec{x}) + \Phi(\vec{y})$
\item $\Phi(\lambda \vec{x}) = \lambda \Phi(\vec{x})$
\eci

\item \defi A mapping $\Phi: \set{V} \mapsto \set{W}$ is called
\bci
\item \bluef{Injective} (단사), if $\forall \vec{x},\vec{y} \in \set{V},$ $\Phi(\vec{x}) = \Phi(\vec{y}) \implies \vec{x} = \vec{y}$

\item \bluef{Surjective} (전사), if $\Phi(\set{V}) = \set{W}$

\item \bluef{Bijective} (전단사), if it is injenctive and surjective. 
\eci

\eci

\end{frame}

%%%%%%%%%%%%%%%%%%%%%%%%%%%%%%%%%%%%%%%%%%%%%%%%%%%%%%
\begin{frame}{Linear Mapping (2)}

\plitemsep 0.1in

\bci 

\item For bjective mapping, there exists an inverse mapping $\Phi^{-1}.$

\item \bluef{Isomorphism} if $\Psi$ is linear and bijective.

\item \redf{Theorem.} Vector spaces $V$ and $W$ are isomorphic, iff $\dimm{V} = \dimm{W}.$
\bci
\item Vector spaces of the same dimension are kind of the same thing. 
\eci

\item Other properties
\bci
\item For two linear mappings $\Phi$ and $\Psi,$ $\Phi \circ \Psi$ is also a linear mapping.
\item If $\Phi$ is an isomorphism, so is $\Phi^{-1}.$
\item For two linear mappings $\Phi$ and $\Psi,$ $\Phi + \Psi$ and $\lambda \Psi$ for $\lambda \in \real$ are linear. 
\eci
\eci

\end{frame}

%%%%%%%%%%%%%%%%%%%%%%%%%%%%%%%%%%%%%%%%%%%%%%%%%%%%%%
\begin{frame}{Coordinates}

\plitemsep 0.1in

\bci 

\mysmalltwocols{0.25}
{
\item A basis defines a coordinate system.
}
{
\vspace{-0.2cm}
\begin{center}
\mypic{0.8}{L2_coordinate.png}
\end{center}
}

\item Consider an ordered basis $B=(\vec{b}_1, \vec{b}_2, \ldots, \vec{b}_n)$ of vector space $V.$ Then, for any $\vec{x} \in V,$ there exists a unique linear combination
\aleq{
\vec{x} = \alpha_1 \vec{b}_1 + \ldots + \alpha_n \vec{b}_n.
}

\item We call $\vec{\alpha} = \colvec{\alpha_1 \\ \vdots \\ \alpha_n}$ the coordinate of $\vec{x}$ with respect to $B =(\vec{b}_1, \vec{b}_2, \ldots, \vec{b}_n) .$

\item Basis change $\implies$ Coordinate change
\eci

\end{frame}

%%%%%%%%%%%%%%%%%%%%%%%%%%%%%%%%%%%%%%%%%%%%%%%%%%%%%%
\begin{frame}{Basis Change}

\plitemsep 0.1in

\bci 

\item Consider a vector space $V$ and two coordinate systems defined by $B =(\vec{b}_1, \ldots, \vec{b}_n)$ and $B' =(\vec{b}'_1, \ldots, \vec{b}'_n).$

\item \question For $(x_1, \ldots, x_n)_B $ $\rightarrow$ $(y_1, \ldots, y_n)_{B'},$ what is $(y_1, \ldots, y_n)_{B'}$?

\item \thm 
$
\colvec{y_1 \\ \vdots \\ y_n} = \inv{\rowvec{\vec{b}'_1 & \ldots &\vec{b}'_n}} \rowvec{\vec{b}_1 & \ldots &\vec{b}_n} \colvec{x_1 \\ \vdots \\ x_n}
$

\item Regard $\mat{A}_{\Phi} = \inv{\rowvec{\vec{b}'_1 & \ldots &\vec{b}'_n}} \rowvec{\vec{b}_1 & \ldots &\vec{b}_n}$ as a linear map
\eci

\end{frame}

%%%%%%%%%%%%%%%%%%%%%%%%%%%%%%%%%%%%%%%%%%%%%%%%%%%%%%
\begin{frame}{Example}

\plitemsep 0.1in

\bci 

\item $B=((1,0), (0,1)$ and $B'=((2,1), (1,2))$

\item $(4,2)_B$ $\rightarrow$ $(x,y)_{B'}$?
\item Using
$
\colvec{y_1 \\ \vdots \\ y_n} = \inv{\rowvec{\vec{b}'_1 & \ldots &\vec{b}'_n}} \rowvec{\vec{b}_1 & \ldots &\vec{b}_n} \colvec{x_1 \\ \vdots \\ x_n},
$
\aleq{
\colvec{x \\ y} = 
\inv{\begin{nmat}
2 & 1 \cr
1 & 2
\end{nmat}}
\begin{nmat}
1 & 0 \cr
0 & 1
\end{nmat}
\colvec{4 \\ 2}
 = 
\begin{nmat}
\frac{2}{3} & -\frac{1}{3} \cr
-\frac{1}{3} & \frac{2}{3}
\end{nmat}
\colvec{4 \\ 2} = \colvec{2 \\ 0}
}
\eci

\end{frame}

%%%%%%%%%%%%%%%%%%%%%%%%%%%%%%%%%%%%%%%%%%%%%%%%%%%%%%
\begin{frame}{Transformation Matrix}

\plitemsep 0.05in

\bci 
\item Two vector spaces 
\bci
\item $V$ with basis $B=(\vec{b}_1, \ldots, \vec{b}_n)$ and $W$ with basis $C=(\vec{c}_1, \ldots, \vec{c}_m)$
\eci

\item What is the coordinate in $C$-system for each basis $\vec{b}_j$? For $j=1, \ldots, n,$
{\small
\aleq{
\vec{b}_j = \alpha_{1j} \vec{c}_1 + \cdots + \alpha_{mj}\vec{c}_m &\Longleftrightarrow 
\vec{b}_j = \rowvec{\vec{c}_1 & \cdots & \vec{c}_m} \colvec{\alpha_{1j} \\ \vdots \\ \alpha_{mj}}\cr
& \implies \rowvec{\vec{b}_1 & \cdots & \vec{b}_n} = \rowvec{\vec{c}_1 & \cdots & \vec{c}_m}  
\overbrace{\begin{nmat}
\alpha_{11} & \cdots & \alpha_{1n} \cr
\vdots & &\vdots \cr
\alpha_{m1} & \cdots & \alpha_{mn}
\end{nmat}}^{\mat{A}_{\Phi}}
}}

\item $\hat{x} = \mat{A}_{\Phi} \hat{y}$, where $\hat{x}$ is the vector w.r.t $B$ and $\hat{y}$ is the vector w.r.t. $C$
% \item There exists a linear mapping $\Phi: V \mapsto W$ which can be represented by a matrix, called 
% \bluef{transformation matrix} $\mat{A}_{\Phi}$.

\eci

\end{frame}

%%%%%%%%%%%%%%%%%%%%%%%%%%%%%%%%%%%%%%%%%%%%%%%%%%%%%%
\begin{frame}{Basis Change: General Case}

\plitemsep 0.1in

\bci 
\item For linear mapping $\Phi: V \mapsto W,$ consider bases $B,B'$ of $V$ and $C,C'$ of $W$ 
\aleq{
B = \rowvec{\vec{b}_1 & \cdots & \vec{b}_n}, \ B' = \rowvec{\vec{b}'_1 & \cdots & \vec{b}'_n}\quad 
C = \rowvec{\vec{c}_1 & \cdots & \vec{c}_m}, \ C' = \rowvec{\vec{c}'_1 & \cdots & \vec{c}'_m}.
}
\item (inter) transformation matrices $\mat{A}_{\Phi}$ from $B$ to $C$ and $\mat{A}'_{\Phi}$ from $B'$ to $C'$
\item (intra) transformation matrices $S$ from $B'$ to $B$ and $T$ from $C'$ to $C$
\item \thm $\mat{A}'_{\Phi} = \inv{T} \mat{A}_{\Phi} S$

\mypic{0.6}{L2_basischange.png}
\eci

\end{frame}

%%%%%%%%%%%%%%%%%%%%%%%%%%%%%%%%%%%%%%%%%%%%%%%%%%%%%%
\begin{frame}{Image and Kernel}

\plitemsep 0.1in

\bci 


\item Consider a linear mapping $\Phi: V \mapsto W.$ The \bluef{kernel} (or \bluef{null space}) is the set of vectors in $V$ that maps to $\vec{0} \in W$ (i.e., neutral element).

\medskip
\defi  $\ker(\Phi) \eqdef \Phi^{-1}(\vec{0}_W) = \{\vec{v} \in V : \Phi(\vec{v}) = \vec{0}_W \}$

\item \bluef{Image/range}: set of vectors $w \in W$ that can be reached by $\Phi$ from any vector in $V$

\item $V$: \bluef{domain}, $W$: \bluef{codomain}

\centering
\mypic{0.45}{L2_image_kernel.png}

\eci

\end{frame}

%%%%%%%%%%%%%%%%%%%%%%%%%%%%%%%%%%%%%%%%%%%%%%%%%%%%%%
\begin{frame}{Image and Kernel: Properties}

\plitemsep 0.2in

\bci 

\item $\vec{0}_V \in \ker(\Phi)$ (because $\Phi(\vec{0}_V) = \vec{0}_W$)
\item Both $\img{\Phi}$ and $\ker(\Phi)$ are subspaces of $W$ and $V,$ respectively. 
\item $\Phi$ is one-to-one (injective) $\Longleftrightarrow$ $\ker(\Phi) = \{0\}$ (i.e., only $\vec{0}$ is mapped to $\vec{0}$)

\item Since $\Phi$ is a linear mapping, there exists $\mat{A} \in \real^{m \times n}$ such that $\Phi: \vec{x} \mapsto \mat{A}\vec{x}.$ Then, $\img{\Phi} = \text{column space of } \mat{A}$ which is the span of column vectors of $\mat{A}.$

\item $\rk{\mat{A}} = \dimm{\img{\Phi}}$

\item $\ker(\Phi)$ is the solution set of the homogeneous system of linear equations $\mat{A}\vec{x} = \vec{0}$
\eci

\end{frame}

%%%%%%%%%%%%%%%%%%%%%%%%%%%%%%%%%%%%%%%%%%%%%%%%%%%%%%
\begin{frame}{Rank-Nullity Theorem}

\plitemsep 0.07in
\myvartwocols{0.7}{0.7}{0.29}
{
\plitemsep 0.07in
\thm 
$$\dimm{\ker(\Phi)} + \dimm{\img{\Phi}} = \dimm{V}$$
\vspace{-0.5cm}
\small
\bci 
\item If $\dimm{\img{\Phi}} < \dimm{V}$, the kernel contains more than just $\vec{0.}$
\item If $\dimm{\img{\Phi}} < \dimm{V}$, $\mat{A}_{\Phi} \vec{x} = \vec{0}$ has infinitely many solutions.

\item If $\dimm{V} = \dimm{W}$ (e.g., $V=W=\real^n$), the followings are equivalent: $\Phi$ is
\plitemsep 0.01in
\bci
\item (1) injective, (2) surjective, (3) bijective, 

\item In this case, $\Phi$ defines $\vec{y} = \mat{A}\vec{x},$ where $\mat{A}$ is regular. 
\eci

\item \redf{Simplified version.} For $\mat{A} \in \real^{m \times n},$
$$
\rk{\mat{A}} + \text{nullity}(\mat{A}) = n
$$
\eci
}
{
\centering
\mypic{0.95}{L2_rank_nullity.png}
}
\footnotetext{Nullity: the dimension of null space (kernel)}
\end{frame}


%%%%%%%%%%%%%%%%%%%%%%%%%%%%%%%%%%%%%%%%%%%%%%%%%%%%%%
\section{L2(8)}
\begin{frame}{Roadmap}

\plitemsep 0.1in

\bce[(1)] 
\item \grayf{Systems of Linear Equations

\item Matrices

\item Solving Systems of Linear Equations

\item Vector Spaces

\item Linear Independence

\item Basis and Rank

\item Linear Mappings}

\item \redf{Affine Spaces}

\ece
\end{frame}



%%%%%%%%%%%%%%%%%%%%%%%%%%%%%%%%%%%%%%%%%%%%%%%%%%%%%%
\begin{frame}{Linear vs. Affine Function}

\plitemsep 0.1in
\mytwocols{0.5}
{
\bci 
\item \redf{linear function}: $f(x) = ax$ 
\item \redf{affine function}: $f(x) = ax + b$ 
\item sometimes (ignorant) people refer to affine functions as linear
\eci
}
{
\centering
\mypic{0.95}{L2_affine_linear.png}
}

\end{frame}

%%%%%%%%%%%%%%%%%%%%%%%%%%%%%%%%%%%%%%%%%%%%%%%%%%%%%%
\begin{frame}{Affine Subspace}

\plitemsep 0.1in
\bci 
\item Spaces that are offset from the origin. Not a vector space.

\item \defi Consider a vector space $V,$ $\vec{x}_0 \in V$ and a subspace $U \subset V.$ Then, the subset $L  = \vec{x_0} + U \eqdef \{\vec{x_0} + \vec{u} : \vec{u} \in U \}$ is called \bluef{affine subspace} or \bluef{linear manifold} of $V.$

\item $U$ is called \bluef{direction} or \bluef{direction space}, and $\vec{x}_0$ is \bluef{support} point.

\item An affine subspace is not a vector subspace of $V$ for $\vec{x}_0 \notin U.$

\item {\bf \em Parametric equation.} A $k$-dimensional affine space $L = \vec{x_0} + U.$ If $(\vec{b}_1, \ldots, \vec{b}_k)$ is an ordered basis of $U,$ any element $\vec{x} \in L$ can be uniquely described as
\aleq{
\vec{x} = \vec{x}_0 + \lambda_1 \vec{b}_1 + \cdots + \lambda_k \vec{b}_k, \quad \lambda_1, \ldots, \lambda_k \in \real
}

\eci

\end{frame}

%%%%%%%%%%%%%%%%%%%%%%%%%%%%%%%%%%%%%%%%%%%%%%%%%%%%%%
\begin{frame}{Example}

\small
\plitemsep 0.05in
\bci 
\item In $\real^2,$ one-dimensional affine subspace: \bluef{line}. $\vec{y} = \vec{x}_0 + \lambda \vec{b}_1.$ $U = \spn{\vec{b}_1}$

\item In $\real^3,$ two-dimensional affine subspace: \bluef{plane}. $\vec{y} = \vec{x}_0 + \lambda_1 \vec{b}_1 + \lambda_2 \vec{b}_2.$ $U = \spn{\vec{b}_1, \vec{b}_2}$

\item In $\real^n,$ $(n-1)$-dimensional affine subspace: \bluef{hyperplane}. $\vec{y} = \vec{x}_0 + \sum_{k=1}^{n-1} \lambda_i \vec{b}_i.$ $U = \spn{\vec{b}_1, \ldots, \vec{b}_n}$

\medskip
\begin{center}
\mypic{0.4}{L2_affine.png}
\end{center}

\item For a linear mapping $\Phi: V \mapsto W$ and a vector $\vec{a} \in W,$ the mapping $\phi: V \mapsto W$ with $\phi(\vec{x}) = \vec{a} + \Phi(\vec{x})$ is an \bluef{affine mapping} from $V$ to $W.$ The vector $\vec{a}$ is called the \bluef{translation vector.}
\eci


\end{frame}


%%%%%%%%%%%%%%%%%%%%%%%%%%%%%%%%%%%%%%%%%%%%%%%%%%%%%%
\begin{frame}{}
\vspace{2cm}
\LARGE Questions?


\end{frame}

\begin{frame}{Review Questions}
% \tableofcontents
%\plitemsep 0.1in
\bce[1)]
\item 

\ece
\end{frame}


\end{document}
