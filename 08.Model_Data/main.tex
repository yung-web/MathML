%\pdfminorversion=4
\documentclass[handout,fleqn,aspectratio=169]{beamer}

% when making printed slides
\usepackage{pgfpages}
\pgfpagesuselayout{resize to}[a4paper,landscape,border shrink=5mm]

\usepackage[english]{babel}
\usepackage{tikz}
\usepackage{courier}
\usepackage{array}
\usepackage{bold-extra}
%\usepackage{minted}
\usepackage[thicklines]{cancel}
\usepackage{fancyvrb}
\usepackage{kotex}
\usepackage{paralist}
\usepackage{collectbox}
\usepackage{bm}

\usepackage{mathrsfs}
\usepackage[reqno,disallowspaces]{mathtools}  % imports amsmath
\usepackage{amsfonts} %for Y&Y BSR AMS fonts
\usepackage{amssymb}
\usepackage{amscd}
%\usepackage{tikz,lipsum,lmodern}
\usepackage[most]{tcolorbox}
\usepackage{verbatim}
\mode<presentation>
{
  \usetheme{default}
  \usecolortheme{default}
  \usefonttheme{default}
  \setbeamertemplate{navigation symbols}{}
  \setbeamertemplate{caption}[numbered]
  \setbeamertemplate{footline}[frame number]  % or "page number"
  \setbeamercolor{frametitle}{fg=yellow}
  \setbeamercolor{footline}{fg=black}
} 

\setbeamercolor{block body alerted}{bg=alerted text.fg!10}
\setbeamercolor{block title alerted}{bg=alerted text.fg!20}
\setbeamercolor{block body}{bg=structure!10}
\setbeamercolor{block title}{bg=structure!20}
\setbeamercolor{block body example}{bg=green!10}
\setbeamercolor{block title example}{bg=green!20}
\setbeamertemplate{blocks}[rounded][shadow]

\xdefinecolor{dianablue}{rgb}{0.18,0.24,0.31}
\xdefinecolor{darkblue}{rgb}{0.1,0.1,0.7}
\xdefinecolor{darkgreen}{rgb}{0,0.5,0}
\xdefinecolor{darkgrey}{rgb}{0.35,0.35,0.35}
\xdefinecolor{darkorange}{rgb}{0.8,0.5,0}
\xdefinecolor{darkred}{rgb}{0.7,0,0}
\definecolor{darkgreen}{rgb}{0,0.6,0}
\definecolor{mauve}{rgb}{0.58,0,0.82}

\usetikzlibrary{shapes.callouts}

\makeatletter
\setbeamertemplate{footline}
{
  \leavevmode%
  \hbox{%
  \begin{beamercolorbox}[wd=.333333\paperwidth,ht=2.25ex,dp=1ex,center]{author in head/foot}%
    \usebeamerfont{author in head/foot}\insertsection
  \end{beamercolorbox}%
  \begin{beamercolorbox}[wd=.333333\paperwidth,ht=2.25ex,dp=1ex,center]{title in head/foot}%
    \usebeamerfont{title in head/foot}\insertsubsection
  \end{beamercolorbox}%
  \begin{beamercolorbox}[wd=.333333\paperwidth,ht=2.25ex,dp=1ex,right]{date in head/foot}%
    \usebeamerfont{date in head/foot}
    \insertshortdate{}\hspace*{2em}
    \insertframenumber{} / \inserttotalframenumber\hspace*{2ex} 
  \end{beamercolorbox}}%

  \vskip0pt%
}
\makeatother

\title[]{Lecture 8: When Models Meet Data}
\author{Yi, Yung (이융)}
\institute{Mathematics for Machine Learning\\ \url{https://yung-web.github.io/home/courses/mathml.html}
\\KAIST EE}
\date{\today}


%%%%%%%%%%%% real, integer notation
\newcommand{\real}{{\mathbb R}}
\newcommand{\realn}{{\mathbb R}^{n}}
\newcommand{\realm}{{\mathbb R}^{m}}
\newcommand{\realD}{{\mathbb R}^{D}}
\newcommand{\realM}{{\mathbb R}^{M}}
\newcommand{\realN}{{\mathbb R}^{N}}
\newcommand{\realnn}{{\mathbb R}^{n \times n}}
\newcommand{\realmm}{{\mathbb R}^{m \times m}}
\newcommand{\realmn}{{\mathbb R}^{m \times n}}
\newcommand{\realnm}{{\mathbb R}^{n \times m}}
\newcommand{\realDM}{{\mathbb R}^{D \times M}}
\newcommand{\realMD}{{\mathbb R}^{M \times D}}
\newcommand{\complex}{{\mathbb C}}
\newcommand{\integer}{{\mathbb Z}}
\newcommand{\natu}{{\mathbb N}}


%%% set, vector, matrix
\newcommand{\set}[1]{\ensuremath{\mathcal #1}}
\newcommand{\sets}[1]{\ensuremath{\{#1 \}}}
\renewcommand{\vec}[1]{\bm{#1}}
\newcommand{\mat}[1]{\bm{#1}}

%%%% vector
\def\vx{\vec{x}}
\def\vy{\vec{y}}
\def\vz{\vec{z}}
\def\vf{\vec{f}}
\def\ve{\vec{e}}
\def\vr{\vec{r}}
\def\vb{\vec{b}}
\def\vc{\vec{c}}
\def\vd{\vec{d}}
\def\vm{\vec{m}}
\def\vu{\vec{u}}
\def\vv{\vec{v}}
\def\vw{\vec{w}}
\def\vX{\vec{X}}
\def\vY{\vec{Y}}
\def\vZ{\vec{Z}}
\def\vth{\vec{\theta}}
\def\vmu{\vec{\mu}}
\def\vnu{\vec{\nu}}
\def\vlam{\vec{\lambda}}
\def\vep{\vec{\epsilon}}
\def\vpi{\vec{\pi}}
\def\vphi{\vec{\phi}}
\def\vxi{\vec{\xi}}
\def\valpha{\vec{\alpha}}
\def\vgamma{\vec{\gamma}}

%%%% Well-used matrices
\def\mA{\mat{A}}
\def\mB{\mat{B}}
\def\mC{\mat{C}}
\def\mD{\mat{D}}
\def\mI{\mat{I}}
\def\mJ{\mat{J}}
\def\mK{\mat{K}}
\def\mE{\mat{E}}
\def\mP{\mat{P}}
\def\mQ{\mat{Q}}
\def\mU{\mat{U}}
\def\mV{\mat{V}}
\def\mR{\mat{R}}
\def\mS{\mat{S}}
\def\mX{\mat{X}}
\def\msig{\mat{\Sigma}}
\def\mPhi{\mat{\Phi}}


\usepackage{amsmath}
%%%%% vector caculus useful macro
% ...\d, which typesets a derivative. ex: \d{y}{x}, instead of \frac{dx}{dy}.
\renewcommand{\d}[2]{\frac{\text{d} #1}{\text{d} #2}}


% ...similar for double-derivatives. ex: \dd{y}{x}.
\newcommand{\dd}[2]{\frac{\text{d}^2 #1}{\text{d} #2^2}}

% ...similar for partial derivatives. ex: \pd{y}{x}.
\newcommand{\pd}[2]{\frac{\partial #1}{\partial #2}}


% ...similar for partial double derivatives. ex: \pdd{y}{x}.
\newcommand{\pdd}[2]{\frac{\partial^2 #1}{\partial #2^2}}
% pdd with argument
\newcommand{\pdda}[3]{\frac{\partial^2 #1}{\partial #2 \partial #3}}

\usepackage{xparse}

%%%% caligraphic fonts
\def\cL{\ensuremath{{\cal L}}}
\def\cN{\ensuremath{{\cal N}}}
\def\cD{\ensuremath{{\cal D}}}
\def\cC{\ensuremath{{\cal C}}}
\def\cX{\ensuremath{{\cal X}}}
\def\cY{\ensuremath{{\cal Y}}}

%%% big parenthesis
\def\Bl{\Bigl}
\def\Br{\Bigr}
\def\lf{\left}
\def\ri{\right}


%%% floor notations
\newcommand{\lfl}{{\lfloor}}
\newcommand{\rfl}{{\rfloor}}
\newcommand{\floor}[1]{{\lfloor #1 \rfloor}}

%%% gradient
\newcommand{\grad}[1]{\nabla #1}
\newcommand{\hess}[1]{\text{H} #1}

%%% definition
%\newcommand{\eqdef}{\ensuremath{\triangleq}}
\newcommand{\eqdef}{\ensuremath{:=}}
%%% imply
\newcommand{\imp}{\Longrightarrow}



\newcommand{\separator}{
%  \begin{center}
    \par\noindent\rule{\columnwidth}{0.3mm}
%  \end{center}
}

\newcommand{\mynote}[1]{{\it \color{red} [#1]}}







%%% equation alignment
\newcommand{\aleq}[1]{\begin{align*}#1\end{align*}}

%%%%%%%%%%%%%%%% colored emphasized font, blanked words

\newcommand{\empr}[1]{{\color{red}\emph{#1}}}
\newcommand{\empb}[1]{{\color{blue}\emph{#1}}}
\newcommand{\redf}[1]{{\color{red} #1}}
\newcommand{\bluef}[1]{{\color{blue} #1}}
\newcommand{\grayf}[1]{{\color{gray} #1}}
\newcommand{\magenf}[1]{{\color{magenta} #1}}
\newcommand{\greenf}[1]{{\color{green} #1}}
\newcommand{\cyanf}[1]{{\color{cyan} #1}}
\newcommand{\orangef}[1]{{\color{orange} #1}}

\newcommand{\blk}[1]{\underline{\mbox{\hspace{#1}}}}


\newcommand{\redblk}[1]{\framebox{\color{red} #1}}
\newcommand{\redblank}[2]{\framebox{\onslide<#1->{\color{red} #2}}}
\newcommand{\blueblk}[1]{\framebox{\color{blue} #1}}
\newcommand{\blueblank}[2]{\framebox{\onslide<#1->{\color{blue} #2}}}



\makeatletter
\newcommand{\mybox}{%
    \collectbox{%
        \setlength{\fboxsep}{1pt}%
        \fbox{\BOXCONTENT}%
    }%
}
\makeatother

\makeatletter
\newcommand{\lecturemark}{%
    \collectbox{%
        \setlength{\fboxsep}{1pt}%
        \fcolorbox{red}{yellow}{\BOXCONTENT}%
    }%
}
\makeatother

\newcommand{\mycolorbox}[1]{
\begin{tcolorbox}[colback=red!5!white,colframe=red!75!black]
#1
\end{tcolorbox}
}
%%%% figure inclusion
\newcommand{\mypic}[2]{
\begin{center}
\includegraphics[width=#1\textwidth]{#2}
\end{center}
}

\newcommand{\myinlinepic}[2]{
\makebox[0cm][r]{\raisebox{-4ex}{\includegraphics[height=#1]{#2}}}
}




%%%% itemized and enumerated list
\newcommand{\bci}{\begin{compactitem}}
\newcommand{\eci}{\end{compactitem}}
\newcommand{\bce}{\begin{compactenum}}
\newcommand{\ece}{\end{compactenum}}


%%%% making 0.5/0.5 two columns
%%%% how to use: first number: length of separation bar
% \mytwocols{0.6}
% {
% contents in the left column
% }
% {
% contents in the right column
% }
%%%%

\newcommand{\mytwocols}[3]{
\begin{columns}[T] \column{.499\textwidth} #2 \column{.001\textwidth} \rule{.3mm}{{#1}\textheight} \column{.499\textwidth} #3 \end{columns}}

\newcommand{\mythreecols}[4]{
\begin{columns}[T] \column{.31\textwidth} #2 \column{.001\textwidth} \rule{.3mm}{{#1}\textheight} \column{.31\textwidth} #3 \column{.001\textwidth} \rule{.3mm}{{#1}\textheight} \column{.31\textwidth} #4  \end{columns}}

\newcommand{\mysmalltwocols}[3]{
\begin{columns}[T] \column{.4\textwidth} #2 \column{.001\textwidth} \rule{.3mm}{{#1}\textheight} \column{.4\textwidth} #3 \end{columns}}

%%%% making two columns with customized ratios
%%%% how to use: 
%first parameter: length of separation bar
%second parameter: ratio of left column
%third parameter: ratio of right column
% \mytwocols{0.6}{0.7}{0.29}
% {
% contents in the left column
% }
% {
% contents in the right column
% }
%%%%
\newcommand{\myvartwocols}[5]{
\begin{columns}[T] \column{#2\textwidth} {#4} \column{.01\textwidth} \rule{.3mm}{{#1}\textheight} \column{#3\textwidth} {#5} \end{columns}}

%%% making my block in beamer
%%% first parameter: title of block
%%% second parameter: contents of block
\newcommand{\myblock}[2]{
\begin{block}{#1} {#2}  \end{block}}

%%% independence notation
\newcommand{\indep}{\perp \!\!\! \perp}

%%%% probability with different shapes (parenthesis or bracket) and different sizes
%%% `i' enables us to insert the subscript to the probability 
\newcommand{\bprob}[1]{\mathbb{P}\Bl[ #1 \Br]}
\newcommand{\prob}[1]{\mathbb{P}[ #1 ]}
\newcommand{\cbprob}[1]{\mathbb{P}\Bl( #1 \Br)}
\newcommand{\cprob}[1]{\mathbb{P}( #1 )}
\newcommand{\probi}[2]{\mathbb{P}_{#1}[ #2 ]}
\newcommand{\bprobi}[2]{\mathbb{P}_{#1}\Bl[ #2 \Br]}
\newcommand{\cprobi}[2]{\mathbb{P}_{#1}( #2 )}
\newcommand{\cbprobi}[2]{\mathbb{P}_{#1}\Bl( #2 \Br)}

%%%% expectation with different shapes (parenthesis or bracket) and different sizes
%%% `i' enables us to insert the subscript to the expectation
\newcommand{\expect}[1]{\mathbb{E}[ #1 ]}
\newcommand{\cexpect}[1]{\mathbb{E}( #1 )}
\newcommand{\bexpect}[1]{\mathbb{E}\Bl[ #1 \Br]}
\newcommand{\cbexpect}[1]{\mathbb{E}\Bl( #1 \Br)}
\newcommand{\bbexpect}[1]{\mathbb{E}\lf[ #1 \ri]}
\newcommand{\expecti}[2]{\mathbb{E}_{#1}[ #2 ]}
\newcommand{\bexpecti}[2]{\mathbb{E}_{#1}\Bl[ #2 \Br]}
\newcommand{\bbexpecti}[2]{\mathbb{E}_{#1}\lf[ #2 \ri]}

%%%% variance
\newcommand{\var}[1]{\text{var}[ #1 ]}
\newcommand{\bvar}[1]{\text{var}\Bl[ #1 \Br]}
\newcommand{\cvar}[1]{\text{var}( #1 )}
\newcommand{\cbvar}[1]{\text{var}\Bl( #1 \Br)}

%%%% covariance
\newcommand{\cov}[1]{\text{cov}( #1 )}
\newcommand{\bcov}[1]{\text{cov}\Bl( #1 \Br)}

%%% Popular pmf, pdf notation to avoid long typing
\newcommand{\px}{\ensuremath{p_X(x)}}
\newcommand{\py}{\ensuremath{p_Y(y)}}
\newcommand{\pz}{\ensuremath{p_Z(z)}}
\newcommand{\pxA}{\ensuremath{p_{X|A}(x)}}
\newcommand{\pyA}{\ensuremath{p_{Y|A}(y)}}
\newcommand{\pzA}{\ensuremath{p_{Z|A}(z)}}
\newcommand{\pxy}{\ensuremath{p_{X,Y}(x,y)}}
\newcommand{\pxcy}{\ensuremath{p_{X|Y}(x|y)}}
\newcommand{\pycx}{\ensuremath{p_{Y|X}(y|x)}}

\newcommand{\fx}{\ensuremath{f_X(x)}}
\newcommand{\Fx}{\ensuremath{F_X(x)}}
\newcommand{\fy}{\ensuremath{f_Y(y)}}
\newcommand{\Fy}{\ensuremath{F_Y(y)}}
\newcommand{\fz}{\ensuremath{f_Z(z)}}
\newcommand{\Fz}{\ensuremath{F_Z(z)}}
\newcommand{\fxA}{\ensuremath{f_{X|A}(x)}}
\newcommand{\fyA}{\ensuremath{f_{Y|A}(y)}}
\newcommand{\fzA}{\ensuremath{f_{Z|A}(z)}}
\newcommand{\fxy}{\ensuremath{f_{X,Y}(x,y)}}
\newcommand{\Fxy}{\ensuremath{F_{X,Y}(x,y)}}
\newcommand{\fxcy}{\ensuremath{f_{X|Y}(x|y)}}
\newcommand{\fycx}{\ensuremath{f_{Y|X}(y|x)}}

\newcommand{\fth}{\ensuremath{f_\Theta(\theta)}}
\newcommand{\fxcth}{\ensuremath{f_{X|\Theta}(x|\theta)}}
\newcommand{\fthcx}{\ensuremath{f_{\Theta|X}(\theta|x)}}

\newcommand{\pkcth}{\ensuremath{p_{X|\Theta}(k|\theta)}}
\newcommand{\fthck}{\ensuremath{f_{\Theta|X}(\theta|k)}}


%%%% indicator
\newcommand{\indi}[1]{\mathbf{1}_{ #1 }}

%%%% exponential rv.
\newcommand{\elambdax}{\ensuremath{e^{-\lambda x}}}

%%%% normal  rv.
\newcommand{\stdnormal}{\ensuremath{\frac{1}{\sqrt{2\pi}} e^{-x^2/2}}}
\newcommand{\gennormal}{\ensuremath{\frac{1}{\sigma\sqrt{2\pi}} e^{-(x-\mu)^2/2}}}

%%%%%% estimator, estimate
\newcommand{\hth}{\ensuremath{\hat{\theta}}}
\newcommand{\hTH}{\ensuremath{\hat{\Theta}}}
\newcommand{\MAP}{\ensuremath{\text{MAP}}}
\newcommand{\LMS}{\ensuremath{\text{LMS}}}
\newcommand{\LLMS}{\ensuremath{\text{L}}}
\newcommand{\ML}{\ensuremath{\text{ML}}}

%%%% colored text
\newcommand{\red}[1]{\color{red}#1} 
\newcommand{\cyan}[1]{\color{cyan}#1} 
\newcommand{\magenta}[1]{\color{magenta}#1} 
\newcommand{\blue}[1]{\color{blue}#1} 
\newcommand{\green}[1]{\color{green}#1} 
\newcommand{\white}[1]{\color{white}#1} 
\newcommand{\gray}[1]{\color{gray}#1} 

%%% definition
\newcommand{\defi}{{\color{red} Definition.} } 
\newcommand{\exam}{{\color{red} Example.} } 
\newcommand{\question}{{\color{red} Question.} } 
\newcommand{\thm}{{\color{red} Theorem.} } 
\newcommand{\background}{{\color{red} Background.} } 
\newcommand{\msg}{{\color{red} Message.} } 


\def\ml{\text{ML}}
\def\map{\text{MAP}}

%%%%%%%%%%%%%%%%%%%%%%% old macros that you can ignore %%%%%%%%%%%%%%%%%%%%%%%%

% \def\un{\underline}
% \def\ov{\overline}


% \newcommand{\beq}{\begin{eqnarray*}}
% \newcommand{\eeq}{\end{eqnarray*}}
% \newcommand{\beqn}{\begin{eqnarray}}
% \newcommand{\eeqn}{\end{eqnarray}}
% \newcommand{\bemn}{\begin{multiline}}
% \newcommand{\eemn}{\end{multiline}}
% \newcommand{\beal}{\begin{align}}
% \newcommand{\eeal}{\end{align}}
% \newcommand{\beas}{\begin{align*}}
% \newcommand{\eeas}{\end{align*}}



% \newcommand{\bd}{\begin{displaymath}}
% \newcommand{\ed}{\end{displaymath}}
% \newcommand{\bee}{\begin{equation}}
% \newcommand{\eee}{\end{equation}}


% \newcommand{\vs}{\vspace{0.2in}}
% \newcommand{\hs}{\hspace{0.5in}}
% \newcommand{\el}{\end{flushleft}}
% \newcommand{\bl}{\begin{flushleft}}
% \newcommand{\bc}{\begin{center}}
% \newcommand{\ec}{\end{center}}
% \newcommand{\remove}[1]{}

% \newtheorem{theorem}{Theorem}
% \newtheorem{corollary}{Corollary}
% \newtheorem{prop}{Proposition}
% \newtheorem{lemma}{Lemma}
% \newtheorem{defi}{Definition}
% \newtheorem{assum}{Assumption}
% \newtheorem{example}{Example}
% \newtheorem{property}{Property}
% \newtheorem{remark}{Remark}

% \newcommand{\separator}{
%   \begin{center}
%     \rule{\columnwidth}{0.3mm}
%   \end{center}
% }

% \newenvironment{separation}
% { \vspace{-0.3cm}
%   \separator
%   \vspace{-0.25cm}
% }
% {
%   \vspace{-0.5cm}
%   \separator
%   \vspace{-0.15cm}
% }

% \def\A{\mathcal A}
% \def\oA{\overline{\mathcal A}}
% \def\S{\mathcal S}
% \def\D{\mathcal D}
% \def\eff{{\rm Eff}}
% \def\bD{\bm{D}}
% \def\cU{{\cal U}}
% \def\bbs{{\mathbb{s}}}
% \def\bbS{{\mathbb{S} }}
% \def\cM{{\cal M}}
% \def\bV{{\bm{V}}}
% \def\cH{{\cal H}}
% \def\ch{{\cal h}}
% \def\cR{{\cal R}}
% \def\cV{{\cal V}}
% \def\cA{{\cal A}}
% \def\cX{{\cal X}}
% \def\cN{{\cal N}}
% \def\cJ{{\cal J}}
% \def\cK{{\cal K}}
% \def\cL{{\cal L}}
% \def\cI{{\cal I}}
% \def\cY{{\cal Y}}
% \def\cZ{{\cal Z}}
% \def\cC{{\cal C}}
% \def\cR{{\cal R}}
% \def\id{{\rm Id}}
% \def\st{{\rm st}}
% \def\cF{{\cal F}}
% \def\bz{{\bm z}}
% \def\cG{{\cal G}}
% \def\N{\mathbb{N}}
% \def\bbh{\mathbb{h}}
% \def\bbH{\mathbb{H}}
% \def\bbi{\mathbb{i}}
% \def\bbI{\mathbb{I}}
% \def\R{\mathbb{R}}
% \def\bbR{\mathbb{R}}
% \def\bbr{\mathbb{r}}
% \def\cB{{\cal B}}
% \def\cP{{\cal P}}
% \def\cS{{\cal S}}
% \def\bW{{\bm W}}
% \def\bc{{\bm c}}

% %\def\and{\quad\mbox{and}\quad}
% \def\ind{{\bf 1}}


% \def\bmg{{\bm{\gamma}}}
% \def\bmr{{\bm{\rho}}}
% \def\bmq{{\bm{q}}}
% \def\bmt{{\bm{\tau}}}
% \def\bmn{{\bm{n}}}
% \def\bmcapn{{\bm{N}}}
% \def\bmrho{{\bm{\rho}}}

% \def\igam{\underline{\gamma}(\lambda)}
% \def\sgam{\overline{\gamma}(\lambda)}
% \def\ovt{\overline{\theta}}
% \def\ovT{\overline{\Theta}}
% \def\PP{{\mathrm P}}
% \def\EE{{\mathrm E}}
% \def\iskip{{\vskip -0.4cm}}
% \def\siskip{{\vskip -0.2cm}}

% \def\bp{\noindent{\it Proof.}\ }
% \def\ep{\hfill $\Box$}



%%%%%%%%%% linear algebra macros %%%%%%%%%%%%%%%%%%%%%%%

%--------linsys
%  Use as \begin{linsys}{3}
%           x &+ &3y &+ &a &= &7 \\
%           x &- &3y &+ &a &= &7
%         \end{linsys}
% Remark: TeXbook pp. 167-170 says to put a medmuskip around a +; and that's
% 4/18-ths of an em.  Why does 2/18-ths of an em work?  I don't know, but
% comparing to a regular displayed equation suggests it is right.
% (darseneau says LaTeX puts in half an \arraycolsep.)
\newenvironment{linsys}[2][m]{%
\setlength{\arraycolsep}{.1111em} % p. 170 TeXbook; a medmuskip
\begin{array}[#1]{@{}*{#2}{rc}r@{}}
}{%
\end{array}}

\newsavebox\boxofmathplus
\sbox{\boxofmathplus}{$+$}
\newcommand{\spaceforemptycolumn}{\makebox[\wd\boxofmathplus]{\ }}

%--------grstep
% For denoting a Gauss' reduction step.
% Use as: \grstep{\rho_1+\rho_3} or \grstep[2\rho_5 \\ 3\rho_6]{\rho_1+\rho_3}
% \newcommand{\grstep}[2][\relax]{%
%    \ensuremath{\mathrel{
%        \mathop{\longrightarrow}\limits^{#2\mathstrut}_{
%                                    \begin{subarray}{l} #1 \end{subarray}}}}}

% Advantage of length formulation is that between adjacent
% \grstep's you can add \hspace{-\grsteplength} to make it look not too wide
\newlength{\grsteplength}
\setlength{\grsteplength}{1.5ex plus .1ex minus .1ex}

\newcommand{\grstep}[2][\relax]{%
   \ensuremath{\mathrel{
       \hspace{\grsteplength}\mathop{\longrightarrow}\limits^{#2\mathstrut}_{
                                     \begin{subarray}{l} #1 \end{subarray}}\hspace{\grsteplength}}}}
% If two or more \grsteps are in a row then they need to be tightened
\newcommand{\repeatedgrstep}[2][\relax]{\hspace{-\grsteplength}\grstep[#1]{#2}}

% row swap operation: \rho_1\swap\rho_2
\newcommand{\swap}{\leftrightarrow}

%-------------amatrix
% Augmented matrix.  Usage (note the argument does not count the aug col):
% \begin{amatrix}{2}
%   1  2  3 \\  4  5  6
% \end{amatrix}
\newenvironment{amatrix}[1]{%
  \left(\begin{array}{@{}*{#1}{c}|c@{}}
}{%
  \end{array}\right)
}



%-------------pmat
% For matrices with arguments.
% Usage: \begin{pmat}{c|c|c} 1 &2 &3 \end{pmat}
\newenvironment{pmat}[1]{
  \left(\begin{array}{@{}#1@{}}
}{\end{array}\right)
}



%-------------misc matrices
% \newenvironment{mat}{\left(\begin{array}}{\end{array}\right)}
\newenvironment{detmat}{\left|\begin{array}}{\end{array}\right|}
\newcommand{\deter}[1]{ \mathchoice{\left|#1\right|}{|#1|}{|#1|}{|#1|} }
\newcommand{\generalmatrix}[3]{ %arg1: low-case letter, arg2: rows, arg3: cols 
               \left(
                  \begin{array}{cccc}
                    #1_{1,1}  &#1_{1,2}  &\ldots  &#1_{1,#2}  \\
                    #1_{2,1}  &#1_{2,2}  &\ldots  &#1_{2,#2}  \\
                              &\vdots                         \\
                    #1_{#3,1} &#1_{#3,2} &\ldots  &#1_{#3,#2}
                  \end{array}
               \right)  }

\newcommand{\generaldet}[3]{ %arg1: low-case letter, arg2: rows, arg3: cols 
               \left|
                  \begin{array}{cccc}
                    #1_{11}  &#1_{12}  &\ldots  &#1_{1 #2}  \\
                    #1_{21}  &#1_{22}  &\ldots  &#1_{2 #2}  \\
                              &\vdots                         \\
                    #1_{#3 1} &#1_{#3 2} &\ldots  &#1_{#3 #2}
                  \end{array}
               \right|  }

% With mathtools we can have column entries right flushed
% There is an optional argument \begin{mat}[r]{3} .. \end{mat} for
% right-flushed columns.  Perhaps the rule is that numbers are better 
% right-flushed but if there are any letters it is better centered?
\newenvironment{nmat}[1][c]{\begin{pmatrix*} % disable optional arg [#1] 
      }{\end{pmatrix*}}
% If mat starts with &\vdots get an error; why?  No apparent macro fix, according to texexchange
\newenvironment{vmat}[1][c]{\begin{vmatrix*} % disable optional arg [#1] 
      }{\end{vmatrix*}}
\newenvironment{amat}[2][c]{%
  % disable optional arg \left(\begin{array}{@{}*{#2}{#1}|#1@{}}
  \left(\begin{array}{@{}*{#2}{c}|#1@{}}
}{%
  \end{array}\right)
}
% \newcommand\vdotswithin[1]{% Taken from mathtools.dtx because my TL is not 2011
%   {\mathmakebox[\widthof{\ensuremath{{}#1{}}}][c]{{\vdots}}}}


%------------colvec and rowvec
% Column vector and row vector.  Usage:
%  \colvec{1  \\ 2 \\ 3 \\ 4} and \rowvec{1  &2  &3}
% Colvec takes an optional argument \colvec[r]{x_1 \\ 0}.  Perhaps 
% digits look better right aligned, but if there are any letters it
% needs to be centered?
\newcommand{\colvec}[2][c]{\begin{nmat}[#1] #2 \end{nmat}}
\newcommand{\smallcolvec}[1]{\left(\begin{smallmatrix} #1 \end{smallmatrix}\right)}
% For row vectors, cannot do \newcommand{\rowvec}[1]{\begin{mat} #1 \end{mat}}
% since the delimiters come out too large.
\newcommand{\rowvec}[1]{\setlength{\arraycolsep}{3pt}\left(\begin{matrix} #1 \end{matrix}\right)}



%-------------making aligned columns
% Usage: \begin{aligncolondecimal}{2} 1.2 \\ .33 \end{aligncolondecimal}
% (negative argument centers decimal pt in column).  Also Usage:
% \begin{aligncolondecimal}[0em]{2} 1.2 \\ .33 \end{aligncolondecimal}
% to make the left and right LaTeX-array padding disappear.
\RequirePackage{array}\RequirePackage{dcolumn}
\newenvironment{aligncolondecimal}[2][.1111em]{%
\setlength{\arraycolsep}{#1}
\newcolumntype{.}{D{.}{.}{#2}}\begin{array}{.}}{%
\end{array}}

% Matrix and vector, with numbers centered on decimal point
% Usage: \begin{dmat}{D{.}{.}{1}D{.}{.}{3}}  0  &.123 \\ .2 &.456 \end{dmat}
%  (in the D{.}{.}{number} that is the number of decimal places)
\newlength{\dmatcolsep}\setlength{\dmatcolsep}{5pt}
\newenvironment{dmat}[2][\dmatcolsep]{%
  \setlength{\arraycolsep}{#1}
  \left(\begin{array}{@{}#2@{}}
}{%
  \end{array}\right)}
% Usage: \dcolvec[2]{1.23 \\ 4.56} where the optional argument is the number
% of decimal places.
\newcommand{\dcolvec}[2][-1]{\left(\begin{array}{@{}D{.}{.}{#1}@{}} #2 \end{array}\right)}

%\newcommand{\trans}[1]{ {{#1}^{\mathsf{T}}} } 
\newcommand{\trans}[1]{ {#1}^{\mathsf{T}} } 
\newcommand{\inv}[1]{ {#1}^{-1} } 
\newcommand{\spn}[1]{\ensuremath{\text{span}[#1]} } 
\newcommand{\rk}[1]{\ensuremath{\text{rk}(#1)} } 
\newcommand{\dimm}[1]{\ensuremath{\text{dim}(#1)} } 
\newcommand{\img}[1]{\ensuremath{\text{Im}(#1)} } 
%\newcommand{\norm}[1]{\ensuremath{\left || #1 \right ||} } 
\newcommand{\norm}[1]{\ensuremath{\left \lVert #1 \right \rVert} } 
% orthogonal complement
\newcommand{\ocomp}[1]{\ensuremath{#1^{\bot}} } 
\newcommand{\inner}[2]{\ensuremath{\left\langle #1, #2 \right\rangle} } 
\DeclareMathOperator{\tr}{tr}


% \NewDocumentCommand{\grad}{e{_^}}{%
%   \mathop{}\!% \mathop for good spacing before \nabla
%   \nabla
%   \IfValueT{#1}{_{\!#1}}% tuck in the subscript
%   \IfValueT{#2}{^{#2}}% possible superscript
% }
% \begin{equation*}
%          \begin{nmat}[r]
% 1 &2 &13 \\
%           4  &5  &6
%          \end{nmat}
%       \end{equation*}

% \begin{equation*}
%          \begin{amat}{2}
%           1  &2  &3  \\
%           4  &5  &6
%          \end{amat}
%       \end{equation*}
       
%       \begin{equation*}
%          \begin{pmat}{c|c|c}
% 1 &2 &3 \\
%           4  &5  &6
%          \end{pmat}
%       \end{equation*}

% \begin{equation*}
%          \begin{vmat}
% a &c \\
%           b  &d
%          \end{vmat}
%          =ad-bc
% \end{equation*}

%  \begin{equation*}
%   \vec{v}=\colvec{-1  \\ -0.5  \\ 0}
% \end{equation*}

%  \begin{equation*}
%   \vec{v}=\rowvec{-1  & -0.5  & 0}
% \end{equation*}




%\addtobeamertemplate{footline}{\rule{0.94\paperwidth}{1pt}}{}

\begin{document}

%itemshape
\setbeamertemplate{itemize item}{\scriptsize\raise1.25pt\hbox{\donotcoloroutermaths$\bullet$}}
\setbeamertemplate{itemize subitem}{\tiny\raise1.5pt\hbox{\donotcoloroutermaths$\circ$}}
\setbeamertemplate{itemize subsubitem}{\tiny\raise1.5pt\hbox{\donotcoloroutermaths$\blacktriangleright$}}
%default value for spacing
\plitemsep 0.1in
\pltopsep 0.03in
\setlength{\parskip}{0.15in}
%\setlength{\parindent}{-0.5in}
\setlength{\abovedisplayskip}{0.07in}
\setlength{\belowdisplayskip}{0.07in}
\setlength{\mathindent}{0cm}
\setbeamertemplate{frametitle continuation}{[\insertcontinuationcount]}

\setlength{\leftmargini}{0.5cm}
\setlength{\leftmarginii}{0.5cm}

\setlength{\fboxrule}{0.05pt}
\setlength{\fboxsep}{5pt}


%%%%%%% This should be placed at the end of this file
\logo{\pgfputat{\pgfxy(0.11, 7.4)}{\pgfbox[right,base]{\tikz{\filldraw[fill=dianablue, draw=none] (0 cm, 0 cm) rectangle (50 cm, 1 cm);}\mbox{\hspace{-8 cm}\includegraphics[height=0.7 cm]{../kaist_ee.png}
}}}}

\begin{frame}
  \titlepage
\end{frame}

\logo{\pgfputat{\pgfxy(0.11, 7.4)}{\pgfbox[right,base]{\tikz{\filldraw[fill=dianablue, draw=none] (0 cm, 0 cm) rectangle (50 cm, 1 cm);}\mbox{\hspace{-8 cm}\includegraphics[height=0.7 cm]{../kaist_ee.png}
}}}}

% rule color - gray
\makeatletter
\let\old@rule\@rule
\def\@rule[#1]#2#3{\textcolor{gray}{\old@rule[#1]{#2}{#3}}}
\makeatother






% START START START START START START START START START START START START START
%%%%%%%%%%%%%%%%%%%%%%%%%%%%%%%%%%%%%%%%%%%%%%%%%%%%%%
\begin{frame}{Roadmap}

\plitemsep 0.1in

\bce[(1)]

\item Data, Models, and Learning 
\item Models as Functions: Empirical Risk Minimization 
\item Models as Probabilistic Models: Parameter Estimation (ML and MAP)
\item Probabilistic Modeling and Inference 
\item Directed Graphical Models 
\item Model Selection

\ece
\end{frame}

%%%%%%%%%%%%%%%%%%%%%%%%%%%%%%%%%%%%%%%%%%%%%%%%%%%%%%
\section{L8(1)}
\begin{frame}{Roadmap}

\plitemsep 0.1in

\bce[(1)]

\item \redf{Data, Models, and Learning}
\item \grayf{Models as Functions: Empirical Risk Minimization 
\item Models as Probabilistic Models: Parameter Estimation (ML and MAP)
\item Probabilistic Modeling and Inference 
\item Directed Graphical Models 
\item Model Selection
}

\ece
\end{frame}

%%%%%%%%%%%%%%%%%%%%%%%%%%%%%%%%%%%%%%%%%%%%%%%%%%%%%%
\begin{frame}{Data, Models, and Learning}

\plitemsep 0.15in

\bci

\item Three major components of a machine learning system

\bce
\item Data:  $\{(\vx_1, y_1), \ldots, (\vx_n,y_n), \ldots, (\vx_N,y_N) \}$
\item Models: deterministic functions or probabilistic models
\item Learning: Training, and prediction/inference
\ece
\item Good machine learning models: Perform well for unseen (untrained) data

\item Machine learning algorithm: training and prediction
\eci
\end{frame}

%%%%%%%%%%%%%%%%%%%%%%%%%%%%%%%%%%%%%%%%%%%%%%%%%%%%%%
\begin{frame}{Data as Vectors}

\plitemsep 0.1in

\bci 

\item Tabular format or not, numerical or not, good feature extraction etc. 

\item Assume that data is given as $D$-dimensional vector $\vx_n$ of real numbers, each called \bluef{features}, \bluef{attributes}, or \bluef{covariates}.

\item Dataset: consisting of data points or examples $\{ \vx_1,$ $\vx_2,$ \ldots, $\vx_N \}$
\item In supervised learning, $\{(\vx_1, y_1), \ldots, (\vx_n,y_n), \ldots, (\vx_N,y_N) \},$ where $y_n$ is the \bluef{label} (or target, response variable, or annotation). 
\item Better representation of data as vectors
\bci
\item finding lower-dimensional approximations of the original feature vector (e.g., PCA via SVD or EVD)
\item using nonlinear higher-dimensional combinations of the original feature vector (e.g., feature map and kernel)
\eci
\eci
\end{frame}

%%%%%%%%%%%%%%%%%%%%%%%%%%%%%%%%%%%%%%%%%%%%%%%%%%%%%%
\begin{frame}{Models: Functions vs. Probabilistic Models}

\myvartwocols{0.7}{0.65}{0.31}
{
\plitemsep 0.07in

\bci 

\item Now, the business of constructing a predictor

\item Models as \bluef{functions}
\bci
\item $f: \realD \mapsto \real.$ 
\item \exam $f(\vx) = \trans{\vth}\vx + \theta_0,$ Unknown parameter: $\vth,\theta_0$

\eci

\item Models as \bluef{probabilistic models}

\bci
\item model our uncertainty due to the \bluef{observation process} and our uncertainty in the \bluef{parameters of our model}

\item predictors should be able to express some sort of uncertainty via probabilistic models
\item Parameters: parameters of a chosen probabilistic model (e.g., mean and variance of Gaussian)
\eci

\eci

}
{
\vspace{-0.3cm}
\mypic{0.99}{L8_model_function.png}
\mypic{0.99}{L8_model_pmodel.png}

}

\end{frame}

%%%%%%%%%%%%%%%%%%%%%%%%%%%%%%%%%%%%%%%%%%%%%%%%%%%%%%
\begin{frame}{Learning Algorithms}

\plitemsep 0.07in

\bci 

\item[] Three algorithmic phases

\item[(1)] Prediction or inference: via function or probabilitic models

\item[(2)] Training or parameters estimation 

\bci
\item fixed parameter assumption (non-probabilistic) or Bayesisan approach (probabilistic)
\item non-probabilistic: e.g., empirical risk minimization
\item probabilistic: e.g., ML (Maximum Likelihood), MAP (Maximum A Posteriori)
\item cross-validation: simulation of performing for unseen data
\item regularization/prior: balancing models between training and unseen data

\eci
\item[(3)] Hyperparameter tuning or model selection
\eci
\end{frame}

%%%%%%%%%%%%%%%%%%%%%%%%%%%%%%%%%%%%%%%%%%%%%%%%%%%%%%
\section{L8(2)}
\begin{frame}{Roadmap}

\plitemsep 0.1in

\bce[(1)] 

\item \grayf{Data, Models, and Learning}
\item \redf{Models as Functions: Empirical Risk Minimization} 
\item \grayf{Models as Probabilistic Models: Parameter Estimation (ML and MAP)
\item Probabilistic Modeling and Inference 
\item Directed Graphical Models 
\item Model Selection
}

\ece
\end{frame}


%%%%%%%%%%%%%%%%%%%%%%%%%%%%%%%%%%%%%%%%%%%%%%%%%%%%%%
\begin{frame}{Empirical Risk Minimization}

\plitemsep 0.07in

\bci 

\item Predictor as a function

\item Given $\{(\vx_1, y_1), \ldots, (\vx_n,y_n), \ldots, (\vx_N,y_N) \},$ estimate a predictor $f(\cdot, \vth): \realD \mapsto \real$

\item Find a good parameter $\vth^*,$ such that $f(\vx_n,\vth^*) = \hat{y}_n \approx y_n,$ for all $n=1,\ldots, N$


\bigskip
\item \exam Affine function: By adding the unit feature $x^{(0)}=1$ and $\theta_0$, i.e., $\vx_n = \trans{[1, x_n^{(1)}, \ldots, x_n^{(D)}]},$ $\vth = \trans{[\theta_0, \theta_1, \ldots, \theta_D]}$
\aleq
{
f(\vx_n,\vth) = \trans{\vth} \vx_n = \theta_0 + \sum_{d=1}^D \theta_d x_n^{(d)}
}

\item \exam Neural network: Complex non-linear function
\eci
\end{frame}

%%%%%%%%%%%%%%%%%%%%%%%%%%%%%%%%%%%%%%%%%%%%%%%%%%%%%%
\begin{frame}{Loss Function}

\plitemsep 0.07in

\bci 

\item Training set: $\{(\vx_1, y_1), \ldots, (\vx_n,y_n), \ldots, (\vx_N,y_N) \},$ an example matrix\footnote{In other chapters, we often use $D \times N$ example matrix by defining it as $\mat{X} \eqdef [\vx_1, \ldots, \vx_N].$ \lecturemark{L10(4)}} 
$\mat{X} \eqdef \trans{[\vx_1, \ldots, \vx_N]} \in \real^{N \times D},$ a label vector $\vy \eqdef 
\trans{[y_1, \ldots, y_N]},$

\item Average loss, empirical risk
$$
\bm{R}_{\text{emp}}(f,\mat{X},\vy) = \frac{1}{N} \sum_{n=1}^N \ell(y_n,\hat{y}_n)
$$

%\bigskip
\item Goal: Minimizing empirical risk

\item \exam The squared loss function $\ell(y_n,\hat{y}_n) = (y_n - \hat{y}_n)^2$ leads to:
$$
\min_{\vth \in \realD} \frac{1}{N} \norm{\vy - \mat{X}\vth}^2
$$

\item \question Ultimgate goal: Minimizing expected risk (for unseen data) $\bm{R}_{\text{true}} = \expecti{\vx,y}{\ell(y,f(\vx))}$? 

\eci
\end{frame}

%%%%%%%%%%%%%%%%%%%%%%%%%%%%%%%%%%%%%%%%%%%%%%%%%%%%%%
\begin{frame}{Overfitting and Regularization}

\plitemsep 0.07in

\bci 

\item The predictor fits too closely to the training data and does not generalize well to new data

\item Need to somehow bias the search for the minimizer of empirical risk by introducing a \bluef{penalty term}

\item \bluef{Regularization}: compromise between accurate solution of empirical risk minimization and the size or complexity of the solution.

\item \exam Regularized Least Squares
$$
\min_{\vth \in \realD} \frac{1}{N} \norm{\vy - \mat{X}\vth}^2 + \lambda \norm{\vth}^2
$$
\bci
\item $\norm{\vth}^2$: regularizer, $\lambda$: regularization parameter
\eci

\eci
\end{frame}

%%%%%%%%%%%%%%%%%%%%%%%%%%%%%%%%%%%%%%%%%%%%%%%%%%%%%%
\begin{frame}{Cross-Validation for Generalization Performance}

\bigskip


\mypic{0.8}{L8_cross_validation.png}

\end{frame}


%%%%%%%%%%%%%%%%%%%%%%%%%%%%%%%%%%%%%%%%%%%%%%%%%%%%%%
\section{L8(3)}
\begin{frame}{Roadmap}

\plitemsep 0.1in

\bce[(1)] 

\item \grayf{Data, Models, and Learning}
\item \grayf{Models as Functions: Empirical Risk Minimization} 
\item \redf{Models as Probabilistic Models: Parameter Estimation (ML and MAP)}
\item \grayf{Probabilistic Modeling and Inference 
\item Directed Graphical Models 
\item Model Selection
}

\ece
\end{frame}


%%%%%%%%%%%%%%%%%%%%%%%%%%%%%%%%%%%%%%%%%%%%%%%%%%%%%%
\begin{frame}{MLE (Maximum Likelihood Estimation): Concept}

\plitemsep 0.07in

\bci 

\item Idea: define a function of the parameters called \bluef{likelihood function}.

\item Negative log-likelihood for data $\vx$ and a family of probability densities $\cprob{\vx \mid \vth}$ parameterized by $\vth$:
$$
\cL_{\vx}(\vth) = \cL(\vth) \eqdef - \log \cprob{\vx \mid \vth}
$$
\bci
\item $\cL(\vth)$: how likely a particular setting of $\vth$ is for the observations $\vx$.
\eci

\bigskip
\item \redf{MLE}: Find $\vth$ such that $\cL(\vth)$ is \bluef{minimized} (i.e., likelihood is \bluef{maximized})
\eci
\end{frame}

%%%%%%%%%%%%%%%%%%%%%%%%%%%%%%%%%%%%%%%%%%%%%%%%%%%%%%
\begin{frame}{MLE: Supervised Learning}

\plitemsep 0.05in

\bci 

\item The set of iid examples $(\vx_1, y_1), \ldots, (\vx_N,y_N)$

\item $\set{X} = \{\vx_1, \ldots, \vx_N \}$ and $\set{Y} = \{y_1, \ldots, y_N \}$

\item Negative log-likelihood
$$
\cL(\vth) = - \log \cprob{\set{Y} \mid \set{X}, \vth} = \sum_{n=1}^N \log \cprob{y_n \mid \vx_n, \vth}
$$

\item \exam Assume independent Gaussian noise $\set{N}(0,\sigma^2)$ and linear model $y_n = \trans{\vx}_n \vth$ for prediction. Then, $Y_n| (\vx_n,\vth) \sim \set{N}(\trans{\vx}_n\vth, \sigma^2).$
{\small
\aleq{
\cL(\vth) &= - \sum_{n=1}^N \log \frac{1}{\sqrt{2\pi\sigma^2}} \exp\left(-\frac{(y_n-\trans{\vx}_n\vth )^2}{2\sigma^2} \right)= \frac{1}{2\sigma^2} \sum_{n=1}^N (y_n-\trans{\vx}_n\vth )^2 - \sum_{n=1}^N \log \frac{1}{\sqrt{2\pi\sigma^2}}
}}

\eci
\end{frame}

%%%%%%%%%%%%%%%%%%%%%%%%%%%%%%%%%%%%%%%%%%%%%%%%%%%%%%
\begin{frame}{MAP (Maximum A Posteriori)}

\plitemsep 0.1in

\bci 

\item What if we have some \bluef{prior knowledge} about $\vth$? Then, how should we change our knowledge about $\vth$ after observing data $\vx$?

\item Compute a posteriori distribution (using Bayes' Theorem) and find $\vth$ that maximizes the distribution:
$$
\max_{\vth} \cprob{\vth \mid \vx} = \max_{\vth} \frac{\cprob{\vx \mid \vth}\cprob{\vth}}{\cprob{\vx}} 
\Longleftrightarrow \min_{\vth}\Big (  -\log \cprob{\vth \mid \vx} \Big )
$$
\bci
\item In finding the optimal $\vth,$ $\cprob{\vx}$ can be ignored
\eci
\item ML and MAP: Bridging the non-probabilistic and probabilistic worlds as it explicitly acknowledges the need for a prior distribution, yet producing a \bluef{point estimate} (one single parameter return). 

%\item We later see the full parameter distributions
\eci
\end{frame}


%%%%%%%%%%%%%%%%%%%%%%%%%%%%%%%%%%%%%%%%%%%%%%%%%%%%%%
\begin{frame}{Model Fitting}

\plitemsep 0.1in

\bci 

\item Model class $M_{\vth}$ vs. Right model $M^*$
\mypic{0.3}{L8_model_class.png}

\item Overfitting vs. Underfitting vs. Good fitting
\mypic{0.7}{L8_fittings.png}
\eci
\end{frame}

%%%%%%%%%%%%%%%%%%%%%%%%%%%%%%%%%%%%%%%%%%%%%%%%%%%%%%
\section{L8(4)}
\begin{frame}{Roadmap}

\plitemsep 0.1in

\bce[(1)] 

\item \grayf{Data, Models, and Learning}
\item \grayf{Models as Functions: Empirical Risk Minimization} 
\item \grayf{Models as Probabilistic Models: Parameter Estimation (ML and MAP)}
\item \redf{Probabilistic Modeling and Inference} 
\item \grayf{Directed Graphical Models 
\item Model Selection
}

\ece
\end{frame}

%%%%%%%%%%%%%%%%%%%%%%%%%%%%%%%%%%%%%%%%%%%%%%%%%%%%%%
\begin{frame}{Modeling Generative Process and Probabilistic Models}

\plitemsep 0.1in

\bci 

\item Many machine learning tasks: prediction of future events and decision making

\item Often build (probabilistic) models that describe the \bluef{generative process} that generates the observed data

\item In probabilistic modeling, the joint distribution $\cprob{\vx,\vth}$ of the observed variables 
$\vx$ and the hidden parameters $\vth$ encapsulate the key information
\bci
\item Given: \orangef{prior} $\cprob{\vth}$ and \orangef{likelihood} $\cprob{\vx | \vth}$
\item \greenf{Joint dist.} from prior and likelihood: $\cprob{\vx,\vth} = \cprob{\vx | \vth} \cprob{\vth}$
\item We get: \redf{marginal likelihood} $\cprob{\vx} = \int \cprob{\vx,\vth} \text{d}\vth$ and \redf{posterior} $\cprob{\vth|\vx} = \frac{\cprob{\vx,\vth}}{\cprob{\vx}}$
\eci

% \item Essentially, if we know the \bluef{joint distribution}, we know all about its probabilistic model
\eci
\end{frame}

%%%%%%%%%%%%%%%%%%%%%%%%%%%%%%%%%%%%%%%%%%%%%%%%%%%%%%
\begin{frame}{Fully Bayesian vs. ML/MAP}
Given the data set $\set{X},$ we want to predict $A,$ i.e., \redblk{$\cprob{A \mid \set{X}}$}
%\vspace{-1.2cm}
\plitemsep 0.05in
\bci 
\item \redf{ML}: Easy (high), Exact (low)
$$
\cprob{A \mid \set{X}} \approx \cprob{A \mid \vth}, \quad \vth = \arg\max \cprob{\set{X} \mid \vth}
$$

\item \redf{MAP}: Easy (mid), Exact (mid)
$$
\cprob{A \mid \set{X}} \approx \cprob{A \mid \vth}, \quad \vth = \arg\max \cprob{\vth \mid \set{X}}
$$

\item \redf{Fully Bayesian}: Easy (low), Exact (high)

\medskip
- predictive inference, use of posterior predictive distribution, bayesian prediction

- remove dependence on the model parameters $\vth$

$$
\cprob{A \mid \set{X}} = \int \cprob{A \mid \vth} \cprob{\vth \mid \set{X}}\text{d}\vth
$$

- Only possible by getting the full posterior distribution $\cprob{\vth \mid \set{X}}$
\eci
\end{frame}


%%%%%%%%%%%%%%%%%%%%%%%%%%%%%%%%%%%%%%%%%%%%%%%%%%%%%%
\begin{frame}{(Fully) Bayesian Inference: Hardness}

\plitemsep 0.2in

\bci 

% \item \bluef{Sinle} Earlier, two ways of estimating the parameter $\vth$: ML and MAP. Essentially, it is solving an optimization problem to get a single best value $\vth^*.$ $\implies$ Prediction through $\cprob{\vx \mid \vth^*}.$

% \item Rather than just a likelihood, having the \bluef{full posterior distribution} can be useful. 

\item For a data set $\set{X},$ a parameter prior $\cprob{\vth},$ and a likelihood function, the posterior is:
$$
\cprob{\vth \mid \set{X}} = \frac{\cprob{\set{X} \mid \vth} \cprob{\vth}}{\cprob{\set{X}}}, \quad 
\cprob{\set{X}} = \int \cprob{\set{X} \mid \vth} \cprob{\vth} \; \text{d}\vth
$$

%\item \question \bluef{Examples of prediction using the posterior distribution?}

\item Implementation hardness
\bci
\item Bayesian inference requires to solve integration, which is often challenging. In particular, a conjugate prior is not chosen, the integration is not analytically tractable.

\item Approximation techniques: MCMC (Markov Chain Monte Carlo), Laplace approximation, variational inference, expectation propagation
\eci
\eci
\end{frame}


%%%%%%%%%%%%%%%%%%%%%%%%%%%%%%%%%%%%%%%%%%%%%%%%%%%%%%
\begin{frame}{Latent-Variable Models (1)}

\plitemsep 0.1in

\bci 

\item Including latent variables in the model $\rightarrow$ contributing to the interpretability of the model

\item General discussions here would be applied the following examples later
\bci
\item PCA for dimensionality reduction \hfill \lecturemark{L10(7)}
\item Gaussian mixture models for density estimation \hfill \lecturemark{L11(3)}
\eci


\item In latent-variable models (LVMs)\footnote{In our note, we express the dependence on the model parameters $\vth$ using subscript notations, e.g., $\cprobi{\vth}{\vx | \vz}$ rather than $\cprob{\vx| \vz, \vth}$ to highlight the role of $\vz.$ }, 
\bci
\item Given: \orangef{prior} $\cprob{\vz}$ and \orangef{likelihood} $\cprobi{\vth}{\vx | \vz}$
\item \greenf{Joint dist.} from prior and likelihood: $\cprobi{\vth}{\vx,\vz} = \cprobi{\vth}{\vx | \vz} \cprob{\vz}$
\item Our interest: \redf{marginal likelihood} $\cprobi{\vth}{\vx}$ 
and \redf{posterior} $\cprobi{\vth}{\vz|\vx}$
\eci


% \item Offers data generation process through parameters: $\cprob{\vx \mid \vth, \vz}$, $\cprob{\vz}$

% \item Marginalization over the latent variables, which allows parameter estimation by ML and MAP (using the prior $\cprob{\vth}$)
% $$
% \cprob{\vx \mid \vth} = \int \cprob{\vx \mid \vth, \vz} \cprob{\vz}\; \text{d}\vz
% $$

\eci
\end{frame}

%%%%%%%%%%%%%%%%%%%%%%%%%%%%%%%%%%%%%%%%%%%%%%%%%%%%%%
\begin{frame}{LVM (2)}

\plitemsep 0.1in

\bci 

\item Assuming we know $\vth$, to generate a data sample from the model (i) sample $\vz$ from $\cprob{\vz}$ and (ii) sample $\vx$ from $\cprobi{\vth}{\vx|\vz}$ 

\item \redf{Inference.} computing the \bluef{posterior distribution} $\cprobi{\vth}{\vz | \vx}$:
$$
\cprobi{\vth}{\vz | \vx} = \frac{\cprobi{\vth}{\vx,\vz}}{\cprobi{\vth}{\vx}} = 
\frac{\cprobi{\vth}{\vx,\vz}}{\int \cprobi{\vth}{\vx,\vz} \text{d}\vz}
$$

\item This requires to solve the sub-problem of computing the \bluef{marginal likelihood} of the observation:
$$\displaystyle \cprobi{\vth}{\vx} = \int \cprobi{\vth}{\vx,\vz} \text{d}\vz$$

\eci
\end{frame}

%%%%%%%%%%%%%%%%%%%%%%%%%%%%%%%%%%%%%%%%%%%%%%%%%%%%%%
\begin{frame}{LVM (3): Why the posterior distribution $\cprobi{\vth}{\vz | \vx}$?}

\plitemsep 0.2in

\bci 

\item \bluef{Explanation of the observation.} Allows us to figure out which latent configurations could have plausibly generated the observation data samples.

\item \bluef{Learning of model parameters $\vth$.} Training LVMs to estimate $\vth$ (e.g., ML) requires $\cprobi{\vth}{\vz | \vx}$ in its inner loops

\item[]
\mycolorbox{
\vspace{-0.3cm}
$$
\text{marginal likelihood $\cprobi{\vth}{\vx}$} \implies \text{posterior distribution $\cprobi{\vth}{\vz | \vx}$} \implies \text{$\vth_{\ml}$}
$$
}
\eci
\end{frame}

%%%%%%%%%%%%%%%%%%%%%%%%%%%%%%%%%%%%%%%%%%%%%%%%%%%%%%
\begin{frame}{LVM (4): How is $\cprobi{\vth}{\vz | \vx}$? used for $\vth_{\ml}$?}

\plitemsep 0.2in

\bci 

\item In ML, we need the gradient of the marginal log-likelihood. For a data sample $\vx,$
\aleq{
\grad_{\vth} \log p_{\vth}(\vx) &= \frac{\grad_{\vth} p_{\vth}(\vx)}{p_{\vth}(\vx)}
= \frac{\int \grad_{\vth} p_{\vth}(\vx,\vz)\text{d}\vz}{p_{\vth}(\vx)} = 
\frac{\int p_{\vth}(\vx,\vz) \grad_{\vth} \log p_{\vth}(\vx,\vz)\text{d}\vz}{p_{\vth}(\vx)} \cr
&= \int \orangef{p_{\vth}(\vz|\vx)} \grad_{\vth} \log p_{\vth}(\vx,\vz)\text{d}\vz
}

\item $\cprobi{\vth}{\vz | \vx}$ performs \bluef{credit assignment} over latent configurations
\eci
\end{frame}

% %%%%%%%%%%%%%%%%%%%%%%%%%%%%%%%%%%%%%%%%%%%%%%%%%%%%%%
% \begin{frame}{Generative Modeling with Latent Variables}

% \plitemsep 0.1in

% \myvartwocols{0.7}{0.29}{0.65}
% {
% \bci 
% \item Generative process
% \bci
% \item $\vz \sim p(\vz)$
% \item $\vx \sim p(\vx | \vz)$
% \eci
% \eci

% \bigskip
% \aleq
% {
% p(\vx) &= \int p(\vx,\vz)\text{d}\vz \cr
% &=\int p(\vx|\vz)p(\vz)\text{d}\vz
% }

% }
% {
% \vspace{-0.5cm}
% \mypic{0.4}{L10_latent.png}
% \vspace{-0.6cm}
% \raggedleft
% {\tiny Source: \url{https://dlvu.github.io/slides/dlvu.lecture06.pdf}}
% }

% \end{frame}

% %%%%%%%%%%%%%%%%%%%%%%%%%%%%%%%%%%%%%%%%%%%%%%%%%%%%%%
% \begin{frame}{Latent-Variable Models (2)}

% \plitemsep 0.1in

% \bci 

% \item We can compute a posterior on the latent variables, but marginalizing over both $\vz$ and $\vth$ is hard:
% $$
% \cprob{\vz \mid \set{X}} = \frac{\cprob{\set{X}\mid \vz} \cprob{\vz}}{\cprob{X}}, \quad 
% \cprob{\set{X} \mid \vz} = \int \cprob{\set{X} \mid \vz, \vth} \cprob{\vth} \text{d}\vth
% $$


% \item Instead, it is easier to compute the latent-variable posterior, but conditioned on the model parameters, i.e., 
% $$
% \cprob{\vz \mid \set{X},\vth} = \frac{\cprob{\set{X}\mid \vz,\vth} \cprob{\vz}}{\cprob{\set{X} \mid \vth}}
% $$

% \item \question How do we use the posteriors $\cprob{\vz \mid \set{X}}$ or $\cprob{\vz \mid \set{X},\vth}$ in practice? Any examples?
% \eci
% \end{frame}

%%%%%%%%%%%%%%%%%%%%%%%%%%%%%%%%%%%%%%%%%%%%%%%%%%%%%%
\section{L8(5)}
\begin{frame}{Roadmap}

\plitemsep 0.1in

\bce[(1)] 

\item \grayf{Data, Models, and Learning}
\item \grayf{Models as Functions: Empirical Risk Minimization} 
\item \grayf{Models as Probabilistic Models: Parameter Estimation (ML and MAP)}
\item \grayf{Probabilistic Modeling and Inference} 
\item \redf{Directed Graphical Models }
\item \grayf{Model Selection}
\ece
\end{frame}

%%%%%%%%%%%%%%%%%%%%%%%%%%%%%%%%%%%%%%%%%%%%%%%%%%%%%%
\begin{frame}{Graphical Models}

\plitemsep 0.1in

\bci 

\item Joint distribution of a probabilistic model: key quantity of interest, but quite complicated without structural properties

\item However, there exist relations of \bluef{independence}, \bluef{conditional independence} among random variables.  

\item (Probabilistic) graphical models: Roughly speaking, a graph of random variables.

\bci
\item Simple ways to visualize the structure of the model
\item Insights into the structural properties, e.g., conditional independence
\item Computations for inference and learning can be expressed in terms of graphical manipulations
\eci
\eci
\end{frame}

%%%%%%%%%%%%%%%%%%%%%%%%%%%%%%%%%%%%%%%%%%%%%%%%%%%%%%
\begin{frame}{Graph Semantics}

\mytwocols{0.4}
{
\aleq{
\cprob{a,b,c} = \cprob{c| a,b} \cprob{b | a}\cprob{a}
}
\mypic{0.4}{L8_gmodel_ex1.png}
}
{
%\vspace{-0.3cm}
\aleq{
&\cprob{x_1, x_2, x_3, x_4, x_5} = \cr
&\cprob{x_1}\cprob{x_5}\cprob{x_2 | x_5}\cprob{x_3 | x_1, x_2} \cprob{x_4 | x_2}}
\mypic{0.6}{L8_gmodel_ex2.png}
}
\vspace{-0.3cm}
\plitemsep 0.03in
\bci 
\item Nodes: random variables
\item Directed edge for direct dependence: $b$ directly depends on $a$: $a \rightarrow b$
\item Graph layout: factorization of the joint distribution
$$
\cprob{x_1, \ldots, x_K} = \prod_{k=1}^K \cprob{x_k \mid \mathbf{Pa}_k}, \quad\text{$\mathbf{Pa}_k$ are the parent nodes of $x_k.$}
$$
\eci
\end{frame}

%%%%%%%%%%%%%%%%%%%%%%%%%%%%%%%%%%%%%%%%%%%%%%%%%%%%%%
\begin{frame}{Example: $N$ coin-flip experiments}

\mypic{0.75}{L8_coinflip.png}
\plitemsep 0.07in
\bci 
\item Shaded nodes: observables, $\mu$: probability of head, a (latent) random variable
\item Joint distribution
$$
\cprob{x_1, \ldots, x_N \mid \mu} = \prod_{n=1}^N \cprob{x_n \mid \mu}
$$
\eci
\end{frame}

%%%%%%%%%%%%%%%%%%%%%%%%%%%%%%%%%%%%%%%%%%%%%%%%%%%%%%
\begin{frame}{Conditional Independence and $d$-Separation}

\plitemsep 0.07in
\bci 
\item \question How can we see conditional independence in the directed graphical models? For example, $\set{A} \indep \set{B} \mid \set{C}$?
\item \bluef{$d$-separation}
\bci
\item All possible trails\footnote{paths that ignore the direction of the arrows} from any node $\set{A}$ to any node in $\set{B}$
\item Any such path is blocked if it includes any node such that either of the following is true:
\bci
\item The arrows on the path meet either head to tail or tail to tail at the node, and the node is in \set{C}
\item The arrows meet head to head at the node, and neither the node nor any of its descendants is in \set{C}
\eci
\item If all the paths are blocked, then $\set{A}$ is $d$-separated from $\set{B}$ by $\set{C}.$
\item If $d$-separated, $\set{A} \indep \set{B} \mid \set{C}$
\eci
\eci

% \myvartwocols{0.4}{0.7}{0.26}
% {
% \small

% }
% {
% \mypic{0.7}{L8_dsep.png}
% }

\end{frame}

%%%%%%%%%%%%%%%%%%%%%%%%%%%%%%%%%%%%%%%%%%%%%%%%%%%%%%
\begin{frame}{Example}

\mypic{0.25}{L8_dsep.png}
% \vspace{-0.8cm}
% \raggedleft{\scriptsize Source: \url{http://www.causality.inf.ethz.ch/data/LUCAS.html}}
\plitemsep 0.1in
\bci 
\item $b \indep d \mid a,c$
\item $a \indep c \mid b$
\item $b \not\indep d \mid c$
\item $a \not\indep c \mid b,e$
\eci
\end{frame}

%%%%%%%%%%%%%%%%%%%%%%%%%%%%%%%%%%%%%%%%%%%%%%%%%%%%%%
\begin{frame}{Example in Healthcare}

\mypic{0.6}{L8_lung_cancer.png}
\vspace{-0.8cm}
\raggedleft{\scriptsize Source: \url{http://www.causality.inf.ethz.ch/data/LUCAS.html}}

\end{frame}

%%%%%%%%%%%%%%%%%%%%%%%%%%%%%%%%%%%%%%%%%%%%%%%%%%%%%%
\begin{frame}{Three Types of Graphical Models}

\mypic{0.7}{L8_all_gmodels.png}

\plitemsep 0.15in
\bci
\item \bluef{Directed graphical models (or Bayesian Networks)}

\item Undirected graphical models (Markov Random Fields)

\item Factor graphs
\eci

\end{frame}

%%%%%%%%%%%%%%%%%%%%%%%%%%%%%%%%%%%%%%%%%%%%%%%%%%%%%%
\begin{frame}{Roadmap}

\plitemsep 0.1in

\bce[(6)]

\item \grayf{Data, Models, and Learning}
\item \grayf{Models as Functions: Empirical Risk Minimization} 
\item \grayf{Models as Probabilistic Models: Parameter Estimation (ML and MAP)}
\item \grayf{Probabilistic Modeling and Inference} 
\item \grayf{Directed Graphical Models }
\item \redf{Model Selection}
\ece
\end{frame}

%%%%%%%%%%%%%%%%%%%%%%%%%%%%%%%%%%%%%%%%%%%%%%%%%%%%%%
\begin{frame}{Nested Cross-Validation}


\mypic{0.65}{L8_nested_cross_validation.png}

\plitemsep 0.1in

\bci 

\item Model selection
\bci
\item Tradeoff between model complexity and data fit

\item \bluef{Occam's razor.} Find the simplest model that explains the data resonably well. 
\eci


\item Test set: estimate the generalization performance

\item Validation set: choose the best model
\eci
\end{frame}

%%%%%%%%%%%%%%%%%%%%%%%%%%%%%%%%%%%%%%%%%%%%%%%%%%%%%%
\begin{frame}{Bayesian Model Selection}

\plitemsep 0.1in

\bci 

\item A set of models $\bm{M} = \{M_1, \ldots, M_k \},$ where each $M_k$ has $\vth_k$ parameters. A prior $\cprob{M}$ on each model $M \in \bm{M}.$ 
$$
M_k \sim \cprob{M}, \quad \vth_k \sim \cprob{\vth \mid M_k}, \quad \set{D} \sim \cprob{\set{D} \mid \vth_k}
$$
\item Posterior distribution $\cprob{M_k \mid \set{D}} \propto \cprob{M_k} \cprob{\set{D} \mid M_k},$ where we have the following \bluef{model evidence} or \bluef{marginal likelihood}:
$$
\cprob{\set{D} \mid M_k}  = \int \cprob{\set{D} \mid \vth_k} \cprob{\vth_k \mid M_k} \text{d}\vth_k \quad \text{(***)}
$$
\item MAP for the model: $M^* = \arg \max_{M_k} \cprob{M_k \mid \set{D}}$
\item With the uniform model prior (i.e., $\cprob{M_k} = 1/k$), the MAP estimate equals to maximization of model evidence. 
\eci
\end{frame}

%%%%%%%%%%%%%%%%%%%%%%%%%%%%%%%%%%%%%%%%%%%%%%%%%%%%%%
\begin{frame}{Bayes Factors for Model Comparison}

\plitemsep 0.1in

\bci 

\item Compare two probabilistic models $M_1$ and $M_2$:
$$
\text{(Posterior odds)} = \frac{\cprob{M_1 \mid \set{D}}}{\cprob{M_2 \mid \set{D}}} = \frac
{
\frac{\cprob{\set{D} \mid M_1}\cprob{M_1}}{\cprob{\set{D}}}
}
{
\frac{\cprob{\set{D} \mid M_2}\cprob{M_2}}{\cprob{\set{D}}}
}
= \underbrace{\frac{\cprob{M_1}}{\cprob{M_2}}}_{\text{Prior odds}} 
\underbrace{\frac{\cprob{\set{D} \mid M_1}}{\cprob{\set{D} \mid M_2}}}_{\text{Bayes factor}}
$$
\item $\cprob{\set{D} \mid M_k}$: How well the data is predicted by the model $M_k$
\item With the uniform model prior, the prior odds $= 1$

\item Computation of Bayes factor requires the complex integration (***) in the previous slide. In this case, we rely on some approximations such as MCMC (Markov Chain Monte Carlo). 
\eci

\end{frame}


%%%%%%%%%%%%%%%%%%%%%%%%%%%%%%%%%%%%%%%%%%%%%%%%%%%%%%
\begin{frame}{Summary}

\plitemsep 0.1in

\bci 

\item 
\eci
\end{frame}



%%%%%%%%%%%%%%%%%%%%%%%%%%%%%%%%%%%%%%%%%%%%%%%%%%%%%%
\begin{frame}{}
\vspace{2cm}
\LARGE Questions?


\end{frame}

%%%%%%%%%%%%%%%%%%%%%%%%%%%%%%%%%%%%%%%%%%%%%%%%%%%%%%
\begin{frame}{Review Questions}
% \tableofcontents
%\plitemsep 0.1in
\bce[1)]
\item 

\ece
\end{frame}


\end{document}
