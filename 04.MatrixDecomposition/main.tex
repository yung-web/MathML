%\pdfminorversion=4
\documentclass[handout,fleqn,aspectratio=169]{beamer}

% when making printed slides
\usepackage{pgfpages}
\pgfpagesuselayout{resize to}[a4paper,landscape,border shrink=5mm]

\usepackage[english]{babel}
\usepackage{tikz}
\usepackage{courier}
\usepackage{array}
\usepackage{bold-extra}
%\usepackage{minted}
\usepackage[thicklines]{cancel}
\usepackage{fancyvrb}
\usepackage{kotex}
\usepackage{paralist}
\usepackage{collectbox}
\usepackage{bm}

\usepackage{mathrsfs}
\usepackage[reqno,disallowspaces]{mathtools}  % imports amsmath
\usepackage{amsfonts} %for Y&Y BSR AMS fonts
\usepackage{amssymb}
\usepackage{amscd}
%\usepackage{tikz,lipsum,lmodern}
\usepackage[most]{tcolorbox}
\usepackage{verbatim}
\mode<presentation>
{
  \usetheme{default}
  \usecolortheme{default}
  \usefonttheme{default}
  \setbeamertemplate{navigation symbols}{}
  \setbeamertemplate{caption}[numbered]
  \setbeamertemplate{footline}[frame number]  % or "page number"
  \setbeamercolor{frametitle}{fg=yellow}
  \setbeamercolor{footline}{fg=black}
} 

\setbeamercolor{block body alerted}{bg=alerted text.fg!10}
\setbeamercolor{block title alerted}{bg=alerted text.fg!20}
\setbeamercolor{block body}{bg=structure!10}
\setbeamercolor{block title}{bg=structure!20}
\setbeamercolor{block body example}{bg=green!10}
\setbeamercolor{block title example}{bg=green!20}
\setbeamertemplate{blocks}[rounded][shadow]

\xdefinecolor{dianablue}{rgb}{0.18,0.24,0.31}
\xdefinecolor{darkblue}{rgb}{0.1,0.1,0.7}
\xdefinecolor{darkgreen}{rgb}{0,0.5,0}
\xdefinecolor{darkgrey}{rgb}{0.35,0.35,0.35}
\xdefinecolor{darkorange}{rgb}{0.8,0.5,0}
\xdefinecolor{darkred}{rgb}{0.7,0,0}
\definecolor{darkgreen}{rgb}{0,0.6,0}
\definecolor{mauve}{rgb}{0.58,0,0.82}

\usetikzlibrary{shapes.callouts}

\makeatletter
\setbeamertemplate{footline}
{
  \leavevmode%
  \hbox{%
  \begin{beamercolorbox}[wd=.333333\paperwidth,ht=2.25ex,dp=1ex,center]{author in head/foot}%
    \usebeamerfont{author in head/foot}\insertsection
  \end{beamercolorbox}%
  \begin{beamercolorbox}[wd=.333333\paperwidth,ht=2.25ex,dp=1ex,center]{title in head/foot}%
    \usebeamerfont{title in head/foot}\insertsubsection
  \end{beamercolorbox}%
  \begin{beamercolorbox}[wd=.333333\paperwidth,ht=2.25ex,dp=1ex,right]{date in head/foot}%
    \usebeamerfont{date in head/foot}
    \insertshortdate{}\hspace*{2em}
    \insertframenumber{} / \inserttotalframenumber\hspace*{2ex} 
  \end{beamercolorbox}}%

  \vskip0pt%
}
\makeatother


\title[]{Lecture 4: Matrix Decompositions}
\author{Yi, Yung (이융)}
\institute{Mathematics for Machine Learning\\ \url{https://yung-web.github.io/home/courses/mathml.html}
\\KAIST EE}
\date{\today}

%%%%%%%%%%%% real, integer notation
\newcommand{\real}{{\mathbb R}}
\newcommand{\realn}{{\mathbb R}^{n}}
\newcommand{\realm}{{\mathbb R}^{m}}
\newcommand{\realD}{{\mathbb R}^{D}}
\newcommand{\realM}{{\mathbb R}^{M}}
\newcommand{\realN}{{\mathbb R}^{N}}
\newcommand{\realnn}{{\mathbb R}^{n \times n}}
\newcommand{\realmm}{{\mathbb R}^{m \times m}}
\newcommand{\realmn}{{\mathbb R}^{m \times n}}
\newcommand{\realnm}{{\mathbb R}^{n \times m}}
\newcommand{\realDM}{{\mathbb R}^{D \times M}}
\newcommand{\realMD}{{\mathbb R}^{M \times D}}
\newcommand{\complex}{{\mathbb C}}
\newcommand{\integer}{{\mathbb Z}}
\newcommand{\natu}{{\mathbb N}}


%%% set, vector, matrix
\newcommand{\set}[1]{\ensuremath{\mathcal #1}}
\newcommand{\sets}[1]{\ensuremath{\{#1 \}}}
\renewcommand{\vec}[1]{\bm{#1}}
\newcommand{\mat}[1]{\bm{#1}}

%%%% vector
\def\vx{\vec{x}}
\def\vy{\vec{y}}
\def\vz{\vec{z}}
\def\vf{\vec{f}}
\def\ve{\vec{e}}
\def\vr{\vec{r}}
\def\vb{\vec{b}}
\def\vc{\vec{c}}
\def\vd{\vec{d}}
\def\vm{\vec{m}}
\def\vu{\vec{u}}
\def\vv{\vec{v}}
\def\vw{\vec{w}}
\def\vX{\vec{X}}
\def\vY{\vec{Y}}
\def\vZ{\vec{Z}}
\def\vth{\vec{\theta}}
\def\vmu{\vec{\mu}}
\def\vnu{\vec{\nu}}
\def\vlam{\vec{\lambda}}
\def\vep{\vec{\epsilon}}
\def\vpi{\vec{\pi}}
\def\vphi{\vec{\phi}}
\def\vxi{\vec{\xi}}
\def\valpha{\vec{\alpha}}
\def\vgamma{\vec{\gamma}}

%%%% Well-used matrices
\def\mA{\mat{A}}
\def\mB{\mat{B}}
\def\mC{\mat{C}}
\def\mD{\mat{D}}
\def\mI{\mat{I}}
\def\mJ{\mat{J}}
\def\mK{\mat{K}}
\def\mE{\mat{E}}
\def\mP{\mat{P}}
\def\mQ{\mat{Q}}
\def\mU{\mat{U}}
\def\mV{\mat{V}}
\def\mR{\mat{R}}
\def\mS{\mat{S}}
\def\mX{\mat{X}}
\def\msig{\mat{\Sigma}}
\def\mPhi{\mat{\Phi}}


\usepackage{amsmath}
%%%%% vector caculus useful macro
% ...\d, which typesets a derivative. ex: \d{y}{x}, instead of \frac{dx}{dy}.
\renewcommand{\d}[2]{\frac{\text{d} #1}{\text{d} #2}}


% ...similar for double-derivatives. ex: \dd{y}{x}.
\newcommand{\dd}[2]{\frac{\text{d}^2 #1}{\text{d} #2^2}}

% ...similar for partial derivatives. ex: \pd{y}{x}.
\newcommand{\pd}[2]{\frac{\partial #1}{\partial #2}}


% ...similar for partial double derivatives. ex: \pdd{y}{x}.
\newcommand{\pdd}[2]{\frac{\partial^2 #1}{\partial #2^2}}
% pdd with argument
\newcommand{\pdda}[3]{\frac{\partial^2 #1}{\partial #2 \partial #3}}

\usepackage{xparse}

%%%% caligraphic fonts
\def\cL{\ensuremath{{\cal L}}}
\def\cN{\ensuremath{{\cal N}}}
\def\cD{\ensuremath{{\cal D}}}
\def\cC{\ensuremath{{\cal C}}}
\def\cX{\ensuremath{{\cal X}}}
\def\cY{\ensuremath{{\cal Y}}}

%%% big parenthesis
\def\Bl{\Bigl}
\def\Br{\Bigr}
\def\lf{\left}
\def\ri{\right}


%%% floor notations
\newcommand{\lfl}{{\lfloor}}
\newcommand{\rfl}{{\rfloor}}
\newcommand{\floor}[1]{{\lfloor #1 \rfloor}}

%%% gradient
\newcommand{\grad}[1]{\nabla #1}
\newcommand{\hess}[1]{\text{H} #1}

%%% definition
%\newcommand{\eqdef}{\ensuremath{\triangleq}}
\newcommand{\eqdef}{\ensuremath{:=}}
%%% imply
\newcommand{\imp}{\Longrightarrow}



\newcommand{\separator}{
%  \begin{center}
    \par\noindent\rule{\columnwidth}{0.3mm}
%  \end{center}
}

\newcommand{\mynote}[1]{{\it \color{red} [#1]}}







%%% equation alignment
\newcommand{\aleq}[1]{\begin{align*}#1\end{align*}}

%%%%%%%%%%%%%%%% colored emphasized font, blanked words

\newcommand{\empr}[1]{{\color{red}\emph{#1}}}
\newcommand{\empb}[1]{{\color{blue}\emph{#1}}}
\newcommand{\redf}[1]{{\color{red} #1}}
\newcommand{\bluef}[1]{{\color{blue} #1}}
\newcommand{\grayf}[1]{{\color{gray} #1}}
\newcommand{\magenf}[1]{{\color{magenta} #1}}
\newcommand{\greenf}[1]{{\color{green} #1}}
\newcommand{\cyanf}[1]{{\color{cyan} #1}}
\newcommand{\orangef}[1]{{\color{orange} #1}}

\newcommand{\blk}[1]{\underline{\mbox{\hspace{#1}}}}


\newcommand{\redblk}[1]{\framebox{\color{red} #1}}
\newcommand{\redblank}[2]{\framebox{\onslide<#1->{\color{red} #2}}}
\newcommand{\blueblk}[1]{\framebox{\color{blue} #1}}
\newcommand{\blueblank}[2]{\framebox{\onslide<#1->{\color{blue} #2}}}



\makeatletter
\newcommand{\mybox}{%
    \collectbox{%
        \setlength{\fboxsep}{1pt}%
        \fbox{\BOXCONTENT}%
    }%
}
\makeatother

\makeatletter
\newcommand{\lecturemark}{%
    \collectbox{%
        \setlength{\fboxsep}{1pt}%
        \fcolorbox{red}{yellow}{\BOXCONTENT}%
    }%
}
\makeatother

\newcommand{\mycolorbox}[1]{
\begin{tcolorbox}[colback=red!5!white,colframe=red!75!black]
#1
\end{tcolorbox}
}
%%%% figure inclusion
\newcommand{\mypic}[2]{
\begin{center}
\includegraphics[width=#1\textwidth]{#2}
\end{center}
}

\newcommand{\myinlinepic}[2]{
\makebox[0cm][r]{\raisebox{-4ex}{\includegraphics[height=#1]{#2}}}
}




%%%% itemized and enumerated list
\newcommand{\bci}{\begin{compactitem}}
\newcommand{\eci}{\end{compactitem}}
\newcommand{\bce}{\begin{compactenum}}
\newcommand{\ece}{\end{compactenum}}


%%%% making 0.5/0.5 two columns
%%%% how to use: first number: length of separation bar
% \mytwocols{0.6}
% {
% contents in the left column
% }
% {
% contents in the right column
% }
%%%%

\newcommand{\mytwocols}[3]{
\begin{columns}[T] \column{.499\textwidth} #2 \column{.001\textwidth} \rule{.3mm}{{#1}\textheight} \column{.499\textwidth} #3 \end{columns}}

\newcommand{\mythreecols}[4]{
\begin{columns}[T] \column{.31\textwidth} #2 \column{.001\textwidth} \rule{.3mm}{{#1}\textheight} \column{.31\textwidth} #3 \column{.001\textwidth} \rule{.3mm}{{#1}\textheight} \column{.31\textwidth} #4  \end{columns}}

\newcommand{\mysmalltwocols}[3]{
\begin{columns}[T] \column{.4\textwidth} #2 \column{.001\textwidth} \rule{.3mm}{{#1}\textheight} \column{.4\textwidth} #3 \end{columns}}

%%%% making two columns with customized ratios
%%%% how to use: 
%first parameter: length of separation bar
%second parameter: ratio of left column
%third parameter: ratio of right column
% \mytwocols{0.6}{0.7}{0.29}
% {
% contents in the left column
% }
% {
% contents in the right column
% }
%%%%
\newcommand{\myvartwocols}[5]{
\begin{columns}[T] \column{#2\textwidth} {#4} \column{.01\textwidth} \rule{.3mm}{{#1}\textheight} \column{#3\textwidth} {#5} \end{columns}}

%%% making my block in beamer
%%% first parameter: title of block
%%% second parameter: contents of block
\newcommand{\myblock}[2]{
\begin{block}{#1} {#2}  \end{block}}

%%% independence notation
\newcommand{\indep}{\perp \!\!\! \perp}

%%%% probability with different shapes (parenthesis or bracket) and different sizes
%%% `i' enables us to insert the subscript to the probability 
\newcommand{\bprob}[1]{\mathbb{P}\Bl[ #1 \Br]}
\newcommand{\prob}[1]{\mathbb{P}[ #1 ]}
\newcommand{\cbprob}[1]{\mathbb{P}\Bl( #1 \Br)}
\newcommand{\cprob}[1]{\mathbb{P}( #1 )}
\newcommand{\probi}[2]{\mathbb{P}_{#1}[ #2 ]}
\newcommand{\bprobi}[2]{\mathbb{P}_{#1}\Bl[ #2 \Br]}
\newcommand{\cprobi}[2]{\mathbb{P}_{#1}( #2 )}
\newcommand{\cbprobi}[2]{\mathbb{P}_{#1}\Bl( #2 \Br)}

%%%% expectation with different shapes (parenthesis or bracket) and different sizes
%%% `i' enables us to insert the subscript to the expectation
\newcommand{\expect}[1]{\mathbb{E}[ #1 ]}
\newcommand{\cexpect}[1]{\mathbb{E}( #1 )}
\newcommand{\bexpect}[1]{\mathbb{E}\Bl[ #1 \Br]}
\newcommand{\cbexpect}[1]{\mathbb{E}\Bl( #1 \Br)}
\newcommand{\bbexpect}[1]{\mathbb{E}\lf[ #1 \ri]}
\newcommand{\expecti}[2]{\mathbb{E}_{#1}[ #2 ]}
\newcommand{\bexpecti}[2]{\mathbb{E}_{#1}\Bl[ #2 \Br]}
\newcommand{\bbexpecti}[2]{\mathbb{E}_{#1}\lf[ #2 \ri]}

%%%% variance
\newcommand{\var}[1]{\text{var}[ #1 ]}
\newcommand{\bvar}[1]{\text{var}\Bl[ #1 \Br]}
\newcommand{\cvar}[1]{\text{var}( #1 )}
\newcommand{\cbvar}[1]{\text{var}\Bl( #1 \Br)}

%%%% covariance
\newcommand{\cov}[1]{\text{cov}( #1 )}
\newcommand{\bcov}[1]{\text{cov}\Bl( #1 \Br)}

%%% Popular pmf, pdf notation to avoid long typing
\newcommand{\px}{\ensuremath{p_X(x)}}
\newcommand{\py}{\ensuremath{p_Y(y)}}
\newcommand{\pz}{\ensuremath{p_Z(z)}}
\newcommand{\pxA}{\ensuremath{p_{X|A}(x)}}
\newcommand{\pyA}{\ensuremath{p_{Y|A}(y)}}
\newcommand{\pzA}{\ensuremath{p_{Z|A}(z)}}
\newcommand{\pxy}{\ensuremath{p_{X,Y}(x,y)}}
\newcommand{\pxcy}{\ensuremath{p_{X|Y}(x|y)}}
\newcommand{\pycx}{\ensuremath{p_{Y|X}(y|x)}}

\newcommand{\fx}{\ensuremath{f_X(x)}}
\newcommand{\Fx}{\ensuremath{F_X(x)}}
\newcommand{\fy}{\ensuremath{f_Y(y)}}
\newcommand{\Fy}{\ensuremath{F_Y(y)}}
\newcommand{\fz}{\ensuremath{f_Z(z)}}
\newcommand{\Fz}{\ensuremath{F_Z(z)}}
\newcommand{\fxA}{\ensuremath{f_{X|A}(x)}}
\newcommand{\fyA}{\ensuremath{f_{Y|A}(y)}}
\newcommand{\fzA}{\ensuremath{f_{Z|A}(z)}}
\newcommand{\fxy}{\ensuremath{f_{X,Y}(x,y)}}
\newcommand{\Fxy}{\ensuremath{F_{X,Y}(x,y)}}
\newcommand{\fxcy}{\ensuremath{f_{X|Y}(x|y)}}
\newcommand{\fycx}{\ensuremath{f_{Y|X}(y|x)}}

\newcommand{\fth}{\ensuremath{f_\Theta(\theta)}}
\newcommand{\fxcth}{\ensuremath{f_{X|\Theta}(x|\theta)}}
\newcommand{\fthcx}{\ensuremath{f_{\Theta|X}(\theta|x)}}

\newcommand{\pkcth}{\ensuremath{p_{X|\Theta}(k|\theta)}}
\newcommand{\fthck}{\ensuremath{f_{\Theta|X}(\theta|k)}}


%%%% indicator
\newcommand{\indi}[1]{\mathbf{1}_{ #1 }}

%%%% exponential rv.
\newcommand{\elambdax}{\ensuremath{e^{-\lambda x}}}

%%%% normal  rv.
\newcommand{\stdnormal}{\ensuremath{\frac{1}{\sqrt{2\pi}} e^{-x^2/2}}}
\newcommand{\gennormal}{\ensuremath{\frac{1}{\sigma\sqrt{2\pi}} e^{-(x-\mu)^2/2}}}

%%%%%% estimator, estimate
\newcommand{\hth}{\ensuremath{\hat{\theta}}}
\newcommand{\hTH}{\ensuremath{\hat{\Theta}}}
\newcommand{\MAP}{\ensuremath{\text{MAP}}}
\newcommand{\LMS}{\ensuremath{\text{LMS}}}
\newcommand{\LLMS}{\ensuremath{\text{L}}}
\newcommand{\ML}{\ensuremath{\text{ML}}}

%%%% colored text
\newcommand{\red}[1]{\color{red}#1} 
\newcommand{\cyan}[1]{\color{cyan}#1} 
\newcommand{\magenta}[1]{\color{magenta}#1} 
\newcommand{\blue}[1]{\color{blue}#1} 
\newcommand{\green}[1]{\color{green}#1} 
\newcommand{\white}[1]{\color{white}#1} 
\newcommand{\gray}[1]{\color{gray}#1} 

%%% definition
\newcommand{\defi}{{\color{red} Definition.} } 
\newcommand{\exam}{{\color{red} Example.} } 
\newcommand{\question}{{\color{red} Question.} } 
\newcommand{\thm}{{\color{red} Theorem.} } 
\newcommand{\background}{{\color{red} Background.} } 
\newcommand{\msg}{{\color{red} Message.} } 


\def\ml{\text{ML}}
\def\map{\text{MAP}}

%%%%%%%%%%%%%%%%%%%%%%% old macros that you can ignore %%%%%%%%%%%%%%%%%%%%%%%%

% \def\un{\underline}
% \def\ov{\overline}


% \newcommand{\beq}{\begin{eqnarray*}}
% \newcommand{\eeq}{\end{eqnarray*}}
% \newcommand{\beqn}{\begin{eqnarray}}
% \newcommand{\eeqn}{\end{eqnarray}}
% \newcommand{\bemn}{\begin{multiline}}
% \newcommand{\eemn}{\end{multiline}}
% \newcommand{\beal}{\begin{align}}
% \newcommand{\eeal}{\end{align}}
% \newcommand{\beas}{\begin{align*}}
% \newcommand{\eeas}{\end{align*}}



% \newcommand{\bd}{\begin{displaymath}}
% \newcommand{\ed}{\end{displaymath}}
% \newcommand{\bee}{\begin{equation}}
% \newcommand{\eee}{\end{equation}}


% \newcommand{\vs}{\vspace{0.2in}}
% \newcommand{\hs}{\hspace{0.5in}}
% \newcommand{\el}{\end{flushleft}}
% \newcommand{\bl}{\begin{flushleft}}
% \newcommand{\bc}{\begin{center}}
% \newcommand{\ec}{\end{center}}
% \newcommand{\remove}[1]{}

% \newtheorem{theorem}{Theorem}
% \newtheorem{corollary}{Corollary}
% \newtheorem{prop}{Proposition}
% \newtheorem{lemma}{Lemma}
% \newtheorem{defi}{Definition}
% \newtheorem{assum}{Assumption}
% \newtheorem{example}{Example}
% \newtheorem{property}{Property}
% \newtheorem{remark}{Remark}

% \newcommand{\separator}{
%   \begin{center}
%     \rule{\columnwidth}{0.3mm}
%   \end{center}
% }

% \newenvironment{separation}
% { \vspace{-0.3cm}
%   \separator
%   \vspace{-0.25cm}
% }
% {
%   \vspace{-0.5cm}
%   \separator
%   \vspace{-0.15cm}
% }

% \def\A{\mathcal A}
% \def\oA{\overline{\mathcal A}}
% \def\S{\mathcal S}
% \def\D{\mathcal D}
% \def\eff{{\rm Eff}}
% \def\bD{\bm{D}}
% \def\cU{{\cal U}}
% \def\bbs{{\mathbb{s}}}
% \def\bbS{{\mathbb{S} }}
% \def\cM{{\cal M}}
% \def\bV{{\bm{V}}}
% \def\cH{{\cal H}}
% \def\ch{{\cal h}}
% \def\cR{{\cal R}}
% \def\cV{{\cal V}}
% \def\cA{{\cal A}}
% \def\cX{{\cal X}}
% \def\cN{{\cal N}}
% \def\cJ{{\cal J}}
% \def\cK{{\cal K}}
% \def\cL{{\cal L}}
% \def\cI{{\cal I}}
% \def\cY{{\cal Y}}
% \def\cZ{{\cal Z}}
% \def\cC{{\cal C}}
% \def\cR{{\cal R}}
% \def\id{{\rm Id}}
% \def\st{{\rm st}}
% \def\cF{{\cal F}}
% \def\bz{{\bm z}}
% \def\cG{{\cal G}}
% \def\N{\mathbb{N}}
% \def\bbh{\mathbb{h}}
% \def\bbH{\mathbb{H}}
% \def\bbi{\mathbb{i}}
% \def\bbI{\mathbb{I}}
% \def\R{\mathbb{R}}
% \def\bbR{\mathbb{R}}
% \def\bbr{\mathbb{r}}
% \def\cB{{\cal B}}
% \def\cP{{\cal P}}
% \def\cS{{\cal S}}
% \def\bW{{\bm W}}
% \def\bc{{\bm c}}

% %\def\and{\quad\mbox{and}\quad}
% \def\ind{{\bf 1}}


% \def\bmg{{\bm{\gamma}}}
% \def\bmr{{\bm{\rho}}}
% \def\bmq{{\bm{q}}}
% \def\bmt{{\bm{\tau}}}
% \def\bmn{{\bm{n}}}
% \def\bmcapn{{\bm{N}}}
% \def\bmrho{{\bm{\rho}}}

% \def\igam{\underline{\gamma}(\lambda)}
% \def\sgam{\overline{\gamma}(\lambda)}
% \def\ovt{\overline{\theta}}
% \def\ovT{\overline{\Theta}}
% \def\PP{{\mathrm P}}
% \def\EE{{\mathrm E}}
% \def\iskip{{\vskip -0.4cm}}
% \def\siskip{{\vskip -0.2cm}}

% \def\bp{\noindent{\it Proof.}\ }
% \def\ep{\hfill $\Box$}



%%%%%%%%%% linear algebra macros %%%%%%%%%%%%%%%%%%%%%%%

%--------linsys
%  Use as \begin{linsys}{3}
%           x &+ &3y &+ &a &= &7 \\
%           x &- &3y &+ &a &= &7
%         \end{linsys}
% Remark: TeXbook pp. 167-170 says to put a medmuskip around a +; and that's
% 4/18-ths of an em.  Why does 2/18-ths of an em work?  I don't know, but
% comparing to a regular displayed equation suggests it is right.
% (darseneau says LaTeX puts in half an \arraycolsep.)
\newenvironment{linsys}[2][m]{%
\setlength{\arraycolsep}{.1111em} % p. 170 TeXbook; a medmuskip
\begin{array}[#1]{@{}*{#2}{rc}r@{}}
}{%
\end{array}}

\newsavebox\boxofmathplus
\sbox{\boxofmathplus}{$+$}
\newcommand{\spaceforemptycolumn}{\makebox[\wd\boxofmathplus]{\ }}

%--------grstep
% For denoting a Gauss' reduction step.
% Use as: \grstep{\rho_1+\rho_3} or \grstep[2\rho_5 \\ 3\rho_6]{\rho_1+\rho_3}
% \newcommand{\grstep}[2][\relax]{%
%    \ensuremath{\mathrel{
%        \mathop{\longrightarrow}\limits^{#2\mathstrut}_{
%                                    \begin{subarray}{l} #1 \end{subarray}}}}}

% Advantage of length formulation is that between adjacent
% \grstep's you can add \hspace{-\grsteplength} to make it look not too wide
\newlength{\grsteplength}
\setlength{\grsteplength}{1.5ex plus .1ex minus .1ex}

\newcommand{\grstep}[2][\relax]{%
   \ensuremath{\mathrel{
       \hspace{\grsteplength}\mathop{\longrightarrow}\limits^{#2\mathstrut}_{
                                     \begin{subarray}{l} #1 \end{subarray}}\hspace{\grsteplength}}}}
% If two or more \grsteps are in a row then they need to be tightened
\newcommand{\repeatedgrstep}[2][\relax]{\hspace{-\grsteplength}\grstep[#1]{#2}}

% row swap operation: \rho_1\swap\rho_2
\newcommand{\swap}{\leftrightarrow}

%-------------amatrix
% Augmented matrix.  Usage (note the argument does not count the aug col):
% \begin{amatrix}{2}
%   1  2  3 \\  4  5  6
% \end{amatrix}
\newenvironment{amatrix}[1]{%
  \left(\begin{array}{@{}*{#1}{c}|c@{}}
}{%
  \end{array}\right)
}



%-------------pmat
% For matrices with arguments.
% Usage: \begin{pmat}{c|c|c} 1 &2 &3 \end{pmat}
\newenvironment{pmat}[1]{
  \left(\begin{array}{@{}#1@{}}
}{\end{array}\right)
}



%-------------misc matrices
% \newenvironment{mat}{\left(\begin{array}}{\end{array}\right)}
\newenvironment{detmat}{\left|\begin{array}}{\end{array}\right|}
\newcommand{\deter}[1]{ \mathchoice{\left|#1\right|}{|#1|}{|#1|}{|#1|} }
\newcommand{\generalmatrix}[3]{ %arg1: low-case letter, arg2: rows, arg3: cols 
               \left(
                  \begin{array}{cccc}
                    #1_{1,1}  &#1_{1,2}  &\ldots  &#1_{1,#2}  \\
                    #1_{2,1}  &#1_{2,2}  &\ldots  &#1_{2,#2}  \\
                              &\vdots                         \\
                    #1_{#3,1} &#1_{#3,2} &\ldots  &#1_{#3,#2}
                  \end{array}
               \right)  }

\newcommand{\generaldet}[3]{ %arg1: low-case letter, arg2: rows, arg3: cols 
               \left|
                  \begin{array}{cccc}
                    #1_{11}  &#1_{12}  &\ldots  &#1_{1 #2}  \\
                    #1_{21}  &#1_{22}  &\ldots  &#1_{2 #2}  \\
                              &\vdots                         \\
                    #1_{#3 1} &#1_{#3 2} &\ldots  &#1_{#3 #2}
                  \end{array}
               \right|  }

% With mathtools we can have column entries right flushed
% There is an optional argument \begin{mat}[r]{3} .. \end{mat} for
% right-flushed columns.  Perhaps the rule is that numbers are better 
% right-flushed but if there are any letters it is better centered?
\newenvironment{nmat}[1][c]{\begin{pmatrix*} % disable optional arg [#1] 
      }{\end{pmatrix*}}
% If mat starts with &\vdots get an error; why?  No apparent macro fix, according to texexchange
\newenvironment{vmat}[1][c]{\begin{vmatrix*} % disable optional arg [#1] 
      }{\end{vmatrix*}}
\newenvironment{amat}[2][c]{%
  % disable optional arg \left(\begin{array}{@{}*{#2}{#1}|#1@{}}
  \left(\begin{array}{@{}*{#2}{c}|#1@{}}
}{%
  \end{array}\right)
}
% \newcommand\vdotswithin[1]{% Taken from mathtools.dtx because my TL is not 2011
%   {\mathmakebox[\widthof{\ensuremath{{}#1{}}}][c]{{\vdots}}}}


%------------colvec and rowvec
% Column vector and row vector.  Usage:
%  \colvec{1  \\ 2 \\ 3 \\ 4} and \rowvec{1  &2  &3}
% Colvec takes an optional argument \colvec[r]{x_1 \\ 0}.  Perhaps 
% digits look better right aligned, but if there are any letters it
% needs to be centered?
\newcommand{\colvec}[2][c]{\begin{nmat}[#1] #2 \end{nmat}}
\newcommand{\smallcolvec}[1]{\left(\begin{smallmatrix} #1 \end{smallmatrix}\right)}
% For row vectors, cannot do \newcommand{\rowvec}[1]{\begin{mat} #1 \end{mat}}
% since the delimiters come out too large.
\newcommand{\rowvec}[1]{\setlength{\arraycolsep}{3pt}\left(\begin{matrix} #1 \end{matrix}\right)}



%-------------making aligned columns
% Usage: \begin{aligncolondecimal}{2} 1.2 \\ .33 \end{aligncolondecimal}
% (negative argument centers decimal pt in column).  Also Usage:
% \begin{aligncolondecimal}[0em]{2} 1.2 \\ .33 \end{aligncolondecimal}
% to make the left and right LaTeX-array padding disappear.
\RequirePackage{array}\RequirePackage{dcolumn}
\newenvironment{aligncolondecimal}[2][.1111em]{%
\setlength{\arraycolsep}{#1}
\newcolumntype{.}{D{.}{.}{#2}}\begin{array}{.}}{%
\end{array}}

% Matrix and vector, with numbers centered on decimal point
% Usage: \begin{dmat}{D{.}{.}{1}D{.}{.}{3}}  0  &.123 \\ .2 &.456 \end{dmat}
%  (in the D{.}{.}{number} that is the number of decimal places)
\newlength{\dmatcolsep}\setlength{\dmatcolsep}{5pt}
\newenvironment{dmat}[2][\dmatcolsep]{%
  \setlength{\arraycolsep}{#1}
  \left(\begin{array}{@{}#2@{}}
}{%
  \end{array}\right)}
% Usage: \dcolvec[2]{1.23 \\ 4.56} where the optional argument is the number
% of decimal places.
\newcommand{\dcolvec}[2][-1]{\left(\begin{array}{@{}D{.}{.}{#1}@{}} #2 \end{array}\right)}

%\newcommand{\trans}[1]{ {{#1}^{\mathsf{T}}} } 
\newcommand{\trans}[1]{ {#1}^{\mathsf{T}} } 
\newcommand{\inv}[1]{ {#1}^{-1} } 
\newcommand{\spn}[1]{\ensuremath{\text{span}[#1]} } 
\newcommand{\rk}[1]{\ensuremath{\text{rk}(#1)} } 
\newcommand{\dimm}[1]{\ensuremath{\text{dim}(#1)} } 
\newcommand{\img}[1]{\ensuremath{\text{Im}(#1)} } 
%\newcommand{\norm}[1]{\ensuremath{\left || #1 \right ||} } 
\newcommand{\norm}[1]{\ensuremath{\left \lVert #1 \right \rVert} } 
% orthogonal complement
\newcommand{\ocomp}[1]{\ensuremath{#1^{\bot}} } 
\newcommand{\inner}[2]{\ensuremath{\left\langle #1, #2 \right\rangle} } 
\DeclareMathOperator{\tr}{tr}


% \NewDocumentCommand{\grad}{e{_^}}{%
%   \mathop{}\!% \mathop for good spacing before \nabla
%   \nabla
%   \IfValueT{#1}{_{\!#1}}% tuck in the subscript
%   \IfValueT{#2}{^{#2}}% possible superscript
% }
% \begin{equation*}
%          \begin{nmat}[r]
% 1 &2 &13 \\
%           4  &5  &6
%          \end{nmat}
%       \end{equation*}

% \begin{equation*}
%          \begin{amat}{2}
%           1  &2  &3  \\
%           4  &5  &6
%          \end{amat}
%       \end{equation*}
       
%       \begin{equation*}
%          \begin{pmat}{c|c|c}
% 1 &2 &3 \\
%           4  &5  &6
%          \end{pmat}
%       \end{equation*}

% \begin{equation*}
%          \begin{vmat}
% a &c \\
%           b  &d
%          \end{vmat}
%          =ad-bc
% \end{equation*}

%  \begin{equation*}
%   \vec{v}=\colvec{-1  \\ -0.5  \\ 0}
% \end{equation*}

%  \begin{equation*}
%   \vec{v}=\rowvec{-1  & -0.5  & 0}
% \end{equation*}



\begin{document}

%itemshape
\setbeamertemplate{itemize item}{\scriptsize\raise1.25pt\hbox{\donotcoloroutermaths$\bullet$}}
\setbeamertemplate{itemize subitem}{\tiny\raise1.5pt\hbox{\donotcoloroutermaths$\circ$}}
\setbeamertemplate{itemize subsubitem}{\tiny\raise1.5pt\hbox{\donotcoloroutermaths$\blacktriangleright$}}
%default value for spacing
\plitemsep 0.1in
\pltopsep 0.03in
\setlength{\parskip}{0.15in}
%\setlength{\parindent}{-0.5in}
\setlength{\abovedisplayskip}{0.07in}
\setlength{\belowdisplayskip}{0.07in}
\setlength{\mathindent}{0cm}
\setbeamertemplate{frametitle continuation}{[\insertcontinuationcount]}

\setlength{\leftmargini}{0.5cm}
\setlength{\leftmarginii}{0.5cm}

\setlength{\fboxrule}{0.05pt}
\setlength{\fboxsep}{5pt}


%%%%%%% This should be placed at the end of this file
\logo{\pgfputat{\pgfxy(0.11, 7.4)}{\pgfbox[right,base]{\tikz{\filldraw[fill=dianablue, draw=none] (0 cm, 0 cm) rectangle (50 cm, 1 cm);}\mbox{\hspace{-8 cm}\includegraphics[height=0.7 cm]{../kaist_ee.png}
}}}}

\begin{frame}
  \titlepage
\end{frame}

\logo{\pgfputat{\pgfxy(0.11, 7.4)}{\pgfbox[right,base]{\tikz{\filldraw[fill=dianablue, draw=none] (0 cm, 0 cm) rectangle (50 cm, 1 cm);}\mbox{\hspace{-8 cm}\includegraphics[height=0.7 cm]{../kaist_ee.png}
}}}}

% rule color - gray
\makeatletter
\let\old@rule\@rule
\def\@rule[#1]#2#3{\textcolor{gray}{\old@rule[#1]{#2}{#3}}}
\makeatother




% START START START START START START START START START START START START START
%%%%%%%%%%%%%%%%%%%%%%%%%%%%%%%%%%%%%%%%%%%%%%%%%%%%%%
\begin{frame}{Roadmap}

\plitemsep 0.1in

\bce[(1)] 
\item Determinant and Trace 

\item Eigenvalues and Eigenvectors

\item Cholesky Decomposition

\item Eigendecomposition and Diagonalization

\item Singular Value Decomposition

\item Matrix Approximation

\item Matrix Phylogeny

\ece
\end{frame}

%%%%%%%%%%%%%%%%%%%%%%%%%%%%%%%%%%%%%%%%%%%%%%%%%%%%%%
\begin{frame}{Summary}

\plitemsep 0.3in

\bci 
\item How to summarize matrices: determinants and eigenvalues

\item How matrices can be decomposed: Cholesky decomposition, diagonalization, singular value decomposition

\item How these decompositions can be used for matrix approximation
\eci
\end{frame}

%%%%%%%%%%%%%%%%%%%%%%%%%%%%%%%%%%%%%%%%%%%%%%%%%%%%%%
\section{L4(1)}
\begin{frame}{Roadmap}

\plitemsep 0.1in

\bce[(1)] 
\item \redf{Determinant and Trace}

\item \grayf{Eigenvalues and Eigenvectors

\item Cholesky Decomposition

\item Eigendecomposition and Diagonalization

\item Singular Value Decomposition

\item Matrix Approximation

\item Matrix Phylogeny}

\ece
\end{frame}

%%%%%%%%%%%%%%%%%%%%%%%%%%%%%%%%%%%%%%%%%%%%%%%%%%%%%%
\begin{frame}{Determinant: Motivation (1)}

\plitemsep 0.1in

\bci 
\item For $\mat{A} = 
\begin{nmat}
a_{11}&a_{12}\cr
a_{21} & a_{22}
\end{nmat},
$
$\inv{\mat{A}} =\frac{1}{a_{11}a_{22} - a_{12}a_{21}}
\begin{nmat}
a_{22}&-a_{12}\cr
-a_{21} & a_{11}
\end{nmat}.
$

\item $\mat{A}$ is invertible iff $a_{11}a_{22} - a_{12}a_{21} \neq 0$

\item  Let's define $\det(\mat{A}) = a_{11}a_{22} - a_{12}a_{21}.$

\item Notation: $\det(\mat{A})$ or $|\text{whole matrix}|$

\item What about $3 \times 3$ matrix? By doing some algebra (e.g., Gaussian elimination), 
\aleq{
\begin{vmat}
a_{11} & a_{12} & a_{13} \cr
a_{21} & a_{22} & a_{23} \cr
a_{31} & a_{32} & a_{33} 
\end{vmat}
&= a_{11}a_{22}a_{33} + a_{21} a_{32}a_{13} + a_{31}a_{12}a_{23} \cr
& \hspace{2cm} - a_{31}a_{22}a_{13} - a_{11}a_{32}a_{23} - a_{21} a_{12} a_{33}
}
\eci
\end{frame}

%%%%%%%%%%%%%%%%%%%%%%%%%%%%%%%%%%%%%%%%%%%%%%%%%%%%%%
\begin{frame}{Determinant: Motivation (2)}

\plitemsep 0.1in

\bci 
\item Try to find some pattern ... 

\mytwocols{0.55}
{
\aleq{
&a_{11}a_{22}a_{33} + a_{21} a_{32}a_{13} + a_{31}a_{12}a_{23} \cr
& - a_{31}a_{22}a_{13} - a_{11}a_{32}a_{23} - a_{21} a_{12} a_{33}=\cr
& a_{11} (-1)^{1+1} \det(\mA_{1,1}) + a_{12}(-1)^{1+2} \det(\mA_{1,2}) \cr
& + a_{13}(-1)^{1+3} \det(\mA_{1,3})
}

- $\mA_{k,j}$ is the submatrix of $\mA$ that we obtain when deleting row $k$ and column $j.$
}
{
\centering
\mypic{0.9}{L4_cofactor_ex.png}
{\scriptsize source: www.cliffsnotes.com}
}

\item This is called \bluef{Laplace expansion}.

\item Now, we can generalize this and provide the formal definition of determinant. 

\eci
\end{frame}

%%%%%%%%%%%%%%%%%%%%%%%%%%%%%%%%%%%%%%%%%%%%%%%%%%%%%%
\begin{frame}{Determinant: Formal Definition }

\plitemsep 0.1in

\myblock{Determinant}
{
For a matrix $\mA \in \real^{n \times n},$ for all $j=1, \ldots, n,$
\bce
\item Expansion along column $j$: 
$
\det(\mA) = \sum_{k=1}^n (-1)^{k+j} a_{kj} \det(\mA_{k,j})
$
\item Expansion along row $j$: 
$
\det(\mA)= \sum_{k=1}^n (-1)^{k+j} a_{jk} \det(\mA_{j,k})
$
\ece
}

\bci 
\item All expansion are equal, so no problem with the definition.  
\item \thm $\det(\mA) \neq 0 \Longleftrightarrow \rk{\mA} = n \Longleftrightarrow$ $\mA$ is invertible. 
\eci
\end{frame}

%%%%%%%%%%%%%%%%%%%%%%%%%%%%%%%%%%%%%%%%%%%%%%%%%%%%%%
\begin{frame}{Determinant: Properties}

\plitemsep 0.05in

\bce[(1)]
\item $\det(\mA\mB) = \det(\mA) \det(\mB)$
\item $\det(\mA) = \det(\trans{\mA})$
\item For a regular $\mA$, $\det(\inv{\mA}) = 1/\det(\mA)$
\item For two similar matrices $\mA,\mA'$ (i.e., $\mA' = \inv{\mat{S}}\mA \mat{S}$ for some $\mat{S}$), $\det(\mA) = \det(\mA')$
\item For a triangular matrix\footnote{This includes diagonal matrices.} $\mat{T},$ $\det(\mat{T}) = \prod_{i=1}^n T_{ii}$
\item Adding a multiple of a column/row to another one does not change $\det(\mA)$
\item Multiplication of a column/row with $\lambda$ scales $\det(\mA)$: $\det(\lambda \mA) = \lambda^n \mA$
\item Swapping two rows/columns changes the sign of $\det(\mA)$
\item[$\circ$] Using (5)-(8), Gaussian elimination (reaching a triangular matrix) enables to compute the determinant. 
\ece
\end{frame}

%%%%%%%%%%%%%%%%%%%%%%%%%%%%%%%%%%%%%%%%%%%%%%%%%%%%%%
\begin{frame}{Trace}

\plitemsep 0.1in

\bci
\item \defi The trace of a square matrix $\mA \in \real^{n \times n}$ is defined as
$$
\tr(\mA) \eqdef \sum_{i=1}^n a_{ii}
$$

\item $\tr(\mA + \mB) = \tr(\mA) + \tr(\mB)$
\item $\tr(\alpha \mA) = \alpha \tr(\mA)$
\item $\tr(\vec{I}_n) = n$
\eci
\end{frame}

%%%%%%%%%%%%%%%%%%%%%%%%%%%%%%%%%%%%%%%%%%%%%%%%%%%%%%
\begin{frame}{Invariant under Cyclic Permutations}

\plitemsep 0.1in

\bci
\item $\tr(\mA\mB) = \tr(\mB \mA)$ for $\mA \in \real^{n \times k}$ and $\mB \in \real^{k \times n}$
\item $\tr(\mA \mat{K} \mat{L}) = \tr(\mat{K} \mat{L} \mat{A}),$ for $\mat{A} \in \real^{a \times k},$ $\mat{K} \in \real^{k \times l},$ $\mat{L} \in \real^{l \times a}$
\item $\tr(\vec{x}\trans{\vec{y}}) = \tr(\trans{\vec{y}} \vec{x}) = \trans{\vec{y}} \vec{x} \in \real$

\medskip
\item A linear mapping $\Phi: V \mapsto V$, represented by a matrix $\mA$ and another matrix $\mB.$ 
\bci
\item $\mA$ and $\mB$ use different bases, where $\mB = \inv{\mat{S}} \mA \mat{S}$ 
$$
\tr(\mB) = \tr(\inv{\mat{S}} \mA \mat{S}) = \tr( \mA \mat{S}\inv{\mat{S}}) = \tr(\mA)
$$
\item \redf{Message.} While matrix representations of linear mappings are basis dependent, but their traces are not. 
\eci
\eci
\end{frame}

%%%%%%%%%%%%%%%%%%%%%%%%%%%%%%%%%%%%%%%%%%%%%%%%%%%%%%
\begin{frame}{Background: Characteristic Polynomial}

\plitemsep 0.1in

\bci
\item \defi For $\lambda \in \real$ and a matrix $\mA \in \real^{n \times n},$ the characteristic polynomial of $\mA$ is defined as:
\aleq{
p_{\mA}(\lambda) &\eqdef \det(\mA - \lambda \mI) \cr
& = c_0 + c_1 \lambda + c_2\lambda^2 + \cdots + c_{n-1} \lambda^{n-1} + (-1)^n \lambda^n,
}
where $c_0 = \det(\mA)$ and $c_{n-1} = (-1)^{n-1}\tr(\mA).$

\item \exam For $\mA = \begin{nmat}
4&2 \cr
1&3
\end{nmat},$ 
\aleq{
p_{\mA}(\lambda) = 
\begin{vmat}
4-\lambda & 2 \cr
1 & 3-\lambda
\end{vmat}
= (4-\lambda)(3-\lambda) -2 \cdot 1
}
\eci
\end{frame}

%%%%%%%%%%%%%%%%%%%%%%%%%%%%%%%%%%%%%%%%%%%%%%%%%%%%%%
\section{L4(2)}
\begin{frame}{Roadmap}

\plitemsep 0.1in

\bce[(1)] 
\item \grayf{Determinant and Trace}

\item \redf{Eigenvalues and Eigenvectors}

\item \grayf{Cholesky Decomposition

\item Eigendecomposition and Diagonalization

\item Singular Value Decomposition

\item Matrix Approximation

\item Matrix Phylogeny}

\ece
\end{frame}

%%%%%%%%%%%%%%%%%%%%%%%%%%%%%%%%%%%%%%%%%%%%%%%%%%%%%%
\begin{frame}{Eigenvalue and Eigenvector}

\plitemsep 0.1in

\bci
\item \defi Consider a square matrix $\mA \in \real^{n \times n}.$ Then, $\lambda \in \real$ is an eigenvalue of $\mA$ and $\vec{x} \in \real^n \setminus \{0 \}$ is the corresponding eigenvector of $\mA$ if 
\mycolorbox{
\vspace{-0.2cm}
$$
\mA \vec{x} = \lambda \vec{x}
$$
}
\item Equivalent statements
\bci
\item $\lambda$ is an eigenvalue.
\item $(\mA - \lambda \mI_n)\vec{x}=0$ can be solved non-trivially, i.e., $\vec{x} \neq \vec{0}.$
\item $\rk{\mA - \lambda\mI_n} < n.$
\item $\det(\mA - \lambda\mI_n) =0$ $\Longleftrightarrow$ The characteristic polynomial $p_{\mA}(\lambda)=0.$
\eci
\eci
\end{frame}

%%%%%%%%%%%%%%%%%%%%%%%%%%%%%%%%%%%%%%%%%%%%%%%%%%%%%%
\begin{frame}{Example}

\plitemsep 0.1in

\bci
\item For $\mA = (\begin{smallmatrix}
4&2 \cr
1&3
\end{smallmatrix}),$ 
$
p_{\mA}(\lambda) = 
\begin{vmat}
4-\lambda & 2 \cr
1 & 3-\lambda
\end{vmat}
= (4-\lambda)(3-\lambda) -2 \cdot 1 = \lambda^2 -7\lambda +10 
$
\item Eigenvalues $\lambda=2$ or $\lambda=5.$
\item Eigenvector $E_5$ for $\lambda =5$
\aleq{
\begin{nmat}
4-\lambda & 2 \cr
1 & 3-\lambda 
\end{nmat}
\vec{x} =0
\implies 
\begin{nmat}
-1 & 2 \cr
1 & -2 
\end{nmat}
\colvec{x_1 \\ x_2} =0
\implies
E_5 = \spn{\colvec{2 \\ 1}}
}

\item Eigenvector $E_2$ for $\lambda =2.$ Similarly, we get $E_2 = \spn{\colvec{1 \\ -1}}$

\item \redf{Message.} Eigenvectors are not unique. 

\eci
\end{frame}

%%%%%%%%%%%%%%%%%%%%%%%%%%%%%%%%%%%%%%%%%%%%%%%%%%%%%%
\begin{frame}{Properties (1)}

\plitemsep 0.1in

\bci
\item If $\vec{x}$ is an eigenvector of $\mA$, so are all vectors that are collinear\footnote{Two vectors are collinear if they point in the same or the opposite direction.}. 

\item $E_\lambda$: the set of all eigenvectors for eigenvalue $\lambda$, spanning a subspace of $\real^n.$ We call this \bluef{eigensapce} of $\mA$ for $\lambda.$

\item $E_\lambda$ is the solution space of $(\mA - \lambda \mI)\vec{x}=0,$ thus $E_\lambda = \ker(\mA -\lambda \mI)$

\item \redf{Geometric interpretation}
\bci
\item The eigenvector corresponding to a nonzero eigenvalue points in a direction \bluef{stretched} by the linear mapping.

\item The eigenvalue is the factor of stretching.
\eci

\item Identity matrix $\mI$: one eigvenvalue $\lambda =1$ and all vectors $\vec{x} \neq \vec{0}$ are eigenvectors. 
\eci
\end{frame}

%%%%%%%%%%%%%%%%%%%%%%%%%%%%%%%%%%%%%%%%%%%%%%%%%%%%%%
\begin{frame}{Properties (2)}

\plitemsep 0.2in

\bci
\item $\mA$ and $\trans{\mA}$ share the eigenvalues, but not necessarily eigenvectors. 

\item For two similar matrices $\mA,\mA'$ (i.e., $\mA' = \inv{\mat{S}}\mA \mat{S}$ for some $\mat{S}$), they possess the same eigenvalues. 

\bci
\item Meaning: A linear mapping $\Phi$ has eigenvalues that are \bluef{independent} of the choice of basis of its transformation matrix. 


\item Symmetric, positive definite matrices always have \bluef{positive}, \bluef{real} eigenvalues.
\eci
\eci

\mycolorbox{
\centering
determinant, trace, eigenvalues: all \bluef{invariant} under basis change 
}


\end{frame}

%%%%%%%%%%%%%%%%%%%%%%%%%%%%%%%%%%%%%%%%%%%%%%%%%%%%%%
\begin{frame}{Examples for Geometric Interpretation (1)}

%\plitemsep 0.1in
\plitemsep 0.03in
\myvartwocols{0.75}{0.6}{0.37}
{
%\medskip
\small
\bce
\item $\mA = 
(\begin{smallmatrix}
\frac{1}{2} & 0 \cr
0 & 2
\end{smallmatrix})$, $\det(\mA) = 1$ 

\bci
\item $\lambda_1 = \frac{1}{2}, \lambda_2 = 2$
\item eigenvectors: canonical basis vectors
\item area preserving, just vertical horizontal) stretching. 
\eci

\item $\mA = 
(\begin{smallmatrix}
1 & \frac{1}{2} \cr
0 & 1
\end{smallmatrix})$, $\det(\mA) = 1$
\bci
\item $\lambda_1 = \lambda_2 = 1$
\item eigenvectors: colinear over the horiontal line
\item area preserving, shearing
\eci

\item $\mA = 
\left(\begin{smallmatrix}
\cos(\frac{\pi}{6}) & -\sin(\frac{\pi}{6}) \cr
\sin(\frac{\pi}{6}) & \cos(\frac{\pi}{6})
\end{smallmatrix} \right)$, $\det(\mA) = 1$
\bci
\item Rotation by $\pi/6$ counter-clockwise
\item only complex eigenvalues (no eigenvectors)
\item area preserving
\eci

\ece
}
{
\vspace{-0.4cm}
\centering
\mypic{0.99}{L4_ev_ex1.png}

\vspace{0.1cm}

\mypic{0.99}{L4_ev_ex2.png}

\vspace{0.1cm}

\mypic{0.99}{L4_ev_ex3.png}
}

\end{frame}

%%%%%%%%%%%%%%%%%%%%%%%%%%%%%%%%%%%%%%%%%%%%%%%%%%%%%%
\begin{frame}{Examples for Geometric Interpretation (2)}

%\plitemsep 0.1in
\plitemsep 0.03in
\bigskip
\myvartwocols{0.5}{0.6}{0.37}
{
%\medskip
\small
\bce[4.]
\item $\mA = 
(\begin{smallmatrix}
1 & -1 \cr
-1 & 1
\end{smallmatrix})$, $\det(\mA) = 0$ 

\bci
\item $\lambda_1 = 0, \lambda_2 = 2$
\item Mapping that collapses a 2D onto 1D
\item area collapses
\eci

\bigskip
\item[5.] $\mA = 
(\begin{smallmatrix}
1 & \frac{1}{2} \cr
\frac{1}{2} & 1
\end{smallmatrix})$, $\det(\mA) = 3/4$
\bci
\item $\lambda_1 = 0.5, \lambda_2 = 1.5$
\item area scales by 75\%, shearing and stretching
\eci
\ece
}
{
\vspace{-0.5cm}
\centering
\mypic{0.99}{L4_ev_ex4.png}

\vspace{0.5cm}

\mypic{0.99}{L4_ev_ex5.png}
}

\end{frame}


%%%%%%%%%%%%%%%%%%%%%%%%%%%%%%%%%%%%%%%%%%%%%%%%%%%%%%
\begin{frame}{Properties (3)}

\plitemsep 0.05in

\bci
\item For $\mA \in \real^{n \times n},$ $n$ distinct eigenvalues $\implies$ eigenvectors are linearly independent, which form a basis of $\real^n.$
 \bci
 \item Converse is not true.
 \item \bluef{Example of $n$ linearly independent eigenvectors for less than $n$ eigenvalues???} 
 \eci

\item \redf{Determinant}. For (possibly repeated) eigenvalues $\lambda_i$ of $\mA \in \real^{n \times n},$ 
\mycolorbox{
\centering
$
\det(\mA) = \prod_{i=1}^n \lambda_i
$
}
\item \redf{Trace}. For (possibly repeated) eigenvalues $\lambda_i$ of $\mA \in \real^{n \times n},$ 
\mycolorbox{
\centering
$
\tr(\mA) = \sum_{i=1}^n \lambda_i
$
}
\item \msg $\det(\mA)$ is the \bluef{area scaling} and $\tr(\mA)$ is the \bluef{circumference scaling}

\eci
\end{frame}

% %%%%%%%%%%%%%%%%%%%%%%%%%%%%%%%%%%%%%%%%%%%%%%%%%%%%%%
% \begin{frame}{Properties (4): Spectral Theorem}

% \plitemsep 0.1in

% \bci
% \item \thm If $\mA \in \real^{n \times n}$ is symmetric, 
% \bci
% \item the eigenvalues are all real 
% \item the eigenvectors to different eigenvalues are perpendicular. 
% \item there exists an orthogonal eigenbasis
% \eci


% \item \exam $\mA = 
% \left(\begin{smallmatrix}
% 3&2&2 \cr
% 2&3&2 \cr
% 2&2&3 
% \end{smallmatrix}\right)$. $p_{\mA}(\lambda) = -(\lambda-1)^2(\lambda-7),$ thus $\lambda_1=1,\lambda_2=7$
% \aleq{
% E_1 = \spn{\smallcolvec{-1\\1\\0}, \smallcolvec{-1\\0\\1}}, \quad E_7 = \spn{\smallcolvec{1\\1\\1}}
% }
% \vspace{-0.3cm}
% \bci
% \item $\trans{(1 1 1)}$ is perpendicular to $\trans{(-1 1 0)}$ and $\trans{(-1 0 1)}$
% \item $\smallcolvec{-1\\1\\0}$ and $\smallcolvec{-1/2\\ -1/2\\ 1}$ (for $\lambda=1$) and $\smallcolvec{1\\1\\1}$ (for $\lambda=7)$ are the orthogonal basis in $\real^3.$
% \eci
% \eci
% \end{frame}

% %%%%%%%%%%%%%%%%%%%%%%%%%%%%%%%%%%%%%%%%%%%%%%%%%%%%%%
% \begin{frame}{Properties (4): Determinant and Trace}

% \plitemsep 0.1in

% \bci
% \item \redf{Determinant}. For (possibly repeated) eigenvalues $\lambda_i$ of $\mA \in \real^{n \times n},$ 
% $$
% \det(\mA) = \prod_{i=1}^n \lambda_i.
% $$

% \item \redf{Trace}. For (possibly repeated) eigenvalues $\lambda_i$ of $\mA \in \real^{n \times n},$ 
% $$
% \tr(\mA) = \sum_{i=1}^n \lambda_i.
% $$

% \item \msg $\det(\mA)$ is the \bluef{area scaling} and $\tr(\mA)$ is the \bluef{circumference scaling}
% \eci
% \end{frame}


%%%%%%%%%%%%%%%%%%%%%%%%%%%%%%%%%%%%%%%%%%%%%%%%%%%%%%
\section{L4(3)}
\begin{frame}{Roadmap}

\plitemsep 0.1in

\bce[(1)] 
\item \grayf{Determinant and Trace}

\item \grayf{Eigenvalues and Eigenvectors}

\item \redf{Cholesky Decomposition}

\item \grayf{Eigendecomposition and Diagonalization

\item Singular Value Decomposition

\item Matrix Approximation

\item Matrix Phylogeny}

\ece
\end{frame}

%%%%%%%%%%%%%%%%%%%%%%%%%%%%%%%%%%%%%%%%%%%%%%%%%%%%%%
\begin{frame}{LU Decomposition}

\plitemsep 0.05in

% \item Triangular matrix

\mypic{0.35}{L4_UTM_LTM.png}

\vspace{-0.9cm}
\hfill{\scriptsize Source: \url{http://mathonline.wikidot.com/}}
    \vspace{-0.3cm}
\bci
\item The Gaussian elimination is the processing of reaching an upper triangular matrix
\item Gaussian elimination: multiplying the matrices corresponding to two elementary operations 
((i) row multiplication by $a$ and (ii) adding two rows downward)
\item The above elementary operations are the low triangular matrices (LTM), and their inverses and their product are all LTMs.

\item $(\mE_k \mE_{k-1}\cdot \mE_1) \mA = \vec{U}$ $\implies$ $\mA = \underbrace{(\inv{\mE_1}\cdots \inv{\mE_{k-1}} \inv{\mE_{k}} )}_{\vec{L}}\vec{U}$
\eci
\end{frame}

%%%%%%%%%%%%%%%%%%%%%%%%%%%%%%%%%%%%%%%%%%%%%%%%%%%%%%
\begin{frame}{Cholesky Decomposition}

\plitemsep 0.1in

\bci

\item A real number: decomposition of two identical numbers, e.g., $9=3\times 3$

\item \thm For a symmetric, positive definite matrix $\mA,$ $\mA = \mat{L}\trans{\mat{L}},$ where
\bci
\item $\mat{L}$ is a lower-triangular matrix with positive diagonals
\item Such a $\mat{L}$ is unique, called \bluef{Cholesky factor} of $\mA.$
\eci

\item Applications
\bce[(a)]
\item factorization of covariance matrix of a multivariate Gaussian variable
\item linear transformation of random variables
\item fast determinant computation: $\det(\mA) = \det(\mat{L})\det(\trans{\mat{L}}) = \det(\mat{L})^2,$ where $\det(\mat{L}) = \prod_{i} l_{ii}.$ Thus, $\det(\mA) = \prod_{i} l_{ii}^2.$
\ece
\eci

\end{frame}



%%%%%%%%%%%%%%%%%%%%%%%%%%%%%%%%%%%%%%%%%%%%%%%%%%%%%%
\section{L4(4)}
\begin{frame}{Roadmap}

\plitemsep 0.1in

\bce[(1)] 
\item \grayf{Determinant and Trace}

\item \grayf{Eigenvalues and Eigenvectors}

\item \grayf{Cholesky Decomposition}

\item \redf{Eigendecomposition and Diagonalization}

\item \grayf{Singular Value Decomposition

\item Matrix Approximation

\item Matrix Phylogeny}

\ece
\end{frame}

%%%%%%%%%%%%%%%%%%%%%%%%%%%%%%%%%%%%%%%%%%%%%%%%%%%%%%
\begin{frame}{Diagonal Matrix and Diagonalization}

\plitemsep 0.1in

\bci

\item \redf{Diagonal matrix.} zero on all off-diagonal elements, $\mat{D} = 
\begin{nmat}
d_1 & \cdots & 0 \cr
\vdots &  & \vdots \cr
0 & \cdots & d_n \cr
\end{nmat}$
\aleq{
\mD^k = \begin{nmat}
d_1^k & \cdots & 0 \cr
\vdots &  & \vdots \cr
0 & \cdots & d_n^k \cr
\end{nmat}, \quad
\inv{\mD} = 
\begin{nmat}
1/d_1 & \cdots & 0 \cr
\vdots &  & \vdots \cr
0 & \cdots & 1/d_n \cr
\end{nmat},\quad
\det(\mD) = d_1d_2\cdots d_n
}

%\item \question How can we transform $\mA$ into a diagonal form?

\item \defi $\mA \in \realnn$ is \bluef{diagonalizable} if it is similar to a diagonal matrix $\mD,$ i.e., $\exists$ an \bluef{invertible} $\mP \in \realnn,$ such that $\mD = \inv{\mP} \mA \mP.$

\item \defi $\mA \in \realnn$ is \bluef{orthogonally diagonalizable} if it is similar to a diagonal matrix $\mD,$ i.e., $\exists$ an \bluef{orthogonal}  $\mP \in \realnn,$ such that $\mD = \inv{\mP} \mA \mP = \trans{\mP} \mA \mP.$


\eci

\end{frame}

%%%%%%%%%%%%%%%%%%%%%%%%%%%%%%%%%%%%%%%%%%%%%%%%%%%%%%
\begin{frame}{Power of Diagonalization}

\plitemsep 0.2in

\bci


\item $\mA^k = \mP \mD^k \inv{\mP}$

\item $\det(\mA) = \det(\mP)\det(\mD)\det(\inv{\mP}) = \det(\mD) = \prod_i d_{ii}$ 

\item Many other things ...

\item \question \bluef{Under what condition is $\mA$ diagonalizable (or orthogonally diagonalizable) and how can we find $\mP$ (thus $\mD$)?}

\eci

\end{frame}





%%%%%%%%%%%%%%%%%%%%%%%%%%%%%%%%%%%%%%%%%%%%%%%%%%%%%%
\begin{frame}{Diagonalizablity, Algebraic/Geometric Multiplicity}

\plitemsep 0.1in

\bci

\item \defi For a matrix $\mA \in realnn$ with an eigenvalue $\lambda_i,$ 
\bci
\item the \bluef{algebraic multiplicity} $\alpha_i$ of $\lambda_i$ is the number of times the root appears in the characteristic polynomial. 
\item the \bluef{geometric multiplicity} $\zeta_i$ of $\lambda_i$ is the number of linearly independent eigenvectors associated with $\lambda_i$ (i.e., the dimension of the eigenspace spanned by the eigenvectors of $\lambda_i$) 

\eci

\item \exam The matrix $\mA = \begin{nmat}
2 &1 \cr
0 &2
\end{nmat}$ has two repeated eigenvalues $\lambda_1 = \lambda_2 = 2,$ thus $\alpha_1 = 2.$ However, it has only one distinct unit eigenvector $\vec{x} = \colvec{1 \\0},$ thus $\zeta_1 =1.$

\item \thm $\mA \in \realnn$ is \bluef{diagonalizable} $\Longleftrightarrow$ $\sum_{i} \alpha_i = \sum_{i} \zeta_i = n.$

\eci

\end{frame}

%%%%%%%%%%%%%%%%%%%%%%%%%%%%%%%%%%%%%%%%%%%%%%%%%%%%%%
\begin{frame}{Orthogonally Diagonaliable and Symmetric Matrix}

\plitemsep 0.1in

\vspace{0.2cm}
\mycolorbox{
\thm $\mA \in \realnn$ is \bluef{orthogonally diagonalizable} $\Longleftrightarrow$ $\mA$ is symmetric.
}
\vspace{-0.3cm}
\bci

\item \question. How to find $\mP$ (thus $\mD$)?

\item \redf{Spectral Theorem.} If $\mA \in \real^{n \times n}$ is symmetric, 
\bce[(a)]
\item the eigenvalues are all real 
\item the eigenvectors to different eigenvalues are perpendicular. 
\item there exists an orthogonal eigenbasis
\ece

\item For (c), from each set of eigenvectors, say $\{\vec{x}_1, \ldots, \vec{x}_k \}$ associated with a particular eigenvalue, say $\lambda_j$, we can construct another set of eigenvectors $\{\vec{x}'_1, \ldots, \vec{x}'_k \}$ that are orthonormal, using the Gram-Schmidt process.  

\item Then, all eigenvectors can form an orthornormal basis. 
\eci

\end{frame}

%%%%%%%%%%%%%%%%%%%%%%%%%%%%%%%%%%%%%%%%%%%%%%%%%%%%%%
\begin{frame}{Example}

\plitemsep 0.2in

\bci

\item \exam $\mA = 
\left(\begin{smallmatrix}
3&2&2 \cr
2&3&2 \cr
2&2&3 
\end{smallmatrix}\right)$. $p_{\mA}(\lambda) = -(\lambda-1)^2(\lambda-7),$ thus $\lambda_1=1,\lambda_2=7$
\aleq{
E_1 = \spn{\smallcolvec{-1\\1\\0}, \smallcolvec{-1\\0\\1}}, \quad E_7 = \spn{\smallcolvec{1\\1\\1}}
}
\vspace{-0.3cm}
\bci
\item $\trans{(1 1 1)}$ is perpendicular to $\trans{(-1 1 0)}$ and $\trans{(-1 0 1)}$
\item $\smallcolvec{-1\\1\\0}$ and $\smallcolvec{-1/2\\ -1/2\\ 1}$ (for $\lambda=1$) and $\smallcolvec{1\\1\\1}$ (for $\lambda=7)$ are the orthogonal basis in $\real^3.$

\item After normalization, we can make the orthonormal basis. 
\eci
\eci
\end{frame}


%%%%%%%%%%%%%%%%%%%%%%%%%%%%%%%%%%%%%%%%%%%%%%%%%%%%%%
\begin{frame}{Eigendecomposition}

\plitemsep 0.2in

\bci
\item \thm The following is equivalent.
\bci
\item[(a)] A square matrix $\mA \in \realnn$ can be factorized into $\mA = \mP \mD \inv{\mP},$ where $\mP \in \realnn$ and $\mD$ is the diagonal matrix whose diagonal entries are eigenvalues of $\mA.$

\item[(b)] The eigenvectors of $\mA$ form a basis of $\realn$ (i.e., 
The $n$ eigenvectors of $\mA$ are linearly independent) 

\eci

\item The above implies the columns of $\mP$ are the $n$ eigenvectors of $\mA$ (because $\mA \mP = \mP \mD$)

\item $\mP$ is an orthogonal matrix, so $\trans{\mP} = \inv{\mP}$

\item $\mA$ is symmetric, then (b) holds (Spectral Theorem).


\eci

\end{frame}

%%%%%%%%%%%%%%%%%%%%%%%%%%%%%%%%%%%%%%%%%%%%%%%%%%%%%%
\begin{frame}{Example of Orthogonal Diagonalization (1)}

\plitemsep 0.05in

\bci
\item Eigendecomposition for $\mA = 
\begin{nmat}
2&1 \cr
1&2 
\end{nmat}$

\item Eigenvalues: $\lambda_1 = 1, \lambda_2=3$ 

\item (normalized) eigenvectors: $\vec{p}_1 = \frac{1}{\sqrt{2}} \colvec{1 \\ -1},$ $\vec{p}_2 = \frac{1}{\sqrt{2}} \colvec{1 \\ 1}.$ 

\item $\vec{p}_1$ and $\vec{p}_2$ linearly independent, so $A$ is diagonalizable.  

\item $\mP = \rowvec{\vec{p}_1 & \vec{p}_2} = \frac{1}{\sqrt{2}} \begin{nmat}
1 & 1\cr
-1 & 1
\end{nmat} $

\item $\mD = \inv{\mP} \mA \mP = \begin{nmat}
1 & 0\cr
0 & 3
\end{nmat}$. Finally, we get $\mA = \mP \mD \inv{\mP}$
\eci

\end{frame}

%%%%%%%%%%%%%%%%%%%%%%%%%%%%%%%%%%%%%%%%%%%%%%%%%%%%%%
\begin{frame}{Example of Orthogonal Diagonalization (2)}


\mytwocols{0.7}
{
\plitemsep 0.05in
\small
\bci
\item $\mA = 
\begin{nmat}
1&2&2 \cr
2&1&2 \cr
2&2&1 
\end{nmat}$

\item Eigenvalues: $\lambda_1 = -1, \lambda_2=5$ ($\alpha_1 =2, \alpha_2=1$)

\item $E_{-1} = \spn{\colvec{-1\\1\\0}, \colvec{-1\\0\\1}}$
$\xrightarrow{\text{Gram-Schmidt}}$ $\spn{\frac{1}{\sqrt{2}}\colvec{-1\\ 1 \\0}, \frac{1}{\sqrt{6}}\colvec{-1\\ 1\\ 2}}$

\eci
}
{
\plitemsep 0.05in
\bci

\item $E_{5} = \spn{\frac{1}{\sqrt{3}}\colvec{1 \\1\\1 }}$

\item $\mP = \begin{nmat}
-1/\sqrt{2} & -1/\sqrt{6}& 1/\sqrt{3} \cr
1/\sqrt{2} & -1/\sqrt{6}& 1/\sqrt{3} \cr
0 & 2/\sqrt{6}& 1/\sqrt{3} 
\end{nmat}$

\item $\mD = \trans{\mP} \mA \mP = \begin{nmat}
-1 & 0 & 0\cr
0 & -1 & 0\cr
0 & 0 & 5
\end{nmat}$
\eci

}

\end{frame}

%%%%%%%%%%%%%%%%%%%%%%%%%%%%%%%%%%%%%%%%%%%%%%%%%%%%%%
\begin{frame}{Eigendecomposition: Geometric Interpretation}

\mypic{0.5}{L4_eigendecomposition.png}

\vspace{-0.4cm}
\question Can we generalize this beautiful result to a general matrix \bluef{$\mA \in \realmn$}?
\end{frame}


%%%%%%%%%%%%%%%%%%%%%%%%%%%%%%%%%%%%%%%%%%%%%%%%%%%%%%
\section{L4(5)}
\begin{frame}{Roadmap}

\plitemsep 0.1in

\bce[(1)] 
\item \grayf{Determinant and Trace}

\item \grayf{Eigenvalues and Eigenvectors}

\item \grayf{Cholesky Decomposition}

\item \grayf{Eigendecomposition and Diagonalization}

\item \redf{Singular Value Decomposition

\item Matrix Approximation}

\item \grayf{Matrix Phylogeny}

\ece
\end{frame}

%%%%%%%%%%%%%%%%%%%%%%%%%%%%%%%%%%%%%%%%%%%%%%%%%%%%%%
\begin{frame}{Storyline}

\plitemsep 0.1in

\bci 
\item Eigendecomposition (also called EVD: EigenValue Decomposition): (Orthogoanl) Diagonalization for symmetric matrices $\mA \in \realnn.$


\item Extensions: Singular Value Decomposition (SVD)
\bce
\item First extension: diagonalization for non-symmetric, but still square matrices $\mA \in \realnn$
\item Second extension: diagonalization for non-symmeric, and non-square matrices $\mA \in \realmn$
\ece

\medskip
\item \redf{Background.} For $\mA \in \real^{m \times n},$ a matrix $\mat{S} \eqdef \trans{\mA} \mA \in \real^{n \times n}$ is always symmetric, positive semidefinite. 
\bci
\item Symmetric, because $\trans{\mat{S}} = \trans{(\trans{\mA} \mA)} = \trans{\mA} \mA = \mat{S}.$
\item Positive semidefinite, because $\trans{\vec{x}} \mat{S} \vec{x} = \trans{\vec{x}} \trans{\mA}\mA \vec{x} = \trans{(\mA \vec{x})} (\mA \vec{x}) \ge 0.$

\item If $\rk{\mA} = n$, then symmetric and positive definite. 
\eci

\eci
\end{frame}

%%%%%%%%%%%%%%%%%%%%%%%%%%%%%%%%%%%%%%%%%%%%%%%%%%%%%%
\begin{frame}{Singular Value Decomposition}

\plitemsep 0.1in

\bci 
\item \thm $\mA \in \realmn$ with rank $r \in [0,\min(m,n)].$ The SVD of $\mA$ is a decomposition of the form 
\mysmalltwocols{0.1}
{
$$
\mA = \mU \Sigma \trans{\mV},
$$
}
{
\vspace{-0.3cm}
\mypic{0.7}{L4_SVD_matrix.png}
}
with an orthogonal matrix $\mU=\rowvec{\vec{u}_1 &\cdots &\vec{u}_m } \in \realmm$ and an orthogonal matrix $\mV=\rowvec{\vec{v}_1 &\cdots &\vec{v}_n} \in \realnn.$ Moreoever, $\Sigma$ s an $m \times n$ matrix with $\Sigma_{ii} = \sigma_i \ge 0$ and $\Sigma_{ij} =0, \ i\neq j,$ which is uniquely determined for $\mA.$

\medskip
\item Note
\bci
\item The diagonal entries $\sigma_i, \ i=1,\ldots, r$ are called  \bluef{singular values.}  
\item $\vec{u}_i$ and $\vec{v}_j$ are called \bluef{left} and \bluef{right singular vectors}, respectively. 
\eci
\eci
\end{frame}

%%%%%%%%%%%%%%%%%%%%%%%%%%%%%%%%%%%%%%%%%%%%%%%%%%%%%%
\begin{frame}{SVD: How It Works (for $\mA \in \realnn$)}

\mytwocols{0.75}
{
\plitemsep 0.03in
\small
\bci 
\item $\mA \in \realnn$ with rank $r \le n.$ Then, $\trans{\mA}\mA$ is symmetric. 

\item Orthogonal diagonalization of $\trans{\mA}\mA$: 
$$\trans{\mA}\mA = \mV \mD \trans{\mV}.$$
\item $\mD = \left(\begin{smallmatrix}
\lambda_1 & & \cr
 &\ddots & \cr
&& \lambda_n
\end{smallmatrix}\right)$ and an orthogonal matrix $\mV = \rowvec{\vec{v}_1 \cdots \vec{v}_n},$
where $\lambda_1 \ge \cdots \ge \lambda_r \ge \lambda_{r+1} = \cdots \lambda_n =0$ are the eigenvalues of $\trans{\mA}\mA$ and $\{\vec{v}_i\}$ are orthonormal.

\item \bluef{All $\lambda_i$ are positive}
\vspace{-0.2cm}
\aleq{
\forall \vec{x} \in \realn, \norm{\mA \vec{x}}^2 = \trans{\mA \vec{x}} \mA \vec{x} = \trans{\vec{x}} 
\trans{\mA} \mA \vec{x} = \lambda_i \norm{\vec{x}}^2
}
\eci
}
{
\plitemsep 0.03in
\small
\bci
\item \bluef{\rk{\mA} =\rk{\trans{\mA}\mA} = \rk{D} =r}

\item Choose $\mU' = \rowvec{\vec{u}_1 & \cdots \vec{u}_r},$ where 
$$\vec{u}_i= \frac{\mA \vec{v}_i}{\sqrt{\lambda_i}}, \ 1\le i \le r.$$

\item We can construct $\{\vec{u}_i\}, \ i = r+1, \cdots, n,$ so that $\mU = \rowvec{\vec{u}_1 & \cdots \vec{u}_n}$ is an orthonormal basis of $\realn.$

\item Define $\Sigma = \left(\begin{smallmatrix}
\sqrt{\lambda_1} & & \cr
 &\ddots & \cr
&& \sqrt{\lambda_n}
\end{smallmatrix}\right)$

\item Then, we can check that $\mU \Sigma = \mA \mV.$

\item Similar arguments for a general $\mA \realmn$ (see pp. 104) 
\eci
}

\end{frame}

%%%%%%%%%%%%%%%%%%%%%%%%%%%%%%%%%%%%%%%%%%%%%%%%%%%%%%
\begin{frame}{Example}

\mytwocols{0.72}
{
\plitemsep 0.1in
\small
\bci 
\item $\mA = \begin{nmat}
1 & 0 & 1\cr
-2 & 1 & 0
\end{nmat}$

\item $\trans{\mA}\mA = \begin{nmat}
5 & -2 & 1\cr
-2 & 1 & 0\cr
1 & 0 & 1
\end{nmat}=\mV \mD \trans{\mV},$

\hspace{-0.5cm}$\mD = \begin{nmat}
6 & 0 & 0\cr
0 & 1 & 0\cr
0 & 0 & 0
\end{nmat},
\mV = \begin{nmat}
\frac{5}{\sqrt{30}} & \frac{-2}{\sqrt{30}} &\frac{1}{\sqrt{30}}\cr
0 & \frac{1}{\sqrt{5}} &\frac{2}{\sqrt{5}}\cr
\frac{-1}{\sqrt{6}} & \frac{-2}{\sqrt{6}} & \frac{1}{\sqrt{6}}
\end{nmat}
$
\item $\rk{\mA}=2$ because we have two singular values $\sigma_1 = \sqrt{6}$ and $\sigma_2=1$

\item $\Sigma = \begin{nmat}
\sqrt{6} & 0 & 0\cr
0 & 1 & 0
\end{nmat}$
\eci
}
{
\plitemsep 0.1in
\small
\bci
\item $\vec{u}_1 = \mA \vec{v}_1/\sigma_1 = \colvec{\frac{1}{\sqrt{5}} \\ \frac{-2}{\sqrt{5}}}$

\item $\vec{u}_2 = \mA \vec{v}_2/\sigma_2 = \colvec{\frac{2}{\sqrt{5}} \\ \frac{1}{\sqrt{5}}}$

\item $\mU = \rowvec{\vec{u}_1 & \vec{u}_2} = \frac{1}{\sqrt{5}}
\begin{nmat}
1&2 \cr
-2 & 1
\end{nmat}$

\item Then, we can see that $\mA = \mU \Sigma \trans{V}.$
\eci
}

\end{frame}

%%%%%%%%%%%%%%%%%%%%%%%%%%%%%%%%%%%%%%%%%%%%%%%%%%%%%%
\begin{frame}{EVD ($\mA = \mP \mD \inv{\mP}$) vs. SVD ($\mA = \mU \msig \trans{\mV}$)}

\plitemsep 0.07in

\bci 
\item SVD: \bluef{always} exists, EVD: \bluef{square} matrix and exists if we can find \bluef{a basis of eigenvectors} (such as symmetric matrices)

\item $\mP$ in EVD is \bluef{not necessarily orthogonal} (only true for symmetric $\mA$), but $\mU$ and $\mV$ are \bluef{orthogonal} (so representing rotations)

\item Both EVD and SVD: (i) basis change in the domain, (ii) independent scaling of each new basis vector and mapping from domain to codomain, (iii) basis change in the codomain. The difference: for SVD, \bluef{different vector spaces} of domain and codomain. 

\item SVD and EVD are closely related through their projections
\bci
\item The left-singular (resp. right-singular) vectors of $\mA$ are eigenvectors of $\mA \trans{\mA}$ (resp. $\trans{\mA} \mA$)
\item The singular values of $\mA$ are the square roots of eigenvalues of $\mA \trans{\mA}$ and $\trans{\mA} \mA$
\item When $\mA$ is symmetric, EVD = SVD (from spectral theorem)
\eci
\eci
\end{frame}

%%%%%%%%%%%%%%%%%%%%%%%%%%%%%%%%%%%%%%%%%%%%%%%%%%%%%%
\begin{frame}{Different Forms of SVD}

\plitemsep 0.1in

\bci 

\item When $\rk{\mA} = r,$ we can construct SVD as the following with only non-zero diagonal entries in $\msig$:
$$
\mA = \overbrace{\mU}^{m \times r} \overbrace{\msig }^{r \times r} \overbrace{\trans{\mV}}^{r \times n}
$$

\item We can even truncate the decomposed matrices, which can be an approximation of $\mA$: for $k < r$
$$
\mA \approx \overbrace{\mU}^{m \times k} \overbrace{\msig }^{k \times k} \overbrace{\trans{\mV}}^{k \times n}
$$

\medskip
We will cover this in the next slides.
\eci
\end{frame}


%%%%%%%%%%%%%%%%%%%%%%%%%%%%%%%%%%%%%%%%%%%%%%%%%%%%%%
\section{L4(6)}
\begin{frame}{Matrix Approximation via SVD}

\plitemsep 0.1in
\mypic{0.5}{L4_matrix_approx.png}
\vspace{-0.6cm}
\bci 
\item $\mA = \sum_{i=1}^r \sigma_i \overbrace{\vec{u}_i \trans{\vec{v}_i}}^{\mA_i}$, where $\mA_i$ is the outer product\footnote{If $\vu$ and $\vv$ are both nonzero, then the outer product matrix $\vu \trans{vv}$ always has matrix rank 1. Indeed, the columns of the outer product are all proportional to the first column.} of $\vu_i$ and $\vv_i$

\item Rank $k$-approximation: $\hat{\mA}(k) = \sum_{i=1}^k \sigma_i \mA_i,$ $k < r$ 
\eci
\end{frame}

%%%%%%%%%%%%%%%%%%%%%%%%%%%%%%%%%%%%%%%%%%%%%%%%%%%%%%
\section{L4(6)}
\begin{frame}{How Close $\hat{\mA}(k)$ is to $\mA$?}

\plitemsep 0.07in
\bci 
\item \defi \bluef{Spectral Norm of a Matrix.} For $\mA \in \realmn,$ 
$\norm{\mA}_2 \eqdef \max_{\vx} \dfrac{\norm{\mA \vx}_2}{\norm{\vx}_2}$
\bci
\item As a concept of length of $\mA$, it measures how long any vector $\vx$ can at most become, when multiplied by $\mA$
\eci
\item \thm \bluef{Eckart-Young.} For $\mA \in \realmn$ of rank $r$ and $\mB \in \realmn$ of rank $k,$ for any $k \le r$, we have:
$$
\hat{\mA}(k) = \arg \min_{\rk{\mB} = k} \norm{\mA - \mB}_2, \quad \text{and} \quad \norm{\mA - \hat{\mA}(k)}_2 = \sigma_{k+1}
$$
\bci
\item Quantifies how much error is introduced by the SVD-based approximation
\item $\hat{\mA}(k)$ is optimal in the sense that such SVD-based approximation is the best 
one among all rank-$k$ approximations. 

\item In other words, it is a projection of the full-rank matrix $\mA$ onto a lower-dimensional space of rank-at-most-$k$ matrices.
\eci

\eci
\end{frame}

%%%%%%%%%%%%%%%%%%%%%%%%%%%%%%%%%%%%%%%%%%%%%%%%%%%%%%
\section{L4(7)}
\begin{frame}{Roadmap}

\plitemsep 0.1in

\bce[(1)] 
\item \grayf{Determinant and Trace}

\item \grayf{Eigenvalues and Eigenvectors}

\item \grayf{Cholesky Decomposition}

\item \grayf{Eigendecomposition and Diagonalization}

\item \grayf{Singular Value Decomposition}

\item \grayf{Matrix Approximation}

\item \redf{Matrix Phylogeny}

\ece
\end{frame}

%%%%%%%%%%%%%%%%%%%%%%%%%%%%%%%%%%%%%%%%%%%%%%%%%%%%%%
\begin{frame}{Phylogenetic Tree of Matrices}

\mypic{0.5}{L4_matrix_tree.png}
\end{frame}



%%%%%%%%%%%%%%%%%%%%%%%%%%%%%%%%%%%%%%%%%%%%%%%%%%%%%%
\begin{frame}{}
\vspace{2cm}
\LARGE Questions?


\end{frame}

\begin{frame}{Review Questions}
% \tableofcontents
%\plitemsep 0.1in
\bce[1)]
\item 

\ece
\end{frame}


\end{document}
