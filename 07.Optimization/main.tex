%\pdfminorversion=4
\documentclass[handout,fleqn,aspectratio=169]{beamer}

% when making printed slides
\usepackage{pgfpages}
\pgfpagesuselayout{resize to}[a4paper,landscape,border shrink=5mm]

\usepackage[english]{babel}
\usepackage{tikz}
\usepackage{courier}
\usepackage{array}
\usepackage{bold-extra}
%\usepackage{minted}
\usepackage[thicklines]{cancel}
\usepackage{fancyvrb}
\usepackage{kotex}
\usepackage{paralist}
\usepackage{collectbox}
\usepackage{bm}

\usepackage{mathrsfs}
\usepackage[reqno,disallowspaces]{mathtools}  % imports amsmath
\usepackage{amsfonts} %for Y&Y BSR AMS fonts
\usepackage{amssymb}
\usepackage{amscd}
%\usepackage{tikz,lipsum,lmodern}
\usepackage[most]{tcolorbox}
\usepackage{verbatim}
\mode<presentation>
{
  \usetheme{default}
  \usecolortheme{default}
  \usefonttheme{default}
  \setbeamertemplate{navigation symbols}{}
  \setbeamertemplate{caption}[numbered]
  \setbeamertemplate{footline}[frame number]  % or "page number"
  \setbeamercolor{frametitle}{fg=yellow}
  \setbeamercolor{footline}{fg=black}
} 

\setbeamercolor{block body alerted}{bg=alerted text.fg!10}
\setbeamercolor{block title alerted}{bg=alerted text.fg!20}
\setbeamercolor{block body}{bg=structure!10}
\setbeamercolor{block title}{bg=structure!20}
\setbeamercolor{block body example}{bg=green!10}
\setbeamercolor{block title example}{bg=green!20}
\setbeamertemplate{blocks}[rounded][shadow]

\xdefinecolor{dianablue}{rgb}{0.18,0.24,0.31}
\xdefinecolor{darkblue}{rgb}{0.1,0.1,0.7}
\xdefinecolor{darkgreen}{rgb}{0,0.5,0}
\xdefinecolor{darkgrey}{rgb}{0.35,0.35,0.35}
\xdefinecolor{darkorange}{rgb}{0.8,0.5,0}
\xdefinecolor{darkred}{rgb}{0.7,0,0}
\definecolor{darkgreen}{rgb}{0,0.6,0}
\definecolor{mauve}{rgb}{0.58,0,0.82}

\usetikzlibrary{shapes.callouts}

\makeatletter
\setbeamertemplate{footline}
{
  \leavevmode%
  \hbox{%
  \begin{beamercolorbox}[wd=.333333\paperwidth,ht=2.25ex,dp=1ex,center]{author in head/foot}%
    \usebeamerfont{author in head/foot}\insertsection
  \end{beamercolorbox}%
  \begin{beamercolorbox}[wd=.333333\paperwidth,ht=2.25ex,dp=1ex,center]{title in head/foot}%
    \usebeamerfont{title in head/foot}\insertsubsection
  \end{beamercolorbox}%
  \begin{beamercolorbox}[wd=.333333\paperwidth,ht=2.25ex,dp=1ex,right]{date in head/foot}%
    \usebeamerfont{date in head/foot}
    \insertshortdate{}\hspace*{2em}
    \insertframenumber{} / \inserttotalframenumber\hspace*{2ex} 
  \end{beamercolorbox}}%

  \vskip0pt%
}
\makeatother

\title[]{Lecture 7: Optimization}
\author{Yi, Yung (이융)}
\institute{Mathematics for Machine Learning\\ \url{https://yung-web.github.io/home/courses/mathml.html}
\\KAIST EE}
\date{\today}

%%%%%%%%%%%% real, integer notation
\newcommand{\real}{{\mathbb R}}
\newcommand{\realn}{{\mathbb R}^{n}}
\newcommand{\realm}{{\mathbb R}^{m}}
\newcommand{\realD}{{\mathbb R}^{D}}
\newcommand{\realM}{{\mathbb R}^{M}}
\newcommand{\realN}{{\mathbb R}^{N}}
\newcommand{\realnn}{{\mathbb R}^{n \times n}}
\newcommand{\realmm}{{\mathbb R}^{m \times m}}
\newcommand{\realmn}{{\mathbb R}^{m \times n}}
\newcommand{\realnm}{{\mathbb R}^{n \times m}}
\newcommand{\realDM}{{\mathbb R}^{D \times M}}
\newcommand{\realMD}{{\mathbb R}^{M \times D}}
\newcommand{\complex}{{\mathbb C}}
\newcommand{\integer}{{\mathbb Z}}
\newcommand{\natu}{{\mathbb N}}


%%% set, vector, matrix
\newcommand{\set}[1]{\ensuremath{\mathcal #1}}
\newcommand{\sets}[1]{\ensuremath{\{#1 \}}}
\renewcommand{\vec}[1]{\bm{#1}}
\newcommand{\mat}[1]{\bm{#1}}

%%%% vector
\def\vx{\vec{x}}
\def\vy{\vec{y}}
\def\vz{\vec{z}}
\def\vf{\vec{f}}
\def\ve{\vec{e}}
\def\vr{\vec{r}}
\def\vb{\vec{b}}
\def\vc{\vec{c}}
\def\vd{\vec{d}}
\def\vm{\vec{m}}
\def\vu{\vec{u}}
\def\vv{\vec{v}}
\def\vw{\vec{w}}
\def\vX{\vec{X}}
\def\vY{\vec{Y}}
\def\vZ{\vec{Z}}
\def\vth{\vec{\theta}}
\def\vmu{\vec{\mu}}
\def\vnu{\vec{\nu}}
\def\vlam{\vec{\lambda}}
\def\vep{\vec{\epsilon}}
\def\vpi{\vec{\pi}}
\def\vphi{\vec{\phi}}
\def\vxi{\vec{\xi}}
\def\valpha{\vec{\alpha}}
\def\vgamma{\vec{\gamma}}

%%%% Well-used matrices
\def\mA{\mat{A}}
\def\mB{\mat{B}}
\def\mC{\mat{C}}
\def\mD{\mat{D}}
\def\mI{\mat{I}}
\def\mJ{\mat{J}}
\def\mK{\mat{K}}
\def\mE{\mat{E}}
\def\mP{\mat{P}}
\def\mQ{\mat{Q}}
\def\mU{\mat{U}}
\def\mV{\mat{V}}
\def\mR{\mat{R}}
\def\mS{\mat{S}}
\def\mX{\mat{X}}
\def\msig{\mat{\Sigma}}
\def\mPhi{\mat{\Phi}}


\usepackage{amsmath}
%%%%% vector caculus useful macro
% ...\d, which typesets a derivative. ex: \d{y}{x}, instead of \frac{dx}{dy}.
\renewcommand{\d}[2]{\frac{\text{d} #1}{\text{d} #2}}


% ...similar for double-derivatives. ex: \dd{y}{x}.
\newcommand{\dd}[2]{\frac{\text{d}^2 #1}{\text{d} #2^2}}

% ...similar for partial derivatives. ex: \pd{y}{x}.
\newcommand{\pd}[2]{\frac{\partial #1}{\partial #2}}


% ...similar for partial double derivatives. ex: \pdd{y}{x}.
\newcommand{\pdd}[2]{\frac{\partial^2 #1}{\partial #2^2}}
% pdd with argument
\newcommand{\pdda}[3]{\frac{\partial^2 #1}{\partial #2 \partial #3}}

\usepackage{xparse}

%%%% caligraphic fonts
\def\cL{\ensuremath{{\cal L}}}
\def\cN{\ensuremath{{\cal N}}}
\def\cD{\ensuremath{{\cal D}}}
\def\cC{\ensuremath{{\cal C}}}
\def\cX{\ensuremath{{\cal X}}}
\def\cY{\ensuremath{{\cal Y}}}

%%% big parenthesis
\def\Bl{\Bigl}
\def\Br{\Bigr}
\def\lf{\left}
\def\ri{\right}


%%% floor notations
\newcommand{\lfl}{{\lfloor}}
\newcommand{\rfl}{{\rfloor}}
\newcommand{\floor}[1]{{\lfloor #1 \rfloor}}

%%% gradient
\newcommand{\grad}[1]{\nabla #1}
\newcommand{\hess}[1]{\text{H} #1}

%%% definition
%\newcommand{\eqdef}{\ensuremath{\triangleq}}
\newcommand{\eqdef}{\ensuremath{:=}}
%%% imply
\newcommand{\imp}{\Longrightarrow}



\newcommand{\separator}{
%  \begin{center}
    \par\noindent\rule{\columnwidth}{0.3mm}
%  \end{center}
}

\newcommand{\mynote}[1]{{\it \color{red} [#1]}}







%%% equation alignment
\newcommand{\aleq}[1]{\begin{align*}#1\end{align*}}

%%%%%%%%%%%%%%%% colored emphasized font, blanked words

\newcommand{\empr}[1]{{\color{red}\emph{#1}}}
\newcommand{\empb}[1]{{\color{blue}\emph{#1}}}
\newcommand{\redf}[1]{{\color{red} #1}}
\newcommand{\bluef}[1]{{\color{blue} #1}}
\newcommand{\grayf}[1]{{\color{gray} #1}}
\newcommand{\magenf}[1]{{\color{magenta} #1}}
\newcommand{\greenf}[1]{{\color{green} #1}}
\newcommand{\cyanf}[1]{{\color{cyan} #1}}
\newcommand{\orangef}[1]{{\color{orange} #1}}

\newcommand{\blk}[1]{\underline{\mbox{\hspace{#1}}}}


\newcommand{\redblk}[1]{\framebox{\color{red} #1}}
\newcommand{\redblank}[2]{\framebox{\onslide<#1->{\color{red} #2}}}
\newcommand{\blueblk}[1]{\framebox{\color{blue} #1}}
\newcommand{\blueblank}[2]{\framebox{\onslide<#1->{\color{blue} #2}}}



\makeatletter
\newcommand{\mybox}{%
    \collectbox{%
        \setlength{\fboxsep}{1pt}%
        \fbox{\BOXCONTENT}%
    }%
}
\makeatother

\makeatletter
\newcommand{\lecturemark}{%
    \collectbox{%
        \setlength{\fboxsep}{1pt}%
        \fcolorbox{red}{yellow}{\BOXCONTENT}%
    }%
}
\makeatother

\newcommand{\mycolorbox}[1]{
\begin{tcolorbox}[colback=red!5!white,colframe=red!75!black]
#1
\end{tcolorbox}
}
%%%% figure inclusion
\newcommand{\mypic}[2]{
\begin{center}
\includegraphics[width=#1\textwidth]{#2}
\end{center}
}

\newcommand{\myinlinepic}[2]{
\makebox[0cm][r]{\raisebox{-4ex}{\includegraphics[height=#1]{#2}}}
}




%%%% itemized and enumerated list
\newcommand{\bci}{\begin{compactitem}}
\newcommand{\eci}{\end{compactitem}}
\newcommand{\bce}{\begin{compactenum}}
\newcommand{\ece}{\end{compactenum}}


%%%% making 0.5/0.5 two columns
%%%% how to use: first number: length of separation bar
% \mytwocols{0.6}
% {
% contents in the left column
% }
% {
% contents in the right column
% }
%%%%

\newcommand{\mytwocols}[3]{
\begin{columns}[T] \column{.499\textwidth} #2 \column{.001\textwidth} \rule{.3mm}{{#1}\textheight} \column{.499\textwidth} #3 \end{columns}}

\newcommand{\mythreecols}[4]{
\begin{columns}[T] \column{.31\textwidth} #2 \column{.001\textwidth} \rule{.3mm}{{#1}\textheight} \column{.31\textwidth} #3 \column{.001\textwidth} \rule{.3mm}{{#1}\textheight} \column{.31\textwidth} #4  \end{columns}}

\newcommand{\mysmalltwocols}[3]{
\begin{columns}[T] \column{.4\textwidth} #2 \column{.001\textwidth} \rule{.3mm}{{#1}\textheight} \column{.4\textwidth} #3 \end{columns}}

%%%% making two columns with customized ratios
%%%% how to use: 
%first parameter: length of separation bar
%second parameter: ratio of left column
%third parameter: ratio of right column
% \mytwocols{0.6}{0.7}{0.29}
% {
% contents in the left column
% }
% {
% contents in the right column
% }
%%%%
\newcommand{\myvartwocols}[5]{
\begin{columns}[T] \column{#2\textwidth} {#4} \column{.01\textwidth} \rule{.3mm}{{#1}\textheight} \column{#3\textwidth} {#5} \end{columns}}

%%% making my block in beamer
%%% first parameter: title of block
%%% second parameter: contents of block
\newcommand{\myblock}[2]{
\begin{block}{#1} {#2}  \end{block}}

%%% independence notation
\newcommand{\indep}{\perp \!\!\! \perp}

%%%% probability with different shapes (parenthesis or bracket) and different sizes
%%% `i' enables us to insert the subscript to the probability 
\newcommand{\bprob}[1]{\mathbb{P}\Bl[ #1 \Br]}
\newcommand{\prob}[1]{\mathbb{P}[ #1 ]}
\newcommand{\cbprob}[1]{\mathbb{P}\Bl( #1 \Br)}
\newcommand{\cprob}[1]{\mathbb{P}( #1 )}
\newcommand{\probi}[2]{\mathbb{P}_{#1}[ #2 ]}
\newcommand{\bprobi}[2]{\mathbb{P}_{#1}\Bl[ #2 \Br]}
\newcommand{\cprobi}[2]{\mathbb{P}_{#1}( #2 )}
\newcommand{\cbprobi}[2]{\mathbb{P}_{#1}\Bl( #2 \Br)}

%%%% expectation with different shapes (parenthesis or bracket) and different sizes
%%% `i' enables us to insert the subscript to the expectation
\newcommand{\expect}[1]{\mathbb{E}[ #1 ]}
\newcommand{\cexpect}[1]{\mathbb{E}( #1 )}
\newcommand{\bexpect}[1]{\mathbb{E}\Bl[ #1 \Br]}
\newcommand{\cbexpect}[1]{\mathbb{E}\Bl( #1 \Br)}
\newcommand{\bbexpect}[1]{\mathbb{E}\lf[ #1 \ri]}
\newcommand{\expecti}[2]{\mathbb{E}_{#1}[ #2 ]}
\newcommand{\bexpecti}[2]{\mathbb{E}_{#1}\Bl[ #2 \Br]}
\newcommand{\bbexpecti}[2]{\mathbb{E}_{#1}\lf[ #2 \ri]}

%%%% variance
\newcommand{\var}[1]{\text{var}[ #1 ]}
\newcommand{\bvar}[1]{\text{var}\Bl[ #1 \Br]}
\newcommand{\cvar}[1]{\text{var}( #1 )}
\newcommand{\cbvar}[1]{\text{var}\Bl( #1 \Br)}

%%%% covariance
\newcommand{\cov}[1]{\text{cov}( #1 )}
\newcommand{\bcov}[1]{\text{cov}\Bl( #1 \Br)}

%%% Popular pmf, pdf notation to avoid long typing
\newcommand{\px}{\ensuremath{p_X(x)}}
\newcommand{\py}{\ensuremath{p_Y(y)}}
\newcommand{\pz}{\ensuremath{p_Z(z)}}
\newcommand{\pxA}{\ensuremath{p_{X|A}(x)}}
\newcommand{\pyA}{\ensuremath{p_{Y|A}(y)}}
\newcommand{\pzA}{\ensuremath{p_{Z|A}(z)}}
\newcommand{\pxy}{\ensuremath{p_{X,Y}(x,y)}}
\newcommand{\pxcy}{\ensuremath{p_{X|Y}(x|y)}}
\newcommand{\pycx}{\ensuremath{p_{Y|X}(y|x)}}

\newcommand{\fx}{\ensuremath{f_X(x)}}
\newcommand{\Fx}{\ensuremath{F_X(x)}}
\newcommand{\fy}{\ensuremath{f_Y(y)}}
\newcommand{\Fy}{\ensuremath{F_Y(y)}}
\newcommand{\fz}{\ensuremath{f_Z(z)}}
\newcommand{\Fz}{\ensuremath{F_Z(z)}}
\newcommand{\fxA}{\ensuremath{f_{X|A}(x)}}
\newcommand{\fyA}{\ensuremath{f_{Y|A}(y)}}
\newcommand{\fzA}{\ensuremath{f_{Z|A}(z)}}
\newcommand{\fxy}{\ensuremath{f_{X,Y}(x,y)}}
\newcommand{\Fxy}{\ensuremath{F_{X,Y}(x,y)}}
\newcommand{\fxcy}{\ensuremath{f_{X|Y}(x|y)}}
\newcommand{\fycx}{\ensuremath{f_{Y|X}(y|x)}}

\newcommand{\fth}{\ensuremath{f_\Theta(\theta)}}
\newcommand{\fxcth}{\ensuremath{f_{X|\Theta}(x|\theta)}}
\newcommand{\fthcx}{\ensuremath{f_{\Theta|X}(\theta|x)}}

\newcommand{\pkcth}{\ensuremath{p_{X|\Theta}(k|\theta)}}
\newcommand{\fthck}{\ensuremath{f_{\Theta|X}(\theta|k)}}


%%%% indicator
\newcommand{\indi}[1]{\mathbf{1}_{ #1 }}

%%%% exponential rv.
\newcommand{\elambdax}{\ensuremath{e^{-\lambda x}}}

%%%% normal  rv.
\newcommand{\stdnormal}{\ensuremath{\frac{1}{\sqrt{2\pi}} e^{-x^2/2}}}
\newcommand{\gennormal}{\ensuremath{\frac{1}{\sigma\sqrt{2\pi}} e^{-(x-\mu)^2/2}}}

%%%%%% estimator, estimate
\newcommand{\hth}{\ensuremath{\hat{\theta}}}
\newcommand{\hTH}{\ensuremath{\hat{\Theta}}}
\newcommand{\MAP}{\ensuremath{\text{MAP}}}
\newcommand{\LMS}{\ensuremath{\text{LMS}}}
\newcommand{\LLMS}{\ensuremath{\text{L}}}
\newcommand{\ML}{\ensuremath{\text{ML}}}

%%%% colored text
\newcommand{\red}[1]{\color{red}#1} 
\newcommand{\cyan}[1]{\color{cyan}#1} 
\newcommand{\magenta}[1]{\color{magenta}#1} 
\newcommand{\blue}[1]{\color{blue}#1} 
\newcommand{\green}[1]{\color{green}#1} 
\newcommand{\white}[1]{\color{white}#1} 
\newcommand{\gray}[1]{\color{gray}#1} 

%%% definition
\newcommand{\defi}{{\color{red} Definition.} } 
\newcommand{\exam}{{\color{red} Example.} } 
\newcommand{\question}{{\color{red} Question.} } 
\newcommand{\thm}{{\color{red} Theorem.} } 
\newcommand{\background}{{\color{red} Background.} } 
\newcommand{\msg}{{\color{red} Message.} } 


\def\ml{\text{ML}}
\def\map{\text{MAP}}

%%%%%%%%%%%%%%%%%%%%%%% old macros that you can ignore %%%%%%%%%%%%%%%%%%%%%%%%

% \def\un{\underline}
% \def\ov{\overline}


% \newcommand{\beq}{\begin{eqnarray*}}
% \newcommand{\eeq}{\end{eqnarray*}}
% \newcommand{\beqn}{\begin{eqnarray}}
% \newcommand{\eeqn}{\end{eqnarray}}
% \newcommand{\bemn}{\begin{multiline}}
% \newcommand{\eemn}{\end{multiline}}
% \newcommand{\beal}{\begin{align}}
% \newcommand{\eeal}{\end{align}}
% \newcommand{\beas}{\begin{align*}}
% \newcommand{\eeas}{\end{align*}}



% \newcommand{\bd}{\begin{displaymath}}
% \newcommand{\ed}{\end{displaymath}}
% \newcommand{\bee}{\begin{equation}}
% \newcommand{\eee}{\end{equation}}


% \newcommand{\vs}{\vspace{0.2in}}
% \newcommand{\hs}{\hspace{0.5in}}
% \newcommand{\el}{\end{flushleft}}
% \newcommand{\bl}{\begin{flushleft}}
% \newcommand{\bc}{\begin{center}}
% \newcommand{\ec}{\end{center}}
% \newcommand{\remove}[1]{}

% \newtheorem{theorem}{Theorem}
% \newtheorem{corollary}{Corollary}
% \newtheorem{prop}{Proposition}
% \newtheorem{lemma}{Lemma}
% \newtheorem{defi}{Definition}
% \newtheorem{assum}{Assumption}
% \newtheorem{example}{Example}
% \newtheorem{property}{Property}
% \newtheorem{remark}{Remark}

% \newcommand{\separator}{
%   \begin{center}
%     \rule{\columnwidth}{0.3mm}
%   \end{center}
% }

% \newenvironment{separation}
% { \vspace{-0.3cm}
%   \separator
%   \vspace{-0.25cm}
% }
% {
%   \vspace{-0.5cm}
%   \separator
%   \vspace{-0.15cm}
% }

% \def\A{\mathcal A}
% \def\oA{\overline{\mathcal A}}
% \def\S{\mathcal S}
% \def\D{\mathcal D}
% \def\eff{{\rm Eff}}
% \def\bD{\bm{D}}
% \def\cU{{\cal U}}
% \def\bbs{{\mathbb{s}}}
% \def\bbS{{\mathbb{S} }}
% \def\cM{{\cal M}}
% \def\bV{{\bm{V}}}
% \def\cH{{\cal H}}
% \def\ch{{\cal h}}
% \def\cR{{\cal R}}
% \def\cV{{\cal V}}
% \def\cA{{\cal A}}
% \def\cX{{\cal X}}
% \def\cN{{\cal N}}
% \def\cJ{{\cal J}}
% \def\cK{{\cal K}}
% \def\cL{{\cal L}}
% \def\cI{{\cal I}}
% \def\cY{{\cal Y}}
% \def\cZ{{\cal Z}}
% \def\cC{{\cal C}}
% \def\cR{{\cal R}}
% \def\id{{\rm Id}}
% \def\st{{\rm st}}
% \def\cF{{\cal F}}
% \def\bz{{\bm z}}
% \def\cG{{\cal G}}
% \def\N{\mathbb{N}}
% \def\bbh{\mathbb{h}}
% \def\bbH{\mathbb{H}}
% \def\bbi{\mathbb{i}}
% \def\bbI{\mathbb{I}}
% \def\R{\mathbb{R}}
% \def\bbR{\mathbb{R}}
% \def\bbr{\mathbb{r}}
% \def\cB{{\cal B}}
% \def\cP{{\cal P}}
% \def\cS{{\cal S}}
% \def\bW{{\bm W}}
% \def\bc{{\bm c}}

% %\def\and{\quad\mbox{and}\quad}
% \def\ind{{\bf 1}}


% \def\bmg{{\bm{\gamma}}}
% \def\bmr{{\bm{\rho}}}
% \def\bmq{{\bm{q}}}
% \def\bmt{{\bm{\tau}}}
% \def\bmn{{\bm{n}}}
% \def\bmcapn{{\bm{N}}}
% \def\bmrho{{\bm{\rho}}}

% \def\igam{\underline{\gamma}(\lambda)}
% \def\sgam{\overline{\gamma}(\lambda)}
% \def\ovt{\overline{\theta}}
% \def\ovT{\overline{\Theta}}
% \def\PP{{\mathrm P}}
% \def\EE{{\mathrm E}}
% \def\iskip{{\vskip -0.4cm}}
% \def\siskip{{\vskip -0.2cm}}

% \def\bp{\noindent{\it Proof.}\ }
% \def\ep{\hfill $\Box$}



%%%%%%%%%% linear algebra macros %%%%%%%%%%%%%%%%%%%%%%%

%--------linsys
%  Use as \begin{linsys}{3}
%           x &+ &3y &+ &a &= &7 \\
%           x &- &3y &+ &a &= &7
%         \end{linsys}
% Remark: TeXbook pp. 167-170 says to put a medmuskip around a +; and that's
% 4/18-ths of an em.  Why does 2/18-ths of an em work?  I don't know, but
% comparing to a regular displayed equation suggests it is right.
% (darseneau says LaTeX puts in half an \arraycolsep.)
\newenvironment{linsys}[2][m]{%
\setlength{\arraycolsep}{.1111em} % p. 170 TeXbook; a medmuskip
\begin{array}[#1]{@{}*{#2}{rc}r@{}}
}{%
\end{array}}

\newsavebox\boxofmathplus
\sbox{\boxofmathplus}{$+$}
\newcommand{\spaceforemptycolumn}{\makebox[\wd\boxofmathplus]{\ }}

%--------grstep
% For denoting a Gauss' reduction step.
% Use as: \grstep{\rho_1+\rho_3} or \grstep[2\rho_5 \\ 3\rho_6]{\rho_1+\rho_3}
% \newcommand{\grstep}[2][\relax]{%
%    \ensuremath{\mathrel{
%        \mathop{\longrightarrow}\limits^{#2\mathstrut}_{
%                                    \begin{subarray}{l} #1 \end{subarray}}}}}

% Advantage of length formulation is that between adjacent
% \grstep's you can add \hspace{-\grsteplength} to make it look not too wide
\newlength{\grsteplength}
\setlength{\grsteplength}{1.5ex plus .1ex minus .1ex}

\newcommand{\grstep}[2][\relax]{%
   \ensuremath{\mathrel{
       \hspace{\grsteplength}\mathop{\longrightarrow}\limits^{#2\mathstrut}_{
                                     \begin{subarray}{l} #1 \end{subarray}}\hspace{\grsteplength}}}}
% If two or more \grsteps are in a row then they need to be tightened
\newcommand{\repeatedgrstep}[2][\relax]{\hspace{-\grsteplength}\grstep[#1]{#2}}

% row swap operation: \rho_1\swap\rho_2
\newcommand{\swap}{\leftrightarrow}

%-------------amatrix
% Augmented matrix.  Usage (note the argument does not count the aug col):
% \begin{amatrix}{2}
%   1  2  3 \\  4  5  6
% \end{amatrix}
\newenvironment{amatrix}[1]{%
  \left(\begin{array}{@{}*{#1}{c}|c@{}}
}{%
  \end{array}\right)
}



%-------------pmat
% For matrices with arguments.
% Usage: \begin{pmat}{c|c|c} 1 &2 &3 \end{pmat}
\newenvironment{pmat}[1]{
  \left(\begin{array}{@{}#1@{}}
}{\end{array}\right)
}



%-------------misc matrices
% \newenvironment{mat}{\left(\begin{array}}{\end{array}\right)}
\newenvironment{detmat}{\left|\begin{array}}{\end{array}\right|}
\newcommand{\deter}[1]{ \mathchoice{\left|#1\right|}{|#1|}{|#1|}{|#1|} }
\newcommand{\generalmatrix}[3]{ %arg1: low-case letter, arg2: rows, arg3: cols 
               \left(
                  \begin{array}{cccc}
                    #1_{1,1}  &#1_{1,2}  &\ldots  &#1_{1,#2}  \\
                    #1_{2,1}  &#1_{2,2}  &\ldots  &#1_{2,#2}  \\
                              &\vdots                         \\
                    #1_{#3,1} &#1_{#3,2} &\ldots  &#1_{#3,#2}
                  \end{array}
               \right)  }

\newcommand{\generaldet}[3]{ %arg1: low-case letter, arg2: rows, arg3: cols 
               \left|
                  \begin{array}{cccc}
                    #1_{11}  &#1_{12}  &\ldots  &#1_{1 #2}  \\
                    #1_{21}  &#1_{22}  &\ldots  &#1_{2 #2}  \\
                              &\vdots                         \\
                    #1_{#3 1} &#1_{#3 2} &\ldots  &#1_{#3 #2}
                  \end{array}
               \right|  }

% With mathtools we can have column entries right flushed
% There is an optional argument \begin{mat}[r]{3} .. \end{mat} for
% right-flushed columns.  Perhaps the rule is that numbers are better 
% right-flushed but if there are any letters it is better centered?
\newenvironment{nmat}[1][c]{\begin{pmatrix*} % disable optional arg [#1] 
      }{\end{pmatrix*}}
% If mat starts with &\vdots get an error; why?  No apparent macro fix, according to texexchange
\newenvironment{vmat}[1][c]{\begin{vmatrix*} % disable optional arg [#1] 
      }{\end{vmatrix*}}
\newenvironment{amat}[2][c]{%
  % disable optional arg \left(\begin{array}{@{}*{#2}{#1}|#1@{}}
  \left(\begin{array}{@{}*{#2}{c}|#1@{}}
}{%
  \end{array}\right)
}
% \newcommand\vdotswithin[1]{% Taken from mathtools.dtx because my TL is not 2011
%   {\mathmakebox[\widthof{\ensuremath{{}#1{}}}][c]{{\vdots}}}}


%------------colvec and rowvec
% Column vector and row vector.  Usage:
%  \colvec{1  \\ 2 \\ 3 \\ 4} and \rowvec{1  &2  &3}
% Colvec takes an optional argument \colvec[r]{x_1 \\ 0}.  Perhaps 
% digits look better right aligned, but if there are any letters it
% needs to be centered?
\newcommand{\colvec}[2][c]{\begin{nmat}[#1] #2 \end{nmat}}
\newcommand{\smallcolvec}[1]{\left(\begin{smallmatrix} #1 \end{smallmatrix}\right)}
% For row vectors, cannot do \newcommand{\rowvec}[1]{\begin{mat} #1 \end{mat}}
% since the delimiters come out too large.
\newcommand{\rowvec}[1]{\setlength{\arraycolsep}{3pt}\left(\begin{matrix} #1 \end{matrix}\right)}



%-------------making aligned columns
% Usage: \begin{aligncolondecimal}{2} 1.2 \\ .33 \end{aligncolondecimal}
% (negative argument centers decimal pt in column).  Also Usage:
% \begin{aligncolondecimal}[0em]{2} 1.2 \\ .33 \end{aligncolondecimal}
% to make the left and right LaTeX-array padding disappear.
\RequirePackage{array}\RequirePackage{dcolumn}
\newenvironment{aligncolondecimal}[2][.1111em]{%
\setlength{\arraycolsep}{#1}
\newcolumntype{.}{D{.}{.}{#2}}\begin{array}{.}}{%
\end{array}}

% Matrix and vector, with numbers centered on decimal point
% Usage: \begin{dmat}{D{.}{.}{1}D{.}{.}{3}}  0  &.123 \\ .2 &.456 \end{dmat}
%  (in the D{.}{.}{number} that is the number of decimal places)
\newlength{\dmatcolsep}\setlength{\dmatcolsep}{5pt}
\newenvironment{dmat}[2][\dmatcolsep]{%
  \setlength{\arraycolsep}{#1}
  \left(\begin{array}{@{}#2@{}}
}{%
  \end{array}\right)}
% Usage: \dcolvec[2]{1.23 \\ 4.56} where the optional argument is the number
% of decimal places.
\newcommand{\dcolvec}[2][-1]{\left(\begin{array}{@{}D{.}{.}{#1}@{}} #2 \end{array}\right)}

%\newcommand{\trans}[1]{ {{#1}^{\mathsf{T}}} } 
\newcommand{\trans}[1]{ {#1}^{\mathsf{T}} } 
\newcommand{\inv}[1]{ {#1}^{-1} } 
\newcommand{\spn}[1]{\ensuremath{\text{span}[#1]} } 
\newcommand{\rk}[1]{\ensuremath{\text{rk}(#1)} } 
\newcommand{\dimm}[1]{\ensuremath{\text{dim}(#1)} } 
\newcommand{\img}[1]{\ensuremath{\text{Im}(#1)} } 
%\newcommand{\norm}[1]{\ensuremath{\left || #1 \right ||} } 
\newcommand{\norm}[1]{\ensuremath{\left \lVert #1 \right \rVert} } 
% orthogonal complement
\newcommand{\ocomp}[1]{\ensuremath{#1^{\bot}} } 
\newcommand{\inner}[2]{\ensuremath{\left\langle #1, #2 \right\rangle} } 
\DeclareMathOperator{\tr}{tr}


% \NewDocumentCommand{\grad}{e{_^}}{%
%   \mathop{}\!% \mathop for good spacing before \nabla
%   \nabla
%   \IfValueT{#1}{_{\!#1}}% tuck in the subscript
%   \IfValueT{#2}{^{#2}}% possible superscript
% }
% \begin{equation*}
%          \begin{nmat}[r]
% 1 &2 &13 \\
%           4  &5  &6
%          \end{nmat}
%       \end{equation*}

% \begin{equation*}
%          \begin{amat}{2}
%           1  &2  &3  \\
%           4  &5  &6
%          \end{amat}
%       \end{equation*}
       
%       \begin{equation*}
%          \begin{pmat}{c|c|c}
% 1 &2 &3 \\
%           4  &5  &6
%          \end{pmat}
%       \end{equation*}

% \begin{equation*}
%          \begin{vmat}
% a &c \\
%           b  &d
%          \end{vmat}
%          =ad-bc
% \end{equation*}

%  \begin{equation*}
%   \vec{v}=\colvec{-1  \\ -0.5  \\ 0}
% \end{equation*}

%  \begin{equation*}
%   \vec{v}=\rowvec{-1  & -0.5  & 0}
% \end{equation*}





%\renewcommand{\baselinestretch}{2.0}

\begin{document}

%itemshape
\setbeamertemplate{itemize item}{\scriptsize\raise1.25pt\hbox{\donotcoloroutermaths$\bullet$}}
\setbeamertemplate{itemize subitem}{\tiny\raise1.5pt\hbox{\donotcoloroutermaths$\circ$}}
\setbeamertemplate{itemize subsubitem}{\tiny\raise1.5pt\hbox{\donotcoloroutermaths$\blacktriangleright$}}
%default value for spacing
\plitemsep 0.1in
\pltopsep 0.03in
\setlength{\parskip}{0.15in}
%\setlength{\parindent}{-0.5in}
\setlength{\abovedisplayskip}{0.07in}
\setlength{\belowdisplayskip}{0.07in}
\setlength{\mathindent}{0cm}
\setbeamertemplate{frametitle continuation}{[\insertcontinuationcount]}

\setlength{\leftmargini}{0.5cm}
\setlength{\leftmarginii}{0.5cm}

\setlength{\fboxrule}{0.05pt}
\setlength{\fboxsep}{5pt}


%%%%%%% This should be placed at the end of this file
\logo{\pgfputat{\pgfxy(0.11, 7.4)}{\pgfbox[right,base]{\tikz{\filldraw[fill=dianablue, draw=none] (0 cm, 0 cm) rectangle (50 cm, 1 cm);}\mbox{\hspace{-8 cm}\includegraphics[height=0.7 cm]{../kaist_ee.png}
}}}}

\begin{frame}
  \titlepage
\end{frame}

\logo{\pgfputat{\pgfxy(0.11, 7.4)}{\pgfbox[right,base]{\tikz{\filldraw[fill=dianablue, draw=none] (0 cm, 0 cm) rectangle (50 cm, 1 cm);}\mbox{\hspace{-8 cm}\includegraphics[height=0.7 cm]{../kaist_ee.png}
}}}}

% rule color - gray
\makeatletter
\let\old@rule\@rule
\def\@rule[#1]#2#3{\textcolor{gray}{\old@rule[#1]{#2}{#3}}}
\makeatother




% START START START START START START START START START START START START START
%%%%%%%%%%%%%%%%%%%%%%%%%%%%%%%%%%%%%%%%%%%%%%%%%%%%%%
\begin{frame}{Roadmap}

\plitemsep 0.1in

\bce[(1)] 

\item Optimization Using Gradient Descent 
\item Constrained Optimization and Lagrange Multipliers 
\item Convex Sets and Functions
\item Convex Optimization 
\item Convex Conjugate
\ece
\end{frame}

%%%%%%%%%%%%%%%%%%%%%%%%%%%%%%%%%%%%%%%%%%%%%%%%%%%%%%
\begin{frame}{Summary}

\plitemsep 0.1in

\bci 

\item Training machine learning models $=$ finding a good set of parameters 

\item A good set of parameters $=$ Solution (or close to solution) to some optimization problem

\item Directions: Unconstrained optimization, Constrained optimization, Convex optimization

\item High-school math: A necessary condition for the optimal point:  $f'(x)=0$ (stationary point)
\bci
\item Gradient will play an important role
\eci
\eci
\end{frame}



%%%%%%%%%%%%%%%%%%%%%%%%%%%%%%%%%%%%%%%%%%%%%%%%%%%%%%
\section{L7(1)}
\begin{frame}{Roadmap}

\plitemsep 0.1in

\bce[(1)] 

\item \redf{Optimization Using Gradient Descent} 
\item \grayf{Constrained Optimization and Lagrange Multipliers 

\item Convex Sets and Functions
\item Convex Optimization 
\item Convex Conjugate}


\ece
\end{frame}

%%%%%%%%%%%%%%%%%%%%%%%%%%%%%%%%%%%%%%%%%%%%%%%%%%%%%%
\begin{frame}{Unconstrained Optimization and Gradient Algorithms}



\plitemsep 0.1in

\bci 

\item Goal
$$
\min f(\vx), \quad f(\vx): \realn \mapsto \real, \quad f \in C^1
$$

\medskip

\item \bluef{Graident-type} algorithms
$$
\vx_{k+1} = \vx_{k} + \gamma_k \vd_k, \quad k = 0,1,2, \ldots
$$


\item \redf{Lemma.} Any direction $\vd \in \real^{n \times 1}$ that satisfies $\grad f(\vx) \cdot \vd <0$ is a
  descent direction of $f$ at $\vx$. That is, if we let $\vx_{\alpha} = \vx + \alpha
  \vd,$ $\exists \bar{\alpha} >0,$ such that for
  all $\alpha \in (0,\bar{\alpha}],$ $f(\vx_\alpha) < f(\vx).$


\item \bluef{Steepest gradient descent}\footnote{In some cases, just gradient descent often means this steepest gradient descent.}. $\vd_{k} = - \trans{\grad f(\vx_k)}.$


\item Finding a local optimum $f(\vx_{\star}),$ if the step-size $\gamma_k$ is suitably chosen. 

\item \question How do we choose $\vd_k$ for a constrained optimization? 
 %(Note) For constrained opti that the above is for unconstrained optimizations. But, our
%   problem is a constrained optimization, so we have to pick $\vx^k$ that
%   is {\em projected onto} the constraint set, i.e., $p \geq 0.$

\eci
\end{frame}

%%%%%%%%%%%%%%%%%%%%%%%%%%%%%%%%%%%%%%%%%%%%%%%%%%%%%%
\begin{frame}{Example}

\plitemsep 0.1in

\bci 

\item A quadratic function $f: \real^2 \mapsto \real$.
$$
f\left(\colvec{x_1 \\ x_2}\right) = \frac{1}{2} \trans{\colvec{x_1 \\ x_2}} 
\begin{nmat} 
2 & 1 \cr
1 & 20
\end{nmat} \colvec{x_1 \\ x_2} - \trans{\colvec{5 \\ 3}} \colvec{x_1 \\ x_2},
$$
whose gradient is $\trans{\colvec{x_1 \\ x_2}} 
\begin{nmat} 
2 & 1 \cr
1 & 20
\end{nmat} - \trans{\colvec{5 \\ 3}}$
\eci

\medskip
\mysmalltwocols{0.35}
{
\small
\bci
\item $\vx_0 = \trans{(-3 -1)}$
\item constant step size $\alpha = 0.085$
\item Zigzag pattern
\eci
}
{
\vspace{-0.5cm}
\mypic{0.95}{L7_gradient_ex.png}
}

\end{frame}

%%%%%%%%%%%%%%%%%%%%%%%%%%%%%%%%%%%%%%%%%%%%%%%%%%%%%%
\begin{frame}{Taxonomy}

\plitemsep 0.1in

\bci 

\item Goal: $\min L(\vth)$ for $n$ training data

\item Based on the \bluef{amount of training data} used for \bluef{each} iteration
\bci
\item Batch gradient descent (the entire $n$)
\item Mini-batch gradient descent($k< n$ data )
\item Stochastic gradient descent (one sampled data)
\eci

\item Based on the adaptive method of update
\bci
\item Momentum, NAG, Adagrad, RMSprop, Adam, etc
\eci

\item \url{https://ruder.io/optimizing-gradient-descent/} 
\eci
\end{frame}

%%%%%%%%%%%%%%%%%%%%%%%%%%%%%%%%%%%%%%%%%%%%%%%%%%%%%%
\begin{frame}{Stochastic Gradient Descent (SGD)}

\plitemsep 0.1in

\bci 

\item Assume $L(\vth) = \sum_{i=1}^n L_n(\vth)$ (which happens in many cases in machine learning, e.g., negative log-likelihood in regression)

\item Gradient update
\aleq{
\vth_{k+1} = \vth_{k} -\gamma_k \trans{\grad L(\vth_k)} = \vth_{k} -\gamma_k \sum_{n=1}^N \trans{\grad L_n(\vth_k)}
}
\bci
\item Batch gradient: $\sum_{n=1}^N \trans{\grad L_n(\vth_k)}$
\item Mini-batch gradient: $\sum_{n \in \set{K}} \trans{\grad L_n(\vth_k)}$ for a suitable choice of $\set{K}, |\set{K}| < n$
\item Stochastic gradient: $\trans{\grad L_n(\vth_i)}$ for some (randomly chosen) $i$. Noisy approximation to the real gradient. 
\eci

\item Tradeoff: computation burden vs. exactness
\eci
\end{frame}

%%%%%%%%%%%%%%%%%%%%%%%%%%%%%%%%%%%%%%%%%%%%%%%%%%%%%%
\begin{frame}{Adaptivity for Better Convergence: Momemtum}

\plitemsep 0.1in

\bci 

\item Step size.
\bci
\item Too small: slow update, Too big: overshoot, zig-zag, often fail to converge
\eci

\item Adaptive update: smooth out the erratic behavior and dampens oscillations

\item Gradient descent with \bluef{momentum}
\aleq{
\vx_{k+1} & = \vx_k - \gamma_i \trans{\grad f(\vx_k)} + \alpha \Delta \vx_k, \quad \alpha \in [0,1] \cr
\Delta \vx_k & = \vx_k - \vx_{k-1}
}
\vspace{-0.5cm}
\bci
\item Memory term: $\alpha \Delta \vx_k,$ where $\alpha$ is the degree of how much we remember the past
\item Next update $=$ a linear combination of current and previous updates
\eci
\eci
\end{frame}




%%%%%%%%%%%%%%%%%%%%%%%%%%%%%%%%%%%%%%%%%%%%%%%%%%%%%%
\section{L7(2)}
\begin{frame}{Roadmap}

\plitemsep 0.1in

\bce[(1)] 

\item \grayf{Optimization Using Gradient Descent} 
\item \redf{Constrained Optimization and Lagrange Multipliers} 

\item \grayf{Convex Sets and Functions
\item Convex Optimization 
\item Convex Conjugate}

\ece
\end{frame}

%%%%%%%%%%%%%%%%%%%%%%%%%%%%%%%%%%%%%%%%%%%%%%%%%%%%%%
\begin{frame}{Standard Constrained Optimization Problem}

\plitemsep 0.2in

\bci 

\item An optimization problem in standard form:
\aleq{
\text{minimize} \ & f(\vx) \cr
\text{subject to} \  & g_{i}(\vx)\leq 0, \;\; i=1, 2, \ldots, m \quad \text{\em (Inequality constraints)}\cr
& h_{j}(\vx)=0, \;\; j=1,2,\ldots,p \quad \text{\em (Equality constraints)}
}

\item Variables: $\vx \in\realn$. Assume nonempty feasible set

\item Optimal value: $p^{*}$. Optimizer: $\vx^{*}$

\eci
\end{frame}

%%%%%%%%%%%%%%%%%%%%%%%%%%%%%%%%%%%%%%%%%%%%%%%%%%%%%%
\begin{frame}{Problem Solving via Langrange Multipliers}

\plitemsep 0.1in

\bci 

\item Duality Mentality

\bci
\item \bluef{Bound} or \bluef{solve} an optimization problem via a \redf{
different} optimization problem!

\item We'll develop the basic Lagrange duality theory for a \redf{
general} optimization problem, then specialize for convex
optimization
\eci

\item Idea: augment the objective with a weighted sum of constraints

\bci

\item \bluef{Lagrangian}: $$\cL(\vx,\vlam,\vnu)=f(\vx) +
\sum_{i=1}^{m}\lambda_{i}g_{i}(\vx) + \sum_{i=1}^{p}\nu_{i}h_{i}(\vx)$$

\item \bluef{Lagrange multipliers (dual variables)}: $\vlam = (\lambda_i: i = 1, \cdots, m)\succeq
0,$ $\vnu=(\nu_1, \cdots, \nu_p)$

\item \bluef{Lagrange dual function}: $$\cD(\vlam,\vnu)=\inf_{\vx}
\cL(\vx,\vlam,\vnu)$$
\eci

\eci
\end{frame}



%%%%%%%%%%%%%%%%%%%%%%%%%%%%%%%%%%%%%%%%%%%%%%%%%%%%%%
\begin{frame}{Lower Bound on the Optimal Value}

\plitemsep 0.1in

\bci 

\item The dual function $\cD(\vlam,\vnu)$ is the \bluef{lower bound} on the optimal value $p^*.$

\item \redf{Theorem}. $\cD(\vlam,\vnu)\leq p^{*}, \;\; \forall \vlam\succeq
0,\ \vnu$


\item \redf{Proof}. Consider feasible $\tilde{\vx}$. Then, 
\[
\cL(\tilde{\vx},\vlam,\vnu) = f(\tilde{\vx}) +
\sum_{i=1}^{m}\lambda_{i}g_{i}(\tilde{\vx}) +
\sum_{i=1}^{p}\nu_{i}h_{i}(\tilde{\vx}) \leq f(\tilde{\vx})
\]
since $f_{i}(\tilde{\vx})\leq 0$ and $\lambda_{i}\geq 0.$

Hence, $\cD(\vlam,\vnu)\leq \cL(\tilde{\vx},\vlam,\vnu) \leq
f(\tilde{\vx})$ for all feasible $\tilde{\vx}.$
Therefore, $\cD(\vlam,\vnu)\leq p^{*}.$

\eci
\end{frame}

%%%%%%%%%%%%%%%%%%%%%%%%%%%%%%%%%%%%%%%%%%%%%%%%%%%%%%
\begin{frame}{Lagrangian Dual Problem}

\plitemsep 0.1in

\bci 

\item Lower bound from Lagrange dual function depends on
$(\vlam,\vnu)$. 

\item \question What's the best lower bound? 
\[
\begin{array}{lll}
\text{\bf Langrangian dual problem} \quad & \mbox{maximize} & \cD(\vlam,\vnu) \\
 & \mbox{subject to} & \vlam\succeq 0
\end{array}
\]

\item Dual variables: $(\vlam,\vnu)$

\item \redf{Always a convex optimization}, because $\cD(\vlam,\vnu)$ is always concave over $\vlam,\vnu.$

\bci
\item Infimum over $\vx$ of a family of affine functions in $(\vlam,\vnu)$ (we will see this later)
\eci

\item Denote the optimal value of Lagrange dual problem by \bluef{$d^{*}$}.

\eci
\end{frame}

%%%%%%%%%%%%%%%%%%%%%%%%%%%%%%%%%%%%%%%%%%%%%%%%%%%%%%
\begin{frame}{Weak Duality}

\plitemsep 0.15in

\bci 

\item What's the relationship between $d^{*}$ and $p^{*}$?

\myblock{Weak Duality}
{
\[
d^{*}\leq p^{*}
\]
\vspace{-0.3cm}
}

\item Weak duality \redf{always} hold (even if the primal problem is
not convex):

\item Optimal duality gap: \bluef{$p^{*}-d^{*}$}


\item Efficient generation of the lower bounds through the dual problem


\eci
\end{frame}





%%%%%%%%%%%%%%%%%%%%%%%%%%%%%%%%%%%%%%%%%%%%%%%%%%%%%%
\section{L7(3)}
\begin{frame}{Roadmap}

\plitemsep 0.1in

\bce[(1)] 

\item \grayf{Optimization Using Gradient Descent} 
\item \grayf{Constrained Optimization and Lagrange Multipliers} 

\item \redf{Convex Sets and Functions}
\item \grayf{Convex Optimization 
\item Convex Conjugate}


\ece
\end{frame}

%%%%%%%%%%%%%%%%%%%%%%%%%%%%%%%%%%%%%%%%%%%%%%%%%%%%%%
\begin{frame}{Convex Optimization}

\plitemsep 0.1in

\bci 

\item Convex optimization problem
\aleq{
\text{minimize} \quad & f(\vx) \cr
\text{subject to} \quad  & \vx \in \set{X},
}
where $f(\vx): \realn \mapsto \real$ is a convex function, and $\set{X}$ is a convex set. 

\item The watershed between easily solvable problem and
intractable ones is not `linearity', but {\red `convexity'}

\item Let's overview the background of convex functions, convex sets, and their basic properties. 

\eci
\end{frame}


%%%%%%%%%%%%%%%%%%%%%%%%%%%%%%%%%%%%%%%%%%%%%%%%%%%%%%
\begin{frame}{Convex Set}

\plitemsep 0.1in

\bci 

\item Set $\set{C}$ is a {\blue convex set} if the line segment between any
two points in $\set{C}$ lies in $\set{C}$, i.e.,  if for any $x_{1},x_{2}\in \set{C}$
and any $\theta\in[0,1]$, we have $\theta x_{1} + (1-\theta)x_{2} \in \set{C}$


\item {\blue Convex hull} of $\set{C}$ is the set of all {\red convex combinations}
of points in $\set{C}$:
\[
\left\{ \sum_{i=1}^{k}\theta_{i}x_{i} \mid x_{i}\in \set{C}, \theta_{i}\geq
0, i=1,2,\ldots,k, \sum_{i=1}^{k}\theta_{i}=1 \right\}
\]

- What is $k$? For all $k$? For some $k$?

\item Generalize to infinite sums and integrals:

$$
\sum_{i=1}^\infty \theta_i x_i \in \set{C},  \quad \int_{\set{C}} p(x) x dx \in \set{C},
$$
where $\sum_{i=1}^\infty \theta_i = 1$ and $p(x)$ is a pdf of some random variable.

\eci
\end{frame}

%%%%%%%%%%%%%%%%%%%%%%%%%%%%%%%%%%%%%%%%%%%%%%%%%%%%%%
\begin{frame}{Examples}

- Convex and Non-convex sets
\mypic{0.45}{L7_convex_set_ex1}


- Convex hulls
\mypic{0.45}{L7_convex_set_ex2}


\end{frame}

%%%%%%%%%%%%%%%%%%%%%%%%%%%%%%%%%%%%%%%%%%%%%%%%%%%%%%
\begin{frame}{Examples of Convex Sets}

\plitemsep 0.1in

\bci 

\item \bluef{Hyperplane} in $\realn$ is a set:
$\{x \mid \trans{a}x=b\}$ where $a\in\realn, a\neq 0, b\in\real$

In other words, $\{ x \mid \trans{a}(x-x_0) =0\},$ where $x_0$ is any point in
the hyperplane, i.e., $\trans{a} x_0 = b.$

\mysmalltwocols{0.2}
{
\item Divides $\realn$ into two {\blue halfspaces}: 
$\{x|\trans{a}x\leq b\}$ and $\{x|\trans{a}x>b\}$
}
{
\vspace{-0.3cm}
\mypic{0.7}{L7_halfspace.png}
}

\item \bluef{Polyhedron} is the solution set of a finite
number of linear equalities and inequalities (intersection of
finite number of halfspaces and hyperplanes)
\aleq{
\set{P} = \{ x \mid \trans{a}_jx \leq b_j, j = 1, \ldots, m, \trans{c}_j x = d_j, j
=1, \ldots, p\}  = \{ x \mid Ax \leq b, Cx = d\}
}

\item \bluef{Polytope}: a bounded polyhedron

\eci
\end{frame}

%%%%%%%%%%%%%%%%%%%%%%%%%%%%%%%%%%%%%%%%%%%%%%%%%%%%%%
\begin{frame}{Separating Hyperplane Theorem}

\mypic{0.3}{L7_separating.png}

\plitemsep 0.1in

\bci 

\item $\set{C}$ and $\set{D}$: non-intersecting convex sets, i.e., 
$\set{C}\bigcap \set{D}=\phi$. 

\item Then, there exist $a\neq 0$ and $b$ such that
$\trans{a}x\leq b$ for all $x\in \set{C}$ and $\trans{a}x\geq b$ for all $x\in
\set{D}$.

\eci
\end{frame}

%%%%%%%%%%%%%%%%%%%%%%%%%%%%%%%%%%%%%%%%%%%%%%%%%%%%%%
\begin{frame}{Supporting Hyperplane Theorem}

\mypic{0.25}{L7_supporting.png}

\plitemsep 0.1in

\bci 

\item Given a set $\set{C}\in\realn$ and a point $x_{0}$ on its
  boundary, if $a\neq 0$ satisfies $\trans{a}x \leq \trans{a}x_{0}$ for all
  $x\in \set{C}$, then $\{x|\trans{a}x=\trans{a}x_{0}\}$ is called a \bluef{
    supporting hyperplane} to $\set{C}$ at $x_{0}$
  
\item For any nonempty convex set $\set{C}$ and \redf{any} $x_{0}$ on
  boundary of $\set{C}$, there exists a supporting hyperplane to $\set{C}$ at
  $x_{0}$

\item What happens if $\set{C}$ is non-convex?

\eci
\end{frame}

%%%%%%%%%%%%%%%%%%%%%%%%%%%%%%%%%%%%%%%%%%%%%%%%%%%%%%
\begin{frame}{Convex Functions}

\plitemsep 0.1in

\bci 

\item $f:\realn \mapsto \real$ is a {\blue convex function} if
$\text{dom} \ f$ is a convex set and for all $x,y\in \text{dom} \ f$ and
$\theta\in[0,1]$, we have
\[
f(\theta x+(1-\theta)y) \leq \theta f(x) + (1-\theta)f(y)
\]

\item $f$ is {\blue strictly convex} if the strict inequality in the above holds for all
$x\neq y$ and $0<\theta<1.$

\eci
\mysmalltwocols{0.35}
{
\bci

\item $f$ is {\blue concave} if $-f$ is convex


\item Affine functions are convex and concave


\item \bluef{Jensen's inequality.} For a rv $X,$ $f(\expect{X}) \leq \expect{f(X)}.$

\eci
}
{
\vspace{-0.4cm}
\mypic{0.8}{L7_convex_fn.png}
}

\end{frame}



%%%%%%%%%%%%%%%%%%%%%%%%%%%%%%%%%%%%%%%%%%%%%%%%%%%%%%
\begin{frame}{Conditions of Convex Functions (1)}

\plitemsep 0.1in

\bci 

\item \bluef{First-order condition.} For differentiable functions, $f$ is convex iff
\[
f(y)-f(x)\geq \trans{\grad f(x)}(y-x), \quad \forall x,y\in \text{dom} \ f, \text{and} \ \text{dom} \ f \ \text{is convex}
\]

\mypic{0.35}{L7_first_condition.png}

\item \exam $f(y) = y^2.$

\item $f(y)\geq \tilde{f}_{x}(y)$ where $\tilde{f}_{x}(y)$ is
  the first order Taylor expansion of $f(y)$ at $x$.

\item {\red Local} information (first order Taylor
  approximation) about a convex function provides {\red global}
  information (global underestimator).
\item If $\grad f(x)=0$, then $f(y)\geq f(x), \; \forall y.$ Thus, $x$ is a global minimizer of $f$

\eci
\end{frame}



%%%%%%%%%%%%%%%%%%%%%%%%%%%%%%%%%%%%%%%%%%%%%%%%%%%%%%
\begin{frame}{Conditions of Convex Functions (2)}

\plitemsep 0.1in

\bci 

\item \bluef{Second-order condition.} For twice differentiable functions, $f$ is convex iff
\[
\grad^{2}f(x)\succeq 0
\]
for all $x\in\text{dom} \ f$ (upward slope) and $\text{dom} \ f$ is convex


\item Example: $f(x) = x^2.$
% 3. $f$ is convex iff for all $x\in\dom f$ and all $v$,
% \[
% g(t)=f(x+tv)
% \]
% is convex on its domain $\{t\in\reals|x+tv\in\dom f\}$

\item Meaning: The graph of the function have positive (upward) curvature at $x.$

\eci

\end{frame}



%%%%%%%%%%%%%%%%%%%%%%%% Slide 1 %%%%%%%%%%%%%%%%%%%%%%%%%%%%

\begin{frame}{Examples of Convex or Concave Functions}

\plitemsep 0.1in

\bci 

\item $e^{ax}$ is convex on $\real$, for any $a\in\real$

\item $x^{a}$ is convex on $\real_{++}$ when $a\geq 1$ or
$a\leq 0$, and concave for $0\leq a\leq 1$

\item $|x|^{p}$ is convex on $\real$ for $p\geq 1$

\item $\log x$ is concave on $\real_{++}$

\item $x\log x$ is strictly convex on $\real_{++}$

\item Every norm on $\realn$ is convex

\item $f(x)=\max\{x_{1},\ldots,x_{n}\}$ is convex on
$\realn$

\item $f(x)=\log\sum_{i=1}^{n}e^{x_{i}}$ is convex on
$\realn$

\item $f(x)=\left(\prod_{i=1}^{n}x_{i}\right)^{\frac{1}{n}}$
is concave on $\real_{++}^{n}$

\eci
\end{frame}


%%%%%%%%%%%%%%%%%%%%%%%% Slide 1 %%%%%%%%%%%%%%%%%%%%%%%%%%%%

\begin{frame}{Convexity-Preserving Operations}

\plitemsep 0.1in

\bci 

\item $f=\sum_{i=1}^{n}w_{i}f_{i}$ convex if $f_{i}$ are all
convex and $w_{i}\geq 0$

\item $g(x)=f(Ax+b)$ is convex iff $f(x)$ is convex

\item $f(x)=\max\{f_{1}(x),f_{2}(x)\}$ convex if $f_{i}$
convex, e.g., sum of $r$ largest components is convex

\item $f(x)=h(g(x))$, where $h:\real^{k} \mapsto \real$
and $g:\realn \mapsto \real^{k}$.

\medskip
If $k=1$: $f''(x) = h''(g(x))g'(x)^{2} + h'(g(x))g''(x)$. So,
$f$ is convex if $h$ is convex and nondecreasing and $g$ is
convex, or if $h$ is convex and nonincreasing and $g$ is concave
...

\item $g(x)=\inf_{y\in \set{C}} f(x,y)$ is convex if $f$ is convex in $(x,y)$
and $\set{C}$ is convex

\eci
\end{frame}

%%%%%%%%%%%%%%%%%%%%%%%% Slide 1 %%%%%%%%%%%%%%%%%%%%%%%%%%%%

\begin{frame}{Point-wise Supremum}

\plitemsep 0.1in

\bci 

\item If $f(x,y)$ is convex in $x$ for each $y \in \set{A},$ then
$$g(x) = \sup_{y \in \set{A}} f(x,y)$$
is \bluef{convex}. Similarly, if $f(x,y)$ is concave in $x$ for each $y \in \set{A},$ then
$$g(x) = \inf_{y \in \set{A}} f(x,y)$$
is \bluef{concave}.

\item \exam distance to farthest point in a set $\set{C}$: $f(x) = \sup_{y \in \set{C}} || x-y ||$ is \bluef{convex}.

\item \exam Lagrange dual function 
$$\cD(\vlam,\vnu)=\inf_{\vx}
\cL(\vx,\vlam,\vnu)$$
is \bluef{concave}. 
\eci
\end{frame}

%%%%%%%%%%%%%%%%%%%%%%%%%%%%%%%%%%%%%%%%%%%%%%%%%%%%%%
\section{L7(4)}
\begin{frame}{Roadmap}

\plitemsep 0.1in

\bce[(1)] 

\item \grayf{Optimization Using Gradient Descent} 
\item \grayf{Constrained Optimization and Lagrange Multipliers} 

\item \grayf{Convex Sets and Functions}
\item \redf{Convex Optimization} 
\item \grayf{Convex Conjugate}


\ece
\end{frame}

%%%%%%%%%%%%%%%%%%%%%%%% Slide 1 %%%%%%%%%%%%%%%%%%%%%%%%%%%%

\begin{frame}{Standard Convex Optimization}

\plitemsep 0.1in

\bci 

\item A {\blue standard convex optimization} problem with variables $\vx$:
\[
\begin{array}{ll}
\mbox{minimize} & f(\vx)\\
\mbox{subject to} & g_{i}(\vx)\leq 0, \;\; i=1,2,\ldots,m\\
& \trans{a_{i}}\vx= b_{i}, \;\; i=1,2,\ldots,p
\end{array}
\]
where $f,f_{1},\ldots,f_{m}$ are convex functions.

\item {\red Minimize} {\blue convex} objective function (or
maximize concave objective function)

\item {\red Upper bound inequality} constraints on {\blue
convex} functions ($\Rightarrow$ Constraint set is convex)

\item {\red Equality} constraints must be {\blue affine} (Only affine functions leads to a convex set for the 
equality constraints)
\eci
\end{frame}

%%%%%%%%%%%%%%%%%%%%%%%%%%%%%%%%%%%%%%%%%%%%%%%%%%%%

\begin{frame}{Properties for Optimality}

\plitemsep 0.05in

\bci 

\item \bluef{Local optimality implies global optimality.}

\bci

\item Given $x$ is {\red locally optimal} for a convex optimization problem, i.e., 
$\vx$ is feasible and for some $R>0$,
\[
f(\vx) = \inf\{f(\vz) \mid \vz \; \text{is feasible}\;,
\|\vz-\vx\|_{2}\leq R\}
\]

\item {\red Theorem.} if $\vx$ is locally optimal in convex program, then
  globally optimal.

\eci

\item \bluef{Optimal condition for differentiable $f$}

\bci
\item $\vx$ is optimal for a convex optimization problem iff $\vx$ is
feasible and for all feasible $\vy$:
\[
\grad \trans{f(\vx)}(\vy-\vx)\geq 0
\]

\item $-\grad f(\vx)$ defines a supporting hyperplane to the feasible set
($\{ \vy \mid - \grad \trans{f(\vx)}\vy \leq - \grad \trans{f(\vx)}\vx \}$). 

\item (Note) Unconstrained convex optimization: condition reduces to: $\grad f(\vx)=0$

\eci
\eci
\end{frame}

%%%%%%%%%%%%%%%%%%%%%%%%%%%%%%%%%%%%%%%%%%%%%%%%%%%%
\begin{frame}{Strong Duality}

\plitemsep 0.1in

\bci 

\item {\blue Strong duality} (zero optimal duality gap):
\[
d^{*}=p^{*}
\]

\item If strong duality holds, solving dual is `equivalent' to solving
primal. But strong duality does {\red not} always hold

\item Convexity and {\blue constraint qualifications} $\implies$
Strong duality

\item A simple constraint qualification: {\blue Slater's condition}
(there exists strictly feasible primal variables $f_{i}(\vx)<0$ for
non-affine $f_{i}$) (see Section 5.3.2 of Boyd's book). 

\item Another reason why convex optimization is `easy'

\eci
\end{frame}


%%%%%%%%%%%%%%%%%%%%%%%%%%%%%%%%%%%%%%%%%%%%%%%%%%%%
\begin{frame}{Complementary Slackness}

\plitemsep 0.05in

\bci 

\item Assume {\red strong duality} holds. Then, 
\aleq{
f(\vx^{*}) & =  \cD(\vlam^{*},\vnu^{*}) =  \inf_{\vx}\left(f(\vx) + \sum_{i=1}^{m}\lambda_{i}^{*}f_{i}(\vx)
+ \sum_{i=1}^{p}\nu_{i}^{*}h_{i}(\vx) \right)\\
& \leq  f(\vx^{*}) + \sum_{i=1}^{m}\lambda_{i}^{*}f_{i}(\vx^{*})
+ \sum_{i=1}^{p}\nu_{i}^{*}h_{i}(\vx^{*}) \leq  f(\vx^{*})
}

\item Thus, the two inequalities must hold with equality, implying:
$
\lambda_{i}^{*}f_{i}(\vx^{*}) = 0, \ \forall i
$

\item \bluef{Complementary slackness} condition:
\aleq{
\lambda_{i}^{*}>0 & \implies  f_{i}(\vx^{*})=0 \cr
f_{i}(\vx^{*})<0 & \implies  \lambda_{i}^{*}=0
}

\item $i$-th optimal Lagrange multiplier is zero unless the $i$th constraint is
active at the optimum. 

\eci

\end{frame}

%%%%%%%%%%%%%%%%%%%%%%%%%%%%%%%%%%%%%%%%%%%%%%%%%%%%
\begin{frame}{KKT Condition}

\plitemsep 0.05in
\bci 
% Assume that all the functions in the objective and the constraint are
% differentiable. 

\item Since $\vx^{*}$ minimizes $\cL(\vx,\vlam^{*},\vnu^{*})$ over $\vx$, 
\[
\grad f(\vx^{*}) + \sum_{i=1}^{m}\lambda_{i}^{*}\nabla
f_{i}(\vx^{*}) + \sum_{i=1}^{p}\nu_{i}^{*}\grad h_{i}(\vx^{*}) = 0
\]
\vspace{-0.3cm}
\myblock{Karush-Kuhn-Tucker optimality condition}
{
\vspace{-0.3cm}
\begin{eqnarray*}
& f_{i}(\vx^{*})\leq 0, \;\; h_{i}(\vx^{*})=0, \;\;
\lambda_{i}^{*}\succeq 0 & \\
& \lambda_{i}^{*}f_{i}(\vx^{*}) = 0 &\\
& \grad f(\vx^{*}) + \sum_{i=1}^{m}\lambda_{i}^{*}\grad f_{i}(\vx^{*}) + \sum_{i=1}^{p}\nu_{i}^{*}\grad h_{i}(\vx^{*}) = 0 &
\end{eqnarray*}
\vspace{-0.3cm}
}

\item {\red Any} optimization  with strong duality, KKT condition is necessary for primal-dual optimality

\item {\red Convex} optimization with Slater's condition, KKT is
also {\red sufficient} for primal-dual optimality. 

\eci

\end{frame}

%%%%%%%%%%%%%%%%%%%%%%%%%%%%%%%%%%%%%%%%%%%%%%%%%%%%
\begin{frame}{Useful Tips}

\plitemsep 0.03in
\bci 
\item Minimization problem ({\red min-min-max} rule)
\bci

  \item Problem: $\text{min }f(x) \quad \text{s.t.}  \quad f_i(x) \leq 0, \quad
      g_i(x)  = 0, \quad x$ 
  \item $f(x)$: convex, $f_i(x)$: convex, $g_i(x)$: affine \medskip
  \item $L(x,\lambda,\mu) = f(x) + \sum_i \lambda_i f_i(x) + \sum_i
    \mu_i g_i(x)$
  \item $\inf_{x} L(x,\lambda,\mu) = \cD(\lambda,\mu)$
  \item  $\max_{\lambda \geq 0} \cD(\lambda,\mu)$

\eci

\item Maximization problem ({\red max-max-min} rule)
  
\bci
  \item Problem: $\text{max } f(x) \quad \text{s.t.}  \quad f_i(x) \geq 0, \quad
      g_i(x)  = 0, \quad x$ 
  \item $f(x)$: concave, $f_i(x)$: concave, $g_i(x)$: affine \medskip
  \item $L(x,\lambda,\mu) = f(x) + \sum_i \lambda_i f_i(x) + \sum_i
    \mu_i g_i(x)$
  \item $\sup_{x} L(x,\lambda,\mu) = \cD(\lambda,\mu)$
  \item  $\min_{\lambda \geq 0} \cD(\lambda,\mu)$
\eci

\eci

\end{frame}

%%%%%%%%%%%%%%%%%%%%%%%%%%%%%%%%%%%%%%%%%%%%%%%%%%%%
\begin{frame}{Linear Programming}

\mysmalltwocols{0.35}
{
- Primal problem

$$
\begin{array}{ll}
\mbox{min}_{\vx \in \real^d} & \trans{\vc}\vx \\
\mbox{subject to} & \mA \vx \preceq \vb,  
\end{array}
$$
where $\mA \in \real^{m \times d}$ and $\vb \in \realm.$
}
{
- Dual problem

$$
\displaystyle
\begin{array}{ll}
\text{max}_{\vlam \in \real^m} & -\trans{\vb}\vlam \\
\mbox{subject to} & \vc + \trans{\mA}\vlam = \vec{0}, \ \vlam \succeq \vec{0},
\end{array}
$$
where $\vlam \in \realm.$
}

\plitemsep 0.1in
\bci 

\item The Lagrangian: $\cL(\vx, \vlam) = \trans{(\vc + \trans{\mA}\vlam)}\vx - \trans{\vlam} \vb,$ 
whose derivative w.r.t. $\vx$ becomes zero, when $\vc + \trans{\mA}\vlam = \vec{0}.$

\item The dual function: $\cD(\vlam) = -\trans{\vlam} \vb$
\eci

\end{frame}

%%%%%%%%%%%%%%%%%%%%%%%%%%%%%%%%%%%%%%%%%%%%%%%%%%%%
\begin{frame}{Quadratic Programming}

\plitemsep 0.1in
\bci 

\item Primal problem
$$\begin{array}{ll}
\mbox{min}_{\vx \in \real^d} & \dfrac{1}{2} \trans{\vx}\mQ \vx + \trans{c}\vx \\
\mbox{subject to} & \mA \vx \preceq \vb,  
\end{array}
$$
where $\mA \in \real^{m \times d},$ $\vb \in \realm,$ $\vc \in \real^d,$ the square matrix $\mQ$ is symmetric, positive definite. 

\item Dual problem
$$
\begin{array}{ll}
\text{max}_{\vlam \in \real^m} & \Bigg(-\dfrac{1}{2}\trans{(\vc + \trans{\mA}\vlam)}\mA\inv{\mQ}(\vc + \trans{\mA}\vlam) -\trans{\vlam}\vb \Bigg )\\
% & -\trans{\vlam}\vb \Bigg )\\
\text{subject to} & \vlam \succeq \vec{0},
\end{array}
$$

where $\vlam \in \realm.$

\eci

\end{frame}

%%%%%%%%%%%%%%%%%%%%%%%%%%%%%%%%%%%%%%%%%%%%%%%%%%%%%%
\section{L7(5)}
\begin{frame}{Roadmap}

\plitemsep 0.1in

\bce[(1)] 

\item \grayf{Optimization Using Gradient Descent} 
\item \grayf{Constrained Optimization and Lagrange Multipliers} 

\item \grayf{Convex Sets and Functions}
\item \grayf{Convex Optimization} 
\item \redf{Convex Conjugate}


\ece
\end{frame}


%%%%%%%%%%%%%%%%%%%%%%%%%%%%%%%%%%%%%%%%%%%%%%%%%%%%%%
\begin{frame}{Conjugate Function: Definition and Meaning}

\plitemsep 0.1in

\bci 

\item Given $f: \realn \mapsto \real$, the {\blue conjugate function}
$f^{*}:\realn \mapsto \real$ defined as:
\[
f^{*}(\vy) = \sup_{\vx\in\text{dom} \ f}(\trans{\vy}\vx-f(\vx))
\]
with domain consisting of $\vy\in\realn$ for which the supremum
is finite

\eci

\mysmalltwocols{0.5}
{
\small
\bci
\item Assume $\real^1.$
\item For a given slope of $y,$ $yx-f(x)$ is the vertical distance between the line $yx$ and $f(x).$
\item Thus, $f^*(y)$ is the maximum distance
\eci
}
{
\mypic{0.9}{L7_convex_conjugate.png}
}

\end{frame}

%%%%%%%%%%%%%%%%%%%%%%%%%%%%%%%%%%%%%%%%%%%%%%%%%%%%%%
\begin{frame}{Conjugate Function: Properties}

\plitemsep 0.1in

\bci 

\item 
\grayf{Given $f: \realn \mapsto \real$, the conjugate function
$f^{*}:\realn \mapsto \real$ defined as:
\[
f^{*}(\vy) = \sup_{\vx\in\text{dom} \ f}(\trans{\vy}\vx-f(\vx))
\]
with domain consisting of $\vy\in\realn$ for which the supremum
is finite}

\item $f^{*}(\vy)$:  {\red always convex} (the pointwise
supremum of a family of affine functions of $\vy$)

\item $f^{*}=f$ if $f$ is convex and closed

\item {\blue Fenchel's inequality}: $f(\vx)+f^{*}(\vy)\geq \trans{\vx} \vy$
for all $\vx,\vy$ (by definition)
\bci
\item \exam $f(x) = |x|^2/2$. Then, $f^*(y) = |y|^2/2.$ Thus, F-inequality tells us: 
$$
\frac{1}{2}(|x|^2 + |y|^2) \ge xy
$$
\eci

\eci
\end{frame}

%%%%%%%%%%%%%%%%%%%%%%%%%%%%%%%%%%%%%%%%%%%%%%%%%%%%%%
\begin{frame}{Examples of Conjugate Functions}

\plitemsep 0.1in

\bci 

\item $f(x)=ax+b$, $f^{*}(a)=-b$

\item $f(x)=-\log x$, $f^{*}(y)=-\log(-y)-1$ for $y<0$

\item $f(x)=e^{x}$, $f^{*}(y)=y\log y-y$

\item $f(x)=x\log x$, $f^{*}(y)=e^{y-1}$

\item $f(x)=\frac{1}{2}\trans{x}Qx$,
$f^{*}(y)=\frac{1}{2}\trans{y}Q^{-1}y$ ($Q$ is positive definite)

\item $f(x)=\log\sum_{i=1}^{n}e^{x_{i}}$,
\aleq{
f^{*}(y) = \begin{cases}
\sum_{i=1}^{n}y_{i}\log y_{i} & \text{if} \ y\succeq 0 \ \text{and} \ \sum_{i=1}^{n}y_{i}=1, \cr
\infty & \text{otherwise}
\end{cases}
}

\eci
\end{frame}

%%%%%%%%%%%%%%%%%%%%%%%%%%%%%%%%%%%%%%%%%%%%%%%%%%%%%%
\begin{frame}{Conjugate Function and Lagrangian Dual Function}

\plitemsep 0.1in

\bci 

\item They are closely related. Consider the following problem:
\[
\begin{array}{ll}
\mbox{minimize} & f(\vx)\\
\mbox{subject to} & \mA \vx \preceq \vb, \\
& \mC \vx = \vd
\end{array}
\]

\item Using the conjugate of $f,$ we can write the dual function as:
\aleq{
\cD(\vlam,\vnu) &= \inf_{\vx}\left( f(\vx) + \trans{\vlam}(\mA \vx - \vb) + 
\trans{\vnu}(\mC \vx - \vd) \right) \cr
&= -\trans{\vb}\vlam - \trans{\vd}\vnu + \inf_{\vx} \left(f(\vx) + 
\trans{(\trans{\mA}\vlam + \trans{\mC} \vnu)}\vx \right )\cr
&= -\trans{\vb}\vlam - \trans{\vd}\vnu -f^{*}\left(-\trans{\mA}\vlam
- \trans{\mC}\vnu \right)
}
\eci
\end{frame}


%%%%%%%%%%%%%%%%%%%%%%%%%%%%%%%%%%%%%%%%%%%%%%%%%%%%%%
\begin{frame}{}
\vspace{2cm}
\LARGE Questions?


\end{frame}

%%%%%%%%%%%%%%%%%%%%%%%%%%%%%%%%%%%%%%%%%%%%%%%%%%%%%%
\begin{frame}{Review Questions}
% \tableofcontents
%\plitemsep 0.1in
\bce[1)]
\item 

\ece
\end{frame}


\end{document}
