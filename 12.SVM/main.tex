%\pdfminorversion=4
\documentclass[handout,fleqn,aspectratio=169]{beamer}

% when making printed slides
\usepackage{pgfpages}
\pgfpagesuselayout{resize to}[a4paper,landscape,border shrink=5mm]

\usepackage[english]{babel}
\usepackage{tikz}
\usepackage{courier}
\usepackage{array}
\usepackage{bold-extra}
%\usepackage{minted}
\usepackage[thicklines]{cancel}
\usepackage{fancyvrb}
\usepackage{kotex}
\usepackage{paralist}
\usepackage{collectbox}
\usepackage{bm}

\usepackage{mathrsfs}
\usepackage[reqno,disallowspaces]{mathtools}  % imports amsmath
\usepackage{amsfonts} %for Y&Y BSR AMS fonts
\usepackage{amssymb}
\usepackage{amscd}
%\usepackage{tikz,lipsum,lmodern}
\usepackage[most]{tcolorbox}
\usepackage{verbatim}
\mode<presentation>
{
  \usetheme{default}
  \usecolortheme{default}
  \usefonttheme{default}
  \setbeamertemplate{navigation symbols}{}
  \setbeamertemplate{caption}[numbered]
  \setbeamertemplate{footline}[frame number]  % or "page number"
  \setbeamercolor{frametitle}{fg=yellow}
  \setbeamercolor{footline}{fg=black}
} 

\setbeamercolor{block body alerted}{bg=alerted text.fg!10}
\setbeamercolor{block title alerted}{bg=alerted text.fg!20}
\setbeamercolor{block body}{bg=structure!10}
\setbeamercolor{block title}{bg=structure!20}
\setbeamercolor{block body example}{bg=green!10}
\setbeamercolor{block title example}{bg=green!20}
\setbeamertemplate{blocks}[rounded][shadow]

\xdefinecolor{dianablue}{rgb}{0.18,0.24,0.31}
\xdefinecolor{darkblue}{rgb}{0.1,0.1,0.7}
\xdefinecolor{darkgreen}{rgb}{0,0.5,0}
\xdefinecolor{darkgrey}{rgb}{0.35,0.35,0.35}
\xdefinecolor{darkorange}{rgb}{0.8,0.5,0}
\xdefinecolor{darkred}{rgb}{0.7,0,0}
\definecolor{darkgreen}{rgb}{0,0.6,0}
\definecolor{mauve}{rgb}{0.58,0,0.82}

\usetikzlibrary{shapes.callouts}

\makeatletter
\setbeamertemplate{footline}
{
  \leavevmode%
  \hbox{%
  \begin{beamercolorbox}[wd=.333333\paperwidth,ht=2.25ex,dp=1ex,center]{author in head/foot}%
    \usebeamerfont{author in head/foot}\insertsection
  \end{beamercolorbox}%
  \begin{beamercolorbox}[wd=.333333\paperwidth,ht=2.25ex,dp=1ex,center]{title in head/foot}%
    \usebeamerfont{title in head/foot}\insertsubsection
  \end{beamercolorbox}%
  \begin{beamercolorbox}[wd=.333333\paperwidth,ht=2.25ex,dp=1ex,right]{date in head/foot}%
    \usebeamerfont{date in head/foot}
    \insertshortdate{}\hspace*{2em}
    \insertframenumber{} / \inserttotalframenumber\hspace*{2ex} 
  \end{beamercolorbox}}%

  \vskip0pt%
}
\makeatother

\title[]{Lecture 12: Classification with Support Vector Machines}
\author{Yi, Yung (이융)}
\institute{Mathematics for Machine Learning\\ \url{https://yung-web.github.io/home/courses/mathml.html}
\\KAIST EE}
\date{\today}


%%%%%%%%%%%% real, integer notation
\newcommand{\real}{{\mathbb R}}
\newcommand{\realn}{{\mathbb R}^{n}}
\newcommand{\realm}{{\mathbb R}^{m}}
\newcommand{\realD}{{\mathbb R}^{D}}
\newcommand{\realM}{{\mathbb R}^{M}}
\newcommand{\realN}{{\mathbb R}^{N}}
\newcommand{\realnn}{{\mathbb R}^{n \times n}}
\newcommand{\realmm}{{\mathbb R}^{m \times m}}
\newcommand{\realmn}{{\mathbb R}^{m \times n}}
\newcommand{\realnm}{{\mathbb R}^{n \times m}}
\newcommand{\realDM}{{\mathbb R}^{D \times M}}
\newcommand{\realMD}{{\mathbb R}^{M \times D}}
\newcommand{\complex}{{\mathbb C}}
\newcommand{\integer}{{\mathbb Z}}
\newcommand{\natu}{{\mathbb N}}


%%% set, vector, matrix
\newcommand{\set}[1]{\ensuremath{\mathcal #1}}
\newcommand{\sets}[1]{\ensuremath{\{#1 \}}}
\renewcommand{\vec}[1]{\bm{#1}}
\newcommand{\mat}[1]{\bm{#1}}

%%%% vector
\def\vx{\vec{x}}
\def\vy{\vec{y}}
\def\vz{\vec{z}}
\def\vf{\vec{f}}
\def\ve{\vec{e}}
\def\vr{\vec{r}}
\def\vb{\vec{b}}
\def\vc{\vec{c}}
\def\vd{\vec{d}}
\def\vm{\vec{m}}
\def\vu{\vec{u}}
\def\vv{\vec{v}}
\def\vw{\vec{w}}
\def\vX{\vec{X}}
\def\vY{\vec{Y}}
\def\vZ{\vec{Z}}
\def\vth{\vec{\theta}}
\def\vmu{\vec{\mu}}
\def\vnu{\vec{\nu}}
\def\vlam{\vec{\lambda}}
\def\vep{\vec{\epsilon}}
\def\vpi{\vec{\pi}}
\def\vphi{\vec{\phi}}
\def\vxi{\vec{\xi}}
\def\valpha{\vec{\alpha}}
\def\vgamma{\vec{\gamma}}

%%%% Well-used matrices
\def\mA{\mat{A}}
\def\mB{\mat{B}}
\def\mC{\mat{C}}
\def\mD{\mat{D}}
\def\mI{\mat{I}}
\def\mJ{\mat{J}}
\def\mK{\mat{K}}
\def\mE{\mat{E}}
\def\mP{\mat{P}}
\def\mQ{\mat{Q}}
\def\mU{\mat{U}}
\def\mV{\mat{V}}
\def\mR{\mat{R}}
\def\mS{\mat{S}}
\def\mX{\mat{X}}
\def\msig{\mat{\Sigma}}
\def\mPhi{\mat{\Phi}}


\usepackage{amsmath}
%%%%% vector caculus useful macro
% ...\d, which typesets a derivative. ex: \d{y}{x}, instead of \frac{dx}{dy}.
\renewcommand{\d}[2]{\frac{\text{d} #1}{\text{d} #2}}


% ...similar for double-derivatives. ex: \dd{y}{x}.
\newcommand{\dd}[2]{\frac{\text{d}^2 #1}{\text{d} #2^2}}

% ...similar for partial derivatives. ex: \pd{y}{x}.
\newcommand{\pd}[2]{\frac{\partial #1}{\partial #2}}


% ...similar for partial double derivatives. ex: \pdd{y}{x}.
\newcommand{\pdd}[2]{\frac{\partial^2 #1}{\partial #2^2}}
% pdd with argument
\newcommand{\pdda}[3]{\frac{\partial^2 #1}{\partial #2 \partial #3}}

\usepackage{xparse}

%%%% caligraphic fonts
\def\cL{\ensuremath{{\cal L}}}
\def\cN{\ensuremath{{\cal N}}}
\def\cD{\ensuremath{{\cal D}}}
\def\cC{\ensuremath{{\cal C}}}
\def\cX{\ensuremath{{\cal X}}}
\def\cY{\ensuremath{{\cal Y}}}

%%% big parenthesis
\def\Bl{\Bigl}
\def\Br{\Bigr}
\def\lf{\left}
\def\ri{\right}


%%% floor notations
\newcommand{\lfl}{{\lfloor}}
\newcommand{\rfl}{{\rfloor}}
\newcommand{\floor}[1]{{\lfloor #1 \rfloor}}

%%% gradient
\newcommand{\grad}[1]{\nabla #1}
\newcommand{\hess}[1]{\text{H} #1}

%%% definition
%\newcommand{\eqdef}{\ensuremath{\triangleq}}
\newcommand{\eqdef}{\ensuremath{:=}}
%%% imply
\newcommand{\imp}{\Longrightarrow}



\newcommand{\separator}{
%  \begin{center}
    \par\noindent\rule{\columnwidth}{0.3mm}
%  \end{center}
}

\newcommand{\mynote}[1]{{\it \color{red} [#1]}}







%%% equation alignment
\newcommand{\aleq}[1]{\begin{align*}#1\end{align*}}

%%%%%%%%%%%%%%%% colored emphasized font, blanked words

\newcommand{\empr}[1]{{\color{red}\emph{#1}}}
\newcommand{\empb}[1]{{\color{blue}\emph{#1}}}
\newcommand{\redf}[1]{{\color{red} #1}}
\newcommand{\bluef}[1]{{\color{blue} #1}}
\newcommand{\grayf}[1]{{\color{gray} #1}}
\newcommand{\magenf}[1]{{\color{magenta} #1}}
\newcommand{\greenf}[1]{{\color{green} #1}}
\newcommand{\cyanf}[1]{{\color{cyan} #1}}
\newcommand{\orangef}[1]{{\color{orange} #1}}

\newcommand{\blk}[1]{\underline{\mbox{\hspace{#1}}}}


\newcommand{\redblk}[1]{\framebox{\color{red} #1}}
\newcommand{\redblank}[2]{\framebox{\onslide<#1->{\color{red} #2}}}
\newcommand{\blueblk}[1]{\framebox{\color{blue} #1}}
\newcommand{\blueblank}[2]{\framebox{\onslide<#1->{\color{blue} #2}}}



\makeatletter
\newcommand{\mybox}{%
    \collectbox{%
        \setlength{\fboxsep}{1pt}%
        \fbox{\BOXCONTENT}%
    }%
}
\makeatother

\makeatletter
\newcommand{\lecturemark}{%
    \collectbox{%
        \setlength{\fboxsep}{1pt}%
        \fcolorbox{red}{yellow}{\BOXCONTENT}%
    }%
}
\makeatother

\newcommand{\mycolorbox}[1]{
\begin{tcolorbox}[colback=red!5!white,colframe=red!75!black]
#1
\end{tcolorbox}
}
%%%% figure inclusion
\newcommand{\mypic}[2]{
\begin{center}
\includegraphics[width=#1\textwidth]{#2}
\end{center}
}

\newcommand{\myinlinepic}[2]{
\makebox[0cm][r]{\raisebox{-4ex}{\includegraphics[height=#1]{#2}}}
}




%%%% itemized and enumerated list
\newcommand{\bci}{\begin{compactitem}}
\newcommand{\eci}{\end{compactitem}}
\newcommand{\bce}{\begin{compactenum}}
\newcommand{\ece}{\end{compactenum}}


%%%% making 0.5/0.5 two columns
%%%% how to use: first number: length of separation bar
% \mytwocols{0.6}
% {
% contents in the left column
% }
% {
% contents in the right column
% }
%%%%

\newcommand{\mytwocols}[3]{
\begin{columns}[T] \column{.499\textwidth} #2 \column{.001\textwidth} \rule{.3mm}{{#1}\textheight} \column{.499\textwidth} #3 \end{columns}}

\newcommand{\mythreecols}[4]{
\begin{columns}[T] \column{.31\textwidth} #2 \column{.001\textwidth} \rule{.3mm}{{#1}\textheight} \column{.31\textwidth} #3 \column{.001\textwidth} \rule{.3mm}{{#1}\textheight} \column{.31\textwidth} #4  \end{columns}}

\newcommand{\mysmalltwocols}[3]{
\begin{columns}[T] \column{.4\textwidth} #2 \column{.001\textwidth} \rule{.3mm}{{#1}\textheight} \column{.4\textwidth} #3 \end{columns}}

%%%% making two columns with customized ratios
%%%% how to use: 
%first parameter: length of separation bar
%second parameter: ratio of left column
%third parameter: ratio of right column
% \mytwocols{0.6}{0.7}{0.29}
% {
% contents in the left column
% }
% {
% contents in the right column
% }
%%%%
\newcommand{\myvartwocols}[5]{
\begin{columns}[T] \column{#2\textwidth} {#4} \column{.01\textwidth} \rule{.3mm}{{#1}\textheight} \column{#3\textwidth} {#5} \end{columns}}

%%% making my block in beamer
%%% first parameter: title of block
%%% second parameter: contents of block
\newcommand{\myblock}[2]{
\begin{block}{#1} {#2}  \end{block}}

%%% independence notation
\newcommand{\indep}{\perp \!\!\! \perp}

%%%% probability with different shapes (parenthesis or bracket) and different sizes
%%% `i' enables us to insert the subscript to the probability 
\newcommand{\bprob}[1]{\mathbb{P}\Bl[ #1 \Br]}
\newcommand{\prob}[1]{\mathbb{P}[ #1 ]}
\newcommand{\cbprob}[1]{\mathbb{P}\Bl( #1 \Br)}
\newcommand{\cprob}[1]{\mathbb{P}( #1 )}
\newcommand{\probi}[2]{\mathbb{P}_{#1}[ #2 ]}
\newcommand{\bprobi}[2]{\mathbb{P}_{#1}\Bl[ #2 \Br]}
\newcommand{\cprobi}[2]{\mathbb{P}_{#1}( #2 )}
\newcommand{\cbprobi}[2]{\mathbb{P}_{#1}\Bl( #2 \Br)}

%%%% expectation with different shapes (parenthesis or bracket) and different sizes
%%% `i' enables us to insert the subscript to the expectation
\newcommand{\expect}[1]{\mathbb{E}[ #1 ]}
\newcommand{\cexpect}[1]{\mathbb{E}( #1 )}
\newcommand{\bexpect}[1]{\mathbb{E}\Bl[ #1 \Br]}
\newcommand{\cbexpect}[1]{\mathbb{E}\Bl( #1 \Br)}
\newcommand{\bbexpect}[1]{\mathbb{E}\lf[ #1 \ri]}
\newcommand{\expecti}[2]{\mathbb{E}_{#1}[ #2 ]}
\newcommand{\bexpecti}[2]{\mathbb{E}_{#1}\Bl[ #2 \Br]}
\newcommand{\bbexpecti}[2]{\mathbb{E}_{#1}\lf[ #2 \ri]}

%%%% variance
\newcommand{\var}[1]{\text{var}[ #1 ]}
\newcommand{\bvar}[1]{\text{var}\Bl[ #1 \Br]}
\newcommand{\cvar}[1]{\text{var}( #1 )}
\newcommand{\cbvar}[1]{\text{var}\Bl( #1 \Br)}

%%%% covariance
\newcommand{\cov}[1]{\text{cov}( #1 )}
\newcommand{\bcov}[1]{\text{cov}\Bl( #1 \Br)}

%%% Popular pmf, pdf notation to avoid long typing
\newcommand{\px}{\ensuremath{p_X(x)}}
\newcommand{\py}{\ensuremath{p_Y(y)}}
\newcommand{\pz}{\ensuremath{p_Z(z)}}
\newcommand{\pxA}{\ensuremath{p_{X|A}(x)}}
\newcommand{\pyA}{\ensuremath{p_{Y|A}(y)}}
\newcommand{\pzA}{\ensuremath{p_{Z|A}(z)}}
\newcommand{\pxy}{\ensuremath{p_{X,Y}(x,y)}}
\newcommand{\pxcy}{\ensuremath{p_{X|Y}(x|y)}}
\newcommand{\pycx}{\ensuremath{p_{Y|X}(y|x)}}

\newcommand{\fx}{\ensuremath{f_X(x)}}
\newcommand{\Fx}{\ensuremath{F_X(x)}}
\newcommand{\fy}{\ensuremath{f_Y(y)}}
\newcommand{\Fy}{\ensuremath{F_Y(y)}}
\newcommand{\fz}{\ensuremath{f_Z(z)}}
\newcommand{\Fz}{\ensuremath{F_Z(z)}}
\newcommand{\fxA}{\ensuremath{f_{X|A}(x)}}
\newcommand{\fyA}{\ensuremath{f_{Y|A}(y)}}
\newcommand{\fzA}{\ensuremath{f_{Z|A}(z)}}
\newcommand{\fxy}{\ensuremath{f_{X,Y}(x,y)}}
\newcommand{\Fxy}{\ensuremath{F_{X,Y}(x,y)}}
\newcommand{\fxcy}{\ensuremath{f_{X|Y}(x|y)}}
\newcommand{\fycx}{\ensuremath{f_{Y|X}(y|x)}}

\newcommand{\fth}{\ensuremath{f_\Theta(\theta)}}
\newcommand{\fxcth}{\ensuremath{f_{X|\Theta}(x|\theta)}}
\newcommand{\fthcx}{\ensuremath{f_{\Theta|X}(\theta|x)}}

\newcommand{\pkcth}{\ensuremath{p_{X|\Theta}(k|\theta)}}
\newcommand{\fthck}{\ensuremath{f_{\Theta|X}(\theta|k)}}


%%%% indicator
\newcommand{\indi}[1]{\mathbf{1}_{ #1 }}

%%%% exponential rv.
\newcommand{\elambdax}{\ensuremath{e^{-\lambda x}}}

%%%% normal  rv.
\newcommand{\stdnormal}{\ensuremath{\frac{1}{\sqrt{2\pi}} e^{-x^2/2}}}
\newcommand{\gennormal}{\ensuremath{\frac{1}{\sigma\sqrt{2\pi}} e^{-(x-\mu)^2/2}}}

%%%%%% estimator, estimate
\newcommand{\hth}{\ensuremath{\hat{\theta}}}
\newcommand{\hTH}{\ensuremath{\hat{\Theta}}}
\newcommand{\MAP}{\ensuremath{\text{MAP}}}
\newcommand{\LMS}{\ensuremath{\text{LMS}}}
\newcommand{\LLMS}{\ensuremath{\text{L}}}
\newcommand{\ML}{\ensuremath{\text{ML}}}

%%%% colored text
\newcommand{\red}[1]{\color{red}#1} 
\newcommand{\cyan}[1]{\color{cyan}#1} 
\newcommand{\magenta}[1]{\color{magenta}#1} 
\newcommand{\blue}[1]{\color{blue}#1} 
\newcommand{\green}[1]{\color{green}#1} 
\newcommand{\white}[1]{\color{white}#1} 
\newcommand{\gray}[1]{\color{gray}#1} 

%%% definition
\newcommand{\defi}{{\color{red} Definition.} } 
\newcommand{\exam}{{\color{red} Example.} } 
\newcommand{\question}{{\color{red} Question.} } 
\newcommand{\thm}{{\color{red} Theorem.} } 
\newcommand{\background}{{\color{red} Background.} } 
\newcommand{\msg}{{\color{red} Message.} } 


\def\ml{\text{ML}}
\def\map{\text{MAP}}

%%%%%%%%%%%%%%%%%%%%%%% old macros that you can ignore %%%%%%%%%%%%%%%%%%%%%%%%

% \def\un{\underline}
% \def\ov{\overline}


% \newcommand{\beq}{\begin{eqnarray*}}
% \newcommand{\eeq}{\end{eqnarray*}}
% \newcommand{\beqn}{\begin{eqnarray}}
% \newcommand{\eeqn}{\end{eqnarray}}
% \newcommand{\bemn}{\begin{multiline}}
% \newcommand{\eemn}{\end{multiline}}
% \newcommand{\beal}{\begin{align}}
% \newcommand{\eeal}{\end{align}}
% \newcommand{\beas}{\begin{align*}}
% \newcommand{\eeas}{\end{align*}}



% \newcommand{\bd}{\begin{displaymath}}
% \newcommand{\ed}{\end{displaymath}}
% \newcommand{\bee}{\begin{equation}}
% \newcommand{\eee}{\end{equation}}


% \newcommand{\vs}{\vspace{0.2in}}
% \newcommand{\hs}{\hspace{0.5in}}
% \newcommand{\el}{\end{flushleft}}
% \newcommand{\bl}{\begin{flushleft}}
% \newcommand{\bc}{\begin{center}}
% \newcommand{\ec}{\end{center}}
% \newcommand{\remove}[1]{}

% \newtheorem{theorem}{Theorem}
% \newtheorem{corollary}{Corollary}
% \newtheorem{prop}{Proposition}
% \newtheorem{lemma}{Lemma}
% \newtheorem{defi}{Definition}
% \newtheorem{assum}{Assumption}
% \newtheorem{example}{Example}
% \newtheorem{property}{Property}
% \newtheorem{remark}{Remark}

% \newcommand{\separator}{
%   \begin{center}
%     \rule{\columnwidth}{0.3mm}
%   \end{center}
% }

% \newenvironment{separation}
% { \vspace{-0.3cm}
%   \separator
%   \vspace{-0.25cm}
% }
% {
%   \vspace{-0.5cm}
%   \separator
%   \vspace{-0.15cm}
% }

% \def\A{\mathcal A}
% \def\oA{\overline{\mathcal A}}
% \def\S{\mathcal S}
% \def\D{\mathcal D}
% \def\eff{{\rm Eff}}
% \def\bD{\bm{D}}
% \def\cU{{\cal U}}
% \def\bbs{{\mathbb{s}}}
% \def\bbS{{\mathbb{S} }}
% \def\cM{{\cal M}}
% \def\bV{{\bm{V}}}
% \def\cH{{\cal H}}
% \def\ch{{\cal h}}
% \def\cR{{\cal R}}
% \def\cV{{\cal V}}
% \def\cA{{\cal A}}
% \def\cX{{\cal X}}
% \def\cN{{\cal N}}
% \def\cJ{{\cal J}}
% \def\cK{{\cal K}}
% \def\cL{{\cal L}}
% \def\cI{{\cal I}}
% \def\cY{{\cal Y}}
% \def\cZ{{\cal Z}}
% \def\cC{{\cal C}}
% \def\cR{{\cal R}}
% \def\id{{\rm Id}}
% \def\st{{\rm st}}
% \def\cF{{\cal F}}
% \def\bz{{\bm z}}
% \def\cG{{\cal G}}
% \def\N{\mathbb{N}}
% \def\bbh{\mathbb{h}}
% \def\bbH{\mathbb{H}}
% \def\bbi{\mathbb{i}}
% \def\bbI{\mathbb{I}}
% \def\R{\mathbb{R}}
% \def\bbR{\mathbb{R}}
% \def\bbr{\mathbb{r}}
% \def\cB{{\cal B}}
% \def\cP{{\cal P}}
% \def\cS{{\cal S}}
% \def\bW{{\bm W}}
% \def\bc{{\bm c}}

% %\def\and{\quad\mbox{and}\quad}
% \def\ind{{\bf 1}}


% \def\bmg{{\bm{\gamma}}}
% \def\bmr{{\bm{\rho}}}
% \def\bmq{{\bm{q}}}
% \def\bmt{{\bm{\tau}}}
% \def\bmn{{\bm{n}}}
% \def\bmcapn{{\bm{N}}}
% \def\bmrho{{\bm{\rho}}}

% \def\igam{\underline{\gamma}(\lambda)}
% \def\sgam{\overline{\gamma}(\lambda)}
% \def\ovt{\overline{\theta}}
% \def\ovT{\overline{\Theta}}
% \def\PP{{\mathrm P}}
% \def\EE{{\mathrm E}}
% \def\iskip{{\vskip -0.4cm}}
% \def\siskip{{\vskip -0.2cm}}

% \def\bp{\noindent{\it Proof.}\ }
% \def\ep{\hfill $\Box$}



%%%%%%%%%% linear algebra macros %%%%%%%%%%%%%%%%%%%%%%%

%--------linsys
%  Use as \begin{linsys}{3}
%           x &+ &3y &+ &a &= &7 \\
%           x &- &3y &+ &a &= &7
%         \end{linsys}
% Remark: TeXbook pp. 167-170 says to put a medmuskip around a +; and that's
% 4/18-ths of an em.  Why does 2/18-ths of an em work?  I don't know, but
% comparing to a regular displayed equation suggests it is right.
% (darseneau says LaTeX puts in half an \arraycolsep.)
\newenvironment{linsys}[2][m]{%
\setlength{\arraycolsep}{.1111em} % p. 170 TeXbook; a medmuskip
\begin{array}[#1]{@{}*{#2}{rc}r@{}}
}{%
\end{array}}

\newsavebox\boxofmathplus
\sbox{\boxofmathplus}{$+$}
\newcommand{\spaceforemptycolumn}{\makebox[\wd\boxofmathplus]{\ }}

%--------grstep
% For denoting a Gauss' reduction step.
% Use as: \grstep{\rho_1+\rho_3} or \grstep[2\rho_5 \\ 3\rho_6]{\rho_1+\rho_3}
% \newcommand{\grstep}[2][\relax]{%
%    \ensuremath{\mathrel{
%        \mathop{\longrightarrow}\limits^{#2\mathstrut}_{
%                                    \begin{subarray}{l} #1 \end{subarray}}}}}

% Advantage of length formulation is that between adjacent
% \grstep's you can add \hspace{-\grsteplength} to make it look not too wide
\newlength{\grsteplength}
\setlength{\grsteplength}{1.5ex plus .1ex minus .1ex}

\newcommand{\grstep}[2][\relax]{%
   \ensuremath{\mathrel{
       \hspace{\grsteplength}\mathop{\longrightarrow}\limits^{#2\mathstrut}_{
                                     \begin{subarray}{l} #1 \end{subarray}}\hspace{\grsteplength}}}}
% If two or more \grsteps are in a row then they need to be tightened
\newcommand{\repeatedgrstep}[2][\relax]{\hspace{-\grsteplength}\grstep[#1]{#2}}

% row swap operation: \rho_1\swap\rho_2
\newcommand{\swap}{\leftrightarrow}

%-------------amatrix
% Augmented matrix.  Usage (note the argument does not count the aug col):
% \begin{amatrix}{2}
%   1  2  3 \\  4  5  6
% \end{amatrix}
\newenvironment{amatrix}[1]{%
  \left(\begin{array}{@{}*{#1}{c}|c@{}}
}{%
  \end{array}\right)
}



%-------------pmat
% For matrices with arguments.
% Usage: \begin{pmat}{c|c|c} 1 &2 &3 \end{pmat}
\newenvironment{pmat}[1]{
  \left(\begin{array}{@{}#1@{}}
}{\end{array}\right)
}



%-------------misc matrices
% \newenvironment{mat}{\left(\begin{array}}{\end{array}\right)}
\newenvironment{detmat}{\left|\begin{array}}{\end{array}\right|}
\newcommand{\deter}[1]{ \mathchoice{\left|#1\right|}{|#1|}{|#1|}{|#1|} }
\newcommand{\generalmatrix}[3]{ %arg1: low-case letter, arg2: rows, arg3: cols 
               \left(
                  \begin{array}{cccc}
                    #1_{1,1}  &#1_{1,2}  &\ldots  &#1_{1,#2}  \\
                    #1_{2,1}  &#1_{2,2}  &\ldots  &#1_{2,#2}  \\
                              &\vdots                         \\
                    #1_{#3,1} &#1_{#3,2} &\ldots  &#1_{#3,#2}
                  \end{array}
               \right)  }

\newcommand{\generaldet}[3]{ %arg1: low-case letter, arg2: rows, arg3: cols 
               \left|
                  \begin{array}{cccc}
                    #1_{11}  &#1_{12}  &\ldots  &#1_{1 #2}  \\
                    #1_{21}  &#1_{22}  &\ldots  &#1_{2 #2}  \\
                              &\vdots                         \\
                    #1_{#3 1} &#1_{#3 2} &\ldots  &#1_{#3 #2}
                  \end{array}
               \right|  }

% With mathtools we can have column entries right flushed
% There is an optional argument \begin{mat}[r]{3} .. \end{mat} for
% right-flushed columns.  Perhaps the rule is that numbers are better 
% right-flushed but if there are any letters it is better centered?
\newenvironment{nmat}[1][c]{\begin{pmatrix*} % disable optional arg [#1] 
      }{\end{pmatrix*}}
% If mat starts with &\vdots get an error; why?  No apparent macro fix, according to texexchange
\newenvironment{vmat}[1][c]{\begin{vmatrix*} % disable optional arg [#1] 
      }{\end{vmatrix*}}
\newenvironment{amat}[2][c]{%
  % disable optional arg \left(\begin{array}{@{}*{#2}{#1}|#1@{}}
  \left(\begin{array}{@{}*{#2}{c}|#1@{}}
}{%
  \end{array}\right)
}
% \newcommand\vdotswithin[1]{% Taken from mathtools.dtx because my TL is not 2011
%   {\mathmakebox[\widthof{\ensuremath{{}#1{}}}][c]{{\vdots}}}}


%------------colvec and rowvec
% Column vector and row vector.  Usage:
%  \colvec{1  \\ 2 \\ 3 \\ 4} and \rowvec{1  &2  &3}
% Colvec takes an optional argument \colvec[r]{x_1 \\ 0}.  Perhaps 
% digits look better right aligned, but if there are any letters it
% needs to be centered?
\newcommand{\colvec}[2][c]{\begin{nmat}[#1] #2 \end{nmat}}
\newcommand{\smallcolvec}[1]{\left(\begin{smallmatrix} #1 \end{smallmatrix}\right)}
% For row vectors, cannot do \newcommand{\rowvec}[1]{\begin{mat} #1 \end{mat}}
% since the delimiters come out too large.
\newcommand{\rowvec}[1]{\setlength{\arraycolsep}{3pt}\left(\begin{matrix} #1 \end{matrix}\right)}



%-------------making aligned columns
% Usage: \begin{aligncolondecimal}{2} 1.2 \\ .33 \end{aligncolondecimal}
% (negative argument centers decimal pt in column).  Also Usage:
% \begin{aligncolondecimal}[0em]{2} 1.2 \\ .33 \end{aligncolondecimal}
% to make the left and right LaTeX-array padding disappear.
\RequirePackage{array}\RequirePackage{dcolumn}
\newenvironment{aligncolondecimal}[2][.1111em]{%
\setlength{\arraycolsep}{#1}
\newcolumntype{.}{D{.}{.}{#2}}\begin{array}{.}}{%
\end{array}}

% Matrix and vector, with numbers centered on decimal point
% Usage: \begin{dmat}{D{.}{.}{1}D{.}{.}{3}}  0  &.123 \\ .2 &.456 \end{dmat}
%  (in the D{.}{.}{number} that is the number of decimal places)
\newlength{\dmatcolsep}\setlength{\dmatcolsep}{5pt}
\newenvironment{dmat}[2][\dmatcolsep]{%
  \setlength{\arraycolsep}{#1}
  \left(\begin{array}{@{}#2@{}}
}{%
  \end{array}\right)}
% Usage: \dcolvec[2]{1.23 \\ 4.56} where the optional argument is the number
% of decimal places.
\newcommand{\dcolvec}[2][-1]{\left(\begin{array}{@{}D{.}{.}{#1}@{}} #2 \end{array}\right)}

%\newcommand{\trans}[1]{ {{#1}^{\mathsf{T}}} } 
\newcommand{\trans}[1]{ {#1}^{\mathsf{T}} } 
\newcommand{\inv}[1]{ {#1}^{-1} } 
\newcommand{\spn}[1]{\ensuremath{\text{span}[#1]} } 
\newcommand{\rk}[1]{\ensuremath{\text{rk}(#1)} } 
\newcommand{\dimm}[1]{\ensuremath{\text{dim}(#1)} } 
\newcommand{\img}[1]{\ensuremath{\text{Im}(#1)} } 
%\newcommand{\norm}[1]{\ensuremath{\left || #1 \right ||} } 
\newcommand{\norm}[1]{\ensuremath{\left \lVert #1 \right \rVert} } 
% orthogonal complement
\newcommand{\ocomp}[1]{\ensuremath{#1^{\bot}} } 
\newcommand{\inner}[2]{\ensuremath{\left\langle #1, #2 \right\rangle} } 
\DeclareMathOperator{\tr}{tr}


% \NewDocumentCommand{\grad}{e{_^}}{%
%   \mathop{}\!% \mathop for good spacing before \nabla
%   \nabla
%   \IfValueT{#1}{_{\!#1}}% tuck in the subscript
%   \IfValueT{#2}{^{#2}}% possible superscript
% }
% \begin{equation*}
%          \begin{nmat}[r]
% 1 &2 &13 \\
%           4  &5  &6
%          \end{nmat}
%       \end{equation*}

% \begin{equation*}
%          \begin{amat}{2}
%           1  &2  &3  \\
%           4  &5  &6
%          \end{amat}
%       \end{equation*}
       
%       \begin{equation*}
%          \begin{pmat}{c|c|c}
% 1 &2 &3 \\
%           4  &5  &6
%          \end{pmat}
%       \end{equation*}

% \begin{equation*}
%          \begin{vmat}
% a &c \\
%           b  &d
%          \end{vmat}
%          =ad-bc
% \end{equation*}

%  \begin{equation*}
%   \vec{v}=\colvec{-1  \\ -0.5  \\ 0}
% \end{equation*}

%  \begin{equation*}
%   \vec{v}=\rowvec{-1  & -0.5  & 0}
% \end{equation*}




%\addtobeamertemplate{footline}{\rule{0.94\paperwidth}{1pt}}{}

\begin{document}

%itemshape
\setbeamertemplate{itemize item}{\scriptsize\raise1.25pt\hbox{\donotcoloroutermaths$\bullet$}}
\setbeamertemplate{itemize subitem}{\tiny\raise1.5pt\hbox{\donotcoloroutermaths$\circ$}}
\setbeamertemplate{itemize subsubitem}{\tiny\raise1.5pt\hbox{\donotcoloroutermaths$\blacktriangleright$}}
%default value for spacing
\plitemsep 0.1in
\pltopsep 0.03in
\setlength{\parskip}{0.15in}
%\setlength{\parindent}{-0.5in}
\setlength{\abovedisplayskip}{0.07in}
\setlength{\belowdisplayskip}{0.07in}
\setlength{\mathindent}{0cm}
\setbeamertemplate{frametitle continuation}{[\insertcontinuationcount]}

\setlength{\leftmargini}{0.5cm}
\setlength{\leftmarginii}{0.5cm}

\setlength{\fboxrule}{0.05pt}
\setlength{\fboxsep}{5pt}


%%%%%%% This should be placed at the end of this file
\logo{\pgfputat{\pgfxy(0.11, 7.4)}{\pgfbox[right,base]{\tikz{\filldraw[fill=dianablue, draw=none] (0 cm, 0 cm) rectangle (50 cm, 1 cm);}\mbox{\hspace{-8 cm}\includegraphics[height=0.7 cm]{../kaist_ee.png}
}}}}

\begin{frame}
  \titlepage
\end{frame}

\logo{\pgfputat{\pgfxy(0.11, 7.4)}{\pgfbox[right,base]{\tikz{\filldraw[fill=dianablue, draw=none] (0 cm, 0 cm) rectangle (50 cm, 1 cm);}\mbox{\hspace{-8 cm}\includegraphics[height=0.7 cm]{../kaist_ee.png}
}}}}

% rule color - gray
\makeatletter
\let\old@rule\@rule
\def\@rule[#1]#2#3{\textcolor{gray}{\old@rule[#1]{#2}{#3}}}
\makeatother






% START START START START START START START START START START START START START

%%%%%%%%%%%%%%%%%%%%%%%%%%%%%%%%%%%%%%%%%%%%%%%%%%%%%%
\begin{frame}{Warm-Up}

{\Large Please watch this tutorial video by Luis Serrano on Support Vector Machine.}

\bigskip

\bigskip

\url{https://youtu.be/Lpr__X8zuE8}

\end{frame}


%%%%%%%%%%%%%%%%%%%%%%%%%%%%%%%%%%%%%%%%%%%%%%%%%%%%%%
\begin{frame}{Roadmap}

\plitemsep 0.1in

\bce[(1)] 

\item Story and Separating Hyperplanes 
\item Primal SVM: Hard SVM 
\item Primal SVM: Soft SVM 
\item Dual SVM 
\item Kernels 
\item Numerical Solution 

\ece
\end{frame}

%%%%%%%%%%%%%%%%%%%%%%%%%%%%%%%%%%%%%%%%%%%%%%%%%%%%%%
\section{L12(1)}
\begin{frame}{Roadmap}

\plitemsep 0.1in

\bce[(1)] 

\item \redf{Story and Separating Hyperplanes}
\item \grayf{Primal SVM: Hard SVM 
\item Primal SVM: Soft SVM 
\item Dual SVM 
\item Kernels 
\item Numerical Solution}

\ece
\end{frame}

%%%%%%%%%%%%%%%%%%%%%%%%%%%%%%%%%%%%%%%%%%%%%%%%%%%%%%
\begin{frame}{Storyline}

\plitemsep 0.1in

\bci

\item (Binary) classification vs. regression

\item A Classification predictor $f:\realD \mapsto \{+1, -1 \},$ where $D$ is the dimension of features.
\item Suppervised learning as in the regression with a given dataset $\{(\vx_1,y_1), \ldots, (\vx_N,y_N) \},$ where our task is to learn the model parameters which produces the smallest classification errors. 

\item SVM
\bci
\item Geometric way of thinking about supvervised learning
\item Relying on empirical risk minimization
\item Binary classification = Drawing a separating hyperplane
\item Various interpretation from various perspectives: geometric view, loss function view, the view from convex hulls of data points 
\eci
\eci
\end{frame}

%%%%%%%%%%%%%%%%%%%%%%%%%%%%%%%%%%%%%%%%%%%%%%%%%%%%%%
\begin{frame}{Hard SVM vs. Soft SVM}

\mypic{0.55}{L12_soft_hard_svm.png}

\plitemsep 0.1in

\bci

\item Hard SVM: Linearly separable, and thus, allow  no classification error 

\item Soft SVM: Non-linearly separable, thus, allow some classification error
\eci
\end{frame}

%%%%%%%%%%%%%%%%%%%%%%%%%%%%%%%%%%%%%%%%%%%%%%%%%%%%%%
\begin{frame}{Separating Hyperplane}

\plitemsep 0.07in

\bci

\item \bluef{Hyperplane} in $\realD$ is a set:
$\{x \mid \trans{a}x=b\}$ where $a\in\realn, a\neq 0, b\in\real$ \hfill \lecturemark{L7(3)}

In other words, $\{ x \mid \trans{a}(x-x_0) =0\},$ where $x_0$ is any point in
the hyperplane, i.e., $\trans{a} x_0 = b.$

\mysmalltwocols{0.2}
{
\item Divides $\realD$ into two {\blue halfspaces}: 
$\{x|\trans{a}x\leq b\}$ and $\{x|\trans{a}x>b\}$
}
{
\vspace{-0.3cm}
\mypic{0.7}{L12_halfspace.png}
}
\vspace{-0.2cm}
\item In our problem, we consider the hyperplane $\trans{\vw}\vx + b=0,$ where $\vw$ and $b$ are the parameters of the model.

\item Classification logic
\aleq{
\begin{cases}
\trans{\vw}\vx_n + b \geq 0 & \ \text{when} \ y_n = +1\cr
\trans{\vw}\vx_n + b < 0 & \ \text{when} \ y_n = -1
\end{cases}
\implies \redf{y_n \big(\trans{\vw}\vx_n +b \big) \geq 0}
}

% \bci
% \item $\trans{\vw}\vx_n + b \geq 0$ when $y_n = +1$
% \item $\trans{\vw}\vx_n + b < 0$ when $y_n = -1$
% \eci
\eci
\end{frame}

%%%%%%%%%%%%%%%%%%%%%%%%%%%%%%%%%%%%%%%%%%%%%%%%%%%%%%
\begin{frame}{Distance bertween Two Hyperplanes}

\plitemsep 0.07in

\bci

\item Consider two hyperplanes $\trans{\vw}\vx - b =0$ and $\trans{\vw}\vx - b= r$, where assume $r >0.$

\item \question What is the distance\footnote{Shortested distance between two hyperplanes.} between two hyperplanes? Answer: \bluef{$\dfrac{r}{\norm{w}}$}
\eci

\vspace{-0.7cm}
\mypic{0.5}{L12_disthyper.png}

% \mysmalltwocols{0.4}
% {
% }
% {

% }


\end{frame}

%%%%%%%%%%%%%%%%%%%%%%%%%%%%%%%%%%%%%%%%%%%%%%%%%%%%%%
\section{L12(2)}
\begin{frame}{Roadmap}

\plitemsep 0.1in

\bce[(1)] 

\item \grayf{Story and Separating Hyperplanes}
\item \redf{Primal SVM: Hard SVM} 
\item \grayf{Primal SVM: Soft SVM 
\item Dual SVM 
\item Kernels 
\item Numerical Solution} 

\ece
\end{frame}

%%%%%%%%%%%%%%%%%%%%%%%%%%%%%%%%%%%%%%%%%%%%%%%%%%%%%%
\begin{frame}{Hard Support Vector Machine}

\plitemsep 0.07in

\bci

\item Assume that the data points are linearly separable.

\item Goal: Find the hyperplane that maximizes the margin between the positive and the negative samples

\item Given the training dataset $\{(\vx_1,y_1), \ldots, (\vx_N,y_N) \}$ 
and a hyperplane $\trans{\vw}\vx + b =0,$ what is the constraint that all data points are $\frac{r}{\norm{w}}$-away from the hyperplane?
$$
y_n \big(\trans{\vw}\vx_n +b \big) \geq \frac{r}{\norm{\vw}}
$$

\item Note that $r$ and $\norm{w}$ are scaled together, so if we fix $\norm{w}=1$, then 
$$
y_n \big(\trans{\vw}\vx_n +b \big) \geq r
$$

\eci
\end{frame}

%%%%%%%%%%%%%%%%%%%%%%%%%%%%%%%%%%%%%%%%%%%%%%%%%%%%%%
\begin{frame}{Hard SVM: Formulation 1}

\plitemsep 0.07in

\bci

\item Maximize the margin, such that all the training data points are well-classified into their classes ($+$ or $-$)
\mycolorbox
{
\vspace{-0.3cm}
\aleq{
\max_{\vw, b, r} \quad &r \cr
\text{subject to} \quad & y_n \big(\trans{\vw}\vx_n +b \big) \geq r, \ \text{for all} \ n=1,\ldots, N, \quad \norm{\vw}=1, \quad r>0
}
}

\eci
\end{frame}

%%%%%%%%%%%%%%%%%%%%%%%%%%%%%%%%%%%%%%%%%%%%%%%%%%%%%%
\begin{frame}{Formulation 2 (1)}

\mycolorbox
{
\aleq{
\max_{\vw, b, r} \quad &r \cr
\text{subject to} \quad & y_n \big(\trans{\vw}\vx_n +b \big) \geq r, \ \text{for all} \ n=1,\ldots, N, \quad \norm{\vw}=1, \quad r>0
}
}
\plitemsep 0.07in
\bci

\item Since $\norm{\vw}=1,$ reformulate $\vw$ by $\vw'$ as: 
$y_n \Big(\dfrac{\trans{\vw'}}{\norm{\vw'}}\vx_n +b \Big) \geq r$
\item Change the objective from $r$ to $r^2.$
\item Define $\vw''$ and $b''$ by rescaling the constraint:
\aleq{
y_n \Big(\frac{\trans{\vw'}}{\norm{\vw'}}\vx_n +b \Big) \geq r \Longleftrightarrow
y_n \Big(\trans{\vw''}\vx_n +b'' \Big) \geq 1, \quad 
\vw'' = \frac{\vw'}{\norm{\vw'}r} \ \text{and} \ b'' = \frac{b}{r}
}
\eci
\end{frame}

%%%%%%%%%%%%%%%%%%%%%%%%%%%%%%%%%%%%%%%%%%%%%%%%%%%%%%
\begin{frame}{Formulation 2 (2)}

\plitemsep 0.07in
\bci

\item Note that $\norm{\vw''} = \frac{1}{r}$
\item Thus, we have the following reformulated problem:
\mycolorbox
{
\vspace{-0.3cm}
\aleq{
\max_{\vw'', b''} \quad &\frac{1}{\norm{\vw''}^2} \cr
\text{subject to} \quad & y_n \big(\trans{\vw''}\vx_n +b'' \big) \geq 1, \ \text{for all} \ n=1,\ldots, N,
}
}
=

\mycolorbox
{
\vspace{-0.3cm}
\aleq{
\min_{\vw, b} \quad &\frac{1}{2} \norm{\vw}^2 \cr
\text{subject to} \quad & y_n \big(\trans{\vw}\vx_n +b \big) \geq 1, \ \text{for all} \ n=1,\ldots, N,
}
}


\eci
\end{frame}

%%%%%%%%%%%%%%%%%%%%%%%%%%%%%%%%%%%%%%%%%%%%%%%%%%%%%%
\begin{frame}{Understanding Formulation 2 Intuitively}

\plitemsep 0.07in
\bci

\item Given the training dataset $\{(\vx_1,y_1), \ldots, (\vx_N,y_N) \}$ 
and a hyperplane $\trans{\vw}\vx + b =0,$ what is the constraint that all data points are $\frac{r}{\norm{w}}$-away from the hyperplane?
$$
y_n \big(\trans{\vw}\vx_n +b \big) \geq \frac{r}{\norm{\vw}}
$$

\item \redf{Formulation 1.} Note that $r$ and $\norm{w}$ are scaled together, so if we fix $\norm{w}=1$, then 
$$
y_n \big(\trans{\vw}\vx_n +b \big) \geq r.
$$
And, \bluef{maximize $r.$}

\item \redf{Formulation 2.} If we fix $r=1,$ then  
$$
y_n \big(\trans{\vw}\vx_n +b \big) \geq 1.
$$
And, minimize $\norm{\vw}$
\eci
\end{frame}


%%%%%%%%%%%%%%%%%%%%%%%%%%%%%%%%%%%%%%%%%%%%%%%%%%%%%%
\section{L12(3)}
\begin{frame}{Roadmap}

\plitemsep 0.1in

\bce[(1)] 

\item \grayf{Story and Separating Hyperplanes}
\item \grayf{Primal SVM: Hard SVM} 
\item \redf{Primal SVM: Soft SVM} 
\item \grayf{Dual SVM 
\item Kernels 
\item Numerical Solution} 

\ece
\end{frame}


%%%%%%%%%%%%%%%%%%%%%%%%%%%%%%%%%%%%%%%%%%%%%%%%%%%%%%
\begin{frame}{Soft SVM: Geometric View}

\plitemsep 0.07in
\bci

\item Now we allow some classification errors, because it's not linearly separable. 

\item Introduce a slack variable that quantifies how much errors will be allowed in my optimization problem
\mytwocols{0.6}
{
\small
\item $\vxi = (\xi_n: n=1, \ldots, N)$
\item $\xi_n$: slack for the $n$-th sample $(\vx_n,y_n)$
\begin{tcolorbox}[colback=red!5!white,colframe=red!75!black]
\vspace{-0.3cm}
\aleq{
\min_{\vw, b} \quad &\frac{1}{2} \norm{\vw}^2 +C\sum_{n=1}^N \xi_n \cr
\text{subject to} \quad & y_n \big(\trans{\vw}\vx_n +b \big) \geq 1 - \xi_n,\cr
& \xi_n \geq 0, \qquad \text{for all} \ n
}
\end{tcolorbox}

\item $C$: Trade-off between width and slack
}
{
%\vspace{-0.4cm}
\mypic{0.75}{L12_softsvm_geo.png}
}

\eci
\end{frame}

%%%%%%%%%%%%%%%%%%%%%%%%%%%%%%%%%%%%%%%%%%%%%%%%%%%%%%
\begin{frame}{Soft SVM: Loss Function View (1)}

\plitemsep 0.07in
\bci

\item From the perspective of empirical risk minimizaiton

\item Loss function design
\bci
\item \bluef{zero-one loss} $\mathbf{1}(f(x_n) \neq y_n)$: \# of mismatches between the prediction and the label $\implies$ combinatorial optimization (typically NP-hard)

\item \bluef{hinge loss}
$$
\ell(t) = \max(0,1-t), \ \text{where} \ t = y f(\vx) = y(\trans{\vw}\vx + b)
$$

\mysmalltwocols{0.4}
{
\bci
\item If $\vx$ is really at the correct side, $t \geq 1$ $\rightarrow$ $\ell(t) =0$
\item If $\vx$ is at the correct side, but too close to the boundary, $0 < t < 1$ \\$\rightarrow$ $0< \ell(t) =1-t <1$
\item If $\vx$ is at the wrong side, $ t < 0$ \\$\rightarrow$ $1 < \ell(t) =1-t$
\eci
}
{
\mypic{0.8}{L12_hingeloss.png}
}

\eci

\eci
\end{frame}

%%%%%%%%%%%%%%%%%%%%%%%%%%%%%%%%%%%%%%%%%%%%%%%%%%%%%%
\begin{frame}{Soft SVM: Loss Function View (2)}

\mycolorbox{
\vspace{-0.3cm}
\aleq{
\min_{\vw, b} \ \text{(regularizer + loss)} = \min_{\vw, b} \quad \frac{1}{2} \norm{\vw}^2 +C\sum_{n=1}^N \max \{0,1- y(\trans{\vw}\vx + b) \}
}
}
\plitemsep 0.1in
\bci

\item $\frac{1}{2}\norm{\vw}^2$: L2-regularizer (margin maximization = regularization)

\item $C$: regularization parameter, which moves from the regularization term to the loss term
\item Why this loss function view = geometric view?
\aleq{
\min_t \max(0,1-t) \Longleftrightarrow \min_{\xi,t} \xi, \ \text{subject to} \ \xi \geq 0, \ \xi \geq 1-t
}

\eci
\end{frame}

%%%%%%%%%%%%%%%%%%%%%%%%%%%%%%%%%%%%%%%%%%%%%%%%%%%%%%
\section{L12(4)}
\begin{frame}{Roadmap}

\plitemsep 0.1in

\bce[(1)] 

\item \grayf{Story and Separating Hyperplanes}
\item \grayf{Primal SVM: Hard SVM} 
\item \grayf{Primal SVM: Soft SVM} 
\item \red{Dual SVM} 
\item \grayf{Kernels 
\item Numerical Solution} 

\ece
\end{frame}


%%%%%%%%%%%%%%%%%%%%%%%%%%%%%%%%%%%%%%%%%%%%%%%%%%%%%%
\begin{frame}{Dual SVM: Idea}

\begin{tcolorbox}[colback=red!5!white,colframe=red!75!black]
\vspace{-0.3cm}
\aleq{
\min_{\vw, b} \quad &\frac{1}{2} \norm{\vw}^2 +C\sum_{n=1}^N \xi_n \cr
\text{subject to} \quad & y_n \big(\trans{\vw}\vx_n +b \big) \geq 1 - \xi_n, \ \xi_n \geq 0, \quad \text{for all} \ n
}
\end{tcolorbox}

\vspace{-0.3cm}
\plitemsep 0.05in
\bci

\item The above primal problem is a convex optimization problem. 

\item Let's apply Lagrange multipliers, find another formulation, and see what other nice properties are shown \hfill \lecturemark{L7(2), L7(4)}

\item Convert the problem into "$\leq$" constraints, so as to apply \redf{min-min-max} rule
\mycolorbox{
\vspace{-0.3cm}
\aleq{
\min_{\vw, b} \ \frac{1}{2} \norm{\vw}^2 +C\sum_{n=1}^N \xi_n, \  
\text{s.t.} \  -y_n \big(\trans{\vw}\vx_n +b \big) \leq -1 + \xi_n, \ -\xi_n \leq 0, \quad \text{for all} \ n
}
}

% \item Lagrangian
% \aleq{
% \cL(\vw, b, \vxi, \valpha, \vgamma) = \frac{1}{2} \norm{\vw}^2 +C\sum_{n=1}^N \xi_n
% - \sum_{n=1}^N \alpha_n\Big[y_n \big(\trans{\vw}\vx_n +b \big) -1 + \xi_n \Big] - \sum_{n=1}^N \gamma_n \xi_n
% }

\eci


\end{frame}

%%%%%%%%%%%%%%%%%%%%%%%%%%%%%%%%%%%%%%%%%%%%%%%%%%%%%%
\begin{frame}{Applying Lagrange Multipliers (1)}

\mycolorbox{
\vspace{-0.3cm}
\aleq{
\min_{\vw, b} \ \frac{1}{2} \norm{\vw}^2 +C\sum_{n=1}^N \xi_n, \  
\text{s.t.} \  -y_n \big(\trans{\vw}\vx_n +b \big) \leq -1 + \xi_n, \ -\xi_n \leq 0, \quad \text{for all} \ n
}
}
\vspace{-0.5cm}
\plitemsep 0.05in
\bci

\item Lagrangian with multipliers $\alpha_n \geq 0$ and $\gamma_n \geq 0$
\aleq{
\cL(\vw, b, \vxi, \valpha, \vgamma) = \frac{1}{2} \norm{\vw}^2 +C\sum_{n=1}^N \xi_n
- \sum_{n=1}^N \alpha_n\Big[y_n \big(\trans{\vw}\vx_n +b \big) -1 + \xi_n \Big] - \sum_{n=1}^N \gamma_n \xi_n
}

\item Dual function: $\cD(\valpha,\vgamma) = \inf_{\vw, b, \vxi} \cL(\vw, b, \vxi, \valpha, \vgamma)$ for which the followings should be met:
\small
\aleq{
\text{\blue (D1)} \ \pd{\cL}{\vw} = \trans{\vw} - \sum_{n=1}^N \alpha_n y_n \trans{\vx}_n = 0, \ \text{\blue (D2)} \ \pd{\cL}{b} = \sum_{n=1}^N \alpha_n y_n =0 , \ \text{(\blue D3)} \ \pd{\cL}{\xi_n} = C - \alpha_n - \gamma_n = 0
}
\eci


\end{frame}

%%%%%%%%%%%%%%%%%%%%%%%%%%%%%%%%%%%%%%%%%%%%%%%%%%%%%%
\begin{frame}{Applying Lagrange Multipliers (2)}

\plitemsep 0.07in
\bci

\item Dual function $\cD(\valpha,\vgamma) = \inf_{\vw, b, \vxi} \cL(\vw, b, \vxi, \valpha, \vgamma)$ with \bluef{(D1)} is given by:
\aleq{
\cD(\valpha,\vgamma) &= \frac{1}{2} \sum_{i=1}^N \sum_{j=1}^N y_i y_j \alpha_i \alpha_j 
\inner{\vx_i}{\vx_j} - \redf{\sum_{i=1}^N y_i \alpha_i} \inner{\sum_{j=1}^N y_j \alpha_j \vx_j}{\vx_i} -b \redf{\sum_{i=1}^N y_i \alpha_i} \cr
&  + \sum_{i=1}^N \alpha_i + \sum_{i=1}^N \magenf{(C-\alpha_i -\gamma_i)}\xi_i
}

\item From \redf{(D2)} and \magenf{(D3)}, the above is simplified into:
\aleq{
\cD(\valpha,\vgamma) = \frac{1}{2} \sum_{i=1}^N \sum_{j=1}^N y_i y_j \alpha_i \alpha_j 
\inner{\vx_i}{\vx_j} + \sum_{i=1}^N \alpha_i 
}

\item $\alpha_i, \gamma_i \geq 0$ and $C-\alpha_i-\gamma_i =0$ $\implies$ $ 0 \le \alpha_i \le C$
\eci


\end{frame}

%%%%%%%%%%%%%%%%%%%%%%%%%%%%%%%%%%%%%%%%%%%%%%%%%%%%%%
\begin{frame}{Dual SVM}

\plitemsep 0.07in
\bci

\item (Lagrangian) Dual Problem: \redf{maximize $\cD(\valpha,\vgamma)$}
\mycolorbox
{
\vspace{-0.3cm}
\aleq{
\min_{\valpha} \quad & \frac{1}{2} \sum_{i=1}^N \sum_{j=1}^N y_i y_j \alpha_i \alpha_j 
\inner{\vx_i}{\vx_j} + \sum_{i=1}^N \alpha_i \cr
\text{subject to} \quad& \sum_{i=1}^N y_i \alpha_i =0, \quad 0 \le \alpha_i \le C, \ \forall i=1, \ldots, N
}
\vspace{-0.2cm}
}
\item Primal SVM: the number of parameters scales as \bluef{the number of features ($D$)}

\item Dual SVM
\bci
\item the number of parameters scales as \bluef{the number of training data ($N$)}
\item only depends on the inner products of individual training data points $\inner{\vx_i}{\vx_j}$ $\rightarrow$ allow the application of \redf{kernel}
\eci

\eci
\end{frame}

%%%%%%%%%%%%%%%%%%%%%%%%%%%%%%%%%%%%%%%%%%%%%%%%%%%%%%
\section{L12(5)}
\begin{frame}{Roadmap}

\plitemsep 0.1in

\bce[(1)] 

\item \grayf{Story and Separating Hyperplanes}
\item \grayf{Primal SVM: Hard SVM} 
\item \grayf{Primal SVM: Soft SVM} 
\item \grayf{Dual SVM} 
\item \redf{Kernels 
\item Numerical Solution} 

\ece
\end{frame}

%%%%%%%%%%%%%%%%%%%%%%%%%%%%%%%%%%%%%%%%%%%%%%%%%%%%%%
\begin{frame}{Kernel}

\mytwocols{0.7}
{
\bigskip

\plitemsep 0.1in
\bci

\item Modularity: Using the feature transformation $\vphi(\vx),$ dual SVMs can be modularized
$$
\inner{\vx_i}{\vx_j} \implies \inner{\vphi(\vx_i)}{\vphi(\vx_j)}
$$

\item Similarity function $k: \cX \times \cX \mapsto \real$, $k(\vx_i,\vx_j) = \inner{\vphi(\vx_i)}{\vphi(\vx_j)}$

\item Kernel matrix, Gram matrix: must be symmetric and positive semidifinite 

\item Examples: polynomial kernel, Gaussian radial basis function, rational quadratic kernel
\eci
}
{
\mypic{0.9}{L12_kernel_ex.png}
}

\end{frame}


%%%%%%%%%%%%%%%%%%%%%%%%%%%%%%%%%%%%%%%%%%%%%%%%%%%%%%
\begin{frame}{Numerical Solution}

\plitemsep 0.07in
\bci

\item 

\eci
\end{frame}


%%%%%%%%%%%%%%%%%%%%%%%%%%%%%%%%%%%%%%%%%%%%%%%%%%%%%%
\begin{frame}{}
\vspace{2cm}
\LARGE Questions?


\end{frame}

%%%%%%%%%%%%%%%%%%%%%%%%%%%%%%%%%%%%%%%%%%%%%%%%%%%%%%
\begin{frame}{Review Questions}
% \tableofcontents
%\plitemsep 0.1in
\bce[1)]
\item 

\ece
\end{frame}


\end{document}
